%auto-ignore
\providecommand{\MainFolder}{..}
\documentclass[\MainFolder/Text.tex]{subfiles}

\begin{document}
\section{Poincar\'e DGA's and Poincar\'e duality models}\label{SubSec:PoincModel}

\Correct[noline,caption={DONE Hom to weak Hom}]{Homotopy to weak homotopy}
\Correct[noline,caption={DONE Add orientation to the data of PDGA}]{Add orientation to the data of PDGA}
\Correct[noline,caption={DONE Change integral to or}]{Change integral to or}


In this section, we restrict to non-negatively graded unital commutative $\DGA$'s, which we denote by $\nnuCDGA$. In this case, the notions of orientation and of cyclic structure agree by Proposition~\ref{Prop:OrAndCyc}.

We modify and combine definitions from~\cite{Van2019} and~\cite{Lambrechts2007} as follows.

\begin{Definition}[Dif.~Poincar\'e duality algebra, $\PDGA$ and formality]\label{Def:PDGA}
A \emph{differential Poincar\'e duality algebra of degree $n$} is a $\nnuCDGA$ $(V,\Dd,\wedge)$ of finite type with orientation~$\Or$ in degree~$n$ such that the induced pairing on $V$ satisfies Poincar\'e duality. If $\Dd = 0$, we call it just \emph{Poincar\'e duality algebra.}

A \emph{Poincar\'e~$\DGA$ (shortly $\PDGA$) of degree $n$} is a $\nnuCDGA$ $(V,\Dd,\wedge)$ together with an orientation $\Or^\H: \H(V) \rightarrow \R$  of degree $n$ which makes $\H(V)$ into a Poincar\'e duality algebra.

A \emph{morphism of $\PDGA$'s} $(V_1,\Dd_1,\wedge_1,\Or^\H_1)$ and $(V_2,\Dd_2,\wedge_2,\Or^\H_2)$ is a $\DGA$-morphism $f: V_1 \rightarrow V_2$ such that the induced map $f_*: \H(V_1) \rightarrow \H(V_2)$ preserves orientation, i.e., it holds $\Or_2^\H \circ f_* = \Or_1^\H$.

A \emph{quasi-isomorphism (or weak equivalence)} of $\PDGA$'s is a morphism of $\PDGA$'s $f: V_1 \rightarrow V_2$ such that $f_*: \H(V_1) \rightarrow \H(V_2)$ is an isomorphism of oriented $\DGA$'s.

Two $\PDGA$'s are \emph{weakly homotopy equivalent (or isomorphic in the homotopy category)} if they are connected by a zig-zag of $\PDGA$-quasi-isomorphisms.

A $\PDGA$ is \emph{formal} if it is weakly homotopy equivalent (as a $\PDGA$) to its homology.
\end{Definition}

It follows from Proposition~\ref{Prop:OrAndCyc} that a differential Poincar\'e duality algebra according to Definition~\ref{Def:PDGA} is precisely a cyclic $\DGA$ from Part~I. In particular, it is finite dimensional. However, if we relax finite type, unitality or commutativity, we obtain a different notion.

\begin{Remark}[Frobenius algebra]
A differential Poincar\'e duality algebra, resp.~a cyclic $\DGA$, is precisely a finite-dimensional symmetric dg-Frobenius algebra from \cite[p.~13]{Vallette2012} or \cite[Theorem~1.1]{Cohen2006}. 
\end{Remark}
%Notice that $\DGA$-morphisms of differential Poincar\'e duality algebras preserve $\Or^V$ if and only if they preserve $\Or^\H$, and in this case, they are injective by \eqref{Lem:AutomaticInjectivity}.

\begin{Definition}[Poincar\'e duality model]\label{Def:PDModel}
A \emph{Poincar\'e duality model} of a $\PDGA$ $(V,\Dd,\wedge,\Or^\H)$ is a differential Poincar\'e duality algebra $(\Model,\Dd^\Model,\wedge^\Model,\Or^\Model)$ which is weakly homotopy equivalent to $V$ as a $\PDGA$.

We call a Poincar\'e duality model \emph{small} if $\VansQuotient(\VansSmall(\Model)) \simeq \Model$ for every Hodge decomposition.
\end{Definition}

\begin{Remark}[On Poincar\'e duality models]
\begin{RemarkList}
\item The definition of a Poincar\'e duality model in~\cite{Lambrechts2007} requires only weak homotopy equivalence of $\DGA$'s, i.e., it does not require quasi-isomorphisms to preserve orientation on homology.

\item In general, a Sullivan minimal model $\Lambda U$ fails easily to be a Poincar\'e duality model because it often has non-zero elements in degree $>n$, e.g., powers of even generators; see Example~\ref{Ex:SphereModel} for $\Sph{2}$. 

For a compact connected Lie group $G$, the subalgebra of harmonic forms $\Harm$ for any biinvariant Riemannian metric is isomorphic to a free algebra on odd generators, see \cite[Chapter~1]{Felix2008}. Therefore, $\Harm$ with zero differential and the induced cyclic structure is at the same time the Sullivan minimal model and a Poincar\'e duality model for $\DR(G)$.

\item Poincar\'e duality models are not ``strongly unique'' in the sense that two Poincar\'e duality models of the same algebra must not be isomorphic; see Example~\ref{Ex:SUsix} for $\mathrm{SU}(6)$. However, there is a ``weak uniqueness'' statement in Proposition~\ref{Prop:LambrechtUnique} below. The situation is similar to the situation with Sullivan and minimal Sullivan models; see \cite{Felix2008}. We introduced ``smallness'' in Definition~\ref{Def:PDModel} as a candidate for a minimality condition on a Poincar\'e duality model which might imply its ``strong uniqueness''; see Question~\ref{Q:QuestionsPonc}.
\qedhere
\end{RemarkList}
\end{Remark}

Consider the functor from $\PDGA$'s to $\DGA$'s which forgets the orientation on homology. We have the following trivial yet somewhat surprising observation.

\begin{Proposition}[$\PDGA$-formality is the same as $\DGA$-formality]\label{Prop:PoincModelOfFormal}
A Poincar\'e $\DGA$ $(V,\Dd,\wedge,\Or^\H)$ is formal (as a $\PDGA$) if and only if it is formal as a $\DGA$.
%In this case, $(\H(V),\wedge,\int)$ is its minimal Poincar\'e duality model.
\end{Proposition}
\begin{proof}
The ``only if'' part is clear.

As for the ``if'' part, let 
\begin{equation}\label{Eq:ZZ}
V \longleftarrow \bullet \dotsb \bullet \longrightarrow \H(V)
\end{equation}
be a weak homotopy equivalence of $\DGA$'s. Denote by $f: \H(V) \rightarrow \H(V)$ the isomorphism on homology induced by \eqref{Eq:ZZ} from the left to the right. We adjoin $\H(V)$ to the right of~\eqref{Eq:ZZ} to obtain the homotopy
\begin{equation}\label{Eq:ZZII}
V \longleftarrow \bullet \dotsb \bullet \longrightarrow \H(V) \xrightarrow{f^{-1}} \H(V)
\end{equation}
whose induced map on homology from the left to the right is the identity. Therefore, we can orient homologies of the inner nodes of~\eqref{Eq:ZZII} so that all maps preserve orientation on homology.
\end{proof}

The next proposition will be used to show the existence of Poincar\'e duality models.

\begin{Proposition}[Extension of Hodge type]\label{Prop:ExtensionOfHodgeType}
Let $V$ be a $\PDGA$ of degree $n\ge 5$ which is of finite type and satisfies $V^0=\Span\{1\}$ and $V^1 = 0$. Then it is a retract of an oriented $\DGA$ $\LambrechtsExtension(V)$ of Hodge type in the category of $\PDGA$'s.
\end{Proposition}

\begin{proof}
Pick an arbitrary harmonic subspace $\Harm$ and an arbitrary complement $C$ of $\ker \Dd$ in $V$.
If $C$ is not perpendicular to $\Harm$, replace it with $\{c - \pi(c)\mid c\in C\}$, where $\pi: V \rightarrow \Harm$ is the orthogonal projection.
We start with $l=\lceil \frac{n}{2} \rceil$ and apply Lemma~\ref{Lemma:Exte} inductively to get an extension $\hat{V} = V \otimes \Lambda$ which admits a decomposition 
\begin{equation}\label{Eq:HatDecomp}
\hat{V} = \hat{\Harm}\oplus\Dd\hat{V}\oplus\hat{C}
\end{equation}
of type \eqref{Eq:DecompOfV} such that there is a complement $\hat{E}$ of $\hat{C}^\perp$ in $\hat{C}$ and a linear map $\hat{\rho}: \hat{E}^{\lceil n/2\rceil}\oplus\dotsb\oplus\hat{E}^n\rightarrow\Dd\hat{V}$ such that \eqref{Eq:ConditionTemp} and \eqref{Eq:ConditionTempII} hold.
We consider a linear map
\[
\kappa: \hat{C} \longrightarrow \Dd \hat{V}
\]
such that
\[
\kappa(e)=
\begin{cases}
	0 & \text{for }e\in \hat{E}^i\text{ with }i<\lceil\frac{n}{2}\rceil,\\
	\hat{\rho}(e) & \text{for }e\in \hat{E}^i\text{ with } i>\lceil\frac{n}{2}\rceil, \\
\end{cases}\quad\text{and}\quad \kappa(c^\perp) = 0\quad\text{for }c^\perp\in\hat{C}^\perp.
\]
The case $n = 2k$ and $e\in \hat{E}^k$ is specified as follows.
If $k$ is even, then $\langle \cdot,\cdot \rangle: \hat{E}^{k}\otimes \hat{E}^{k} \rightarrow \R$ is an inner product, and there is an orthonormal basis $\eta_1$, $\dotsc$, $\eta_m$ for some $m\in\N$.
We require
\begin{equation}\label{Eq:InnerProdCase}
 \kappa(\eta_i) = \frac{1}{2}\hat{\rho}(\eta_i)\quad\text{for all }i=1, \dotsc, m.
\end{equation}
If $k$ is odd, then $\langle \cdot,\cdot \rangle: \hat{E}^{k}\otimes \hat{E}^{k} \rightarrow \R$ is a symplectic form, and there is a symplectic basis $\eta_1$, $\theta_1$, $\dotsc$, $\eta_m$, $\theta_m$ for some $m\in\N$.
We use the convention $\langle \theta_i,\eta_j\rangle = \delta_{ij}$ for $i$, $j=1$,~$\dotsc$, $m$.
We require 
\begin{equation}\label{Eq:SymplCase}
 \kappa(\eta_i) = \hat{\rho}(\eta_i)\quad\text{and}\quad\kappa(\theta_i)= 0\quad\text{for }i=1,\dotsc,m.
\end{equation}
Let 
\[
\hat{C}' \coloneqq \{c - \kappa(c) \mid c\in \hat{C}\}.
\]
This is a complement of $\ker \Dd$ in $\hat{V}$ perpendicular to $\hat{\Harm}$ because $\hat{C}$ is and $\im \kappa \subset\Dd\hat{V}$.
Given homogenous $c_1$, $c_2\in \hat{C}$ with $\deg c_1 + \deg c_2 = n$ and $\deg c_1 \le \deg c_2$, write $c_1 = c^\perp_1 + e_1$ and $c_2 = c^\perp_2 + e_2$ for $c_1^\perp$, $c_2^\perp \in \hat{C}^\perp$ and $e_1$, $e_2\in\hat{E}$, and compute
\begin{align*}
\langle c_1 - \kappa(c_1), c_2 - \kappa(c_2) \rangle &= \langle c_1, c_2 \rangle - \langle \kappa(c_1), c_2 \rangle - \langle c_1, \kappa(c_2) \rangle\\
&=\begin{aligned}[t]
&\underbrace{\langle e_1, e_2 \rangle - \langle \kappa(e_1), e_2 \rangle - \langle e_1, \kappa(e_2) \rangle}_{\eqqcolon(*)} \\ &{}-\underbrace{\langle\kappa(e_1),c_2^\perp\rangle - \langle c_1^\perp, \kappa(e_2)\rangle}_{\eqqcolon(**)}.
\end{aligned}
\end{align*}
Now, $(**)=0$ because of \eqref{Eq:ConditionTempII}.
As for $(*)$, if $\deg c_1 < \deg c_2$, then
\begin{align*}
(**) &= \langle e_1, e_2 \rangle - \langle e_1,\kappa(e_2)\rangle \\
     &= \langle e_1, e_2 \rangle - \langle e_1,\hat{\rho}(e_2)\rangle \\ 
     &= 0
\end{align*}
because of \eqref{Eq:ConditionTemp}.
If $\deg c_1 = \deg c_2 = k$ and $k$ is even, we plug in the orthonormal basis and get using \eqref{Eq:InnerProdCase} that
\begin{align*}
e_1 = \eta_i,\ e_2 = \eta_j:  && (**) &= \langle \eta_i, \eta_j \rangle - \langle \kappa(\eta_i),\eta_j\rangle - \langle \eta_i,\kappa(\eta_j)\rangle \\
&& & = \langle \eta_i, \eta_j\rangle - \langle \eta_j, \kappa(\eta_i)\rangle - \langle \eta_i, \kappa(\eta_j)\rangle \\
&& & = \langle \eta_i, \eta_j \rangle - \frac{1}{2}\langle \eta_j, \hat{\rho}(\eta_i)\rangle - \frac{1}{2}\langle\eta_i,\hat{\rho}(\eta_j)\rangle\\
&& & = \langle \eta_i, \eta_j \rangle - \frac{1}{2}\langle \eta_j, \eta_i \rangle - \frac{1}{2}\langle \eta_i, \eta_j \rangle \\
&& & = 0.
\end{align*}
If $k$ is odd, we plug in the symplectic basis and get using \eqref{Eq:SymplCase} that
\begin{align*}
e_1 = \eta_i,\ e_2 = \eta_j: && (**) &= \langle \eta_j, \kappa(\eta_i) \rangle - \langle \eta_i, \kappa(\eta_j) \rangle \\
&& &= \langle \eta_j, \hat{\rho}(\eta_j) \rangle - \langle \eta_i, \hat{\rho}(\eta_j) \rangle \\
&& &= 0, \\
e_1 = \theta_i,\ e_2 = \eta_j: && (**) &= \langle \theta_i, \eta_j\rangle - \langle \theta_i, \kappa(\eta_j) \rangle \\
&& &= \langle \theta_i, \eta_j\rangle - \langle \theta_i, \hat{\rho}(\eta_j) \rangle \\
&& &= 0, \\
e_1 = \theta_i,\ e_2 = \theta_j: && (**) &= 0.
\end{align*}
This shows that $\hat{C}\perp\hat{C}$, and hence \eqref{Eq:HatDecomp} is a Hodge decomposition.

Finally, because $\hat{V} = V \otimes \Lambda$ as a $\DGA$, both the inclusion $\iota: V \rightarrow \hat{V}$ of $V$ into $\hat{V}_0$ and the projection $\pi: \hat{V} \rightarrow V$ from $\hat{V}_0$ onto $V$ are $\DGA$ morphisms.
Because $\pi \circ \iota = \Id$ and because $\iota_*$ is an orientation preserving isomorphism, $\pi_*$ is an orientation preserving isomorphism as well.
Therefore, $V$ is a retract of $\LambrechtsExtension(V)\coloneqq\hat{V}$ in the category of $\PDGA$'s.
\end{proof}

The next example shows that the Sullivan minimal model is sometimes of Hodge type.

\begin{Example}[Sullivan minimal model of $\Sph{2}$ is of Hodge type]\label{Ex:SphereModel}
The Sullivan minimal model of $\Sph{2}$ is the free graded commutative algebra $\Model\coloneqq\Lambda(\eta_2,\eta_3)$ with $\Abs{\eta_2}=2$, $\Abs{\eta_3}=3$, $\Dd \eta_2 = 0$ and $\Dd\eta_3=\eta_2\wedge\eta_2$.
From degree reasons, it holds $\Model = \Span\{\eta_2^k \eta_3^l \mid k\ge 0, l \in \{0,1\}\}$ as a graded vector space.
We have a canonical decomposition $\Model = \Harm \oplus \Dd \Model \oplus C$, where $\Harm = \Span\{\eta_2\}$, $\Dd \Model = \Span\{\eta_2^k \mid k \ge 2 \}$ and $C = \Span\{\eta_2^{k}\eta_3 \mid k \ge 0\}$.
We define an orientation $\Or : \Model \rightarrow \R$ in degree $2$ by $\Or(\eta_2) \coloneqq 1$ on $\Harm$ and by $0$ on $\Dd \Model$ and $C$.
It is easy to see that $C\perp \Harm$ and $C\perp C$ with respect to the induced cyclic structure $\langle \cdot,\cdot\rangle$. 
Consider the $\DGA$-quasi-isomorphism $f: \Model \rightarrow \DR(\Sph{2})$ defined by $f(\eta_2)\coloneqq \Vol$ and $f(\eta_3)\coloneqq 0$.
Clearly, it is orientation preserving.
We have $\Model/\Model^\perp\simeq \Lambda(\eta_2)$
\end{Example}

The following proposition about the existence of a Poincar\'e dualiy model of a Poincar\'e $\DGA$ $V$ with $\H^1(V)=0$ was originally proven in \cite[Theorem~1.1]{Lambrechts2007}.
It was formulated in the category of $\DGA$'s, i.e., not checking whether the arrows are orientation preserving on homology.
The idea was to construct an extension $\LambrechtsExtension(\Lambda U )$ of the Sullivan minimal model $\Lambda U$ of~$V$ and an orientation on it such that the degenerate subspace is acyclic; the Poincar\'e duality model is then obtained by taking the quotient.
The extension is constructed by adding elements which kill the so called orphans.

By Proposition~\ref{Prop:HodgeAcyc}, we know that $V^\perp$ is acyclic if and only if $V$ is of Hodge type ($V$ needs to be of finite type for the direct implication).
Based on this, we give a new construction of $\LambrechtsExtension(\Lambda U)$ using Lemma~\ref{Lemma:Exte}, i.e., by adding exact partners to non-degenerates.
Our construction works for $n\ge 5$, whereas the assumption of \cite{Lambrechts2007} is $n\ge 7$.
We also do not need $\Dd (\Lambda U)^2 = 0$, although it follows from $\H^1(V) = 0$.
It is also clear from our construction that the arrows preserve orientation on homology.
However, this can be checked for the construction of \cite{Lambrechts2007} as well.

\begin{Proposition}[Existence of Poincar\'e duality model for $\H^1 = 0$]\label{Prop:ExOfLambrStan}
A~Poincar\'e $\DGA$ $V$ with $\H^0(V) = \Span\{1\}$ and $\H^1(V)=0$ admits a Poincar\'e duality model $\Model$.
%If moreover $\H^2 = \H^3 = 0$, then for any two finite dimensional Poincar\'e duality models $M_1$ and $M_2$ there is another finite dimensional Poincar\'e duality model $M_3$ and the zig-zag of orientation preserving quasi-isomorphisms
%$$\begin{tikzcd}
%& M_3 & \\
%M_1\arrow{ur} & & M_2.\arrow{ul}
%\end{tikzcd}$$
\end{Proposition}
\begin{proof}
If $\Or^\H$ comes from a pairing on $V$ which is of Hodge type, then we can take $\Model=\VansQuotient(\VansSmall(V))$. The weak homotopy equivalence of $\PDGA$'s looks like
\begin{equation}\label{Eq:ModOne}
\begin{tikzcd}
 &  \VansSmall(V) \arrow[two heads]{dl}\arrow[hook]{dr} & \\
 \VansQuotient(\VansSmall(V)) & & V.
\end{tikzcd}
\end{equation}
If $V$ is not of Hodge type, we proceed as follows. Let $n$ be the degree of the orientation on $\H(V)$.
If $n\le 6$, then $V$ is formal as a $\DGA$ by~\cite{Miller1979}, and we can take $\Model=\H(V)$ by Proposition~\ref{Prop:PoincModelOfFormal}.
The weak homotopy equivalence of $\PDGA$'s looks like
\begin{equation}\label{Eq:ModTwo}
\begin{tikzcd}
 & \Lambda U \arrow{dl}\arrow{dr} & \\
 \H(V) & & V,
\end{tikzcd}
\end{equation}
where $\Lambda U$ is the Sullivan minimal model of $V$.
The Sullivan minimal model $W\coloneqq \Lambda U$ is of finite type and satisfies $W^0 = \R$, $W^1 = 0$ and $\Dd W^2 = 0$.
Suppose that $n\ge 7$.
Let $\LambrechtsExtension(W)$ be the extension of $W$ of Hodge type either from Proposition~\ref{Prop:ExtensionOfHodgeType} or from \cite[Section~4]{Lambrechts2007}.
This extension is of finite type, the inclusion $W\hookrightarrow\LambrechtsExtension(W)$ is a quasi-isomorphism of $\DGA$'s and $\LambrechtsExtension(W)^\perp$ is acyclic.
Moreover, $W\hookrightarrow\LambrechtsExtension(W)$ preserves orientation on homology.
We take $\Model = \VansQuotient(\LambrechtsExtension(\Lambda U))$ and obtain the following weak homotopy equivalence of $\PDGA$'s:
\begin{equation}\label{Eq:ModThree}\begin{tikzcd}
 & \Lambda U \arrow{dl}\arrow{dr} & \\
\VansQuotient(\LambrechtsExtension(\Lambda V))& & V.
\end{tikzcd}\end{equation}
This proves the proposition.
\end{proof}

The following is mostly \cite[Theorem~7.1]{Lambrechts2007}.
In addition, we check that the orientation on homology is preserved.
Also, by using our extension of Hodge type, we can improve from $n\ge 7$ to $n\ge 5$.

\begin{Proposition}[``Weak uniqueness'' of Poincar\'e duality model]\label{Prop:LambrechtUnique}
Let $V_1$ and $V_2$ be differential Poincar\'e duality algebras of degree $n$ which are weakly homotopy equivalent as $\PDGA$'s.
Suppose that $\H^0(V_1)=\H^0(V_2)=\Span\{1\}$ and $\H^1(V_1) = \H^1(V_2) = 0$.
In addition, suppose that $\H^2(V_1) = \H^2(V_2) = 0$, $V_1^1 = V_2^1 = 0$ and $n\ge 5$.
Then there is a differential Poincar\'e duality algebra $V_3$ and $\PDGA$-quasi-isomorphisms\footnote{These are automatically injective and orientation preserving.}
\begin{equation}\label{Eq:LambrechtsZigZag}
\begin{tikzcd}
& V_3 & \\
V_1\arrow{ur}& & \arrow{ul}V_2.
\end{tikzcd}
\end{equation}
%Moreover, $V_3$ is of finite type, $V_3^0 = \R$, $V_3^1 = 0$ and $\Dd V^3 = 0$.
\end{Proposition}
\begin{proof}
By the assumption, there is $k\ge 1$ and a zig-zag of $\PDGA$-quasi-isomorphisms
\begin{equation}\label{Eq:ZigZag}
V_1 \longleftarrow Z_1 \longrightarrow Z_2 \longleftarrow Z_3 \longrightarrow Z_4 \longleftarrow \dotsb \longleftarrow Z_k \longrightarrow V_2.
\end{equation}
Consider the Sullivan minimal model $\Lambda U \rightarrow Z_2$ and use the Lifting Lemma \cite[Lemma~2.15]{Felix2008} to construct $\DGA$-quasi-isomorphisms $\Lambda U \rightarrow Z_1$ and $\Lambda U \rightarrow Z_3$ such that the diagram
$$\begin{tikzcd}
& \Lambda U \arrow{d} \arrow{ld} \arrow{rd} & \\
Z_1 \arrow{r} & Z_2 & \arrow{l} Z_3
\end{tikzcd}$$
commutes up to homotopy of $\DGA$'s. It is easy to see that there is an orientation on $\H(\Lambda U)$ such that all morphisms preserve orientation on homology. Therefore, we can replace the segment $V_1 \longleftarrow Z_1 \longrightarrow Z_2 \longleftarrow Z_3 \longrightarrow Z_4$ in \eqref{Eq:ZigZag} by $V_1 \longleftarrow \Lambda U \longrightarrow Z_4$. Repeating this process, we can shorten \eqref{Eq:ZigZag} to 
\begin{equation}\label{Eq:DiagDiag}
\begin{tikzcd}
& \Lambda U \arrow{dr}{f_2} \arrow[swap]{dl}{f_1} & \\
V_1 & & V_2,
\end{tikzcd}
\end{equation}
where $f_1$ and $f_2$ are $\PDGA$-quasi-isomorphisms. In order to take $\VansQuotient(\LambrechtsExtension(\Lambda U))$ and obtain a zig-zag with three terms, we have to revert the arrows in \eqref{Eq:DiagDiag}. The trick from~\cite{Lambrechts2007} is the following:

Consider the relative minimal model of the multiplication $\mu: \Lambda U \otimes \Lambda U \rightarrow \Lambda U$; from \cite[Example~2.48]{Felix2008}, it is given by
\begin{equation}\label{Eq:RelMinMod}
\begin{tikzcd}
\Lambda U \otimes \Lambda U\arrow{r}{\mu} \arrow{rd}{i} & \Lambda U \\
& M(\mu)\coloneqq \Lambda U \otimes \Lambda U \otimes \Lambda(U[1]), \arrow{u}{p}
\end{tikzcd}
\end{equation}
where $i$ is the inclusion into the first two factors, which is a cofibration, and $p$ is a surjective quasi-isomorphism. Let $\iota_i : \Lambda U \rightarrow \Lambda U \otimes \Lambda U$ for $i=1$, $2$ be the inclusions to the first and the second factor, respectively. Because $\mu \circ \iota_i = \Id$ and $p$ is a quasi-isomorphism, the maps $i \circ \iota_i : \Lambda U \rightarrow M(\mu)$ for $i=1$, $2$ are quasi-isomorphisms. Moreover, it is easy to see that $\H(M(\mu))$ inherits an orientation such that $p_*$ and $(i\circ \iota_i)_*$ are orientation preserving. To transfer this situation to $V_i$, we use the diagram
\begin{equation}\label{Eq:Pushout}
\begin{tikzcd}
\Lambda U \arrow{r}{f_i}\arrow{d}{\iota_i}& V_1 \arrow{d}{\iota_i^V} \\
\Lambda U \otimes \Lambda U \arrow{r}{f_1\otimes f_2} \arrow{d}{i} & V_1 \otimes V_2 \arrow{d}{g_2} \\
M(\mu) \arrow{r}{g_1} & \tilde{V}_3 \coloneqq  M(\mu) \otimes_{\Lambda U \otimes \Lambda U} (V_1\otimes V_2),
\end{tikzcd}
\end{equation}
where $\iota_i^V : V_i \rightarrow V_1 \otimes V_2$ for $i=1$, $2$ are inclusions. The lower square, i.e., the maps $g_1$, $g_2$ and the $\DGA$ $\tilde{V}_3$, is a pushout diagram (see \cite[Example~1.4]{LoopSpaces}). According to~\cite{MO204414}, the model category of $\nnuCDGA$ is proper, and hence pushouts along cofibrations preserve quasi-isomorphisms. Therefore, $f_1\otimes f_2$ being a quasi-isomorphism implies that $g_1$ is a quasi-isomorphism. We push the orientation to~$\H(\tilde{V}_3)$ via $g_{1*}$. Since $i\circ \iota_i$, $g_1$ and $f_i$ are quasi-isomorphisms preserving orientation on homology, it follows that $h_i\coloneqq g_2\circ \iota_i^V:  V_i \rightarrow V_3$ are quasi-isomorphisms preserving orientation of homology as well. It holds
\[
\tilde{V}_3 \simeq \Lambda(U[1]) \otimes V_1 \otimes V_2.
\]
Clearly, $\tilde{V}_3$ is of finite type.
Using $\H^1(V_i) = \H^2(V_i) = 0$, we have $U^1 = U^2 = 0$, and hence $(\Lambda(U[1]))^1 = 0$.
This together with $V_i^1 = 0$ implies that $\tilde{V}_3^1 = 0$.
%$\tilde{V}_3^1 = 0$ and $\tilde{V}_3^2 = \tilde{V}_1^0 \otimes \tilde{V}_2^0 \otimes U^{3} = 0$ hold due to the additional assumptions.
Therefore, all conditions for an application of the Hodge extension $\LambrechtsExtension$ from Proposition~\ref{Prop:ExtensionOfHodgeType} are satisfied, and we can set
$$ V_3\coloneqq \VansQuotient(\LambrechtsExtension(\tilde{V}_3)). $$
This finishes the proof.
%
%I could have started with Poincar\'e $\DGA$'s (i.e., Poincar\'e duality algebra on homology) and obtain a Poincar\'e DGA $\tilde{V}_3$ and subsequently a Poincar\'e model. Hence quasi-isomorphic $\PDGA$'s, then there exists a Poincar\'e model $V_3$ and the arrows to them.
%
%Suppose we have started with differential Poincar\'e duality algebras, then we obtain isomorphisms of small algebras. 
\end{proof}

\begin{Conjecture}\label{Conj:PDGALST}
The additional assumptions of Proposition~\ref{Prop:LambrechtUnique} can be dropped.
\end{Conjecture}

\begin{Remark}[Weak uniqueness in the case of $\H^1(V)=0$ and $n\le 3$]
 For $n=1$, there is no differential Poincar\'e duality algebra $V$ with $\H^1(V) = 0$.
 
 For $n=2$, a general differential Poincar\'e duality algebra can be written in terms of its Hodge decomposition as
 \begin{align*}
 	V^2 & = \Span\{\Vol\}\oplus \Dd C^1\\
	V^1 & = \Dd C^0 \oplus C^1\\
	V^0 & = \Span\{1\} \oplus C^0.
 \end{align*}
 Now, $\Harm = \Span\{1\}\oplus \Span\{\Vol\}$ is a dg-subalgebra which is itself a Poincar\'e duality algebra. Therefore, two differential Poincar\'e duality algebras with $\H(V_1)\simeq \H(V_2)$ and $\H^1(V_i) = 0$ are connected via the zig-zag
\[
\begin{tikzcd}
& \Harm \arrow{dr}{} \arrow[swap]{dl}{} & \\
V_1 & & V_2.
\end{tikzcd}
\]
For $n=3$, we have 
\begin{align*}
	V^3 & = \Span\{\Vol\} \oplus \Dd C^2\\
 	V^2 & =\Dd C^1 \oplus C^2 \\
	V^1 & = \Dd C^0 \oplus C^1\\
	V^0 & = \Span\{1\} \oplus C^0,
\end{align*}
and the same situation as for $n=2$ occurs.

For $n=4$ and $V^1 = 0$, we have $V \simeq \Harm$.
\end{Remark}

We would like to define a ``minimal Poincar\'e duality model''. We motivate this notion in the following remark.

\begin{Remark}[Model and minimal model]\label{Rem:Models}
We shall understand models and minimal models in terms of model categories and their homotopy categories.

Let us illustrate this on Sullivan models. A Sullivan $\DGA$ is a free graded commutative algebra $\Lambda U$ over a positively graded vector space $U$ which admits a well-ordered homogenous basis $(v_\alpha)$ such that $\Dd v_\alpha \in \Lambda(v_\beta \mid \beta < \alpha)$ ($\coloneqq$\,the subalgebra of $\Lambda U$ generated by the $v_\beta$'s) for all $\alpha$. A Sullivan $\DGA$ is called minimal if $\im \Dd \subset \Lambda_{\ge 2} U$ ($\coloneqq$\,the set of decomposable elements).

According to \cite[Theorem~4.3]{Bousfield1976}, the category $\nnuCDGA$ is a model category with weak equivalences being $\DGA$-quasi-isomorphisms, fibrations being degree-wise surjective $\DGA$-morphisms and cofibrations being retracts of relative Sullivan algebras (see \cite[Proposition~2.22 and Proposition~2.28]{Felix2008}). Cofibrant objects are then precisely Sullivan algebras. 

So, the homotopy extension property holds already for Sullivan $\DGA$'s. To see the role of minimality, we shall descent to the homotopy category. The homotopy category is constructed from a model category by localizing morphisms at weak equivalences. An isomorphism in the homotopy category, called weak homotopy equivalence, corresponds to a zig-zag of weak equivalences. If $V_1$ is weakly homotopy equivalent to $V_2$, we say that $V_2$ is a model of $V_1$. If there is a weak equivalence $V_2 \rightarrow V_1$, we say that $V_2$ is a resolution of~$V_1$. We understand minimality as a condition which is in each weak homotopy equivalence class satisfied by at most one object up to isomorphism in the model category. If minimal models exist, they form a skeleton of the homotopy category. This is precisely the case of $\nnuCDGA$ and minimal Sullivan algebras. Indeed, by \cite[Theorem~2.24]{Felix2008}, every connected $\nnuCDGA$ is resolved by a minimal Sullivan algebra. Next, by \cite[Proposition~2.26]{Felix2008}, a $\DGA$-morphism lifts to resolutions by Sullivan algebras, and by \cite[Corollary~2.13]{Felix2008}, quasi-isomorphic minimal Sullivan $\DGA$'s are isomorphic. Finally, this implies that weakly homotopy equivalent minimal Sullivan algebras are isomorphic, and hence are minimal in the sense above.

As another example, for an operad (or properad) $\Operad$, one wants to construct a dg-operad~$\Operad_\infty$ which is a quasi-free resolution of $\Operad$ (see \cite{Vallette2012}). Quasi-free means that after forgetting the differential, the operad $\Operad_\infty$ is free over $\Perm$-bimodules. This is similar to Sullivan models which are free over vector spaces. For quadratic operads, $\Operad_\infty$ is often constructed as the cobar construction $\Omega$ of the Koszul dual cooperad $\Operad^{\mbox{!`}}$. This is the case of $\AInfty$, $\LInfty$ or of the properad $\IBLInfty$. The differential on $\Omega \Operad^{\mbox{!`}}$ is the extension of the decomposition on~$\Operad^{\mbox{!`}}$ to a derivative, and hence it has decomposable image (c.f., the explicit formula~\cite[Formula~(2)]{Peksova2018}). This is similar to minimal Sullivan models. It would be interesting to know whether $\Operad_\infty$ can be constructed using the same inductive method of ``killing'' and ``adding'' generators of homology as Sullivan minimal models. 

Finally, let us note that in \cite{Cirici2019}, they use the inductive method to construct (minimal) models of $\Operad$-algebras for a wide class of operads $\Operad$. Note that cyclic $\AInfty$-algebras can be formulated in the language of cyclic operads. However, in the case of $\PDGA$'s, we have the non-degenerate pairing on homology and an operadic description is not clear.
\end{Remark}
%\begin{Lemma}
%Let $f: V_1 \rightarrow V_2$ be an orientation preserving quasi-isomorphism of differential Poincar\'e duality algebras $(V_1,\Dd_1,\wedge_1,\Or_1)$ and $(V_2,\Dd_2,\wedge_2,\Or_2)$. Then $f$ is injective and... \todo{What more properties? Dies there exist a dg-ideal $Z$ s.t. $V_2 = Z \oplus f(V_1)$?}  
%\end{Lemma}
%\begin{proof}
%It follows easily that $f$ is injective. Now we have $V = W \oplus W_\perp$. From the cyclicity of $\Dd$ we get $\Dd W_\perp \subset V$. We would like to construct a complementary differential graded ideal. We still haven't used the fact that $W$ is a subalgebra and the cyclicity.
%\end{proof}
%
%\begin{Lemma}
%A cyclic cochain complex which is of finite-type (degreewise finite dimensional) and satisfies Poincar\'e duality is of Hodge type. 
%\end{Lemma}
%\begin{proof}
%
%\end{proof}
%
%\begin{Lemma}
%Let $V$ be of finite type. Then Poincar\'e duality is equivalent to non-degeneracy.
%\end{Lemma}
%
%Therefore, Poincar\'e duality algebras are contained in degrees $0$ $\dots$ $n$.
%
%
%
%\begin{Lemma}
%Any two small algebras are isomorphic.
%\end{Lemma}
%\begin{proof}
%\end{proof}
%
%\begin{Lemma}
%If $V$ is differential Poincar\'e duality algebra of finite type, then $Q(V_{\text{small}})$ is a weakly equivalent differential Poincar\'e duality algebra of finite type.
%\end{Lemma}
%
%\begin{Lemma}
%Let $V$ be a differential Poincar\'e duality algebra of finite type. Then the sequence of taking $Q(\bullet_{smallal}$ stabilizes. I.e. there are going to be isomorphic.
%\end{Lemma}
%\begin{proof}
%Clear for finite dimensional.
%\end{proof}
%
%\begin{Lemma}
%Any two minimal Poincar\'e duality models are isomorphic.
%\end{Lemma}

%Consider small Poincar\'e duality models. Let $(V,\Dd,\wedge,\Or^\H)$ be a $\PDGA$ with $\H^0(V) = \R$ and $\H^1(V)=0$. Proposition~\ref{Prop:ExOfLambrStan} gives a Poincar\'e duality model $\Model$. Proposition~\ref{Prop:PropPropertiessd} asserts that $\VansQuotient(\VansSmall(\Model))$ is small and Proposition~\ref{Prop:HodgeAcyc} asserts that it is weakly homotopy equivalent to $\Model$ and hence to $V$ as a $\PDGA$. Therefore, small Poincar\'e duality models exist. Let $\Model_1$ and $\Model_2$ be two small Poincar\'e duality models of $V$. Pick any of their Hodge decompositions. Proposition~\ref{Prop:LambrechtUnique}, more generally Conjecture~\ref{Conj:PDGALST}, gives the diagram \eqref{Eq:LambrechtsZigZag}, where $V_1 = \Model_1$, $V_2 = \Model_2$ and $V_3$ are differential Poincar\'e duality algebras and the maps, which we denote by $f_1: V_1 \rightarrow V_3$ and $f_2: V_2\rightarrow V_3$, are $\PDGA$-quasi-isomorphisms. Lemma~\ref{Eq:LemSmallSub} asserts that there are Hodge decompositions of $V_3$ with small subalgebras $\VansSmall_1(V_3)$ and $\VansSmall_2(V_3)$ such that $f_1$ and $f_2$ induce pairing preserving isomorphisms $\VansSmall(V_1) \simeq \VansSmall_1(V_3)$ and $\VansSmall(V_2) \simeq \VansSmall_2(V_3)$, respectively. Conjecture~\ref{Conj:UnieqSmal} asserts that $\VansQuotient(\VansSmall_1(V_3)) \simeq \VansQuotient(\VansSmall_2(V_3))$. In total, we have
%$$ \Model_1 \simeq \VansQuotient(\VansSmall(\Model_1)) \simeq \VansQuotient(\VansSmall_1(V_3))\simeq \VansQuotient(\VansSmall_2(V_3)) \simeq \VansQuotient(\VansSmall(\Model_2)) \simeq \Model_2 $$
%as differential Poincar\'e duality algebras.
%\begin{Corollary}[Small Poincar\'e duality models are minimal]\label{Cor:SmalPoinc}
%Small Poincar\'e duality models exist when $\H^0 = \R$ and $\H^1 = 0$ and are unique up to an isomorphism provided that the conjectures from Sections~\ref{SubSec:CycStr} and~\ref{SubSec:PoincModel} are true.
%\end{Corollary}

The following example shows that the zig-zag of a Poincar\'e duality model can not be shortened to one arrow.

\begin{Example}[No Poincar\'e dualiy model with one arrow]\label{Ex:NoOneArrow}
\begin{ExampleList}
\item The closed genus $2$ surface $\Sigma_2$ does not admit a Poincar\'e duality model $A$ with just one arrow $A \rightarrow \DR(\Sigma_2)$. Suppose the contrary. We consider the quasi-isomorphism $f:A\rightarrow \DR(\Sigma_2)$ and compute for homogenous $v_1$, $v_2\in A$ the following:
\begin{align*}
\langle f(v_1),f(v_2)\rangle &= \pm \langle f(v_1)\wedge f(v_2),1 \rangle\\
&= \pm \langle f(v_1\wedge v_2),1 \rangle\\
&= \pm \langle [f(v_1\wedge v_2)],[1]\rangle\qquad\text{(*)}\\
&= \pm \langle f_*[v_1\wedge v_2], f^*[1] \rangle\\
&= \pm \langle [v_1\wedge v_2],[1]\rangle\\
&= \pm \langle v_1\wedge v_2, 1 \rangle\qquad(*) \\
&= \langle v_1, v_2 \rangle.
\end{align*}
Stars hold because the only non-zero case is when $\deg(v_1) + \deg(v_2) = \deg(\langle\cdot,\cdot\rangle)$, and hence $\Dd (v_1\wedge v_2) = 0$ because non-degeneracy implies vanishing of higher degrees. We see that $f$ preserves the pairing on the chain-level and is injective by non-degeneracy. We can thus assume that $A\subset\DR(\Sigma_2)$ is a dg-subalgebra equipped with the restriction of the intersection pairing. Since $\dim(A)<\infty$, there is a Hodge decomposition
\begin{align*}
	A^2 &= \Harm^2 \oplus \Dd C^1\\
	A^1 &= \Harm^1 \oplus \Dd C^0 \oplus C^1\\
	A^0 &= \Harm^0 \oplus C^0.
\end{align*}
If $1\neq f\in C^{\infty}(\Sigma_2)$, then $f^k$, $k\in \N_0$ are linearly independent over $\R$. Therefore, it must hold $\Harm^0 = \Span\{1\}$ and $C^0 = 0$. Duality implies $\Dd C^1 = 0$, and hence $\Harm^2$ is spanned by $\omega \in \DR^2(\Sigma_2)$ with $\int_{\Sigma_2}\omega = 1$. It holds even $C^1=0$ as $\ker \Dd \cap C = 0$ in a Hodge decomposition. Since $A\simeq \H(\Sigma_2)$ as a $\DGA$, there are closed $\alpha_1$, $\beta_1$, $\alpha_2$, $\beta_2\in\DR^1(\Sigma_2)$ such that $\Harm^1 = \Span\{\alpha_1, \beta_1, \alpha_2, \beta_2\}$ and such that for all $x\in\Sigma_2$ the following holds:
\begin{align*}
	\alpha_1(x) \wedge \alpha_2(x) &=  \alpha_1(x) \wedge \beta_2(x) = 0\\
	\quad\alpha_1(x)\wedge\beta_1(x) &= \alpha_2(x)\wedge\beta_2(x) = \omega(x).
\end{align*}
Taking an $x\in \Sigma_2$ with $\omega(x) \neq 0$ gives a contradiction.

\item  The simply-connected $4$-manifold $\CP^{2\# 7}$, where $\#$ denotes the connected sum, does not admit a Poincar\'e duality model $A$ with just one arrow $A \rightarrow \DR(M)$. Similarly as in the proof for $\Sigma_2$, we can restrict to the case $A\subset\DR(\CP^{2 \# 7})$. We obtain $A^4 = \Harm^4 = \langle \omega \rangle$ for a somewhere non-vanishing $4$-form $\omega$ and $\Harm^2 = \langle K_0, \dotsc, K_6 \rangle$ such that for all $x\in \CP^{2\#7}$ the following holds:
\begin{align*}
	K_i(x)\wedge K_i(x) &= \pm \omega(x) \\
	K_i(x) \wedge K_j(x) &= 0.
\end{align*}
We now view $K_i(x)$ as vectors in $\R^6$ so that $K_0(x)\wedge K_j(x) = 0$ corresponds to taking the scalar product. If $\sum_{i=0}^6 \lambda_i K_i(x) = 0$ for some $\lambda_i\in \R$ with $\lambda_{i_0} \neq 0$, then
\[
0 = K_{i_0}(x) \wedge \Bigl(\sum_{i=0}^6 \lambda_i K_i(x)\Bigr) = \pm \lambda_{i_0} \omega(x).
\]
Therefore, $\omega(x) \neq 0$ implies that $K_i(x)$ are linearly independent. Hence, in this case, $K_0(x) \wedge K_i(x) = 0$ for all $i=1$, $\dotsc$, $6$ implies $K_0(x) = 0$, which is a contradiction with $K_0(x) \wedge K_0(x) = \omega(x)$.\qedhere
\end{ExampleList}
\end{Example}


\begin{Questions}\label{Q:QuestionsPonc}
\begin{RemarkList}
%\item Is it possible to prove Proposition~\ref{Prop:LambrechtUnique} for $n\le 6$ when formality holds in an easier way?
\item Give an example of a Sullivan minimal model $f: \Lambda U \rightarrow V$ of a $\PDGA$ $V$ which is not of Hodge type for any orientation such that $f_*: \H(\Lambda U) \rightarrow \H(V)$ is orientation preserving.
\item Can some of the additional assumptions on $V$ in Lemma~\ref{Lemma:Exte} and Proposition~\ref{Prop:ExtensionOfHodgeType} be dropped?
\item Can we describe a model category structure on Poincar\'e $\DGA$'s ?
%\item Sullivan's inductive construction gives a minimal model which resolves $V$. The construction of Proposition~\ref{Prop:ExOfLambrStan} does not resolve $V$ in any case \eqref{Eq:ModOne}, \eqref{Eq:ModTwo} and \eqref{Eq:ModThree}. Is it possible to resolve a $\PDGA$ by a Poincar\'e duality model? It is clearly possible in the geometrically formal case, i.e., when there is a $\DGA$-quasi-isomorphism $\H(V)\rightarrow V$.
%For the sake of comparison with Sullivan's inductive construction, it might be interesting to know whether $\VansQuotient(\VansSmall(\cdot))$ is free over $\DGA$'s of Hodge type and whether its differential has decomposable image.
\item Is smallness or minimal dimension the correct notion of minimality of a Poincar\'e duality model?  \qedhere
\end{RemarkList}
\end{Questions}


\end{document}
