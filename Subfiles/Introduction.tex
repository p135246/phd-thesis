%auto-ignore
\providecommand{\MainFolder}{..}
\documentclass[\MainFolder/Text.tex]{subfiles}

\begin{document}

An \emph{$\IBLInfty$-algebra} is essentially a collection of multilinear operations $\OPQ_{klg}$ with~$k$ inputs,~$l$ outputs and ``genus''~$g$ satisfying certain relations; in particular,~$\OPQ_{110}$ is a boundary operator, and the pair $\OPQ_{210}$, $\OPQ_{120}$ induces the structure of an involutive Lie bialgebra on the homology of $\OPQ_{110}$.
It was introduced in \cite{Cieliebak2015} and applications to string topology, symplectic field theory and higher genus Lagrangian Floer theory were proposed.

This part of the thesis is an attempt to understand its application to \emph{string topology}.
The idea was to carry out some explicit computations according to the plan sketched in~\cite[Section~13]{Cieliebak2015} and test the string topology conjecture (see below).
%It turned out that the main degree argument used in the computation for $\Sph{n}$ extends to a more general case thanks to a tricky Hodge propagator modification from~\cite{Mnev2009}. 

%The operation $\OPQ_{klg}$ can be thought of as being associated to a Riemannian surface with $k$ incoming boundary components,~$l$ outgoing boundary components and genus $g$. The partial composition then corresponds to gluing of surfaces, and the relations encode the boundary structure of a certain compactification of the moduli space of such surfaces. 


The following results from~\cite[Corollary~11.9]{Cieliebak2015} are our starting point.
Precise definitions of all the notions will be given in the next chapters.
$\IBLInfty$-algebras in Part~I will be strict and filtered in the terminology of \cite{Cieliebak2015}.
We denote by $[d]$ the degree shift by $d$; we handle the additional signs in \cite{Cieliebak2015} by working with degree shifted objects.
%
\begin{enumerate}[listparindent=\parindent,label=\textbf{(\Alph*)}]
\item For a finite-dimensional cyclic cochain complex $(V,\Pair,m_1)$ of degree $2-n$, where~$\Pair: V[1]\otimes V[1]\rightarrow\R$ is a pairing of degree $2-n$ and $m_1: V[1] \rightarrow V[1]$ a differential,
%with a non-degenerate graded antisymmetric pairing $\Pair:V[1]\otimes V[1] \rightarrow \R$ (the square bracket denotes the degree shift by $1$) of degree $2-n$ for some $n\in \N$ and a differential $m_1: V[1] \rightarrow V[1]$ satisfying 
%\[ \Pair(m_1(v_1),v_2) = (-1)^{\Abs{v_1}\Abs{v_2}}\Pair(m_1(v_2),v_1)\quad \text{for all }v_1, v_2\in V[1], \]
there is a canonical $\dIBL$-structure $\OPP_{110}$, $\OPP_{210}$, $\OPP_{120}$ of bidegree $(n-3,2)$ on the degree shifted dual cyclic bar complex
\[ \CycC(V)\coloneqq \DBCyc V[2-n] \simeq \Bigl(\bigoplus_{k\ge 1} \bigl(V[1]^{\otimes k} / \text{cyc}\bigr)^{\GD} \Bigr)[2-n], \]
where $\mathrm{cyc}$ stands for cyclic permutations with the Koszul sign and $\GD$ denotes the graded dual.
This structure is denoted by $\dIBL(\CycC(V))$.
\item Let $(\Harm,\Pair,m_1) \subset (V,\Pair,m_1)$ be a subcomplex such that the restriction $\Pair: \Harm[1]\otimes\Harm[1]\rightarrow\R$ is non-degenerate. We apply (A) to $(\Harm,\Pair,m_1)$ to get the canonical $\dIBL$-algebra $\dIBL(\CycC(\Harm)) = (\CycC(\Harm),\OPQ_{110},\OPQ_{210},\OPQ_{120})$. 
Suppose that $\pi: V[1] \rightarrow V[1]$ is a projection to $\Harm[1]$ which satisfies
\begin{equation*}
 \begin{aligned}
 \pi \circ m_1 &= m_1 \circ \pi \quad\text{and}\\ \Pair(\pi(v_1),v_2) &= \Pair(v_1,\pi(v_2))
\end{aligned}
\end{equation*}
for all $v_1$, $v_2 \in V[1]$, and let $\iota: \Harm[1]\rightarrow V[1]$ be the inclusion. A linear map $\Htp: V[1]\rightarrow V[1]$ of degree $-1$ such that
\begin{equation}\label{Eq:ConditionOnG}
\begin{aligned}
m_1 \circ \Htp + \Htp \circ m_1 &= \iota\circ \pi - \Id_{V[1]}\quad\text{and} \\ 
\Pair(\Htp(v_1),v_2) &= (-1)^{\Abs{v_1}}\Pair(v_1,\Htp(v_2)) 
\end{aligned}
\end{equation}
for all $v_1$, $v_2 \in V[1]$ induces an $\IBLInfty$-homotopy equivalence 
\[\HTP=(\HTP_{klg}): \dIBL(\CycC(V)) \longrightarrow \dIBL(\CycC(\Harm)) \]
such that $\HTP_{110}: \CycC(V)[1] \rightarrow \CycC(\Harm)[1]$ is the map given by the precomposition with~$\iota$ in every component.
We recall from \cite{Cieliebak2015} that $\HTP_{klg}: \Ext_k \CycC(V) \rightarrow \Ext_l \CycC(\Harm)$ is a linear map between exterior powers of $\CycC$ which are realized as symmetric powers of the degree shift $\CycC[1]$.
%Recall from \cite{Cieliebak2015} that $\HTP_{klg}: \hat{\Ext}_k \CycC(V)\rightarrow \hat{\Ext}_l \CycC(\Harm)$ are linear maps between completed exterior powers.
%(Theorem 11.3) 
\end{enumerate}

The map $\HTP_{klg}$ is constructed as a sum of contributions coming from isomorphism classes of \emph{ribbon graphs} (=:\,multigraphs with a cyclic ordering of half-edges at every internal vertex) with~$k$ internal vertices, $l$ boundary components and genus $g$.
To compute the contribution of a labeled ribbon graph~$\Gamma$ to the value 
\[ \HTP_{klg}(\Psi_1\otimes \dotsb \otimes \Psi_k)(\W_1 \otimes \dotsb\otimes \W_l)\] 
for $\Psi_1$, $\dotsc$, $\Psi_k \in \DBCyc V[3-n]$ and  $\W_1$, $\dotsc$, $\W_l\in \BCyc \Harm [3-n]$, we decorate the~$i$-th internal vertex of $\Gamma$ with $\Psi_i$, external vertices lying on the $i$-th boundary component with components $v_{i1}$, $\dotsc$, $v_{i s_i}\in V[1]$ of $\W_i = \Susp (v_{i 1} \otimes \dotsb \otimes v_{i s_i} / \text{cyc})$, where $\Susp$ is a formal symbol of degree $n-3$, and internal edges with the Schwartz kernel $\Prpg$ of $\Htp$ with respect to $\Pair$.
Decorated ribbon graphs are then evaluated in a consistent way to obtain real numbers (see Appendix \ref{Section:Appendix} for an invariant formalism or \cite[Section 10]{Cieliebak2015} for the original definition in coordinates).


We will also use the following results from \cite[Proposition~12.5 and Theorem~12.9]{Cieliebak2015} about deformations of $\IBLInfty$-algebras:
%(ignoring technical difficulties coming from algebraic completions):
\begin{enumerate}[resume,listparindent=\parindent,label=\textbf{(\Alph*)}]
 \item If in addition to (A) there is a product $m_2 : V[1]\otimes V[1] \rightarrow V[1]$ making $(V,m_1,m_2)$ into a cyclic dga, then $(-1)^{n-2} m_2^+\in \CycC(V)$, where $m_2^+ = \Pair(m_2(\cdot,\cdot),\cdot)$, defines a canonical Maurer-Cartan element $\MC\coloneqq(\MC_{10})$ for $\dIBL(\CycC(V))$.
The twisted $\IBLInfty$-algebra is again a $\dIBL$-algebra of bidegree $(n-3,2)$; it is denoted by $\dIBL^\MC(\CycC(V))$ and satisfies
\begin{equation} \label{Eq:CanonMCTwist}
\begin{aligned}
&\dIBL^\MC(\CycC(V)) \\ 
&\qquad = (\CycC(V),\OPP^\MC_{110}=\OPP_{110}+\OPP_{210}\circ_1\MC_{10},\ \OPP^\MC_{210}=\OPP_{210},\ \OPP^\MC_{120} = \OPP_{120}).
\end{aligned}
\end{equation}
\item The $\IBLInfty$-morphism $\HTP$ from (B) can be used to pushforward $\MC$ and obtain a Maurer-Cartan element $\PMC = (\PMC_{lg})$ for $\dIBL(\CycC(\Harm))$.
The twist by $\PMC$ is an $\IBLInfty$-algebra of bidegree $(n-3,2)$; it is denoted by $\dIBL^\PMC(\CycC(\Harm))$ and satisfies
\[\begin{aligned}
& \dIBL^\PMC(\CycC(\Harm)) \\
& \quad = \begin{multlined}[t]
\bigl(\CycC(\Harm), \OPQ_{110}^\PMC = \OPQ_{110} + \OPQ_{210}\circ_1 \PMC_{10},\ \OPQ_{210}^\PMC=\OPQ_{210},\ \OPQ_{120}^\PMC = \OPQ_{120} \\+ \OPQ_{210} \circ_1 \PMC_{20},
\text{ plus higher operations }\OPQ_{1lg}^\PMC = \OPQ_{210}\circ_1 \PMC_{lg} \bigr).
\end{multlined} \end{aligned} \]
This $\IBLInfty$-algebra is $\IBLInfty$-homotopy equivalent to $\dIBL^\MC(\CycC(V))$ via a twisted $\IBLInfty$-morphism 
\[\HTP^\MC=(\HTP^\MC_{klg}): \dIBL^\MC(\CycC(V)) \longrightarrow \dIBL^\PMC(\CycC(\Harm)). \]
\end{enumerate}

The pushforward Maurer-Cartan element $\PMC = \HTP_* \MC$ can be expressed as a sum of contributions of isomorphism classes of \emph{trivalent ribbon graphs} ($m_2^+$ has namely three inputs), where a labeled ribbon graph $\Gamma$ is decorated with $m_2^+$ at internal vertices, with the components of the $i$-th argument of $\PMC_{lg}$, i.e., elements of $\Harm[1]$, at the $i$-th boundary component and with $\Prpg$ at internal edges.
%Whereas  (A) -- (C) can be formulated without completions, infinite sums appear in~$\PMC_{lg}$, and filtration and completions come into play.

The application to string topology of an oriented closed manifold $M$ of dimension~$n$ comes from studying generalizations of (A) -- (D) to the infinite-dimensional oriented dga $(\DR,\allowbreak \Pair,\allowbreak m_1,\allowbreak m_2)$. Here, $\DR = \DR(M)$ is the de Rham complex of $M$ and the maps $\Pair: \DR[1]^{\otimes 2} \rightarrow \R$, $m_1: \DR[1]\rightarrow \DR[1]$ and $m_2: \DR[1]^{\otimes 2} \rightarrow \DR[1]$ are defined for all $\eta$, $\eta_1$, $\eta_2 \in \DR$ as follows:
\begin{equation} \label{Eq:DeRhamDGA}
 \qquad \mathllap{\text{de Rham cyc.~dga}}\left\{ \begin{aligned}
 \Pair(\SuspU\eta_1,\SuspU\eta_2) &\coloneqq (-1)^{\deg \eta_1} \int_M \eta_1\wedge \eta_2, \\ 
 m_1(\SuspU\eta) &\coloneqq \SuspU \Dd \eta,\\[\jot] 
 \quad m_2(\SuspU\eta_1,\SuspU\eta_2) &\coloneqq (-1)^{\deg \eta_1} \SuspU(\eta_1\wedge\eta_2), \end{aligned}\right.
\end{equation}
where $\Dd$ is the de Rham differential, $\wedge$ the wedge product, $\SuspU$ a formal symbol of degree $-1$ and $\deg \eta_1$ is the form-degree of $\eta_1$.

By picking a Riemannian metric on $M$, we obtain the subcomplex of harmonic forms $(\Harm, \Pair, m_1 \equiv 0)$ with the projection $\pi_{\Harm}: \DR \rightarrow \Harm$ coming from the Hodge decomposition.
This cyclic cochain complex shall be taken as the subcomplex in (B).

Because the intersection pairing on $\DR[1]$ is not perfect, one has to restrict the construction in (A) to the subspace $\DBCyc \DR_\infty$ of elements with smooth Schwartz kernel.
Then (A) and (B) work in the setting of the so called Fr\'echet $\IBLInfty$-algebras introduced in \cite[Section 13]{Cieliebak2015}.
However, the element $\MC_{10} \in \DBCyc \DR[3-n]$, which translates into the \emph{Chern-Simons term}
\begin{equation*} \label{Eq:ChernSimons}
 m_2^+(\SuspU\eta_1,\SuspU\eta_2,\SuspU\eta_3) \coloneqq (-1)^{\deg \eta_2}  \int_M \eta_1\wedge \eta_2\wedge\eta_3\quad\text{for all }\eta_1, \eta_2, \eta_3 \in \DR(M),
\end{equation*}
does not define the canonical Maurer-Cartan element $\MC$ in (C) directly because $m_2^+ \not\in \CDBCyc \DR_\infty$.
This also means that one cannot use (D) to conclude the existence of the pushforward Maurer-Cartan element $\PMC$.

Nevertheless, it was proposed to define $\PMC$ formally using a summation over trivalent ribbon graphs as in the finite-dimensional case.
We call such $\PMC$ the \emph{Chern-Simons} or \emph{formal pushforward Maurer-Cartan element.} In order to compute the contribution of a labeled trivalent ribbon graph $\Gamma$ with $k$ internal vertices, $l$ boundary components and genus $g$ to the value
\[ \PMC_{lg}(\Omega_1\otimes \dotsb \otimes \Omega_l), \]
where $\Omega_i = \Susp \omega_i$ for $\omega_1$,~$\dotsc$, $\omega_l \in \BCyc\Harm$, one starts by decorating internal vertices with integration variables $x_1$, $\dotsc$, $x_k$ on the $k$-fold product $M\times \dotsb \times M$, external vertices on the $i$-th boundary component with the components $\alpha_{i1}$,~$\dotsc$, $\alpha_{i s_i} \in\Harm[1]$ of $\omega_i$ and internal edges with the Hodge propagator $\Prpg$.
In the setting of $\DR$ and $\Harm$, $\Prpg$ shall be the Schwartz kernel of a homotopy $\Htp$ in the sense of pseudo-differential operators (it is necessarily singular at the diagonal and smooth outside of it, i.e., $\Prpg \in \DR^{n-1}(M\times M \backslash \Diag)$).
One then takes the wedge product of all forms in the decorated graph in the order and with the sign deduced from the labeling of~$\Gamma$ and computes the integral over $x_1$, $\dotsc$, $x_k$.
Similar integrals appear in \emph{perturbative Chern-Simons quantum field theory}.


Because of the singularity of $\Prpg$ at $\Diag$, the integrand described above is smooth only on the $k$-th configuration space of $M$.
It is not clear that all the integrals converge and that the resulting $(\PMC_{lg})$ are well-defined and satisfy the Maurer-Cartan equation.
The idea of work in progress~\cite{Cieliebak2018} of K.~Cieliebak and E.~Volkov is to use iterated spherical blow-ups of multiple diagonals to resolve the singularities and obtain integrals of smooth forms on compact manifolds with corners; this guarantees integrability. The Maurer-Cartan equation for $\PMC = (\PMC_{lg})$ is then proven with the help of a Stokes' formula on stratified spaces; the key part is to show that contributions of hidden codimension-$1$ faces cancel. This method is similar to the method from~\cite{Axelrod1991} and~\cite{Axelrod1993}, where Feynman integrals of perturbative Chern-Simons theory were considered.

Having $\PMC$, the twisted $\IBLInfty$-algebra $\dIBL^\PMC(\CycC(\Harm))$, which can be equivalently written as $\dIBL^\PMC(\CycC(\H_{\mathrm{dR}}))$ using the Hodge isomorphism $\Harm\simeq \H_{\mathrm{dR}}$, should satisfy the following conjecture:

\newtheorem*{Conject}{String topology conjecture}
\begin{Conject}[{Quotation of \cite[Conjecture~1.12]{Cieliebak2015}}]
Let $M$ be a closed oriented manifold of dimension $n$ and $\H_{\mathrm{dR}}$ its de Rham cohomology. Then there exists an $\IBLInfty$-structure on (a suitable version of) $\DBCyc \H_{\mathrm{dR}}[2-n]$ whose homology equals the cyclic cohomology of the de Rham complex of $M$.
\end{Conject}

The idea is as follows. The \emph{$\Sph{1}$-equivariant homology of the free loop space} $\StringH(\Loop M)$ is for simply-connected $M$ isomorphic to a version of \emph{Connes' cyclic cohomology of the de Rham algebra} $\CycCoH^*(\DR)$. The precise relation will be established in yet another work in progress~\cite{Cieliebak2018} of K.~Cieliebak and E.~Volkov using a chain-map coming from a cyclic version of Chen's iterated integrals. Now, a suitable degree shift of $\CycCoH^*(\DR)$ is isomorphic to the homology of the boundary operator~$\OPQ_{110}^\MC$ of the hypothetical $\dIBL$-algebra $\dIBL^\MC(\CycC(\DR))$. This hypothetical $\dIBL$-algebra is then by (D) hypothetically quasi-isomorphic to the $\IBLInfty$-algebra $\dIBL^\PMC(\CycC(\Harm))$ via the twisted $\IBLInfty$-morphism $\HTP^\MC$.

The space $\StringH(\Loop M, M)$ carries an $\IBL$-structure consisting of the Chas-Sullivan string bracket $\StringOp_2$ and string cobracket $\StringCoOp_2$; \Add[caption={DONE Reduced loop space}]{Write that they are actually defined on the homology relative to constant loops. Actually, write here modulo problems with modding out constant loops or one constant loop.} these operations were defined geometrically on suitably transverse smooth chains in \cite{Sullivan1999} and \cite{Sullivan2002}, respectively. A natural question is: How is the $\IBL$-structure $\StringOp_2$, $\StringCoOp_2$ related to the $\IBL$-structure $\OPQ_{210}^\PMC$, $\OPQ_{120}^\PMC$ induced on $\StringH(\Loop M)$ via the isomorphism from the string topology conjecture? An extended string topology conjecture asserts that these structures agree, and hence the operations $\OPQ_{210}^\PMC$, $\OPQ_{120}^\PMC$ defined on cyclic cochains provide a \emph{chain model} for~$\StringOp_2$, $\StringCoOp_2$. Based on our observations and explicit computations, we formulate an up-to-date version of the string topology conjecture for simply-connected manifolds (see Conjecture~\ref{Conj:StringTopology}).

%A large part of this text consists of setting up the algebraic base for the work with $\dIBL^\PMC(\CycC(\Harm))$. In addition to repeating the theory from~\cite{Cieliebak2015} in a slightly different formalism, we also include the following topics:
%
%\begin{itemize}
%\item A formula for the partial composition $\circ_s$ in terms of operations of the canonical associative bialgebra on the symmetric algebra (Definition~\ref{Def:CircS}); formulas for $\OPQ_{klg}^\PMC$ (Proposition~\ref{Prop:Formulafortwisted}).
%%Definition of the homology of an $\IBLInfty$-algebra and the induced $\IBL$-structure on it (Definition~\ref{Def:HomIBL});
%\item Definition of the cyclic cohomology of $\AInfty$-algebras (Definition~\ref{Def:CycHom}) and its relation to the homology of $\OPQ_{110}^\PMC$ (Proposition~\ref{Prop:CyclicHom}); definitions of the reduced versions (Definitions~\ref{Def:ReducedDual}, \ref{Def:ReduceddIBL} and \ref{Def:StrictlyReduced}) and their relation to the unreduced versions (Propositions~\ref{Prop:Ones} and \ref{Prop:Reduced}).
%\item A coordinate-free framework for the evaluation of labeled ribbon graphs (Definition~\ref{Def:EvalRibGraph} and Proposition~\ref{Prop:GraphPairing}); formal analogy of the finite-dimensional and the de Rham case which we use to obtain signs for the definition of~$\PMC$ (Proposition~\ref{Prop:FinDimAnalog}).
%\item Definition of the Hodge propagator (Definition~\ref{Def:GreenKernel}) and of the Chern-Simons Maurer-Cartan element $\PMC$ (Definition~\ref{Def:PushforwardMCdeRham}). 
%\end{itemize}

Our first result is an explicit computation of $\dIBL^\PMC(\CycC(\HDR({\Sph{n}})))$ by finding a particular Hodge propagator and showing that all integrals which contribute to $\PMC$ vanish for $n\ge 3$; for $n=1$, there is a non-vanishing integral whose value we compute (see Section~\ref{Sec:GreenSphere}); for $n=2$, the existence of a non-vanishing integral remains open.

\begin{IntroThm}[Explicit computation for $\Sph{n}$]\label{IntroThm:A}
Consider the round sphere $\Sph{n}\subset \R^{n+1}$. Define $\NOne\coloneqq \SuspU 1$, $\NVol\coloneqq \SuspU \Vol \in \HDR(\Sph{n})[1]$, where $\Vol$ is the volume form, $1$ the constant one and $\SuspU$ a formal symbol of degree $-1$. There exists a Hodge propagator $\Prpg$ such that the following holds. For the homology of the twisted boundary operator $\OPQ_{110}^\PMC$, we have:
\begin{equation*}
\begin{aligned}
& \HIBL^\PMC(\CycC(\HDR(\Sph{n})))[1] \coloneqq \H(\CDBCyc \HDR(\Sph{n})[3-n],\OPQ_{110}^\PMC) \\
& \qquad =\begin{cases}
\langle \Susp \NVol^{i*}, \Susp \NOne^{2j-1*} \mid i, j \ge 1\rangle & \text{for }n\ge 3 \text{ odd}, \\
\langle \Susp \NVol^{2 i-1*}, \Susp \NOne^{2j-1*} \mid i, j \ge 1\rangle &\text{for }n\text{ even}, \\
 \bigl\langle \Susp\sum_{k=1}^\infty c_k \NVol^{k*}, \Susp \NOne^{2j-1*}\mid c_k\in \R, j \ge 1\bigr\rangle & \text{for }n=1. 
\end{cases}
\end{aligned}
\end{equation*}
Here $\langle \cdot \rangle$ denotes the linear span over $\R$, $*$ the dual and $\Susp$ is a formal symbol of degree $n-3$. The product $\OPQ_{210}^\PMC$ vanishes on $\HIBL^\PMC$ except for the following relations for $n\ge 3$ odd
\[ \OPQ_{210}^\PMC(\Susp \NOne^* \otimes \Susp \NVol^{k*}) = \OPQ_{210}^\PMC(\Susp \NVol^{k*} \otimes \Susp \NOne^*) = -(k-1) \NVol^{k-1*} \]
and the following relations for $n=1$:
\[ \OPQ_{210}^\PMC\Bigl(\Susp \NOne^* \otimes \Susp\sum_{k=1}^\infty c_k \NVol^{k*}\Bigr) = - \Susp \sum_{k=1}^\infty k c_{k+1} \NVol^{k*}.  \]
The coproduct $\OPQ_{120}^\PMC$ as well as all higher operations $\OPQ_{1lg}^\PMC$ vanish on $\HIBL^\PMC$ in every  dimension $n$. For $\Sph{1}$, we have $\OPQ_{120}^\PMC \neq \OPQ_{120}$ on the chain level; i.e., the twisting is non-trivial. For $n\neq 2$, all higher operations vanish on the chain level.


%For every $k\in \N$ and $i=1$, $\dotsc$, $n$, there is a cyclic word $t_{2k-1,i} \in \BCyc \Harm(\CP^n)$ of length $2k-1$ and degree $2i+ 2(k-1)n -1$ such that
%\begin{equation*}
%\HIBL^\PMC(\CycC(\HDR(\CP^n)))[1] = \langle \Susp t_{2k-1,i}^*, \Susp\NOne^{2j-1*} \mid i=1,\dotsc, n; j, k\in \N \rangle,
%\end{equation*}
%where $\Susp$ is a formal symbol of degree $2n-3$. All operations $\OPQ_{210}^\PMC$, $\OPQ_{120}^\PMC$ and $\OPQ_{1lg}^\PMC$ vanish on $\HIBL^\PMC$ in any dimension $2n$.
\end{IntroThm}

If we mod out $\Susp \NOne^{2j-1*}$, i.e., if we consider the point-reduced version\Modify[caption={DONE Point reduced}]{Write Point reduced version.}, then, after dropping~$\Susp$, the results agree with the string topology of $M$ relative to one constant loop and with Chas-Sullivan operations. The only exception is $M=\Sph{1}$. This supports the string topology conjecture for simply-connected manifolds and provides a counterexample for non-simply connected manifolds.

\Correct[caption={DONE Degree shift}]{Write here that we have to drop $\Susp$!}

%The case of $\CP^n$ becomes a standard exercise on the cyclic homology of the graded truncated polynomial algebra (and the operations vanish from degree reasons) after one shows that $\OPQ_{110}^\MC = \OPQ_{110}^\PMC$. This follows from the following result.

%The following claim can be proven independently of the convergence results from yet unfinished \cite{Cieliebak2018}:
%
%
%\begin{IntroThm}[``Baby'' theorem for geometrically formal manifolds]
%Let $M$ be an oriented closed Riemannian manifold which is geometrically formal. Then there exists a Hodge propagator~$\Prpg$ such that all contributions to $\PMC$ coming from trivalent ribbon trees with at least two internal vertices vanish, and hence we have $\PMC_{10} = \MC_{10}$.
%\end{IntroThm}

%A more interesting observation is Proposition~\ref{Prop:PMCEqualsMC}, which claims the following. If the so called assumption $(V_{\NOne})$ is satisfied, which means that contributions of graphs decorated with $\NOne\in \Harm(M)[1]$ at an external vertex vanish, then the following holds for a closed oriented $n$-dimensional manifold $M$:
%
%\begin{description}
%\item[$\IBL$-structure assuming $(V_{\NOne})$:] If $\HDR^1(M) = 0$, then $\PMC_{20} = 0$, and for $n\le 6$ also $\PMC_{10} = \MC_{10}$. Therefore, if both conditions are satisfied, then we do not have to compute any integrals to get the twisted $\IBL$-structure. The~critical dimension is in concordance with the result of \cite{Miller1979} that every simply-connected manifold of dimension at most $6$ is formal.
%
%\item[Higher operations assuming $(V_{\NOne})$:] For $n>3$, all higher operations $\OPQ_{1lg}^\PMC$ with $(l,g)\neq (1,0)$, $(2,0)$ vanish on the chain level. The case $n=1$ is considered in this text, and the surfaces will be handled in \cite{MyPhD}. Therefore, the case $n=3$ is the only dimension where a non-trivial higher genus operation may exist.
%\end{description}
%
%The assumption $(V_{\NOne})$ can be achieved by the choice of a particular Hodge propagator~$\PrpgStd$. We will show in \cite{MyPhD} that $\PrpgStd$ exists at least for flat manifolds; however, we think that it might exist in general (see Conjecture~\ref{Conj:GStd}).


%which is the Schwartz kernel of the operator~$-\CoDd \Htp_\Delta$, where $\CoDd$ is the codifferential and $\Htp_\Delta$ the generalized inverse of the Hodge-de Rham Laplacian~$\Delta = \Dd \CoDd + \CoDd \Dd$; i.e., it is an operator $\DR^*(M) \rightarrow \DR^*(M)$ such that $\Delta \Htp_\Delta = \Id - \pi_\Harm$. The author can prove that $\PrpgStd$ satisfies all the properties of the Hodge propagator except for the fact that it extends smoothly to the spherical blow-up of the diagonal. This holds in the case of flat manifolds, which will be shown in \cite{MyPhD} (there we may also address the general case, which for the time being remains open).

%
%Moreover, the restrictions of $\OPQ_{120}^\PMC$, $\OPQ_{210}^\PMC$ to $\tilde{H}$ vanish.
%
%
%The author is currently working on the following conjecture:

%\begin{Conjecture} \label{Conjecture:One}
%Let $M$ be a closed oriented Riemannian manifold with $H^1_{dR}(M)=0$. If $M$ is geometrically formal, then~\eqref{Eq:ComparisonOfMC} holds.
%%Denote by $\MC = (\MC_{10})$ the canonical Maurer-Cartan element coming from $\wedge$ on $\Harm(M)$. For $n\neq 2$ holds $\PMC = \MC$. For $n=2$ holds $\PMC_{10} = \MC_{10}$ and $\PMC_{l0} = 0$ for $l\neq 1$. 
%%\item{Suppose $M$ is simply connected connected and formal. Denote by $\MC = (\MC_{10})$ the canonical Maurer-Cartan element coming from $\wedge$ on $H_{\mathrm{dR}}(M)$. Then $(\DBCyc\Harm(M)[2-n],(\OPQ_{klg}^\PMC))$ is $\IBLInfty$-homotopy equivalent to $(\DBCyc H_{\mathrm{dR}}(M)[2-n], \OPQ_{110}^\MC, \OPQ^{\MC}_{210}, \OPQ^{\MC}_{120})$.}
%%\end{enumerate}
%%$\OPQ_{110}^\PMC=\OPQ_{110}^\MC$, $\OPQ_{210}^\PMC=\OPQ_{210}$, $\OPQ_{120}^\PMC=\OPQ_{120}$.  
%\end{Conjecture}

%\noindent The proof is based on a generalization of Lemma~\ref{Lemma:ABVanishing} and instead of the rotational symmetry, which we have at hand for $\Sph{n}$, uses some properties of the canonical coexact Schwartz kernel of $\Dd^{-1}$ constructed from the heat kernel. The work reduces to proving that this particular Schwartz kernel extends smoothly to the blow-up, and hence we can use it as a Hodge propagator in our theory.

%
%\noindent Proposition~\ref{Proposition:MCSphere} for $\Sph{1}$ is a counter-example as $\PMC_{20}\neq 0$.

%\noindent A generalization to formal manifolds is also being studied and a similar result as for the pushforward $\AInfty$-structure is expected.

%The main property used to study integrals associated to ribbon graphs for $\Sph{n}$ is rotational symmetry. However, we did not prove vanishing of all of them for $n=2$, $3$:

%\begin{Open}
%Are there any non-zero integrals for $\Sph{2}$ defining non-trivial operations $\OPQ_{klg}$ on the chain level? Are there any non-zero integrals for $\Sph{3}$ with no external vertices?
%\end{Open}
%
%\noindent See Section~\ref{Section:Proof2} for details.

Our second result generalizes the previous explicit computation and shows that in many cases, the twists with $\PMC$ and $\MC$ coincide. Its proof is a combination of facts from Section~\ref{Sec:Vanishing}.

%This is surprising and kind of unfortunate. On the other hand, if we assume the string topology conjecture, we obtain a canonical chain model for Chas-Sullivan string topology operations on the dual cyclic bar complex of $\HDR(M)$ for, e.g., geometrically formal and simply-connected $M$ with $M\neq \Sph{2}$.

\begin{IntroThm}[Triviality of the twist with $\PMC$ on the chain level] \label{IntroThm:B}
Let $M$ be a closed oriented $n$-manifold. There exists a Hodge propagator $\Prpg$, the so called special Hodge propagator, such that the following holds for the twisted $\IBLInfty$-structure $\dIBL^\PMC(\CycC(\HDR(M)))$:
\begin{enumerate}[label=(\arabic*)]
\item For the basic operations $\OPQ_{110}^\PMC = \OPQ_{210}\circ_1 \PMC_{10}$, $\OPQ_{210}^\PMC = \OPQ_{210}$, $\OPQ_{120}^\PMC = \OPQ_{120} + \OPQ_{210}\circ_1 \PMC_{20}$, we have:
\begin{enumerate}
\item If $\HDR^1(M)=0$, then $\PMC_{20}=0$, and hence $\OPQ_{120}^\PMC = \OPQ_{120}$.
\item If $M$ is geometrically formal, then $\PMC_{10} = \MC_{10}$, and hence $\OPQ_{110}^\PMC = \OPQ_{110}^\MC$. (In fact, if in addition $\HDR^1(M)=0$, then $\PMC = \MC$, at least for $n\ne 2$.)
\end{enumerate}
\item For the higher operations $\OPQ_{1lg}^\PMC=\OPQ_{210}\circ_1 \PMC_{20}$ with $(l,g)\neq (1,0)$, $(2,0)$, we have $\PMC_{lg}=0$, and hence $\OPQ_{1lg}^\PMC=0$ with the possible exception of surfaces and $3$-manifolds with $\HDR^1(M)\neq 0$.
\end{enumerate}
\end{IntroThm}

The following are some interesting directions to work on sorted from the most concrete to the most unclear:

\begin{enumerate}[label=(\arabic*)]
\item Improve Theorem~\ref{IntroThm:B} by showing that the higher operations for $\Sph{2}$ vanish. If this is the case, then the statement that for every manifold~$M$ with $\HDR^1(M)=0$ there is a propagator such that all higher operations vanish is true.
\item Let $M$ be a formal and simply-connected manifold. Are  $\dIBL^\PMC(\CycC(\HDR(M)))$ and $\dIBL^\MC(\CycC(\HDR(M)))$ homotopy equivalent as $\IBLInfty$-algebras? If not, we would like to understand the obstruction. 
\item Does the Schwartz kernel $\PrpgStd$ of $\HtpStd=-\CoDd \Delta^{-1}$, the so called standard Hodge propagator, where $\CoDd$ is the codifferential and~$\Delta$ the Hodge-de~Rham Laplacian, extends smoothly to a blow-up? If yes, then it is a canonical special Hodge propagator for which the statement of Theorem~\ref{IntroThm:B} holds.
\item Compute $\dIBL^\PMC(\CycC(\HDR(M)))$ for surfaces $\Sigma_g$ with $g\ge 1$ and formulate and proof a version of string topology conjecture for non-simply connected manifolds. One can consider differential forms with values in a Lie algebra and obtain an $\IBLInfty$-theory with gauge group. How is this structure related to \cite{Goldman1986} and \cite{Andersen1996}?
\item The $\IBLInfty$-theory uses only graphs with at least one external vertex to construct~$\PMC_{lg}$. Let $W_{lg}$ be the real number obtained by summing over graphs with $l$ boundary components, genus $g$ and with no external vertex. How is this related to the partition function and Chern-Simons invariants and does it fit into a ``weak non-reduced $\IBLInfty$-formalism'' and possibly satisfy some relations?
\item What physical field theory has $\Prpg$ as a propagator and trivalent ribbon graphs with harmonic forms as Feynman graphs?
Due to the polarization $\Prpg = \sum_{k=0}^{n-1}\Prpg^{(k)}(x,y)$, such theory must have a field in every degree (the standard $3d$ Chern-Simons theory before gauge fixing has just one $1$-form).
\item Can the $\IBLInfty$-theory be interpreted from the point of view of string field theory or topological strings? Together with symplectic field theory, what is the global picture?
\end{enumerate}

%In the end, let us summarize some existing work on $\IBLInfty$-algebras which helped us to understand $\IBLInfty$-algebras in broader context: In \cite{Muenster2011}, they find an $\IBLInfty$-structure in open-closed string field theory. In~\cite{Doubek2017}, they view $\IBLInfty$-algebras as algebras over the cobar construction of the closed Frobenius properad.\Correct[caption={DONE cobar complex of properad}]{Algebra over the cobar complex of the closed Frobenius properad.}In~\cite{Markl2015}, they consider $\IBLInfty$-algebras as a particular case of $\BVInfty$- or,  more generally, $\mathrm{MV}$-algebras.

%\vspace{2ex}
%\textbf{Acknowledgements:} 
%I thank my Ph.D.~supervisor Prof.~Dr.~Kai Cieliebak for his continuous support, helpful discussions and encouragement during the research. I thank Dr.~Evgeny Volkov for providing me \cite{Cieliebak2018} and \cite{Cieliebak2018b}, for explaining me his work and for helpful discussions. I~thank Prof.~Robert Bryant for suggesting a better notation for the Hodge propagator for spheres in an online discussion. I thank my colleagues Dr.~Alexandru Doicu for checking a tedious sign computation, Alexei Kudryashov for discussing the introductory paragraphs, English and notation, and Thorsten Hertl for answering my questions about algebraic topology, which he understands astonishingly well. I~thank the University of Augsburg for financial support and for being a nice place to pursue my Ph.D..

% I would like to express my deepest gratitude for his help and encouragement.  

\end{document}
