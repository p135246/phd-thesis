%auto-ignore
\providecommand{\MainFolder}{..}
\documentclass[\MainFolder/Text.tex]{subfiles}

\begin{document}
In this chapter, we tackle the question whether the standard Hodge propagator $\StdPrpg$ extends smoothly to the blow-up.

In Section~\ref{Sec:SchwFrom}, we recall the Schwartz kernel theorem (Proposition~\ref{Prop:SchwKer}) and define the Schwartz form of an operator between sections of the exterior bundle of the cotangent bundle (Proposition~\ref{Prop:SchwForm} and Definition~\ref{Def:SDFDF}). We mention that the Schwartz form of a pseudo-differential operator is smooth outside of the diagonal (Proposition~\ref{Prop:ASD}). On the example of the identity (Example~\ref{Ex:IdFE}) we illustrate that the smooth part does not always determine the pseudo-differential operator. We recall basic facts about regularity of the Laplace Green kernel and standard Hodge propagator (Proposition~\ref{Prop:BasicFactsGP}). We formulate the problem of smooth extension to the blow-up (Definition~\ref{Def:Esdas}) and ask whether it is satisfied by $\GKer$ and $\StdPrpg$.

In Section \ref{Sec:TZ}, we study uniqueness of the Hodge homotopy (Proposition~\ref{Prop:UniHO}). We would like to characterize the standard Hodge propagator as a unique primitive to the harmonic kernel satisfying certain properties. We give some ideas how to do it. We also illustrate that one has to be careful with blow-ups (Proposition~\ref{Prop:PatEm}).

In Section \ref{Sec:Hwe}, we consider the heat form approximation. We define the heat form and sum up its computational properties (Proposition~\ref{Prop:Heasd}). We write down formulas for the heat form approximations (Proposition~\ref{Sec:Hwe}). We postulate that the standard Hodge propagator is the codifferential of the Laplacian Green kernel with respect to one variable (Proposition~\ref{Prop:StdCodifInt}).

In Section \ref{Sec:HeatRN}, we study the standard Hodge propagator on $\R^n$. We recall the well-known formulas for the heat form and the Green form on $\R^n$ (Proposition~\ref{Prop:GreenKernelRn}). We compute the standard Hodge propagator in two ways --- as a coderivative of the Green form and as an integral using heat form approximation (Proposition~\ref{Prop:StdHodgePropRn} for $n\ge 2$ and Example~\ref{Ex:SDFSDF} for $n=1$).

In Section \ref{Sec:GrSpgh}, we study the standard Hodge propagator on $\Sph{n}$. We compute it explicitly for $\Sph{1}$ (Example~\ref{Ex:SADQQ}). Next, we study the structure of the space of Hodge propagators on $\Sph{n}$ (Proposition~\ref{Prop:SpaceOfSolnSn}). We show that the Hodge propagator for $\Sph{n}$ constructed in Part~I is coexact (Proposition~\ref{Prop:ArtProsCoexact}). Finally, using the previous results, we prove that the standard Hodge propagator for $\Sph{2}$ extends smoothly to the blow-up and give an explicit formula with two unknown constants (Proposition~\ref{Prop:StdS2}).
\end{document}
