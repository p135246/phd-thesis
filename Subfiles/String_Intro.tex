%auto-ignore
\providecommand{\MainFolder}{..}
\documentclass[\MainFolder/Text.tex]{subfiles}


\begin{document}
%
%In this chapter, we take the canonical $\dIBL$-algebra $\dIBL(\HDR(M))$ of the de Rham cohomology of a  closed oriented Riemannian manifold $M$ and twist it with a Maurer-Cartan element $\PMC$ arising as a formal pushforward of the Chern-Simons term $m_2^+$ from $\DR(M)$ to harmonic forms $\Harm(M)$.

In Section~\ref{Sec:Manifold1}, we consider the cyclic dga's $\DR(M)$, $\H_{\mathrm{dR}}(M)$ and $\Harm(M)$ for a closed oriented $n$-manifold $M$ (Proposition~\ref{Prop:DGAs}) and apply the theory from Section~\ref{Sec:Alg3} to the last two, which are finite-dimensional.

In Section~\ref{Section:Proof1}, we define the admissible Hodge propagator $\Prpg$ (Definition~\ref{Def:GreenKernel}). It is a primitive to the Schwartz kernel $\HKer$ of the harmonic projection~$\pi_\Harm$ (see Proposition~\ref{Lemma:HKer}) outside the diagonal and extends smoothly to the spherical blow-up of the diagonal. These ideas come from an early version of~\cite{Cieliebak2018}. We consider conditions (P1)--(P5) on a linear operator $\Htp$ and its Schwartz kernel $\Prpg$ (see p.~\pageref{ConditionsG}) and show that $\Prpg$ satisfying all these conditions always exists (Proposition~\ref{Prop:ExistenceG}). We also mention the standard Hodge propagator $\PrpgStd$ (see \eqref{Eq:GStdStd}), which might be a canonical Hodge propagator satisfying (P1)--(P5). 

In Section~\ref{Section:Proof2}, we review ribbon graphs, labelings, compatibility of the order and orientation of internal edges, and the edge and vertex order from~\cite{Cieliebak2015} (Definitions~\ref{Def:Ribbon}, \ref{Def:Labeling}, \ref{Def:CompatLabeling} and~\ref{Def:EdgeVertex}). We then define $\PMC$ as a signed sum of integrals of products of Hodge propagators and harmonic forms which are associated to labeled trivalent ribbon graphs (Definition~\ref{Def:PushforwardMCdeRham}). We do not show that these integrals converge and that $\PMC$ satisfies the Maurer-Cartan equation, but we do show all other properties of a Maurer-Cartan element (Lemma~\ref{Lem:MCCond} and Proposition~\ref{Prop:FormalPushforwardProp}). We define the $Y$-graph, trees, circular graphs, vertices of types $A$, $B$, $C$ and their contributions $A_{\alpha_1, \alpha_2}$, $B_\alpha$, $C$, respectively (Definitions~\ref{Def:Graphs} and~\ref{Def:Contributions}). 

In Section~\ref{Sec:Vanishing}, we observe that vanishing of some special vertices in the graphs implies $\PMC_{lg} = \MC_{lg}$. For example, if all graphs, except for the $Y$-graph, with $\NOne$ at an external vertex vanish, which holds if $\Prpg$ satisfies (P4) and (P5) (Proposition~\ref{Prop:COne}), then all higher operations $\OPQ_{1lg}^\PMC$ vanish on the chain level in dimensions $n>3$ (Proposition~\ref{Prop:PMCEqualsMC}). Next, if all graphs with an $A$-vertex vanish, then $\PMC_{10}=\MC_{10}$, and hence $\OPQ_{110}^\PMC = \OPQ_{110}^\MC$ (Proposition~\ref{Prop:Avertexvanish}). We show that $\PMC = \MC$ for simply-connected geometrically formal manifolds with $n\neq 2$ (Proposition~\ref{Prop:GeomForm}). Using the results of \cite{Cieliebak2018}, we argue that the chain complexes of $\OPQ_{110}^\MC$ and~$\OPQ_{110}^\PMC$ are quasi-isomorphic provided~$M$ is simply-connected and formal (Proposition~\ref{Prop:Formal}).

%We suspect that if $M$ is formal and $\HDR^1(M) = 0$, then the $\IBL$-structures on homologies are isomorphic (Conjecture~\ref{Conj:Formality}).

In Section~\ref{Sec:StringTopology}, we recall basic facts about the Chas-Sullivan operations $\StringOp_2$ and $\StringCoOp_2$ on the $\Sph{1}$-equivariant homology of the free loop space and formulate a version of the string topology conjecture for simply-connected manifolds (Conjecture~\ref{Conj:StringTopology}). \Correct[caption={DONE Mistake in reduced notation}]{Correct the notation here, if it is not already corrected. Anyway, define something like RedNTwistHIBL and replace the construction.}
 
\end{document}
