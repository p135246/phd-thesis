%auto-ignore
\providecommand{\MainFolder}{.}
\documentclass[\MainFolder/Text.tex]{subfiles}

\begin{document}
\label{Sec:Alg0}
%In this part, we collect some definitions and facts from linear algebra. 

In Section~\ref{Sec:Alg1a}, we recall weight-grading (Definition \ref{Def:Grading}), Koszul sign (Definition~\ref{Def:Koszul}), degree shift (Definition~\ref{Def:DegreeShift}), filtrations (Definition~\ref{Def:Filtrations}) and completions (Definition~\ref{Def:Completion}).
We prove the K\"unneth formula for completed symmetric (co)homology (Proposition~\ref{Prop:Kuenneth}).

In Section~\ref{Sec:Alg1}, we review basics of $\IBLInfty$-algebras from \cite{Cieliebak2015}.
We define the exterior algebra $\Ext C$ over a graded vector space $C$ as the symmetric algebra~$\Sym$ over~$C[1]$ (Definition~\ref{Def:ExtAlg}) and use the operations $\mu$ and $\Delta$ from the structure of an associative bialgebra on $\Sym(C[1])$ to give explicit formulas for the partial compositions $\circ_{h_1, \dotsc, h_k}$  (Definition~\ref{Def:CircS}).
We use the compositions to define the notion of an $\IBLInfty$-algebra $(\OPQ_{klg})$ on $C$ (Definition~\ref{Def:IBLInfty}), a~Maurer-Cartan element~$(\PMC_{lg})$ (Definition~\ref{Def:MaurerCartan}) and twisted operations $(\OPQ_{klg}^\PMC)$ (Definition~\ref{Def:TwistedOperations}).
We mention that an $\IBL$-algebra according to our definition is an odd degree shift of a classical $\IBL$-algebra (Proposition~\ref{Prop:ClasModIBL}).
We define the induced $\IBL$-structure on homology (Definition~\ref{Def:HomIBL}) and briefly discuss the $\mathrm{BV}$-formalism and mention weak $\IBLInfty$-algebras (Remark~\ref{Rem:BVForm}).
Finally, we summarize the situation for twisted $\dIBL$-algebras (Proposition~\ref{Prop:dIBL}) and briefly discuss higher operations (Remark~\ref{Rem:Higher}).

In Section~\ref{Sec:Alg2}, we define the (weight-reduced) dual cyclic bar-complex $\DBCyc V$ of a graded vector space $V$ (Definition~\ref{Def:BarComplex}) and introduce some notation (Notation~\ref{Def:Notation}).
We then summarize some facts about the completions $\CDBCyc V$ and $\hat{\Ext}_k \DBCyc V$ (Proposition~\ref{Prop:Compl}).
We define the notion of a cyclic $\AInfty$-structure on~$V$ (Definition~\ref{Def:CyclicAinfty}) and its Hochschild and cyclic homology (Definition~\ref{Def:CycHom}).
We recall strict units and strict augmentations (Definition~\ref{Def:AugUnit}), define the reduced dual cyclic bar complex $\RedDBCyc V$ (Definition~\ref{Def:ReducedDual}) and sketch a proof of the fact that the cyclic homology is a direct sum of the reduced cyclic homology and the cyclic homology of the ground field (Proposition~\ref{Prop:Reduced}).
Our version of cyclic homology of a dga $V$ is based on the degree shift $V[1]$.
We undo the degree shift and obtain a version based on $V$, which can be compared to the standard version from~\cite{LodayCyclic} (Proposition~\ref{Prop:DGA}).
We also show that the reduced spaces for a simply connected and connected $V$ are complete (Proposition~\ref{Prop:SimplCon}).



In Section~\ref{Sec:Alg3}, we review the construction of the canonical $\dIBL$-structure $\dIBL(\CycC(V))$ (Definition~\ref{Def:CanonicaldIBL}) and the canonical Maurer-Cartan element $\MC$ (Definition~\ref{Def:CanonMC}) starting from a cyclic dga $(V,\Pair,m_1,m_2)$.
We give formulas for the operations $(\OPQ_{1lg}^\PMC)$ of the $\IBLInfty$-algebra $\dIBL^\PMC(\CycC(V))$ twisted by a Maurer-Cartan element $\PMC$ (Proposition~\ref{Prop:Formulafortwisted}).
We consider the $\AInfty$-structure induced on $V$ by $\PMC_{10}$ (Definition~\ref{Def:MukDef}) and relate its cyclic homology to the homology of $\OPQ_{110}^\PMC$ (Proposition~\ref{Prop:CyclicHom}).
We define the reduced canonical $\dIBL$-algebra $\dIBL(\RedCycC(V))$ (Definition~\ref{Def:ReduceddIBL}) and the notion of a strictly reduced Maurer-Cartan element (Definition~\ref{Def:StrictlyReduced}).
The twisted $\IBLInfty$-structure then splits into the reduced part and the part generated by $\NOne^{i*}$, which we can explicitly compute (Proposition~\ref{Prop:Ones}).
\end{document}
