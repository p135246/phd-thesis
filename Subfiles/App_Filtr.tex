%auto-ignore
%! TEX root = ../Text.tex
\providecommand{\MainFolder}{..}
\documentclass[\MainFolder/Text.tex]{subfiles}
%\setuptodonotes{disable}
\begin{document}
\allowdisplaybreaks
%\chapter{$\BV$-formalism for weak $\IBLInfty$-algebras}
%\label{Section:Appendix}
%
%
%Relevant literature is Markl, Fukaya and possibly recent work of B. Valette dealing with curved properads.
%
%Notice that completion is a functor 
%
%Is the twist of a srict $\IBLInfty$-algebra strict? This is for $\AInfty$ but probably not here. Maybe I am missing somethng, correct that! There are operations with no input present after twisting with $\PMC$!!! They may disturb the $\IBLInfty$-relations.
%
%
%Notice that by twisting something with operations with no output, we get numerical invariants!
%
%
%Let us mention some facts about about operations with no inputs or outputs.
%\begin{itemize}
%\item It is possible to define a weak $\IBLInfty$-algebra without using filtrations? We need maybe to make sense of the operator $\BVOp$. Check that it is really equivalent to the sequence of compositions.
%\item 
%\end{itemize}
%
%IN THE FORMULA FOR MORPHISMS, THERE ARE NO ELEMENTS WITH NO OUTPUTS APPEARING ON THE RIGHT OF $p$ BECAUSE THE SURFACE WOULD BE DISCONNECTED. THEY DO APPEAR, HOWEVER, ON THE LEFT, WHERE AGAIN NO SURFACES WITH NO INPUT APPEAR. ELEMENTS OF NO INPUT AND OUTPUT DO NOT APPEAR AT ALL.
%
%
%THIS HAS TO BE DONE MORE CAREFULY WHAT IS HAPPENING WITH ALL THE STUFF!
%
%
%WHEN WE TWIST WE OBTAIN OERATIOSNS WITH ZERO INPUT WHICH ARE NOT NECESSARY 0 UNLIKE THE SITUATION WITH $\AInfty$
%
%THE OPERATIONS $p_{00g}$ VANISH. HOWEVER, $f_{00g}$ DO NOT HAVE TO VANISH. IS THERE SOME EQUATION WHICH THEY SATISFY?
%
%LET US RECALL WHAT THE UNSTABLE CURVES CAUSE:
%\begin{itemize}
%\item (0,2,0), ()
%\item (1,1,0)
%\end{itemize}
%
%PROBABLY I NEED TO IMPOSE THE UNSTABLE GENUS STRICT FILTRATION CONDITION ONLY ON MORPHISMS, NOT ON THE OPERATIONS !!
%
%NOTICE THAT WE HAVE the part without any 
%\[  \frac{1}{\hbar}\sum_{g\ge 0} \PMC_{0g}(1)\hbar^g \]
%effect on the twisting. What shit is happening here? THESE ARE NOT DEFINED AS ELEMENTS OF THE MC ELEMENT. BUT WE MAY CONSIDER THEM. 
%
%WHY DO I NOT COUNT THE ELEMENTS $\PMC_{0g}$ WITHIN THE TWISTING?
%
%FORGETING OPERATIONS WITH NO INPUT OR WHATEVER DOES NOT CHNANGE IBL?
%
%
%Let us recall that 
%\[ \chi_{klg} = \chi_{k^- l^- g^-} + \sum_{i=1}^r \chi_{k_i l_i g_i} \]
%and this can be used to bound the number $r_1$ of surfaces $\chi_{k_i l_i g_i}\ge 1$. The number of $(1,1,0)$ is bounded by $k = k_1 + k_2 + \dotsb + k_r$. The number of the rest unstable guys is bounded using the norm inequality and the special $\Norm>0$. It is necessary for motphisms but it does not seem necessary for weak $\IBLInfty$-algebras.
%
%
%
%We would like to give sense to this equation also for weak filtered $\IBLInfty$-algebra and in particular Maurer-Cartan elements. The problem is the negative infinity powers.
%
%
%\begin{itemize}
%\item strict 
%\item weak (curved)
%\item filtered
%\item regular
%\end{itemize}
%
%Filtrations are necessary to define weak morphisms. All other cases can be done without.
%
%We have the following two descriptions of the $\IBLInfty$-theory:
%\begin{itemize}
%\item The surface calculus
%\item The $\BV$ formalism
%\item The properadic description by Peksova.
%\end{itemize}
%The surface calculus 
%
%Perhaps work by B. Valette on curved properads will bring order in the jungle.
%
%
%\section{Some details on filtrations and completions}
%
%\section{Some computations in bialgebra calculus}

\chapter{Filtered IBL-infinity-algebras as filtered MV-algebras}\label{App:IBLMV}
\Correct[caption={$BV$ and $BVinfty$}]{Change $\BV$ to $\BVInfty$.}
\Correct[caption={DONE Degree of BV},noline]{Probably $|\BVOp| = -1$ in our formalism. Indeed, it is $-1$. Correct!}
In this appendix, we want to understand the $\BV$-formalism for $\IBLInfty$-algebras and the equivalence
\[\IBLInfty:\quad \begin{gathered}
\underline{\text{Surface calculus}} \\
(\OPQ_{klg}), (f_{klg}), \text{gluing relations}
\end{gathered}
\;\Longleftrightarrow\;
\begin{gathered}
\underline{\BV\text{-formalismus}} \\
 \BVOp, \MVMorF,\text{\,relations }\BVOp^2 = 0, e^\MVMorF \BVOp^+ = \BVOp^- e^\MVMorF.
\end{gathered}\]
%Having this, we want to interpret a Maurer-Cartan element $\PMC$ for $\BVOp$ as a weak morphism of $\BVOp$ and the twisted algebra $\BVOp^\PMC$. Because~$\PMC$ has no inputs, it is given by the multiplication with the actual exponential $e^{\PMC(1)}$. Invertibility of the exponential induces the following seemingly paradoxical situation:\Modify[noline]{Induce all morphisms $\MVMorF$ the isomorphism $e^\MVMorF$?}
%
%\emph{``The twisted operations~$(\OPQ_{klg}^\PMC)$ undoubtedly contain additional information coming from~$\PMC$, yet the $\IBLInfty$-algebras $(\OPQ_{klg})$ and~$(\OPQ_{klg}^\PMC)$ are isomorphic in the weak sense, i.e., allowing no inputs for the morphism.''}
This was originally done in \cite{Cieliebak2015} using filtrations and formal power series in $\hbar$ as a ``bookkeeping''. Nevertheless, the author thinks that some details were not fully addressed and was also curious about extending the $\MV$-formalism from \cite{Markl2015} to the filtered setting.
%In \cite[Section~8]{Cieliebak2015}, a $\BV$-formalism based on formal power series in the Planck's constant~$\hbar$ was developed. However, as stated on~\cite[p.\,10]{Cieliebak2015}, they use it merely as a ``bookkeeping device'' for the surface calculus; in other words, they seem to use the $\BV$-relations only as a mnemonic tool to derive the gluing relations which are then taken as definitions (c.f., \cite[Lemma~8.3]{Cieliebak2015} which deals with the morphisms).
%However, interpreting this ``bookkeeping'' literally might be problematic.\footnote{For example, a Maurer-Cartan element $\PMC$ contains not only positive powers of $\hbar$ but also $\hbar^{-1}$; thus,~$e^{\PMC(1)}$ contains~$\hbar^{n}$ for all $n\in \Z$, and it is not clear that it is possible to multiply with~$e^{\PMC(1)}$. Generally, it is namely not possible to multiply a series with a double-infinite series $\sum_{i=-\infty}^\infty a_i \hbar^i$ because there can be infinitely many terms at a given power of $\hbar$.}

As a matter of fact, the main motivation for better understanding of these details was a brief discussion with Kyler Siegel at a workshop about $\BV$-quantization in Stony Brook, 2019. \Add[noline,caption={DONE Kyler's surname}]{Look up Kyler's surname.}It was pointed out that a twisted $\infty$-algebra is, in some sense, ``isomorphic'' to the untwisted one because the twisting is just conjugation with the exponential of the Maurer-Cartan element. The case of $\AInfty$-algebras from~\cite{FOOOI} was mentioned. The author of this text was confused because, by Part~I, the twisting with the Chern-Simons Maurer-Cartan element ``adds'' the information about string topology to the canonical structure and one expects to obtain a non-isomorphic structure. The precise notions of isomorphisms needed to be clarified.
%It was also not clear from \cite{Cieliebak2015} 

In Section~\ref{Sec:DetailsOnFiltr}, we start by dealing with filtrations and completions in terms of series in more details. We formulate the Resummation Lemma (Lemma~\ref{Lem:TechLem}), define (complete) filtered algebras (Definition~\ref{Def:FiltAlg}) and comment on units and augmentations (Remark~\ref{Rem:FiltrUnitAug}). We then consider combinations of two filtrations (Equation~\eqref{Eq:CombinedFiltr}) and show that the symmetric bialgebra with the filtration which is the union of the induced filtration and the filtration by weights might not be a filtered bialgebra (Example~\ref{Ex:CombinedOnSymetric}); it is, however, under a boundedness assumption on the filtration (Lemma~\ref{Lem:BoundCondOnFiltr}). We then consider the exponential and the logarithm on a complete filtered algebra (Lemma~\ref{Lem:Exponential}) and generalize it for the convolution product of morphisms with codomain a complete filtered algebra and domain a complete filtered coalgebra satisfying the limit conilpotency property (Proposition~\ref{Prop:ConvPwrSer}). This is the case of completions of conilpotent coalgebras, e.g., the symmetric bialgebra (Lemma~\ref{Lem:SymAlgLimConilp}). We remark that the exponential of a morphism might not be invertible as a map (Remark~\ref{Rem:ExpLogStar}).

In Section~\ref{Sec:FilteredMV}, we recall $\MV$-algebras from \cite{Markl2015} and introduce their complete filtered version (Definition~\ref{Def:FilteredMV}). Based on this, in analogy to \cite{Markl2015}, we define the notion of a complete filtered $\IBLInfty$-algebra in $\MV$-formalism and its morphisms (Definition~\ref{Def:ComplFiltrIBL}). We study the equivalent formulation in components via the surface calculus (Proposition~\ref{Prop:EqCharOfMVIBL}). We show that the notion of a filtered $\IBLInfty$-algebra in $\MV$-formalism is equivalent to the notion of a filtered $\IBLInfty$-algebra from \cite{Cieliebak2015} in some cases, e.g., for the dual cyclic bar complex from Part~I (Proposition~\ref{Prop:BVforIBL}). The formalism of \cite{Cieliebak2015} is symmetric on exchanging inputs and outputs whereas the $\MV$-formalism is not; we illustrate how bubblings at inputs and outputs are handled differently (Example~\ref{Ex:AsymOfMV}). Finally, we consider twisting with a Maurer-Cartan element (Proposition~\ref{Eq:TwistingProp}).
%Finally, we formulate some open questions (Remark~\ref{Rem:OpQueMV}).

In Section~\ref{Sec:BVCompl}, we define the $\BV$-chain complexes associated to a complete filtered $\IBLInfty$-algebra in $\MV$-formalism (Definition~\ref{Def:BVCompl}) and observe that morphisms and the twisting with the Maurer-Cartan element induce chain maps of these chain complexes (Proposition~\ref{Prop:ObservationsMor}). We formulate some open questions (Remark~\ref{Rem:SomeQuestionsFilter}).

In Section~\ref{Sec:CompConvA}, we study the composition of maps which are convolutions of other maps (Lemma~\ref{Lem:ItCompCond}). This leads to the definition of compositions controlling the number of ``veins'' between the individual maps (Definition~\ref{Def:ConComp}). This can be used to formulate the surface calculus algebraically (Proposition~\ref{Prop:PartCompAComp}). We give proofs of some formulas for partial compositions from Section~\ref{Sec:Alg1} in Part~I (Proposition~\ref{Prop:PartCompositions}).


\section{Filtered bialgebras and exponential in convolution product}\label{Sec:DetailsOnFiltr}

We use the definitions of a filtration, completion and the induced filtration on the completion from Section~\ref{Sec:Alg1a} in Part~I. We formulate the statements in the category of vector spaces, but similar statements hold in the category of graded vector spaces (writing ``homogenous'' and ``graded'' everywhere). All filtrations are decreasing, i.e., $\Filtr^{\lambda_1} \supseteq \Filtr^{\lambda_2}$ whenever $\lambda_1 \le \lambda_2$.
%If we say that a map is continuous in the filtered context, we mean that its filtration degree is not $-\infty$.
We work over a field $\K$ of characteristic zero.

We start with the following technical lemma, which shows that convergence in completion is very much like absolute convergence in analysis.\footnote{In fact, for complex numbers $v_k\in \C$ for $k\in\N_0$, property (b) of Lemma~\ref{Lem:TechLem} is equivalent to the absolute convergence of $\sum_{k=0}^\infty v_k$ in $\C$.
%It can  direction uses Fubini's theorem for the counting measure and the other Riemann's Rearrangement Theorem.
Thanks to Ji\v r\'i Zeman for noting that.}
%(over a field $\K$ of characteristic $0$)
 
\begin{Lemma}[Resummation Lemma]\label{Lem:TechLem}
Let $V$ be a vector space filtered by a decreasing filtration~$\Filtr$, and let~$\hat{V}$ be its completion. 
\begin{ClaimList}
\item Let $v_{ij}\in V$ for all $i$, $j\in\N_0$ be vectors such that $\sum_{j=0}^\infty v_{ij}$ converges for every $i\ge 0$ (i.e., $\Norm{v_{ij}}\to\infty$ as $j\to\infty$, so that $\sum_{j=0}^\infty v_{ij}\in\hat{V}$) and $\Norm{\sum_{j=0}^\infty v_{ij}} \to \infty$ as $i\to \infty$. Then for any bijection $r: \N_0\rightarrow \N_0 \times \N_0$, the sum $\sum_{i=0}^\infty v_{r(i)}$ converges and the limit does not depend on $r$; we denote it by $\sum_{i=0}^\infty\sum_{j=0}^\infty v_{ij} \in \hat{V}$.\footnote{It is an exercise to prove a similar statement for nested infinite sums $\sum_{i_1=0}^\infty \dotsb \sum_{i_n=0}^\infty v_{i_1,\dotsc,i_n}$.}
%\item If $\sum_{i=0}^\infty v_i \in \hat{V}$ and $s: \N_0\times \N_0 \rightarrow \N_0$ is a bijection, then $\sum_{i=0}^\infty v_i = \sum_{i=0}^\infty\sum_{j=0}^\infty v_{s(i,j)}$.
\item Suppose that $\sum_{k=0}^\infty v_k$ for $v_k\in V$ converges, and let $s: Z\subset \N_0\times \N_0 \rightarrow \N_0$ be a bijection. If we define $w_{ij} \coloneqq v_{s(i,j)}$ for $(i,j)\in Z$ and $w_{ij} \coloneqq 0$ otherwise, then $\sum_{i=0}^\infty \sum_{j=0}^\infty w_{ij}$ converges (in the sense of (a)) and equals $\sum_{k=0}^\infty v_k$.
\item Let $V$ and $V'$ be vector spaces filtered by exhaustive filtrations, and let $\sum_{i=0}^\infty v_{i}$ for $v_i\in V$ and $\sum_{i=0}^\infty v_{i}'$ for $v_i'\in V'$ converge. Then for any bijection $r: \N_0 \rightarrow \N_0 \times \N_0$, $i\to (r_1(i),r_2(i))$, the sum $\sum_{i=0}^\infty v_{r_1(i)}\otimes v_{r_2(i)}'$ converges and the limit does not depend on $r$; we denote it by $\sum_{i=0}^\infty v_i \otimes \sum_{j=0}^\infty v_j' \in V\COtimes V'$.
\end{ClaimList}
Moreover, we have the following statements about maps of filtered vector spaces:
\begin{ClaimList}[resume]
\item If $f: \hat{V}\rightarrow\hat{V}$ is a linear map of finite filtration degree, i.e., $\Norm{f}>-\infty$, and $\sum_{i=0}^\infty v_i$ with $v_i\in V$ a convergent series, then $\sum_{i=0}^\infty f(v_i)$ converges to $f(\sum_{i=0}^\infty v_i)$.
\item Given linear maps $f_i: \hat{V}\to \hat{V}$ for $i\in \N_0$, we say that the sum $\sum_{i=0}^\infty f_i$ converges if $\sum_{i=0}^\infty f_i(v)$ converges for all $v\in \hat{V}$. In this case, $f(v) \coloneqq \sum_{i=0}^\infty f_i(v)$ defines a linear map $\hat{V}\rightarrow\hat{V}$ which satisfies $\Norm{f}\ge \inf_{i\in\N_0} \Norm{f_i}$.
\end{ClaimList}
\end{Lemma}
\begin{proof}
\begin{ProofList}
\item Let $r:\N_0\rightarrow\N_0\times\N_0$ be a bijection. We first prove the convergence of $\sum_{i=0}^\infty v_{r(i)}$. Given $K>0$, let $i_0\in\N_0$ be such that $\Norm{v_{ij}}\ge K$ for $(i,j)\in \{i\ge i_0, j\ge 0\}$. Such $i_0$ exists by the assumption that $\Norm{\sum_{j=0}^\infty v_{ij}}\to \infty$ as $i\to \infty$. Let $j_0\in \N_0$ be such that $\Norm{v_{ij}}\ge K$ for $(i,j)\in \{i \le i_0, j\ge j_0\}$. Such $j_0$ exists from the convergence of $\sum_{j=0}^\infty v_{0j}$, $\dotsc$, $\sum_{j=0}^\infty v_{i_0 j}$. We can now pick $k_0\in \N$ such that $r(\{k< k_0\}) \supset \{i\le i_0, j\le j_0\}$, and hence $\Norm{v_{r(k)}}\ge K$ for all $k\ge k_0$. This implies the convergence of $\sum_{i=0}^\infty v_{r(i)}$.

Let now $r': \N_0 \rightarrow \N_0 \times \N_0$ be another bijection. Given $K>0$, we find $i_0$, $j_0\in \N_0$ so that $\Norm{v_{ij}}\ge K$ for all $(i,j)\not\in\{i\le i_0, j\le j_0\}$ (like in the previous paragraph). There exits $k_0\in \N$ such that $r(\{k < k_0\}), r'(\{k \le k_0\}) \supset \{i\le i_0, j\le j_0\}$. Hence, for any $k\ge k_0$, the contribution of $v_{ij}$ with $(i,j)\in\{i\le i_0, j\le j_0\}$ cancels in $\Delta_k \coloneqq \sum_{i=0}^k v_{r(i)} - \sum_{i=0}^k v_{r'(i)}$, and we get $\Norm{\Delta_k} \ge K$. It follows that the sums are equal in the limit.
\item It is easy to see that $w_{ij}$ satisfy the assumptions of (a). The claim follows from~(a) by defining $r\coloneqq s^{-1}\sqcup \Id: \N_0 = \N_0 \sqcup (\N_0\times \N_0\backslash Z) \rightarrow \N_0\times \N_0$, where the notation $\N_0 = \N_0 \sqcup (\N_0\times \N_0\backslash Z)$ means an infinite shuffle permutation. Clearly, $\sum_{i=0}^\infty w_{r(i)}$ is precisely $\sum_{i=0}^\infty v_i$ after crossing out possibly infinitely many zeros.

\item We start with convergence. Let $K>0$. Pick indices $i_0\in \N_0$ and $i_0'\in\N_0$ such that $\Norm{v_i} \ge K - \inf_j \Norm{v_j'}$ and $\Norm{v_i'}\ge K - \inf_j \Norm{v_j}$ for all $i\ge i_0$ and $i\ge i_0'$, respectively. Note that the right-hand sides of the inequalities are finite because the filtrations are exhaustive. Pick $k_0\in\N$ such that $r(\{k<k_0\})\supset\{i\le i_0, j\le j_0\}$. Then, for $k\ge k_0$, we have 
\begin{align*}
\Norm{v_{r_1(k)} \otimes v_{r_2(k)}'} &\ge \Norm{v_{r_1(k)}} + \Norm{v_{r_1(k)}'}\\
&\ge \begin{cases}
(K - \inf_j \Norm{v_j'}) + \inf_j \Norm{v_j'} = K & \text{if }r_1(k)\ge i_0, \\
 \inf_j \Norm{v_j} + (K - \inf_j \Norm{v_j}) = K & \text{if }r_2(k)\ge i_0'.
\end{cases}
\end{align*}
It follows that $\sum_{i=0}^\infty v_{r_1(i)}\otimes v_{r_2(i)}'$ converges.

If $r': \N_0 \rightarrow \N_0\times\N_0$ is another bijection, we write $r' = r \circ s$ for a bijection $s: \N_0 \rightarrow \N_0$ and apply (b) to $V\otimes V'$ with the bijection $s: Z\coloneqq \N_0 \times\{0\}\simeq \N_0\rightarrow \N_0$. It follows that $\sum_{i=0}^\infty v_{r(i)} = \sum_{i=0}^\infty v_{r'(i)}$.
\item This is clear because $f(\sum_{i=0}^\infty v_i) = \sum_{i=0}^n f(v_i) + f(\sum_{i=n+1}^\infty v_i)$ for any $n\in \N_0$ and $\Norm{f(\sum_{i=n+1}^\infty v_i)} \ge \Norm{f} + \inf_{i>n}\Norm{v_i} \to \infty$ as $n\to\infty$.
\item Linearity is clear. The inequality $\Norm{f}\ge \inf_i\Norm{f_i}$ follows immediately from $\Norm{f(v)} \ge \inf_i \Norm{f_i(v)} \ge \inf_i \Norm{f_i} + \Norm{v}$ for all $v\in \hat{V}$.\qedhere
\end{ProofList}
\end{proof}

Note that (a) and (c) of the Resummation Lemma are, in fact, reformulations of the facts that the canonical inclusions induce the isomorphisms $\hat{V}\simeq\hat{\hat{V}}$ and $V\COtimes V\simeq\hat{V}\COtimes \hat{V}$, respectively.

We will work with (complete) filtered algebras, coalgebras and bialgebras, which we now define schematically. They are basically algebras over the corresponding (pr)operad in the category of (complete) filtered vector spaces. 

\begin{Definition}[(Complete) filtered algebras, coalgebras, bialgebras]\label{Def:FiltAlg}
A \emph{filtered algebra, coalgebra or bialgebra} is a filtered vector space $V$ with linear operations $\mu_{klg}: V^{\otimes k} \rightarrow V^{\otimes l}$  of non-negative filtration degree for $k$, $l$, $g\in \N_0$ such that $(V,(\mu_{klg}))$ is an algebra, coalgebra or bialgebra, respectively.

A \emph{complete filtered algebra, coalgebra or bialgebra} is a complete filtered vector space~$V$ with linear operations $\mu_{klg}: V^{\COtimes k} \rightarrow V^{\COtimes l}$ of non-negative filtration degree for $k$, $l$, $g\in \N_0$ such that $\mu_{klg}$ satisfy the relations of an algebra, coalgebra or bialgebra with $\otimes$ replaced by $\COtimes$, respectively.
%(these notions are literally the same due to the canonical induced filtration on $\hat{V}$ and the canonical isomorphism $\hat{\hat{V}} \simeq \hat{V}$).

A morphism of (complete) filtered algebras, coalgebras or bialgebras is a linear map $f:V_1 \rightarrow V_2$ with non-negative filtration degree intertwining $\mu_{klg}$.
\end{Definition}

\begin{Remark}[On (complete) filtered algebras]
\begin{RemarkList}
\item Clearly, completion is a functor from filtered algebras to complete filtered algebras. In deformation theory, we may deal with complete filtered algebras which are not in the image of this functor.
\item  The $\IBLInfty$-algebra on $C$ from Definition~\ref{Def:IBLInfty} in Part~I for $\gamma = 0$ is, in fact, a ``complete filtered $\IBLInfty$-algebra on~$\hat{C}$'' in the sense of Definition~\ref{Def:FiltAlg} plus the data of the filtered vector space~$C$. It might not be the completion of a ``filtered $\IBLInfty$-algebra on~$C$''; in fact, we called an $\IBLInfty$-algebra ``completion-free'' if the operations on~$\hat{C}$ arose as continuous extensions of operations on~$C$. Definition~\ref{Def:IBLInfty} seemed natural when we were trying to compute examples of the twisted structure, hoping that we do not encounter infinite sums. After dealing with completions, however, Definition~\ref{Def:FiltAlg} is more logical as an abstract definition.

Notice that for $\gamma > 0$, the norm $\Norm{\OPQ_{klg}} \ge \gamma(2-2g-k-l)$ is allowed to be negative; in particular, $\Norm{\OPQ_{210}}\ge -\gamma$ and $\Norm{\OPQ_{120}}\ge -\gamma$. Therefore, one has to allow finite filtration degree in Definition~\ref{Def:FiltAlg} in order to accommodate it to $\IBLInfty$-algebras with $\gamma \ge 0$.
\end{RemarkList}
\end{Remark}

\begin{Remark}[Units and augmentations]\label{Rem:FiltrUnitAug}
The ground field $\K$ is filtered by the trivial complete filtration $\Filtr^{\lambda\le 0}\K = \K$ and $\Filtr^{\lambda>0}\K = 0$ (see also \eqref{Eq:TrivFiltr} in Part~I). Suppose that $\MVUnit: \K \rightarrow V$ is a unit and $\MVAug: V \rightarrow \K$ an augmentation of a complete filtered algebra~$V$; in particular, we have $\MVAug \circ \MVUnit = \Id$ and $\Norm{\MVUnit}$, $\Norm{\MVAug}\ge 0$. From this, we deduce the following implications for~$v\in V$:
\begin{align*}
\Norm{v}>0 &\quad\Implies\quad v\in \ker \MVAug =: \bar{V},\\
v\in \im \MVUnit &\quad\Implies\quad \Norm{v} = 0.
\end{align*}
Recall that $V = \im \MVUnit \oplus \bar{V}$.
\end{Remark}

Given a vector space $U$ over $\K$, we will work with the \emph{symmetric bialgebra} $\Sym U = \bigoplus_{k=0}^\infty \Sym_k U$ from Definition~\ref{Def:SymAlgebra} in Part~I. It has the concatenation product $\MVProd: \Sym U \otimes \Sym U \rightarrow \Sym U$ and the shuffle coproduct $\MVCoProd: \Sym U \rightarrow \Sym U \otimes \Sym U$ which are given for all $u_{ij}$, $u_i \in U$ and $k$, $k'\in \N_0$ by
\begin{align*}
\MVProd(u_{11}\dotsb u_{1k} \otimes u_{21} \dotsb u_{2k'}) &= u_{11} \dotsb u_{1k} u_{21} \dotsb u_{2k'}\quad \text{and}\\
\MVCoProd(u_1 \dotsb u_k) &= \sum_{\substack{k_1,\,k_2 \ge 0\\ k_1 + k_2 = k}} \sum_{\sigma\in \Perm_{k_1, k_2}} \varepsilon(\sigma,u) u_{\sigma^{-1}_1} \dotsb u_{\sigma_{k_1}^{-1}}\otimes u_{\sigma_{k_1+1}^{-1}}\dotsb u_{\sigma_{k_1 + k_2}^{-1}},
\end{align*}
respectively. Here $\Perm_{k_1,k_2}$ denotes the shuffle permutations and $\varepsilon(\sigma,u)$ the Koszul sign. The unit $\MVUnit: \K \rightarrow \Sym U$ is given by $\MVUnit(1) = 1 \in \Sym_0 U = \K$, and the augmentation $\MVAug: \Sym U \rightarrow \K$ is determined by its reduced algebra $\RedSym U = \bigoplus_{k=1}^\infty \Sym_k U$. If $U$ is filtered, then $\MVProd$, $\MVCoProd$, $\MVUnit$ and~$\MVAug$ preserve the induced filtration on $\Sym U$ (see Definition~\ref{Def:Filtrations} in Part~I), and hence $\Sym U$ is a filtered bialgebra. We will also use the canonical projection $\pi_k: \Sym U \rightarrow \Sym_k U$ and the canonical inclusion $\iota_l: \Sym_l U \rightarrow \Sym U$ for $k\in\N_0$ and $l\in \N_0$. They too preserve the filtration.

The symmetric bialgebra of a filtered vector space has naturally two filtrations --- the induced filtration and the filtration by weights $\Filtr_\WeightMRM^\lambda \Sym U \coloneqq \bigoplus_{k\ge \lambda} \Sym_k U$.\footnote{Note that the filtration by weights can be also viewed as the induced filtration from the filtration on $U$ defined by $\Filtr^{\le 1}U \coloneqq U$, $\Filtr^{>1}U\coloneqq 0$.} In general, having two filtrations~$\Filtr_1$ and~$\Filtr_2$ on a vector space $V$, we define the filtrations
\begin{equation}\label{Eq:CombinedFiltr}
\Filtr_\cup^\lambda \coloneqq \Filtr_1^\lambda + \Filtr_2^\lambda\quad\text{and}\quad\Filtr_\cap^{\lambda} \coloneqq \Filtr_1^\lambda \cap \Filtr_2^\lambda
\end{equation}
and call them the \emph{combined filtrations} --- the union and the intersection. It is easy to see that \Correct[caption={DONE Inequality for combined fitlrations},noline]{There has to be an inequality!!}
\[ \Norm{\cdot}_\cup \ge \max(\Norm{\cdot}_1,\Norm{\cdot}_{2})\quad\text{and}\quad\Norm{\cdot}_\cap \le \min(\Norm{\cdot}_1,\Norm{\cdot}_{2}) \]
hold for the corresponding filtration degrees.

\begin{figure}[t]\label{Fig:FiltrationTypes}
\centering
%auto-ignore
\begin{tikzpicture}
%
    \def\lI{.8cm}
    \def\lII{0.2cm}
    \def\lIII{1.2cm}
    
    \coordinate (C) at (0,0);
    \coordinate (P2) at ($(C)+(\lIII,\lIII)$);
    \coordinate (P3) at ($(C)+(\lIII,-\lII)$);
    \coordinate (P4) at ($(C)+(\lIII+\lI,-\lII)$);
    \coordinate (P5) at ($(C)+(\lIII+\lI,\lIII+\lI)$);
    \coordinate (P6) at ($(C)+(\lIII,\lIII+\lI)$);
    \coordinate (P1) at (P6);
    \pattern[pattern=north east lines] (P1)--(P2)--(P3)--(P4)--(P5)--(P6)--cycle;
    \coordinate (L2) at ($(C)+(\lIII,0)$); 
    \coordinate (Y1) at ($(C)+(0,\lI+\lIII)$);
    \coordinate (Y2) at ($(C)+(0,-\lII)$);
    \coordinate (X1) at ($(C)+(\lI+\lIII,0)$);
    \coordinate (X2) at ($(C)+(-\lII,0)$);
    \node[circle,fill=white,inner sep=4pt] (L1) at ($(X1)+(0,-1.8ex)$) {$\lambda_2$};
    \node[circle,fill=white,inner sep=4pt] (L2) at ($(Y1)+(-.8em,0)$) {$\lambda_1$};
    \draw[->] (X2)--(X1);
    \draw[->] (Y2)--(Y1);
%
\end{tikzpicture}%
%auto-ignore
\begin{tikzpicture}
%
    \def\lI{.8cm}
    \def\lII{0.2cm}
    \def\lIII{1.2cm}
    
    \coordinate (C) at (0,0);
    \coordinate (P1) at ($(C)+(-\lII,\lIII)$);
    \coordinate (P2) at ($(C)+(\lIII,\lIII)$);
    \coordinate (P4) at ($(C)+(\lIII+\lI,\lIII)$);
    \coordinate (P3) at (P4);
    \coordinate (P5) at ($(C)+(\lIII+\lI,\lIII+\lI)$);
    \coordinate (P6) at ($(C)+(-\lII,\lIII+\lI)$);
    \pattern[pattern=north east lines] (P1)--(P2)--(P3)--(P4)--(P5)--(P6)--cycle;
    \coordinate (L1) at ($(C)+(0,\lIII)$);
    \coordinate (Y1) at ($(C)+(0,\lI+\lIII)$);
    \coordinate (Y2) at ($(C)+(0,-\lII)$);
    \coordinate (X1) at ($(C)+(\lI+\lIII,0)$);
    \coordinate (X2) at ($(C)+(-\lII,0)$);
    \node[circle,fill=white,inner sep=4pt] (L1) at ($(X1)+(0,-1.8ex)$) {$\lambda_2$};
    \node[circle,fill=white,inner sep=4pt] (L2) at ($(Y1)+(-.8em,0)$) {$\lambda_1$};
    \draw[->] (X2)--(X1);
    \draw[->] (Y2)--(Y1);
%
\end{tikzpicture}%
%auto-ignore
\begin{tikzpicture}
%
    \def\lI{.8cm}
    \def\lII{0.2cm}
    \def\lIII{1.2cm}
    
    \coordinate (C) at (0,0);
    \coordinate (P1) at ($(C)+(-\lII,\lIII)$);
    \coordinate (P2) at ($(C)+(\lIII,\lIII)$);
    \coordinate (P3) at ($(C)+(\lIII,-\lII)$);
    \coordinate (P4) at ($(C)+(\lIII+\lI,-\lII)$);
    \coordinate (P5) at ($(C)+(\lIII+\lI,\lIII+\lI)$);
    \coordinate (P6) at ($(C)+(-\lII,\lIII+\lI)$);
    \pattern[pattern=north east lines] (P1)--(P2)--(P3)--(P4)--(P5)--(P6)--cycle;
    \coordinate (L1) at ($(C)+(0,\lIII)$);
    \coordinate (L2) at ($(C)+(\lIII,0)$); 
    \coordinate (Y1) at ($(C)+(0,\lI+\lIII)$);
    \coordinate (Y2) at ($(C)+(0,-\lII)$);
    \coordinate (X1) at ($(C)+(\lI+\lIII,0)$);
    \coordinate (X2) at ($(C)+(-\lII,0)$);
    \node[circle,fill=white,inner sep=4pt] (L1) at ($(X1)+(0,-1.8ex)$) {$\lambda_2$};
    \node[circle,fill=white,inner sep=4pt] (L2) at ($(Y1)+(-.8em,0)$) {$\lambda_1$};
    \draw[->] (X2)--(X1);
    \draw[->] (Y2)--(Y1);
%
\end{tikzpicture}%
%auto-ignore
\begin{tikzpicture}
%
    \def\lI{.8cm}
    \def\lII{0.2cm}
    \def\lIII{1.2cm}
    
    \coordinate (C) at (0,0);
    \coordinate (P1) at ($(C)+(\lI+\lIII,\lIII)$);
    \coordinate (P2) at ($(C)+(\lI+\lIII,\lI+\lIII)$);
    \coordinate (P3) at ($(C)+(\lIII,\lI+\lIII)$);
    \coordinate (P4) at ($(C)+(\lIII,\lIII)$);
    \pattern[pattern=north east lines] (P1)--(P2)--(P3)--(P4)--cycle;
    \coordinate (L1) at ($(C)+(0,\lIII)$);
    \coordinate (L2) at ($(C)+(\lIII,0)$); 
    \coordinate (Y1) at ($(C)+(0,\lI+\lIII)$);
    \coordinate (Y2) at ($(C)+(0,-\lII)$);
    \coordinate (X1) at ($(C)+(\lI+\lIII,0)$);
    \coordinate (X2) at ($(C)+(-\lII,0)$);
    \node[circle,fill=white,inner sep=4pt] (L1) at ($(X1)+(0,-1.8ex)$) {$\lambda_2$};
    \node[circle,fill=white,inner sep=4pt] (L2) at ($(Y1)+(-.8em,0)$) {$\lambda_1$};
    \draw[->] (X2)--(X1);
    \draw[->] (Y2)--(Y1);
%
\end{tikzpicture}%
\caption[Natural filtrations of a bigraded vector space.]{For a bigraded vector space $V = \bigoplus_{i} \bigoplus_{j} V_{ij}$, let $\Filtr^{\lambda_1}_{\VertMRM}\coloneqq\bigoplus_{i\ge \lambda_1}\bigoplus_{j} V_{ij}$ be the vertical and $\smash{\Filtr^{\lambda_2}_{\HorMRM}}\coloneqq\bigoplus_{i}\bigoplus_{j\ge \lambda_2} V_{ij}$ the horizontal filtration. The illustrations above depict $\Filtr_{\VertMRM}$, $\Filtr_{\HorMRM}$, $\Filtr_{\VertMRM}\cup\Filtr_{\HorMRM}$ and $\Filtr_{\VertMRM}\cap \Filtr_{\HorMRM}$, respectively. We imagine the component $V_{ij}$ sitting at the position $(i,j)$ in the plane. Given $v_{ij}\in V_{ij}$, the sum $\sum_{i,j} v_{ij}$ converges with respect to the given filtration if and only if for every $\lambda$, only finitely many $v_{ij}$'s in the white region are non-zero.}
\end{figure}

The following example shows that $\Sym U$ might not be a filtered bialgebra with respect to combined filtrations.

\begin{Example}[Combined filtrations of symmetric bialgebra $\Sym U$]\label{Ex:CombinedOnSymetric}
Let $U$ be a vector space filtered by a filtration $\Filtr$. Consider the symmetric bialgebra $\Sym U$ with the induced filtration $\Filtr$ and the filtration by weights $\Filtr_{\WeightMRM}^\lambda$. It is easy to see that the operations $\MVProd$, $\MVCoProd$, $\MVUnit$ and $\MVAug$ preserve both $\Filtr$ and $\Filtr_\WeightMRM$. Consider the combined filtrations $\Filtr_\cup$ and $\Filtr_\cap$. Because
\begin{align*}
\min(\lambda_1 + \lambda_2,k_1+k_2) &\ge \min(\lambda_1,k_1) + \min(\lambda_2,k_2)\quad\text{and}\\
\max(\lambda_1 + \lambda_2,k_1+k_2) &\le \max(\lambda_1,k_1) + \max(\lambda_2,k_2),
\end{align*}
it holds $\Norm{\MVProd}_\cap \ge 0$ and $\Norm{\MVCoProd}_\cup\ge 0$, respectively.
%Non-negativity of the combined norms of $\MVUnit$ and $\MVAug$ is clear.

We now demonstrate that it might happen that $\Norm{\MVProd}_\cup = - \infty$ and $\Norm{\MVCoProd}_\cap = -\infty$. Let $U = \Ten W$ be the tensor algebra over a non-zero vector space $W$. On $U$, consider the filtration by weights $\Filtr^\lambda U = \bigoplus_{k\ge \lambda} W^{\otimes k}$. Let $0\neq w\in W$, and define 
\[ v_k \coloneqq \underbrace{w\smallotimes \dotsb \smallotimes w}_{k\text{-times}} \in \Filtr^{k} \Sym_1 U\quad\text{for all }k\in\N, \]
where $\smallotimes$ denotes the tensor product on $\Ten W$. For $0\neq u \in \Ten_0 W = \K$, define 
\[ v_k' \coloneqq \underbrace{u\dotsb u}_{k\text{-times}} \in \Filtr^0 \Sym_k U\quad\text{for all }k\in \N, \]
where $\cdot$ denotes the concatenation product on $\Sym U$. It follows that 
\[ \Norm{v_k}_{\cup} = k,\quad\Norm{v_k'}_{\cup} = k\quad\text{and}\quad\Norm{v_k v_k'}_\cup = k+1. \]
Therefore, for all $k\in \N$, it holds
\begin{align*}
\Norm{\MVProd}_\cup &\le \Norm{v_k v_k'}_\cup - \Norm{v_k \otimes v_k'}_{\cup} \\
&\le k + 1 - \Norm{v_k}_\cup - \Norm{v_k'}_\cup \\
&=  1 - k.
\end{align*}
Consequently $\Norm{\mu}_\cup = -\infty$. Next, it holds 
\[ \Norm{v_k v_k'}_\cap = k\quad\text{and}\quad\Norm{v_k\otimes v_k'}_\cap = 1, \]
and we compute
\begin{align*}
\Norm{\MVCoProd}_\cap &\le \Norm{\MVCoProd(v_k v_k')}_\cap - \Norm{v_k v_k'}_\cap \\ 
&=  1 - k.
\end{align*}
We used here that the summands of $\delta(v_k v_k')$ are tensor products of $v_1''=u^{i} v_k \in \Filtr^1 \Sym_{i+1} U$ and $v_2''=u^{k-i} \in \Filtr^0 \Sym_{k-i} U$ for $i=0$, $\dotsc$, $k$, so that $\Norm{v_1'' \otimes v_2''}_\cap = 1$. It follows that $\Norm{\MVCoProd}_\cap = -\infty$. A heuristic explanation of these phenomenons is that the tensor product of combined filtrations is not the combined filtration of the tensor product.

In the next paragraph, we will argue that it might still be possible to extend $\MVProd$ to $\Sym U\COtimes_\cup\Sym U$ --- the completion of $\Sym U \otimes\Sym U$ with respect to the tensor product of the union filtrations --- even though $\Norm{\MVProd}=-\infty$. A similar discussion applies for $\MVCoProd$.

Suppose that the filtration on $U$ is bounded from above. Let $v_i\in\Filtr^{\lambda_i}\Sym_{k_i}U$ and $v_i'\in \Filtr^{\lambda_i'}\Sym_{k_i'}U$ for all $i\in \N_0$ be such that $\max(\lambda_i,k_i) + \max(\lambda_i',k_i') \to \infty$ as $i\to \infty$, so that $\sum_{i=0}^\infty v_i \otimes v_i'$ converges in $\Sym U\COtimes_\cup\Sym U$ (in fact, any element of $\Sym U\COtimes_\cup \Sym U$ can be written in this way). We have $\mu(v_i,v_i') \in\Filtr^{\lambda_i + \lambda_i'}\Sym_{k_i+k_i'}U$ for all $i\in \N_0$. Suppose that $\sum_{i=0}^\infty v_i v_i'$ does not converge with respect to $\Filtr_\cup$. Thus, $\max(\lambda_i + \lambda_i',k_i+k_i')$ is bounded, hence $\lambda_i + \lambda_i'$ and $k_i + k_i'$ are bounded, and since $k_i$, $k_i'\ge 0$, also~$k_i$ and~$k_i'$ are bounded. Nevertheless, in order to comply with the assumption on the convergence of $\sum_{i=0}^\infty v_i \otimes v_i'$, we need one of $\lambda_i$ or $\lambda_i'$, let's say $\lambda_i$, to diverge to $\infty$. Because $\lambda_i + \lambda_i'$ is bounded, $\lambda_i'$ has to diverge to~$-\infty$. But this is not allowed by the boundedness assumption.
%Suppose that
%\begin{equation}\label{Eq:CombFiltrCond}
%\Filtr^\xi U = V\quad\text{for some }\xi>0.
%\end{equation}
%In other words, $\Filtr$ is bounded from above in positive degree. Then we have $\Norm{v} \ge \xi \Norm{v}_{\mathrm{w}}$ for all $v\in \Sym(U)$, and it follows that
%\[\begin{cases}
%\Norm{v}\le\Norm{v}_\cup\le \frac{1}{\xi}\Norm{v} & \text{if }0<\xi\le 1, \\
%\Filtr_\cup = \Filtr & \text{otherwise.}
%\end{cases}\]
%It follows that in both cases, the completions with respect to the combined and induced filtrations are homeomorphic, and we can extend $\mu$ and $\delta$ to $\hat{\Sym}^{\cup}(U)$. However, in order to get an $\MV$-algebra, i.e., in order to have $\Norm{\mu} = \Norm{\delta} = 0$, we require \eqref{Eq:CombFiltrCond} with $\xi = 1$. This is clearly satisfied for the reduced dual cyclic bar complex $\DBCyc W$, which is used throughout part I. However, it makes it unclear how to do it for the non-reduced bar complex. In this case, our $\BV$-formalism fails.
%Let us shortly mention another possibilities of combining $\Filtr$ and $\Filtr_{\mathrm{w}}$. First of all, we have the space $\tilde{\Sym}(U) = \prod_{k=0}^\infty \hat{\Sym}_k(U)$. This space arises by completing using $\Filtr$ for fixed weight (and degree), obtaining $\bigoplus_{k=0}^\infty \hat{\Sym}_k(U)$, and then completing by weights.
\end{Example}

%After imposing the following boundedness condition, $\Norm{\MVProd}_\cup$ becomes finite and even non-negative for $\gamma \ge 1$.
It turns out that the union filtration is often identical with the induced filtration.

\begin{Lemma}[Combined filtration and boundedness condition]\label{Lem:BoundCondOnFiltr}
Let $U$ be a vector space filtered by a decreasing filtration $\Filtr$. Suppose that there is $\gamma > 0$ such that
\begin{equation}\label{Eq:BoundedFromAbove}
\Filtr^\gamma U = U. 
\end{equation}
Then the following holds:
\begin{align*}
 \gamma\ge 1\quad&\Implies\quad\forall\lambda\in\R:\ \Filtr^\lambda_\cup \Sym U = \Filtr^\lambda \Sym U,\\
 0<\gamma < 1\quad&\Implies\quad\forall v\in \Sym U:\ \Norm{v}_\cup \ge \Norm{v}\ge \gamma\Norm{v}_\cup.
\end{align*}
\end{Lemma}
\begin{proof}
Under the assumption \eqref{Eq:BoundedFromAbove}, we have for any $k\in \N$ and $\lambda\in\R$ the following:
\begin{equation}\label{Eq:Condition}
\lambda\le k \gamma\quad \Implies\quad \Filtr^\lambda \Sym_k U = \sum_{\lambda_1 + \dotsb + \lambda_k = \lambda} \Filtr^{\lambda_1}U \otimes \dotsb \otimes \Filtr^{\lambda_k} U \Bigl/ \Perm_k = \Sym_k U.
\end{equation}
We compute
\begin{align*}
\Filtr_\cup^\lambda \Sym U &= \bigoplus_{k<\lambda}\Filtr^{\lambda}\Sym_k U \oplus  \bigoplus_{k\ge \lambda} \Sym_k U \\
&= \begin{cases}
     \displaystyle\bigoplus_{k=0}^\infty \Filtr^\lambda \Sym_k U = \Filtr^\lambda \Sym U & \text{for } \gamma \ge 1, \\
    \displaystyle\bigoplus_{\substack{k<\lambda\text{ or }k\ge\frac{\lambda}{\gamma}}} \Filtr^\lambda \Sym_k U \oplus \bigoplus_{\lambda \le k < \frac{\lambda}{\gamma}}\Sym_k U\subset \Filtr^{\lambda\gamma} \Sym U& \text{for }0<\gamma< 1.
   \end{cases}
\end{align*}
The first case holds because if $\gamma\ge 1$, then $k\ge \lambda$ implies $\gamma k \ge \lambda$, and thus $\Sym_k U = \Filtr^\lambda \Sym_k U$ by \eqref{Eq:Condition}. The second case holds from the following reasons. If $0<\gamma<1$ and $k\ge \frac{\lambda}{\gamma}$, so that $\Sym_k U = \Filtr^\lambda \Sym_k U$ by \eqref{Eq:Condition}, then $k>\lambda$. Now, $\Filtr^\lambda \Sym_k U \subset \Filtr^{\lambda \gamma} \Sym_k U$ because the filtration is decreasing and because $\lambda \ge \lambda \gamma$, and if $v\in \Sym_k U$ with $\lambda \le k < \frac{\lambda}{\gamma}$, then $k\gamma \ge \lambda \gamma$, and hence $\Sym_k U = \Filtr^{\lambda\gamma}\Sym_k U$ by \eqref{Eq:Condition}. The claim follows.
\end{proof}

We consider the exponential and the logarithm on filtered algebras.

\begin{Lemma}[Exponential and logarithm on filtered algebras]\label{Lem:Exponential}
Let $V$ be a complete filtered associative algebra with unit~$1$. Given $v\in V$ with $\Norm{v}>0$, we define the exponential
\[ e^v \coloneqq 1 + v + \frac{1}{2!}v^2 + \frac{1}{3!} v^3 + \dotsb \in V. \]
It holds $e^v e^{-v} = 1$. Given $v\in V$ with $\Norm{v-1}>0$, we define the logarithm
\[ \log(v) = \sum_{r=1}^\infty \frac{(-1)^{r-1}}{r} (v-1)^{r}\in V. \]
It holds $\log(e^v) = v$ for $v\in V$ with $\Norm{v}>0$ and $e^{\log v} = v$ for $v\in V$ with $\Norm{v-1}>0$.
%In order to define the composition, we consider the logarithm
%\[ \log(\MVMorF) = \sum_{r=1}^\infty \frac{(-1)^{r-1}}{r} (\MVMorF - \StarProdOne)^{\Star r}, \]
%where $\StarProdOne$ is the convolution unit and $\MVMorF: \hat{\Sym}U((\hbar))\rightarrow\hat{\Sym}U((\hbar))$ is an $R$-linear map such that 
%\begin{equation}\label{Eq:LogCondFiltr}
%\Norm{\MVMorF}\ge 0\quad\text{and}\quad\Norm{\MVMorF(1) - 1} >0.
%\end{equation}
\end{Lemma}
\begin{proof}
For any $v\in V$, we have
\begin{align*}
e^{-v}e^v &= e^{-v}\Bigl(\sum_{i=0}^\infty \frac{1}{i!} v^i\Bigr)\\
  &=\sum_{i=0}^\infty \frac{1}{i!} e^{-v}v^i \\
  & = \sum_{i=0}^\infty \sum_{j=0}^\infty \frac{1}{i! j!}(-1)^j v^{i+j} \\
  &= \sum_{n=0}^\infty \frac{1}{n!}\underbrace{\sum_{k=0}^n (-1)^k \binom{n}{k}}_{=0\text{ for }n>0} v^n.
\end{align*}
On the second line, we used that the multiplication with $e^{-v}$ from the left has finite filtration degree, and hence it commutes with infinite sums; on the third line, we used that the multiplication with $v^i$ from the right has finite filtration degree; on the fourth line, we used the Resummation Lemma (Lemma~\ref{Lem:TechLem}). The facts about $\log$ are proven similarly, the equations reducing to known combinatorial identities.
\end{proof}

Let $(V,\MVCoProd)$ be a coalgebra and $(V',\MVProd')$ an algebra. The \emph{convolution product} $\Star$ on $\K$-linear maps $f_1$, $f_2: V \rightarrow V'$ is defined by
\[ f_1\Star f_2 \coloneqq \MVProd'\circ(f_1\otimes f_2)\circ\MVCoProd. \]
See also \cite[Section~1.6]{Loday2012}.
%Note that in Section~\ref{Sec:Alg1}, in the context of symmetric bialgebras, we denoted~$\Star$ by $\odot$.
If~$\MVAug: V \rightarrow \K$ is a counit for~$V$ and~$\MVUnit': \K \rightarrow V'$ a unit for~$V'$, then
\begin{equation}\label{Eq:ConvUnit}
\StarProdOne \coloneqq \MVUnit'\circ\MVAug: V \rightarrow V'
\end{equation}
is the unit for $\Star$. The convolution product is associative and commutative provided that the algebra and coalgebra are.

Let us recall the conilpotency property of a coaugmented counital coassociative coalgebra $(V,\MVCoProd,\MVAug,\MVUnit)$, where $\MVAug: V \rightarrow \K$ is the counit and $\MVUnit: \K\rightarrow V$ the coaugmentation. We write
\[ \MVCoProd =\bar{\MVCoProd}+ \MVCoProd_0, \]
where $\MVCoProd_0: V \rightarrow V\otimes V$ is defined by
\begin{equation}\label{Eq:CoProdOne}
\MVCoProd_0(v) \coloneqq \begin{cases} 1 \otimes 1 & \text{if }v=1, \\
1\otimes v + v\otimes 1 & \text{if }v\in \bar{V},\end{cases}
\end{equation}
with respect to the decomposition $V = \langle 1 \rangle \oplus \bar{V}$. The other map is then defined simply as the difference 
\[ \bar{\MVCoProd} \coloneqq \MVCoProd - \MVCoProd_0: V \longrightarrow V\otimes V. \]
We call~$\MVCoProd_0$ the \emph{trivial coproduct} and $\bar{\MVCoProd}$ the \emph{reduced diagonal.} They are both coassociative, and $\MVCoProd_0$ is even cocommutative. The \emph{conilpotency property} reads: for every $v\in V$, there exists an $n\in \N$ such that $\bar{\MVCoProd}^{(n)}(v) = 0$, where $\bar{\MVCoProd}^{(n)}\coloneqq(\bar{\MVCoProd}\otimes\Id^{\otimes n-2})\circ \dotsb\circ\bar{\MVCoProd}$ is the iterated reduced diagonal with $n$ outputs.

Using the conilpotency property, it is possible to weaken the condition $\Norm{f}>0$ on the convergence of a power series in $f\in\Hom(V,V')$ in the convolution algebra $(\Hom(V,V'),\Star,\StarProdOne)$ to $\Norm{f}\ge 0$ and $\Norm{f(1)}>0$. The latter condition can not be weakened because $e^f(1) = e^{f(1)}$ is the exponential in $(V',\MVProd',\MVUnit')$.

\begin{Proposition}[Power series in $\Star$ with coefficients in $\K$]\label{Prop:ConvPwrSer}
Let $(V,\MVCoProd,\MVAug, \MVUnit)$ be a complete filtered coaugmented counital cocommutative coassociative coalgebra such that the following \emph{limit conilpotency property} holds:
\begin{equation}\label{Eq:LimConilp}
\bar{\MVCoProd}^{(n)}(v) \to 0\quad\text{as}\quad n\to\infty\quad\text{for all }v\in V.
\end{equation}
Let $(V',\MVProd',\MVUnit',\MVAug')$ be a complete filtered augmented unital commutative associative algebra. Let~$R$ be a complete filtered $\K$-algebra. Then the following holds:
\begin{ClaimList}
\item The convolution product $\Star$ extends naturally to $R$-linear maps $\MVMorF_i: V\COtimes R\rightarrow V'\COtimes R$ ($\otimes = \otimes_\K$) of finite filtration degrees, and for its iterations, we have
\begin{equation}\label{Eq:ItStarProd}
\MVMorF_1\Star\dotsb\Star\MVMorF_n = {\MVProd'}^{(n)}\circ(\MVMorF_1\COtimes_R\dotsb\COtimes_R\MVMorF_n)\circ \MVCoProd^{(n)},
\end{equation}
where ${\MVProd'}^{(n)}$ and $\MVCoProd^{(n)}$ are the continuous $R$-linear extensions of the iterated product and coproduct, respectively. (Recall that the continuous extension of a linear map~$f$ of finite filtration degree on $V$ to the completion $\hat{V}$ is defined by $f(\sum_{i=0}^\infty v_i) \coloneqq \sum_{i=0}^\infty f(v_i)$ for all $v_i\in V$ with $\Norm{v_i}\to\infty$.)\Modify[caption={DONE Fin filtr degree},noline]{This holds for maps with finite filtration degree.}
\item Let $\MVMorF: V\COtimes R\rightarrow V'\COtimes R$ be an $R$-linear map of finite filtration degree such that 
\begin{equation}\label{Eq:ConditionsPowerSeries}
\Norm{\MVMorF}\ge 0\quad\text{and}\quad\Norm{\MVMorF(1)}>0.
\end{equation}
Then any power series $\sum_{k=0}^\infty \alpha_k \MVMorF^{\Star k}$ with coefficients $\alpha_k\in \K$, where $\MVMorF^{\Star 0}\coloneqq \StarProdOne$ is the unit of the convolution product, converges to an $R$-linear map $V\COtimes R \rightarrow V'\COtimes R$ of non-negative filtration degree, and it holds
\[ \bigl(\sum_{k=0}^\infty \alpha_k \MVMorF^{\Star k}\bigr)(1) = \sum_{k=0}^\infty \alpha_k \MVMorF(1)^k. \]
\end{ClaimList}
\end{Proposition}
\begin{proof}
\begin{ProofList}
\item 
%Using Proposition~\ref{}, the expression for the graded module of a tensor product from the proof of Proposition~\ref{} and the assumption $\hat{R}\simeq R$, one can check that for any $n\in \N$, the following isomorphisms induced by the canonical maps hold (they hold even if $V$ is not complete):
%\begin{equation}\label{Eq:IsomCompl}
%(V\COtimes R)^{\COtimes_R \displaystyle n }\simeq(\hat{V}\otimes R)^{\COtimes_R \displaystyle n} \simeq \hat{V}^{\otimes \displaystyle n} \COtimes R \simeq V^{\otimes \displaystyle n} \COtimes R \simeq (V\otimes R)^{\COtimes_R \displaystyle n}.
%\end{equation}
Given a $\K$-linear map $u: V^{\COtimes k} \rightarrow \hat{V}^{\otimes l}$ of finite filtration degree for some $k$, $l\ge 0$, we consider its $R$-linear extension $u\otimes \Id: V^{\COtimes k}\otimes R \rightarrow V^{\COtimes l}\otimes R$ and extend it continuously to $V^{\COtimes k}\COtimes R\rightarrow V^{\COtimes l}\COtimes R$. Because $V^{\COtimes n}\COtimes R \simeq (V\COtimes R)^{\COtimes_R n}$ via canonical maps (this can be proven explicitly using Proposition~\ref{Prop:IsoCrit} in Part~I), we get the desired continuous $R$-linear extension $(V\COtimes R)^{\COtimes_R k}\rightarrow (V\COtimes R)^{\COtimes_R l}$. We apply this construction to $u = {\mu'}^{(n)}$, $\delta^{(n)}$ and define $\MVMorF_1\Star\dotsb\Star\MVMorF_n$ using \eqref{Eq:ItStarProd} for all $n\ge 2$. It is then easy to see that
\begin{align*}
& (\dotsb((\MVMorF_1\Star\MVMorF_2)\Star\MVMorF_3)\Star\dotsb)\Star\MVMorF_n\\
&\quad= \MVProd'(\MVProd'(\dotsb\MVProd'(\MVMorF_1 \COtimes_R \MVMorF_2)\MVCoProd \COtimes_R \MVMorF_3)\MVCoProd\COtimes_R \dotsb \COtimes_R \MVMorF_n)\MVCoProd  \\
&\quad = \underbrace{\MVProd'(\MVProd'\COtimes_R \Id)\dotsb(\MVProd'\COtimes_R\Id^{n-2})}_{={\MVProd'}^{(n)}}(\MVMorF_1\COtimes_R\dotsb\COtimes_R\MVMorF_n)\underbrace{(\MVCoProd\COtimes_R\Id^{n-2})\dotsb(\MVCoProd\COtimes_R\Id)\MVCoProd}_{\MVCoProd^{(n)}} \\
&\quad = \MVMorF_1\Star\dotsb\Star\MVMorF_n.
\end{align*}
\item \begin{figure}
\centering
\begin{tikzpicture}[point/.style = {draw, circle, fill=black, minimum size=2pt,inner sep=0pt},
mylabel/.style args={at #1 #2  with #3}{
    postaction={decorate,
    decoration={
      markings,
      mark= at position #1
      with  \node [#2] {#3};
 } } } 
]
\def\maxy{4}
\def\maxx{8}
\draw[->,name path=xaxis](-.5,0) -- (\maxx,0) node[below] {$n$};
\draw[->,name path=yaxis](0,-.5) -- (0,\maxy) node[left] {$\lambda$};
\coordinate[point,label={[left]$n_1 \Norm{\MVMorF(1)}$}] (f1) at (0,3);
\coordinate[point,label={[below]$n_1$}] (n1) at (7,0);
\path[name path=f1--n1] (f1) -- (n1);
\draw[name path=plot,mylabel=at 0.9 above left with {$\Norm{\bar{\delta}^{(n)}(w)}$}] (0,0)..controls (3,0.5) and (4,1)..(\maxx-.5,\maxy-.5);
\path[name intersections={of=f1--n1 and plot,by=inter}];
\path[name path=inter--xaxis] (inter) --+(0,-2);
\path[name intersections={of=inter--xaxis and xaxis,by=n0int}];
\coordinate[point,label={[below]$n_0$}] (n0) at (n0int);
\draw[dashed] (n0) -- (inter);
\draw[dashed] (f1) -- (n1);
\path let \p1 = (inter) in coordinate (K0) at (-.5,\y1);
\path let \p1 = (inter) in coordinate (K1) at (\maxx,\y1);
\path let \p1 = (inter) in coordinate[label={[above left]K}] (Kmid) at (0,\y1);
\draw[dotted] (K0) -- (K1);
\end{tikzpicture}
\caption[Bound for the filtration degree in the iterated convolution product.]{Given $K>0$, pick $n_0\in\N$ such that $\Norm{\bar{\delta}^{(n)}(w)}\ge K$ for $n\ge n_0$. Find $n_1\ge n_0$ such that $n_1 \Norm{\MVMorF(1)} \ge K + n_0 \Norm{\MVMorF(1)}$. Then $(n - i)\Norm{\MVMorF(1)} + \Norm{\bar{\delta}^{(i)}(w)} \ge K$ for $i=0$, $\dotsc$, $n$ for any $n\ge n_1$.}
\label{Fig:Convergence}
\end{figure}
Let $w\in V\COtimes R$ and $n\ge 2$. For the map $\MVCoProd^{(n)}: V\COtimes R \rightarrow (V\COtimes R)^{\COtimes_R n}$, we write
\begin{equation}\label{Eq:ItCoprod}
\MVCoProd^{(n)}(w) = \sum_{i=0}^n\sum_{(n-i)\times 1} \sum_{\bar{\MVCoProd^{(i)}}} w_{(1)}\COtimes_R 1 \COtimes_R 1 \COtimes_R w_{(2)}\COtimes_R 1 \COtimes_R \dotsb \COtimes_R w_{(i)}\COtimes_R 1,
\end{equation}
where the second sum is over all insertions of $n-i$ units $1$ into the tensor product and the third sum denotes the sum from the Sweedler's notation
\[ \bar{\delta}^{(i)}(w) = \sum w_{(1)}\COtimes_R \dotsb \COtimes_R w_{(i)}. \]
Formula \eqref{Eq:ItCoprod} is easy to see from $\MVCoProd = \MVCoProd_0 + \bar{\MVCoProd}$.

Before we continue, let us check that $\bar{\MVCoProd}: V\COtimes R \rightarrow (V\COtimes R)^{\COtimes_R 2}$ also satisfies the limit conilpotency property (the proof is similar to the proof of Lemma~\ref{Lem:SymAlgLimConilp} below). Every element of $V\COtimes R$ can be written as $\sum_{i=0}^\infty r_i v_i$ for $r_i\in R$ and $v_i\in V$ such that $\Norm{r_i v_i} \to \infty$ as $i\to \infty$; this follows from the Resummation Lemma (Lemma~\ref{Lem:TechLem}). Given $K>0$, we pick $i_0\in \N$ such that $\Norm{r_iv_i}\ge K$ for all $i\ge i_0$. We pick $n_0\in \N$ such that $\Norm{\bar{\MVCoProd}^{(n)}(v_0)}+\Norm{r_0}$, $\dotsc$, $\Norm{\bar{\MVCoProd}^{(n)}(v_{i_0-1})}+\Norm{r_{i_0-1}} \ge K$ for all $n\ge n_0$. According to the construction in (a), we have 
\[ \bar{\MVCoProd}^{(n)}(\sum_{i=0}^\infty r_i v_i) = \sum_{i=0}^\infty r_i \bar{\MVCoProd}^{(n)}(v_i). \]
For $n\ge n_0$, it holds $\Norm{r_i\bar{\MVCoProd}^{(n)}(v_i)}\ge \Norm{r_i} + \Norm{\bar{\MVCoProd}^{(n)}(v_i)}\ge K$ for $i<i_0$ and $\Norm{r_i\bar{\MVCoProd}^{(n)}(v_i)}\ge \Norm{r_i v_i} \ge K$ for $i\ge i_0$. It follows that $\bar{\MVCoProd}^{(n)}(\sum_{i=0}^\infty r_i v_i) \to 0$ as $n\to \infty$.


Going on with the main proof, using~\eqref{Eq:ItStarProd} and~\eqref{Eq:ItCoprod}, we have
\begin{align*}
\Norm{\MVMorF^{\Star n}(w)} & = \Norm{\mu^{(n)}(\MVMorF\COtimes_R \dotsb \COtimes_R \MVMorF)\delta^{(n)}(w)} \\
 & = \begin{multlined}[t] \bigl\|\sum_{i=0}^n \sum_{(n-i)\times 1} \sum_{\bar{\delta}^{(i)}} \MVMorF(w_{n,(1)})\COtimes_R \MVMorF(1)\COtimes_R \MVMorF(1) \COtimes_R \MVMorF(w_{n,(2)}) \COtimes_R \MVMorF(1)\COtimes_R \dotsb \\ \COtimes_R \MVMorF(w_{n,(n-i)})\COtimes_R \MVMorF(1)\bigr\| \end{multlined}\\
 & \ge \min_{i=0,\dotsc,n}\bigl( (n-i) \Norm{\MVMorF(1)} + i \Norm{\MVMorF} + \Norm{\bar{\delta}^{(i)}(w)} \bigr) \\
& \ge \underbrace{\min_{i=0,\dotsc,n}\bigl( (n-i) \Norm{\MVMorF(1)} + \Norm{\bar{\delta}^{(i)}(w)} \bigr)}_{=:(*)_n}.
\end{align*}
Let $K>0$ be arbitrary. Because $\Norm{\MVMorF}\ge 0$, $\Norm{\MVMorF(1)}>0$ and $\bar{\delta}^{i}(w) \to 0$ as $i\to \infty$ by the assumptions, we obtain an $n_1\in \N$ such that $(*)_n\ge K$ for all $n \ge n_1$ (see Figure~\ref{Fig:Convergence}).
%Because 
%\[\Norm{\alpha_n \MVMorF^{\Star n}(rv)} \ge \Norm{\MVMorF^{\Star n}(v)} + \Norm{r}\quad\text{for any }r\in R\text{ and }v\in V, \]
This implies the convergence of $\sum_{n=0}^\infty \alpha_n\MVMorF^{\Star n}(w)$ in $V\COtimes R$. Using (e) of the Resummation Lemma, we get a well-defined $\K$-linear map $\sum_{k=0}^\infty \alpha_k \MVMorF^{\Star k}: V\COtimes R \rightarrow V\COtimes R$ with $\Norm{\sum_{k=0}^\infty \alpha_k \MVMorF^{\Star k}} \ge \inf_{k=0,\dotsc,\infty} \Norm{\MVMorF^{\Star k}} \ge  0$. It is easy to see that it is $R$-linear as well. \qedhere
\end{ProofList}
\end{proof}

\begin{Lemma}[Completion of conilpotent coalgebra satisfies limit conilpotency]\label{Lem:SymAlgLimConilp}
Let $(V,\MVCoProd,\MVAug,\MVUnit)$ be a filtered coaugmented counital coassociative coalgebra which is conilpotent. Then the completion $(\hat{V},\MVCoProd,\MVAug,\MVUnit)$ is a complete filtered coaugmented counital coassociative coalgebra which satisfies the limit conilpotency property \eqref{Eq:LimConilp}.
\end{Lemma}
\begin{proof}
Let $v\in \hat{V}$, and write $v = \sum_{i=0}^\infty v_i$ for $v_i\in V$ with $\Norm{v_i} \to \infty$. Given $K>0$, find $i_0\in \N$ such that $\Norm{v_i}\ge K$ for all $i\ge i_0$. From the conilpotency, we can find $n_0\in \N$ such that $\bar{\MVCoProd}^{(n)}(v_0) = \dotsc = \Norm{\bar{\MVCoProd}^{(n)}(v_{i_0-1})} = 0$ for all $n\ge n_0$. Because $\bar{\MVCoProd}$ has finite filtration degree, $\bar{\MVCoProd}^{(n)}$ has finite filtration degree too, and so we can permute it with the infinite sum and write 
\[ \bar{\MVCoProd}^{(n)}(v) = \sum_{i=0}^\infty \bar{\MVCoProd}^{(n)}(v_i) = \sum_{i=i_0}^\infty \bar{\MVCoProd}^{(n)}(v_i). \]
Because $\Norm{\bar{\MVCoProd}} \ge 0$, we have $\Norm{\bar{\MVCoProd}^{(n)}} \ge 0$, and hence 
\[ \Norm{\bar{\MVCoProd}^{(n)}(v)} \ge \inf_{i=i_0,\dotsc,\infty}\Norm{\bar{\MVCoProd}^{(n)}(v_i)} \ge \inf_{i=i_0,\dotsc,\infty} \Norm{v_i} \ge K \]
for all $n\ge n_0$. This shows the limit conilpotency property.
%Let $v\in \hat{\Sym}U$, and write $v = \sum_{i = 0}^\infty v_i$ for $v_i \in \hat{\Sym}_i U$ with $\Norm{v_i}\to\infty$. This is always possible due to the Resummation Lemma (Lemma~\ref{Lem:TechLem}). Given $K>0$, we find $i_0\in \N_0$ such that $\Norm{v_i} \ge K$ for all $i\ge i_0$. Because $\bar{\MVCoProd}$ has finite filtration degree, we can propagate it over an infinite sum, and for all $i\ge i_0$, we get
%\[ \bar{\MVCoProd}^{(i)}(v) = \sum_{j=0}^\infty \bar{\MVCoProd}^{(i)}(v_j) = \sum_{j=i_0}^\infty \bar{\MVCoProd}^{(i)}(v_j), \]
%where we used the conilpotency 
%Thus, we have
%\[ \Norm{\bar{\MVCoProd}^{(i)}(v)}\ge \inf_{j=i_0,\dotsc,\infty}\Norm{v_j} \ge K. \]
%This finishes the argument.
\end{proof}


\begin{Remark}[Exponential and logarithm in $\Star$]\label{Rem:ExpLogStar}
\begin{RemarkList}
\item It is obvious that Lemma~\ref{Lem:Exponential} holds for~$\Star$ with the weakened conditions~\eqref{Eq:ConditionsPowerSeries}. To sum up, the conditions for the existence of $\exp(\MVMorF)$ are $\Norm{\MVMorF}\ge 0$ and $\Norm{\MVMorF(1)}>0$, and the conditions for the existence of $\log(\MVMorF)$ are $\Norm{\MVMorF}\ge 0$ and $\Norm{\MVMorF(1) - 1}>0$.

\item Given $\MVMorF: V\COtimes R \rightarrow V\COtimes R$ satisfying~\eqref{Eq:ConditionsPowerSeries}, it holds $e^{-\MVMorF}\Star e^{\MVMorF} = \StarProdOne$ by Lemma~\ref{Lem:Exponential} because $\StarProdOne$ is the unit for the convolution product. However, there might not be any $\MVMorG: V\COtimes R \rightarrow V\COtimes R$ such that $\MVMorG \circ e^{\MVMorF} = \Id$ (or $e^{\MVMorF}\circ\MVMorG = \Id$). To see this, let $V=\hat{\Sym} U$, $R=\K$, and let $\HTP_{110}: U \rightarrow U$ be a linear map with $\Norm{\HTP_{110}}\ge 0$ such that there is $0 \neq v\in U$ with $\HTP_{110}(v)=0$. Consider the trivial extension $\MVMorF\coloneqq \HTP_{110}: \hat{\Sym}U\rightarrow\hat{\Sym}U$ (it equals $\HTP_{110}$ on $\hat{\Sym}_1 U\rightarrow \hat{\Sym}_1 U$ and $0$ otherwise). Because
\[ \HTP^{\Star n}(\underbrace{v \dotsb v}_{k\text{-times}}) = \begin{cases}
 n! \HTP_{110}(v)^n & \text{if } k = n, \\
 0 & \text{otherwise},
\end{cases}\]
for all $k$, $n\in\N_0$, it holds $e^{\MVMorF}(v) = \HTP_{110}(v) = 0$, and so $e^{\MVMorF}: \hat{\Sym}U\rightarrow \hat{\Sym}U$ is not invertible.

\item In the case of $V=\hat{\Sym}U$, one can check that $\log(\Id)$ is the trivial extension of $\Id_{110}: \hat{U}\rightarrow \hat{U}$ (see also \cite[Example~21]{Markl2015} in the non-filtered setting).\qedhere
\end{RemarkList}
\end{Remark}


\section{Filtered IBL-infinity-algebras in filtered MV-formalism}\label{Sec:FilteredMV}

%\item Surface calculus amounts to interpreting $\OPQ_{klg}$ and $\HTP_{klg}$ as connected surfaces with~$k$ inputs, $l$ outputs and genus $g$. Surfaces can be glued together at some boundary components to obtain a \emph{connected} or a \emph{disconnected} surface of signature $(k,l,g)$. The algebraic counterpart can be described using the fine structure of composition of convolution products of linear maps on a bialgebra (see Section~\ref{}).
%\item Filtrations is a way to deal with \emph{bubbling} of unstable surfaces whose gluing does not change the signature $(k,l,g)$, and hence induce infinite sums in the algebraic formalism.
%\item $\BV$-formalism is the construction of operators $\BVOp$ and $e^{\MVMorF}$ 
%\end{itemize}
In \cite{Markl2015}, $\MV$-algebras were introduced; they are precisely the algebras governed by the $\BV$-relations $\BVOp^2 = 0$ and $e^\MVMorF \BVOp^+ = \BVOp^- e^\MVMorF$ (following \cite{Cieliebak2015}, we will denote by ${}^+$ the source and by ${}^-$ the target). Schematically, there is the following inclusion of categories (explanations will be given below):
\begin{equation}\label{Eq:InclOfCat}
\begin{gathered}[t]
\underline{\MV\text{-algebras}} \\
\begin{aligned}
&\text{Obj.:}\ \begin{aligned}[t]
&(V,\MVProd,\MVCoProd,\MVUnit,\MVAug), R, \\
&\BVOp : V\COtimes R \rightarrow V\COtimes R,\\
& \BVOp^2 = 0, \Abs{\BVOp}=1, \\
& \BVOp(1) = 0.
\end{aligned}\\
&\text{Mor.:}\ \begin{aligned}[t]
&\MVMorF: V^+\COtimes R\rightarrow V^-\COtimes R, \\
&e^\MVMorF \circ \BVOp^+ = \BVOp^- \circ e^\MVMorF,  \\
&\MVMorF(1) \subset V^-\COtimes\mathfrak{m}^-.
\end{aligned}
\end{aligned}
\end{gathered}
\ \supset\ 
\begin{gathered}[t]
\underline{\text{$\BVInfty$\text{-algebras}}^{*}} \\
\begin{aligned}[t]
&R = \K[[\hbar]], \\
&\Delta= \BVOp_1 + \hbar \BVOp_2 + \hbar^2 \BVOp_3 + \dotsb, \\
& \begin{aligned}[t]\Delta_i: V \rightarrow V \text{ differential}\\\text{operator of order}\le i, \end{aligned}\\
&\MVMorF= \MVMorF_1 + \hbar \MVMorF_2 + \hbar^2 \MVMorF_3 + \dotsb.
\end{aligned}
\end{gathered}
\ \supset\ 
\begin{gathered}[t]
\underline{\text{$\IBLInfty$\text{-algebras}}^{**}} \\
\begin{aligned}
&V = \Sym U,  \\
&\Sym_{k > i}(U) \subset \ker \MVMorF_i.
\end{aligned}
\end{gathered}
\end{equation}
Here, $R$ is a complete local Noetherian ring with residue field~$\K$ of characteristic $0$ and maximal ideal~$\mathfrak{m}$, $(V,\MVProd,\MVUnit,\MVAug)$ is a graded commutative associative algebra over $\K$ with unit $\MVUnit: \K \rightarrow V$ and augmentation $\MVAug: V\rightarrow\K$, $(V,\MVCoProd,\MVAug,\MVUnit)$ is a \emph{conilpotent} graded cocommutative coassociative coalgebra with counit~$\MVAug$ and coaugmentation~$\MVUnit$ (the same maps as $\MVAug$ and $\MVUnit$ for $(V,\mu)$), $V\COtimes R \coloneqq \varprojlim_n (V\otimes R/V\otimes \mathfrak{m}^n)$, where $\mathfrak{m}^n = \mathfrak{m}\dotsb\mathfrak{m}$, is the completed tensor product, the operators~$\BVOp$ and~$\MVMorF$ are $R$-linear and have non-negative filtration degrees with respect to the filtration by $V\COtimes \mathfrak{m}^n$.

If $R$ is a $\K$-algebra,\footnote{The ring $\Z_4$ is a local ring which is not a $\K$-algebra over its residue field $\K=\Z_2$. The ring of polynomials $\R[x]$ in a single variable $x$ is a $\K$-algebra over its residue field $\K=\R$, but it is not a local ring because both $(x)$ and $(x+1)$ are maximal ideals. In contrast to this, the ring of power series $\K((\hbar))$ is both a local ring and a $\K$-algebra. Thanks to Thorsten Hertl for pointing this out.} then this is precisely the same setting as that of Proposition~\ref{Prop:ConvPwrSer} for the filtrations $\Filtr^{\le 0} V = V$, $\Filtr^{>0} V= 0$ and $\Filtr^{\lambda\le 0} R = R$, $\Filtr^{\lambda>0} R = \mathfrak{m}^{\lceil\lambda\rceil}$. Clearly, $\HTP(1)\subset V^- \COtimes \mathfrak{m}^-$ is equivalent to $\Norm{\HTP(1)}>0$. Therefore, $e^\MVMorF$ exists by Proposition~\ref{Prop:ConvPwrSer}.

Recall from \cite[p.\,5]{Markl2015} that a linear operator $D: V\rightarrow V$ on a commutative associative algebra~$V$ with unit $1$ is called a \emph{differential operator} of order $\le k$ for $k\in \N_0$ if it holds\label{Page:DiffOp}
\[ \psi^D_{k+1}(v_1, \dotsc, v_{k+1}) \coloneqq [[\dotsb[[D,L_{v_1}],L_{v_2}],\dotsc],L_{v_{k+1}}] = 0 \quad\text{for all }v_1, \dotsc, v_{k+1}\in V, \]
where 
\[ L_v(w) \coloneqq v w\quad\text{for all }w\in V \]
is the left-multiplication with $v\in V$ and $[\cdot,\cdot]$ is the graded commutator. 

%
%The convergence of any power series with $\K$-coefficients of an $\MV$-morphism $\MVMorF$ is guaranteed by the completeness of $R$, the conilpotency of $\delta$ and the condition $\MVMorF(1) \subset V_2\COtimes\mathfrak{m}_2$. Continuity then follows from continuity of $\MVMorF$, $\mu$ and $\delta$. To show the convergence, one first deduces the formula
%\begin{equation}\label{Eq:ItStarProd}
%\MVMorF^{\Star n} = \mu^{(n)}\circ(\MVMorF\COtimes_R\dotsb\COtimes_R\MVMorF)\circ \delta^{(n)},
%\end{equation}
%where $\mu^{(n)}$ and $\delta^{(n)}$ are the $R$-linear extensions to $V\COtimes R$ of the iterated products and coproducts with $n$ inputs and outputs, respectively. F
%Therefore, if $n\in\N$ is such that $\bar{\delta}^n(v)=0$ by conilpotence, then 
%\[ \MVMorF^{\Star (n+i)}(v) \subset V\COtimes\mathfrak{m}^i\quad\text{for all }i\in \N \]
%because $\delta^{(n+i)}(v)$ contains at least $i$ ones. See Proposition~\ref{Prop:ConvPwrSer} below for a precise proof in a more general setting.

The categories of $\BVInfty$- and $\IBLInfty$-algebras contained in the $\MV$-formalism are not the most general ones; this is what $*$ and $**$ indicate. Before we explain this, let us agree on calling $(\OPQ_{klg})$ a \emph{strict $\IBLInfty$-algebra} if $\OPQ_{0lg}=\OPQ_{k0g}=0$ for all $k$, $l$, $g\in \N_0$, an \emph{input-strict $\IBLInfty$-algebra} if $\OPQ_{0lg} = 0$ for all $l$, $g\in \N_0$, and a \emph{weak $\IBLInfty$-algebra} if we want to emphasize that operations with no input and output are allowed. The same terminology will be used for morphisms.
\begin{itemize}
\item[*] It is the category of \emph{augmented strictly commutative $\BVInfty$-algebras} (see \cite[Section~5]{Cieliebak2007} for the definition of a commutative $\BVInfty$-algebra). The augmentation $\varepsilon$ is an additional data to get an $\MV$-algebra. The coproduct is defined by $\delta\coloneqq \delta_{0}$ (see~\eqref{Eq:CoProdOne}).
\item[**] It is the category of \emph{(non-filtered) input-strict $\IBLInfty$-algebras} $(\OPQ_{klg})_{k \ge 1, l, g\ge 0}$ and input-strict morphisms~$(\HTP_{klg})_{k\ge 1,l, g\ge 0}$ which satisfy the following \emph{finiteness condition} for all $k\ge 1$ and $g\ge 0$: for any $v\in \Sym U$, we have\Add[caption={DONE Add a reference},noline]{Remark (6) at the beginning of CFL15.}
\begin{equation}\label{Eq:RegCond}
\OPQ_{klg}(v) = 0\quad\text{and}\quad\MVMorF_{klg}(v) = 0 \quad\text{for all but finitely many }l\in \N_0.
\end{equation}
See \cite[Remark~(6), p.\,14]{Cieliebak2015} for the same finiteness condition. This is precisely what symplectic field theory for exact cobordisms gives.

(The transformation formulas between $\OPQ_{klg}$ and $\BVOp_i$ and $\MVMorF_{klg}$ and $\MVMorF_i$ were written down in Remark~\ref{Rem:BVForm} in Part~I. We will repeat them in Proposition~\ref{Prop:EqCharOfMVIBL} below.)
\end{itemize}


The following filtered version of $\MV$-algebras will allow us to describe more general weak $\IBLInfty$-algebras. In particular, we will be able to remove the restriction~\eqref{Eq:RegCond}.\Add[caption={DONE Resolve strict and norm continuity},noline]{We choose to work ie do proof with $\Norm{}$ because it is more easier. However, it can be redone using $\Filtr_\lambda$. There is some class of filtrations for which $\Norm{c} = \lambda$ is equivalent to $c\in \Filtr_\lambda$ and $c\not\in\Filtr_\mu$ for $\mu>\lambda$. Make a remark.}

\begin{Definition}[Complete filtered $\MV$-algebra]\label{Def:FilteredMV}
A \emph{complete filtered $\MV$-algebra} over a complete filtered graded algebra $R$ over a field $\K$ is a complete filtered graded vector space~$V$ over $\K$, a homogenous $R$-linear map $\BVOp: V\COtimes R \rightarrow V\COtimes R$ of finite filtration degree satisfying 
\[ \Abs{\BVOp} = -1, \quad \Norm{\BVOp}\ge 0,\quad\Norm{\BVOp(1)}>0 \quad\text{and}\quad\BVOp\circ \BVOp = 0,\]
and operations $\MVProd: V\COtimes V \rightarrow V$, $\MVCoProd: V \rightarrow V\COtimes V$, $\MVUnit: \K \rightarrow V$ and $\MVAug: V\rightarrow \K$ such that $(V,\mu,\eta,\varepsilon)$ is a complete filtered augmented unital commutative associative algebra and $(V,\delta,\varepsilon,\eta)$ is a complete filtered coaugmented counital cocommutative coassociative coalgebra satisfying the limit conilpotency property \eqref{Eq:LimConilp}. We often denote a complete filtered $\MV$-algebra simply by $(V,\BVOp)$.

A \emph{morphism of complete filtered $\MV$-algebras} $(V^+,\BVOp^+)$ and $(V^-,\BVOp^-)$ over $R$ is a homogenous $R$-linear map $\MVMorF: V^+\COtimes R\rightarrow V^-\COtimes R$ of finite filtration degree \Add[caption={DONE Degree $0$?},noline]{Add that $\HTP$ is of degree $0$?}such that 
\[ \Abs{\MVMorF} = 0,\quad \Norm{\MVMorF}\ge0, \quad \Norm{\MVMorF(1)}>0\quad\text{and}\quad e^{\MVMorF}\circ\BVOp^+ = \BVOp^- \circ e^\MVMorF. \]

The \emph{composition} $\DiamComp$ of two composable morphisms $\MVMorF_1$ and $\MVMorF_2$ (i.e., the target of $\MVMorF_2$ is the source of $\MVMorF_1$) is defined by
\begin{equation}\label{Eq:CompositionMV}
\MVMorF_1\DiamComp\MVMorF_2 \coloneqq \log(e^{\MVMorF_1}\circ e^{\MVMorF_2}).
\end{equation}
(The exponential and logarithm are well-defined by Proposition~\ref{Prop:ConvPwrSer}.)
\end{Definition}

\begin{Remark}[On complete filtered $\MV$-algebras]\begin{RemarkList}
\item The condition $\Norm{\MVMorF(1)}>0$ is required for the exponential to converge; i.e., it might be seen as a technical condition. On the other hand, the condition $\Norm{\BVOp(1)}>0$ is optional, and we impose it in the manner of \cite{Markl2015} and \cite{Cieliebak2015}.
\item We do not define ``non-complete filtered $\MV$-algebras'' because it is not clear whether $\MVMorF_1\DiamComp\MVMorF_2: V^+\COtimes R \rightarrow V^-\COtimes R$ for $V^+\otimes R \xrightarrow{\MVMorF_2} V^{+-}\otimes R \xrightarrow{\MVMorF_1} V^-\otimes R$ restricts to a map $V^+\otimes R \rightarrow V^-\otimes R$. This forces us to define morphisms as maps of completions $V^+\COtimes R \rightarrow V^-\COtimes R$. In such category, two filtered $\MV$-algebras would be isomorphic if and only if their completions were. This is not desired, and thus we define only ``complete filtered $\MV$-algebras''.\qedhere
\end{RemarkList}\end{Remark}

Inspired by \eqref{Eq:InclOfCat}, we define the following version of weak $\IBLInfty$-algebras.

\begin{Definition}[Complete filtered $\IBLInfty$-algebra in $\MV$-formalism]\label{Def:ComplFiltrIBL}
Let $d\in\Z$ and $\gamma>0$. A \emph{complete filtered $\IBLInfty$-algebra of bidegree $(d,\gamma)$ in $\MV$-formalism} is a complete filtered $\MV$-algebra $(V,\BVOp, R, \MVProd, \MVCoProd, \MVAug, \MVUnit)$ satisfying the following conditions:
\begin{enumerate}[label=(\arabic*)]
\item $R=\K((\hbar))$ is the ring of Laurent series\footnote{Another notation for $\K((\hbar))$ is $\K[[\hbar]][\hbar^{-1}]$ to emphasize that it is the localization of the ring of power series $\K[[\hbar]]$ at the powers of $\hbar$. An element of $\K((\hbar))$ is a formal power series $\sum_{i=-\infty}^\infty a_i \hbar^i$ with only finitely many non-zero $a_i\in \K$ for $i\le 0$.} in a formal variable~$\hbar$ of degree $\Abs{\hbar} = 2 d$ equipped with the complete, Hausdorff and exhaustive filtration\Correct[noline,caption={DONE It is The multiplication is not continuous}]{The multiplication is not continuous}
\begin{equation}\label{Eq:MVFiltr}
\Filtr^\lambda \K((\hbar)) \coloneqq \Bigl\{\sum_{i=-\infty}^\infty a_i \hbar^i \in \K((\hbar)) \mid a_i = 0\text{ for }i<\frac{\lambda}{2\gamma}\Bigr\}\quad\text{for }\lambda\in\R.
\end{equation}
\item There is a complete filtered graded vector space $W$ such that if we define
\begin{equation}\label{Eq:DefOfU}
 U\coloneqq W[1]\quad\text{and}\quad\Filtr^\lambda U \coloneqq (\Filtr^{\lambda - \gamma} W)[1]\quad\text{for all }\lambda\in \R,
 \end{equation}
then $(V=\hat{\Sym} U, \MVProd, \MVCoProd, \MVUnit, \MVAug)$ is the completion of the symmetric bialgebra on $U$ filtered by the induced filtration.
\item The $\BV$-operator $\BVOp: \hat{\Sym}U((\hbar))\rightarrow\hat{\Sym}U((\hbar))$, where $\hat{\Sym}U((\hbar)) \coloneqq \Sym U\COtimes\K((\hbar))$,\footnote{In contrast to $\K((\hbar))$, an element of $\hat{\Sym}U((\hbar))$, when seen as a power series, can have a non-zero coefficient at every power $\hbar^i$.} decomposes as
\begin{equation}\label{Eq:BVOpDecompRel}
 \BVOp = \hbar^{-1}\BVOp_0 + \BVOp_1 + \hbar\BVOp_2 + \hbar^2 \BVOp_3 + \dotsb,
\end{equation}
where for all $i\ge 0$ the operator $\BVOp_i: \hat{\Sym}U \rightarrow \hat{\Sym}U$ is a linear differential operator of order $\le i$.\footnote{The definition of a differential operator on p.\,\pageref{Page:DiffOp} generalizes in a straightforward way to complete filtered algebras.}
\end{enumerate}
We often denote the data of a complete filtered $\IBLInfty$-algebra in $\MV$-formalism simply by~$(W,\BVOp)$. 
 
A \emph{morphism of complete filtered $\IBLInfty$-algebras $(W^+,\BVOp^+)$ and $(W^-,\BVOp^-)$ of bidegree $(d,\gamma)$ in $\MV$-formalism} is a morphism of complete filtered $\MV$-algebras $\MVMorF: \hat{\Sym}U^+((\hbar)) \rightarrow \hat{\Sym}U^-((\hbar))$ which decomposes as 
\begin{equation}\label{Eq:MorDecompRel}
 \MVMorF = \hbar^{-1}\MVMorF_0 + \MVMorF_1 + \hbar\MVMorF_2 + \hbar^2\MVMorF_3 + \dotsb,
\end{equation}
where for all $i\ge 0$ the map $\MVMorF_i : \hat{\Sym}U^+\rightarrow\hat{\Sym}U^-$ is $\K$-linear and satisfies 
\begin{equation}\label{Eq:CondOnMor}
\hat{\Sym}_{j}U^+\subset\ker \MVMorF_i\quad\text{for all }j>i.
\end{equation}
\end{Definition}

\begin{Notation}[Replacing $\Star$ with $\odot$ for $V=\Sym U$]
From now on, we will denote the convolution product~$\Star$ on morphisms of $\Sym U$ by $\odot$. It is namely the same operation on morphisms which is denoted by~$\odot$ in Section~\ref{Sec:Alg1} in Part~I and in~\cite{Cieliebak2015}.
\end{Notation}
\renewcommand{\Star}{\odot}
For a map $\MVMorF = \sum_{i=-\infty}^\infty \MVMorF_{i+1}\hbar^i: \hat{\Sym}U^+((\hbar)) \rightarrow \hat{\Sym}U^-((\hbar))$, where $\MVMorF_{i}: \hat{\Sym}U^+\rightarrow\hat{\Sym}U^-$ is $\K$-linear, we denote 
\[ \langle \MVMorF\rangle_{klg} \coloneqq \pi_l \circ \MVMorF_{k+g}\circ \iota_k: \hat{\Sym}_k U^+ \rightarrow \hat{\Sym}_l U^-\quad\text{for all }k, l\ge 0, g\in \Z. \]
This is the notation introduced in \cite[Equation~(2.14)]{Cieliebak2015}.
%\Modify[caption={DONE Filtration preserving},noline]{Perhaps one should require $\MVMorF$ to be filtration preserving instead of just $\Norm{\MVMorF}\ge 0$. When are these two notions equivalent? Something like gapped filtration...}
%\begin{Remark}[On filtered $\MV$-algebras]\label{Rem:FilteredMV}
%\begin{RemarkList}
%\item We recall the following facts from Section~\ref{Sec:Alg1a}. The space $V\COtimes R$ is the completion of $V\otimes_{\K} R$ with respect to the tensor product filtration. It holds $\hat{V}\COtimes\hat{R} \simeq V\COtimes R$. We always take completions in the category of graded $\K$-vector spaces. A map of filtered vector spaces is continuous, by definition, if and only if it preserves filtration. The notation $\Norm{\cdot}$ means the filtration degree. The filtration degree of a continuous map is non-negative. 
%%Also, we recall that we often denote~$\eta(1)$ by~$1$ and~$\Ker \varepsilon$ by~$\bar{V}$.
%\item On $\K$, we consider the trivial filtration
%\begin{equation}\label{Eq:TrivFiltr}
%\Filtr_\lambda \K = \begin{cases} \K & \lambda\le 0, \\ 0 & \lambda >0. \end{cases}
%\end{equation}
%Because $\eta$ and $\varepsilon$ preserve the filtration, and because we require the augmentation property
%\[ \varepsilon(\eta(1))=1, \]
%we see that 
%\[ \Norm{\eta} = \Norm{\varepsilon} = 0,\quad \im \eta \subset \Filtr_0 V\quad\text{and}\quad\ker \varepsilon\supset\sum_{\lambda>0}\Filtr_\lambda V. \]
%\item We can realize a (non-filtered) $\MV$-algebra from \cite{Markl2015} as a filtered $\MV$-algebra by taking the trivial filtration on $V$ and the $\mathfrak{m}$-adic filtration $\Filtr_i\coloneqq \mathfrak{m}^i$ on $R$.
%Notice the additional condition $\Norm{f}\ge 0$.
%However, in contrast to the original definition, we do not require conilpotency of $\delta$ and impose the additional condition $\Norm{\MVMorF}\ge 0$ for morphisms.\Correct[noline]{This is wrong. We need to require some sort of limit conilpotency! DONE}
%
%Let $\MVMorF: V_1\COtimes R \rightarrow V_2\COtimes R$ be a continuous $R$-linear map. For $v=\sum_{k=0}^\infty r_k v_k \in V\COtimes R$ with $r_k\in R$, $v_k\in V$ and $\Norm{r_k v_k} \to \infty$, we have
%\begin{align*}
%&\MVMorF^{\Star n}(v) \\
%&= \sum_{k=0}^\infty \sum_{i=0}^n \underbrace{r_k  \sum_1 \sum \MVMorF(v_{k,(1)})\otimes \MVMorF(1)\otimes \MVMorF(1) \otimes \MVMorF(v_{k,(2)}) \otimes \MVMorF(1)\otimes \dotsb \otimes \MVMorF(v_{k,(n-i)})\otimes \MVMorF(1)}_{\displaystyle =:z_{n,k,i}},
%\end{align*}
%where $\sum_1$ means the summation over all insertions of $1$, $\sum$ is the Sweedler's notation for~$\bar{\delta}$ and $\otimes = \COtimes_R$ is the completed tensor product over $R$. It holds
%\begin{equation}\label{Eq:NormIneq}
%\Norm{z_{n,k,i}} \ge \Norm{r_k} + i \Norm{\MVMorF(1)} + (n-i)\Norm{\MVMorF} + \Norm{\bar{\delta}^{(n-i)}(v_k)}.
%\end{equation} 
%We see that the convergence of a power series of $\MVMorF$ evaluated at $v$ is guaranteed by
%\[ \inf_{\substack{i=0,\dotsc,n \\ k=0,\dotsc, \infty}} \Norm{z_{n,k,i}} \to \infty\quad\text{as }n\to \infty. \]
%From \eqref{Eq:NormIneq}, this is clearly satisfied provided $\Norm{\MVMorF}\ge 0$ and $\Norm{\MVMorF(1)}>0$.
%\qedhere
%\end{RemarkList}
%\end{Remark}
\ToDo[caption={DONE Source and target},noline]{One has to define the product on $\HTP: V \rightarrow V'$, i.e., differenti source and target.}

Based on \cite{Cieliebak2015}, we will now give an equivalent characterization of complete filtered $\IBLInfty$-algebras in $\MV$-formalism and their morphisms in terms of their components $(\OPQ_{klg})$ and $(\HTP_{klg})$ within the surface calculus. We will not repeat the interpretation of the algebraic relations in terms of gluing of surface; for this, see \cite{Cieliebak2015}. We need the following lemma.

\begin{Lemma}[Differential operators on filtered symmetric bialgebras]\label{Lem:ComplSymBialg}
Let $U$ be a complete filtered graded vector space. Then the following holds for the complete filtered symmetric bialgebra $\hat{\Sym} U$ and all $k\in \N_0$:
\begin{ClaimList}
 \item A linear homogenous map $D: \hat{\Sym}U \rightarrow \hat{\Sym}U$ of finite filtration degree is a differential operator of order $\le k$ if and only if it can be written as 
\begin{equation}\label{Eq:DifOpForm}
D = \sum_{i=0}^k D_i \Star \Id
\end{equation}
for linear homogenous maps $D_i : \hat{\Sym}_i U\subset\hat{\Sym}U \rightarrow \hat{\Sym} U$ of finite filtration degrees (here $D_i = 0$ on $\hat{\Sym}_j U$ for $j\neq i$ is the trivial extension). The maps~$D_i$ are uniquely determined by $D$.
 \item A linear homogenous map $\MVMorF:\hat{\Sym}U\rightarrow\hat{\Sym}U$ of finite filtration degree satisfies $\hat{\Sym}_{i>k}U\subset\ker\MVMorF$ if and only if it can be written as $\MVMorF = \MVMorF_0 + \dotsb + \MVMorF_k$ for linear homogenous maps $\MVMorF_i: \hat{\Sym}_iU\rightarrow\hat{\Sym}U$ of finite filtration degrees. The maps $\MVMorF_i$ are uniquely determined by $\MVMorF$.
\end{ClaimList}
\end{Lemma}

\begin{proof}
\begin{ProofList}
%\item This was done in \cite[Proposition~3.2]{Markl1997} with derivatives of order $\le k$ (they are defined by requiring $\psi(v_1,\dotsc,v_{k+1})(1) = 0$ and $D(1)=0$) and in the non-completed case. However, the proof, which we recall below, works in the same way in our case. 
%
%One checks that given $D_i$, \eqref{Eq:DifOpForm} defines a differential operator of order $\le k$. This shows one implication. For the other implication, one checks that vanishing of a differential operator of order $\le k$ on $\hat{\Sym}_{i\le k}U$ implies $D=0$. This shows uniqueness. Finally, given~$D$, one finds $D_i$ by solving a linear equation involving $D\circ \iota_j$ for $0\le j \le i$. This shows the existence.
\item The claim is implied by the following subclaim and proposition, which generalize in a straightforward way to the filtered case. Recall that an operator $D: \Sym U \rightarrow \Sym U$ is called a \emph{derivative} of order $\le k$ if 
\[ D(1)=0\quad\text{and}\quad\psi_i^D(v_1,\dotsc,v_i)(1)=0\quad\text{for all }i\ge k+1\text{ and }v_1, \dotsc, v_i\in V, \]
where $\psi_i^D$ were defined on page~\pageref{Page:DiffOp}.
\begin{SubClaim}[Derivatives and differential operators]\label{SubClaimFiltr}
A homogenous linear operator $D: \Sym U \rightarrow \Sym U$ is a differential operator of order $\le k$ if and only if the linear operator $D' \coloneqq D - D_0\Star \Id$, where $D_0 = D\circ\iota_0$, is a derivative of order $\le k$ (recall that $\iota_k: \Sym_k U \rightarrow \Sym U$ is the inclusion).
\end{SubClaim}
\begin{proof}
Given a derivative $D'$ of order $\le k$ and a linear map $D_0: \Sym_0 U\subset \Sym U \rightarrow \Sym U$, we will check that $D\coloneqq D' + D_0\Star \Id$ is a differential operator of order $\le k$. We have
\begin{align*}
 \underbrace{\psi_{k+2}^{D'}(v_1,\dotsc,v_{k+2})(1)}_{=0}&=
 [\psi^{D'}_{k+1}(v_1,\dotsc,v_{k+1}),L_{v_{k+2}}](1)\\
 &=\psi_{k+1}^{D'}(v_1,\dotsc,v_{k+1})(v_{k+2})-\underbrace{\psi_{k+1}^{D'}(v_1,\dotsc,v_{k+1})(1)}_{=0} v_{k+2}
\end{align*}
for all $v_1$, $\dotsc$, $v_{k+2}\in V$, and hence $D'$ is also a differential operator of order $\le k$. For all $v\in \Sym U$, it holds
\[ (D_0\Star\Id)(v) = D_0(1)v,\]
and hence for all $v$, $v_1\in \Sym U$, we have
\begin{align*}
\psi^{D_0\Star \Id}_1(v_1)(v) &= (D_0\Star \Id)(v_1 v) - (-1)^{D_0 v_1} v_1 (D_0\Star \Id)(v) \\
& = D_0(1)v_1 v - (-1)^{D_0 v_1} v_1 D_0(1) v \\
& = 0.
\end{align*}
It follows that $D_0 \Star \Id$ is a differential operator of order $0$. Clearly, a sum of differential operators of orders $\le k_1$ and $\le k_2$ is a differential operator of order $\le \max(k_1,k_2)$. Therefore, $D = D' + D_0\Star\Id$ is a differential operator of order $\le k$. Moreover, since $D'(1) = 0$ by definition, we have $D(1) = D_0(1)$, i.e., $D_0 = D \circ \iota_0$.

Given a differential operator $D$ of order $\le k$, we define $D_0\coloneqq D\circ\iota_0$ and $D'\coloneqq D - D_0\Star\Id$. Firstly, it holds 
\[ D'(1) = D(1) - (D_0\Star \Id)(1) = 0. \]
Secondly, we have 
\[ \psi_i^D(v_1,\dotsc,v_i) = \psi^{D'}(v_1,\dotsc,v_i) = 0, \]
and hence $\psi_i^D(v_1,\dotsc,v_i)(1) = 0$ for all $i\ge k+1$ and $v_1$, $\dotsc$, $v_i\in V$. Therefore, $D'$ is a derivative of order $\le k$.
\renewcommand{\qed}{\hfill\textit{(Subclaim) }$\square$}
\end{proof}
\begin{ProofProposition}[{\cite[Proposition~3.2]{Markl1997}}]\label{ProofProp:Filtr}
A linear operator $D: \Sym U \rightarrow \Sym U$ is a derivative of order $\le k$ if and only if it can be written as $D = \sum_{i=1}^k D_i \Star \Id$ for unique $D_i : \Sym_i U \rightarrow \Sym U$. 
\end{ProofProposition}
\item This is clear. We have $\MVMorF_i = \MVMorF \circ \iota_i$ for all $i=0$, $\dotsc$, $k$. \qedhere
%Given $\MVMorF$, the maps $\MVMorF_j$ are given by $\MVMorF_j = \MVMorF\circ\iota_j$. Now, $\Norm{\MVMorF_j} \ge \Norm{\MVMorF} + \Norm{\iota_j} \ge \Norm{\MVMorF}$, and $\Norm{\MVMorF}\ge \min_{i=1,\dotsc,k} \Norm{\MVMorF_i}$
\end{ProofList}
\end{proof}


%%The $f_{100}$ do not bubble in the MV-formalism because they are different powers of $\hbar$.

\begin{Proposition}[Filtered $\IBLInfty$-algebras in $\MV$-formalism in components]\label{Prop:EqCharOfMVIBL}
We have the following equivalences of structures:
\begin{ClaimList}
\item The data of a complete filtered $\IBLInfty$-algebra of bidegree $(d,\gamma)$ in $\MV$-formalism $(W,\BVOp)$ is equivalent to the data of a complete filtered graded vector space $W$ and a collection of $\K$-linear maps $\OPQ_{klg}: \hat{\Sym}_k U \rightarrow \hat{\Sym}_l U$ for all $k$, $l$, $g\in \N_0$ which are homogenous, have finite filtration degrees and satisfy the following conditions for all $k$, $l$, $g\ge 0$:
\begin{enumerate}[label=(\arabic*)]
\item  $\Abs{\OPQ_{klg}} = -2d(k+g-1) - 1$.
\item $\Norm{\OPQ_{klg}}\ge -2\gamma(k+g-1)$.
\item The inequality in (2) is strict whenever $k=0$.
\item The sum of maps $Q_{kg} \coloneqq \sum_{l=0}^\infty \OPQ_{klg}: \hat{\Sym}_k U \rightarrow \hat{\Sym}U$ converges (in the sense of Lemma~\ref{Lem:TechLem}).
\item The following equation holds:\footnote{Notice that \eqref{Eq:IBLFormula} does not contain $\OPQ_{00g}$ with $g\ge 0$ for any $k$, $l$, $g\ge 0$ (otherwise all~$\OPQ_{klg}$ appear). In fact, it will be clear from the proof that there is no condition on $\OPQ_{00g}$ coming from $\BVOp^2 = 0$; if we work over a general ring, then $\OPQ_{00g}(1)$ for $g\ge 0$ can be arbitrary odd elements in it. Note that this contradicts \cite[p.\,47, bottom-most paragraph]{Cieliebak2015}.}
\begin{equation}\label{Eq:IBLFormula}
 \sum_{\substack{h \ge 1 \\ k_1, k_2, l_1, l_2, g_1, g_2 \ge 0 \\ k_1 + k_2 - h = k \\ l_1 + l_2 - h = l \\ g_1 + g_2 + h-1 = g}} \OPQ_{k_1 l_1 g_1}\circ_h \OPQ_{k_2 l_2 g_2} = 0.
\end{equation}
\end{enumerate}
\item The data of a strict morphism $\MVMorF: (W^+,\BVOp^+)\rightarrow(W^-,\BVOp^-)$ of complete filtered $\IBLInfty$-algebras of bidegree $(d,\gamma)$ in $\MV$-formalism, strict meaning that $\MVAug^- \circ \MVMorF = \MVMorF \circ \MVUnit^+ = 0$, is equivalent to a collection of $\K$-linear maps $\HTP_{klg}: \hat{\Sym}_k U^+ \rightarrow \hat{\Sym}_l U^-$ for all $k$, $l$, $g\in \N_0$ which are homogenous, have finite filtration degrees and satisfy the following conditions for all $k$, $l$, $g\ge 0$:
\begin{enumerate}[label=(\arabic*)]
\item $\Abs{\HTP_{klg}} = -2d(k+g-1)$.
\item $\Norm{\HTP_{klg}}\ge -2\gamma(k+g-1)$.
\item The inequality in (2) is strict whenever $k=0$.
\item The sum $F_{kg}\coloneqq \sum_{l=0}^\infty \HTP_{klg}: \hat{\Sym}_k U^+ \rightarrow \hat{\Sym}U^-$ converges.
\ToDo[caption={DONE $r=0$ in composition},noline]{Is there $r=0$? Because twisting with $MC$ element has this term, but it does not make much sense in the composition! Do the relations hold also for $k, l =0$ or we do not require that?}
\item It holds $\HTP_{k0g}=\HTP_{0lg}=0$.
\item The following equation holds:
\begin{equation}\label{Eq:WeakIBLMor}\begin{multlined}
\sum_{r=0}^\infty \frac{1}{r!}\sum_{\substack{h_1, \dotsc, h_r \ge 1 \\ k_1, \dotsc, k_r, k^-, l_1, \dotsc, l_r, l^-, g_1, \dotsc, g_r, g^-\ge 0 \\k_1 + \dotsb + k_r = k \\ l_1+ \dotsb + l_r - k^- + l^- = l \\ g_1 + \dotsb + g_r + g^- + k^- - r = g}} \OPQ_{k^- l^- g^-}^-\circ_{h_1,\dotsc,h_r}(\HTP_{k_1 l_1 g_1},\dotsc,\HTP_{k_r l_r g_r}) \\ = \sum_{r=0}^\infty\frac{1}{r!}\sum_{\substack{h_1, \dotsc, h_r \ge 1\\k_1, \dotsc, k_r, k^+, l_1, \dotsc, l_r, l^+, g_1, \dotsc, g_r, g^+ \ge 0 \\k_1 + \dotsb + k_r + k^+ - l^+ = k \\ l_1+ \dotsb + l_r = l \\ g_1 + \dotsb + g_r + g^+ + l^+ - r = g}} (\HTP_{k_1 l_1 g_1},\dotsc,\HTP_{k_r l_r g_r})\circ_{h_1,\dotsc,h_r}\OPQ_{k^+ l^+ g^+}^+.
\end{multlined}\end{equation}
The term $r=0$ on the left- or right-hand side is possible only for $k=0$ or $l=0$ and equals $\OPQ_{0lg}^-\circ\StarProdOne: \hat{\Sym}_0 U^+ \rightarrow \hat{\Sym}_l U^-$ or $\StarProdOne\circ\OPQ_{k0g}^+: \hat{\Sym}_k U^+ \rightarrow \hat{\Sym}_0 U^-$, respectively, where $\StarProdOne$ is the $\hat{\Sym}_0 U^+ \rightarrow \hat{\Sym}_0 U^-$ component of the unit for the convolution product~\eqref{Eq:ConvUnit} (the identity under $\hat{\Sym}_0 U^{\pm} \simeq \K$ via $\MVUnit^{\pm}$). \footnote{If $\HTP$ is not strict, it would be interesting to know whether $e^{\MVMorF} \BVOp^+ = \BVOp^- e^{\MVMorF}$ is still equivalent to the connected calculus \eqref{Eq:WeakIBLMor} or whether there are some other equations possibly involving disconnected gluing. Note that trying to incorporate~$\HTP_{0lg}$ to the right-hand side and~$\HTP_{k0g}$ to the left-hand side of \eqref{Eq:WeakIBLMor} always leads to the disconnected calculus. Also note that $\HTP_{00g}$ for $g\ge 0$ do not appear in \eqref{Eq:WeakIBLMor} for any $k$, $l$, $g\ge 0$; otherwise all~$\HTP_{klg}$ appear. Finally, note that~$\OPQ_{klg}^+$ and~$\OPQ_{klg}^-$ appear for all $k$, $l$, $g\ge 0$ and that it follows from \eqref{Eq:WeakIBLMor} that $\StarProdOne\circ\OPQ_{00g}^+$ and $\OPQ_{00g}^-\circ\StarProdOne$ have to be equal for all $g\ge 0$.}
\end{enumerate}
\end{ClaimList}
The equivalences (a) and (b) are given by the formulas
\begin{equation}\label{Eq:FormulasSums}
\BVOp_i = \sum_{\substack{k, g\ge 0 \\k+g=i}} Q_{kg}\Star\Id \quad\text{and}\quad\MVMorF_i = \sum_{\substack{k,g\ge 0\\k+g=i}} F_{kg}\quad\text{for all }i\ge 0,\text{ respectively.}
\end{equation}
We remark that the operations $\circ_{h_1,\dotsc,h_r}$ are  defined in Definition \ref{Def:ConComp} in the next section or in Definition~\ref{Def:CircS} in Part~I.

Suppose that $\MVMorF^+$ and $\MVMorF^-$ are strict composable morphisms of complete filtered $\IBLInfty$-algebras in $\MV$-formalism. Then we have the following:
\begin{ClaimList}[resume]
\item The composition $\MVMorF\coloneqq \MVMorF^-\DiamComp\MVMorF^+$ is a strict morphism of complete filtered $\IBLInfty$-algebras in $\MV$-formalism and its components $\HTP_{klg}$ for $k$, $l\ge 1$, $g\ge 0$ are given by
\begin{equation}\label{Eq:CompositionOfMorphisms}
\HTP_{klg} = \hspace{-1.8cm}\sum_{\substack{r^-, r^+ \ge 0 \\ k_{1}^-, l_1^-, \dotsc, k_{r^-}^-, l_{r^-}^- \ge 1 \\ k_{1}^+, l_1^+,\dotsc, k_{r^+}^+, l_{r^+}^+ \ge 1  \\ g_1^+, \dotsc, g_{r^+}^+, g_1^-,\dotsc, g_{r^-}^- \ge 0 \\ k_1^- + \dotsb + k^-_{r^-} = l_1^+ + \dotsb + l_{r^+}^+ \\ k_1^+ + \dotsb + k_{r^+}^+ = k \\ l_1^- + \dotsb + l_{r^-}^- = l \\ g_1^+ + \dotsb + g_{r^+}^+ + g_1^- + \dotsb + g_r^- - r^+ - r^- \\ 
+ k_1^- + \dotsb + k_{r^-}^- + 1 = g}}\hspace{-1.5cm}\frac{1}{r^+! r^-!} (\HTP^-_{k_1^- l_1^- g_1^-},\dotsc,\HTP^-_{k_{r^-}^- l_{r^-}^- g_{r^-}^-})\circ_{\mathrm{con}} (\HTP^+_{k_1^+ l_1^+ g_1^+},\dotsc,\HTP^+_{k_{r^+}^+ l_{r^+}^+ g_{r^+}^+}),
\end{equation}
where $\circ_{\mathrm{con}}$ denotes the connected composition (see Definition~\ref{Def:ConComp} in the next section).
\end{ClaimList}
\end{Proposition}
\begin{proof}
\begin{ProofList}
\item Suppose first that $(W,\BVOp)$ is a complete filtered $\IBLInfty$-algebra of bidegree $(d,\gamma)$ in $\MV$-formalism; i.e., we have $\BVOp = \BVOp_0 \hbar^{-1} + \BVOp_1 + \BVOp_2 \hbar + \dotsb$ for $\BVOp_i: \hat{\Sym}U \rightarrow \hat{\Sym}U$ differential operators of order $\le i$, $\Abs{\BVOp} = -1$, $\Norm{\BVOp}\ge 0$ and $\Norm{\BVOp(1)}>0$. Using Lemma~\ref{Lem:ComplSymBialg}, we write $\BVOp_i = \sum_{j=0}^i D_{i j} \Star \Id$ for $\K$-linear maps $D_{ij}:\hat{\Sym}_j U\subset\hat{\Sym}U\rightarrow\hat{\Sym}U$ of finite filtration degrees which are uniquely determined by $\BVOp_i$, and we define $Q_{kg} \coloneqq  D_{k+g,k}: \hat{\Sym}_{k}U \rightarrow \hat{\Sym}U$ for all $k$, $g\in \N_0$. Next, we define $\OPQ_{klg} \coloneqq \pi_l \circ Q_{kg}: \hat{\Sym}_k U \rightarrow \hat{\Sym}_l U$ for all $k$, $l$, $g\in \N_0$.

In the following computations, we will keep in mind that $\Abs{\hbar} = 2d$ and $\Norm{\hbar} = 2\gamma$. Using the formula $\Abs{f \circ g} = \Abs{f} + \Abs{g}$, which is valid for homogenous linear maps $f$ and $g$, we have 
\begin{align*}
\Abs{\BVOp}=-1 &\quad\Equiv\quad\forall i\ge -1: \Abs{\BVOp_{i+1}} = -1 - d i \\
&\quad\Equiv\quad\forall k, g \ge 0, k + g = i+1: \Abs{Q_{kg}} = - 1 - di \\
&\quad\Equiv\quad\forall k, l, g \ge 0, k+g=i+1: \Abs{\OPQ_{klg}} = - 1 - d(k+g-1).
\end{align*}
Thus (1) follows. Using that $\Norm{f\circ g} \ge \Norm{f} + \Norm{g}$ and $\Norm{\sum_i v_i}\ge \inf_i \Norm{v_i}$, we have
\begin{align*}
\Norm{\BVOp}\ge 0 &\quad\Equiv\quad\forall i\ge -1: \Norm{\BVOp_{i+1}} \ge -2\gamma i \\
 &\quad\Equiv\quad\forall k, g \ge 0, k+g=i+1: \Norm{Q_{kg}} \ge -2\gamma i \\
 &\quad\Equiv\quad\forall k, l, g \ge 0, k+g=i+1: \Norm{\OPQ_{klg}} \ge -2\gamma (k+g-1).
\end{align*}
Thus (2) follows. The argument for ``$\Implies$'' on the second line is inductive using that
\[ Q_{kg} = \Bigl(\BVOp_{k+g} - \sum_{j=0}^{k-1} Q_{j, k+g - j} \Star \Id\Bigr)\circ\iota_k \]
for all $k$, $g\ge 0$. Evaluation of $\BVOp(1)$ gives
\begin{align*}
\BVOp(1)&=\BVOp_0(1)\hbar^{-1}+\BVOp_1(1)+\BVOp_2(1)\hbar+\dotsb \\
        &=Q_{00}(1)\hbar^{-1}+Q_{01}(1)+Q_{02}(1)\hbar+\dotsb \\
& = \Bigl(\sum_{l=0}^\infty \OPQ_{0l0}(1) \Bigr)\hbar^{-1} +  \Bigl(\sum_{l=0}^\infty \OPQ_{0l1}(1)\Bigr) + \hbar\Bigl(\sum_{l=0}^\infty \OPQ_{0l2}(1)\Bigr) + \dotsb,
\end{align*}
and we see that $\Norm{\BVOp(1)}>0$ is equivalent to (3). As for (4), it is implied by the Resummation Lemma (Lemma~\ref{Lem:TechLem}). Namely, we have 
\[ \sum_{l=0}^\infty \OPQ_{klg}(v) = \sum_{l=0}^\infty \pi_l(Q_{kg}(v)) \overset{\ref{Lem:TechLem}}{=} Q_{kg}(v)\quad \text{for all }v\in \hat{\Sym}_k U. \]
Thus $\sum_{l=0}^\infty \OPQ_{klg}$ converges to $Q_{kg}$.

On the other hand, given $\OPQ_{klg}$ satisfying (1), (2), (3) and (4), we clearly get an operator $\BVOp: \hat{\Sym}U((\hbar))\rightarrow\hat{\Sym}U((\hbar))$ which has the decomposition \eqref{Eq:BVOpDecompRel} and which satisfies $\Abs{\BVOp} = 0$, $\Norm{\BVOp}\ge 0$ and $\Norm{\BVOp(1)}>0$.

Assuming (1), (2), (3) and (4), it remains to check that $\BVOp^2 = 0$ is equivalent to (5). This is the same computation as in \cite[Section~2]{Cieliebak2015}, just in the filtered setting and allowing $k=0$ or $l=0$ (it will turn out that it does not change anything). Recall the notation from Section~\ref{Sec:Alg1} in Part~I that $\hat{D} \coloneqq D \Star \Id = \mu(D\COtimes \Id)\delta$. We will be using formulas from Remark~\ref{Rem:Compositions} in that section, which are proven in Proposition~\ref{Prop:PartCompositions} in the next section. Note that by the Resummation Lemma, elements of the completion can be summed up in any order, even as nested infinite sums, provided one (and hence all) of these sums converges. Therefore, we could, in principle, write just one sum $\sum$ and think of $\BVOp$ as of the sum $\sum \hat{\OPQ}_{klg} \hbar^{k+g-1}$ over all $k$, $l$, $g\ge 0$.

We have
\[ \begin{aligned} \BVOp\circ\BVOp& = \sum_{i=-2}^\infty \Bigl(\sum_{\substack{k_1, l_1, g_1, k_2, l_2, g_2 \ge 0 \\ k_1 + k_2 + g_1 + g_2 = i + 2}} \hat{\OPQ}_{k_1 l_1 g_1}\circ\hat{\OPQ}_{k_2 l_2 g_2}\Bigr) \hbar^i \\
 & = \sum_{i=-2}^\infty \Bigl(\sum_{\substack{k_1, l_1, g_1, k_2, l_2, g_2 \ge 0 \\ k_1 + k_2 + g_1 + g_2 = i + 2\\ }}\sum_{h=0}^{\min(k_1,l_2)}\reallywidehat{\OPQ_{k_1 l_1 g_1}\circ_h \OPQ_{k_2 l_2 g_2}}\Bigr) \hbar^i, \end{aligned}\]
and we see that 
\[ \langle \BVOp^2 \rangle_{klg} = \sum_{\substack{k_1, l_1, g_1, k_2, l_2, g_2 \ge 0 \\ k_1 + k_2 + g_1 + g_2 = k+g + 1\\ }}\sum_{h=0}^{\min(k_1,l_2)}\pi_l\circ\reallywidehat{\OPQ_{k_1 l_1 g_1}\circ_h \OPQ_{k_2 l_2 g_2}}\circ\iota_k\quad\text{for all }k,l\ge 0, g\ge -1.  \]
Because $\OPQ_{k_1 l_1 g_1}\circ_h \OPQ_{k_2 l_2 g_2}: \hat{\Sym}_{k_1+k_2-h}U\rightarrow\hat{\Sym}_{l_1+l_2-h}U$, it follows from the definition $\hat{f} \coloneqq \mu(f\otimes \Id)\delta$ that $\pi_l \circ \reallywidehat{\OPQ_{k_1 l_1 g_1}\circ_h \OPQ_{k_2 l_2 g_2}}\circ\iota_k = 0$ unless $0\le k_1 + k_2 - h \le k$, $0\le l_1 + l_2 - h\le l$ and $l_1 + l_2 + h - (k_1 + k_2 + h) = l-k$. For fixed $k$, $l$, $h\ge 0$, $g\ge -1$, it holds
\begin{align*}
\left\{\begin{aligned}
&k_1, l_1, g_1, k_2, l_2, g_2\ge 0\\
%&h=0, \dotsc, \min(k_1,l_2)\\
& 0 \le k_1 + k_2 - h \le k \\
& 0 \le l_1 + l_2 - h \le l \\
& l_1 + l_2 - k_1 - k_2 = l - k \\
&k_1 + k_2 + g_1 + g_2 = k + g + 1
\end{aligned}\right\} &= \bigsqcup_{\substack{s=0, \dotsc, k}}\left\{\begin{aligned}
&k_1, k_2, l_1, l_2, g_1, g_2 \ge 0\\
%&h=0, \dotsc, g' + 1\\
&k_1 + k_2 - h = k-s \\
&l_1 + l_2 - h = l-s \\
&g_1 + g_2 + h - 1 = g+s
\end{aligned}\right\},
\end{align*}
and hence\Correct[caption={DONE This is incorrect!!},noline]{The computation is incorrect--it is either $k'<k$ or $l'<l$ and so on.}
\begin{equation}\label{Eq:ComputingComponentsBVOp}
\begin{aligned}
\langle \BVOp^2 \rangle_{klg} &=
\sum_{s=0}^k \sum_{h = 0}^{g + s +1} \sum_{\substack{k_1, k_2, l_1, l_2, g_1, g_2 \ge 0\\k_1+k_2-h=k-s\\l_1+l_2-h=l-s\\ g_1 + g_2 + h - 1 = g+s}} \pi_l \circ \reallywidehat{\OPQ_{k_1 l_1 g_1}\circ_h\OPQ_{k_2 l_2 g_2}}\circ \iota_k \\
&=\underbrace{\sum_{h = 0}^{g+1} \sum_{\substack{k_1, k_2, l_1, l_2, g_1, g_2 \ge 0\\k_1+k_2-h=k\\l_1+l_2-h=l\\ g_1 + g_2 + h - 1 = g}}\OPQ_{k_1 l_1 g_1}\circ_h\OPQ_{k_2 l_2 g_2}}_{\displaystyle=:(\square)_{klg}} + \sum_{s=1}^k  \pi_l \circ \reallywidehat{(\square)_{k-s,l-s,g+s}}\circ \iota_k.
\end{aligned}
\end{equation}
We see immediately that 
\begin{equation}\label{Eq:ComponentsBlaBla}
\langle\BVOp^2 \rangle_{klg} = 0\quad\text{for all }k, l\ge 0, g\ge -1,
\end{equation}
which is equivalent to $\BVOp^2 = 0$, follows from $(\square)_{klg} = 0$ for all $k$, $l\ge 0$, $g\ge -1$. The reverse implication is proven by induction on the linear order $\prec$ on signatures $(k,l,g)$ defined in \cite[Definition~2.4]{Cieliebak2015}. It holds $(k',l',g')\prec (k,l,g)$, by definition, if one of the following  conditions is satisfied:
\begin{enumerate}[label=(\roman*)]\label{Enum:OrderingOfSignatures}
\item $k' + l' + 2g' < k + l + 2g$,
\item $k'+l'+2g' = k + l + 2g$ and $g' > g$, or
\item $k' + l' + 2g' = k + l + 2g$ and $g' = g$ and $k' < k$.
\end{enumerate}
For the last sum in \eqref{Eq:ComputingComponentsBVOp}, we denote $k_s\coloneqq k-s$, $l_s\coloneqq l-s$ and $g_s\coloneqq g+s$ and compute
\[  (k-k_s) + (l-l_s) + 2(g-g_s) = s + s + 2(-s) = 0. \]
Therefore, case (ii) applies, and so $(k_s,l_s,g_s)\prec (k,l,g)$ for all $s=1$, $\dotsc$, $k$. This shows the equivalence of $\BVOp^2 = 0$  and $(\square)_{klg} = 0$ for all $k$, $l\ge 0$, $g\ge -1$. Finally, $(\square)_{k,l,-1} = 0$ holds automatically because $g_1 + g_2 + h =0$ implies $h=0$, and all terms $\OPQ_{k_1 l_1 g_1}\circ_h \OPQ_{k_2 l_2 g_2}$ with $h=0$ vanish. This is because~$\OPQ_{klg}$ are odd and $f_1 \circ_0 f_2 = (-1)^{\Abs{f_1}\Abs{f_2}} f_2\circ_0 f_1$.
%Because $\OPQ_{k_1 l_1 g_1}\circ_h \OPQ_{k_2 l_2 g_2}: \hat{\Sym}_{k_1+k_2-h}U\rightarrow\hat{\Sym}_{l_1+l_2-h}U$, we see from the definition $\hat{f} = \mu(f\otimes \Id)\delta$ that $\pi_l \circ (*)_i \circ\iota_k$ consists of contributions $\reallywidehat{\OPQ_{k_1 l_1 g_1}\circ_h \OPQ_{k_2 l_2 g_2}}$ with $k_1 + k_2 - h \le k$ and $l_1 + l_2 - h\le l$ only. This shows that $\BVOp^2 = 0$ is equivalent to (5).
\item Exactly as in (a), we first prove the equivalence of $\Abs{\MVMorF} = 0$, $\Norm{\MVMorF} \ge 0$, $\Norm{\MVMorF(1)}>0$ and the conditions \eqref{Eq:MorDecompRel} and \eqref{Eq:CondOnMor} to (1), (2), (3) and (4) under the correspondence \eqref{Eq:FormulasSums}. Clearly, $\MVAug^- \circ \MVMorF = \MVMorF \circ \MVUnit^+ = 0$ is equivalent to (5). 

Assuming (1), (2), (3) and (4), it remains to check that the equation $e^\MVMorF \BVOp^+ = \BVOp^- e^\MVMorF$ is equivalent to~(6). This is again the same computation as in \cite[Section~2]{Cieliebak2015}. We will do it in the weak case and then restrict to the strict case for the induction. We have
\begin{align*}
& e^\MVMorF \circ \BVOp^+ \\ 
&\quad = \begin{multlined}[t] \Bigl(\sum_{r=0}^\infty \frac{1}{r!} \sum_{i_1, \dotsc, i_r \ge -1} \sum_{\substack{k_1, l_1, g_1, \dotsc, k_r,l_r,g_r \ge 0\\ k_1 + g_1 = i_1 + 1, \dotsc, k_r + g_r = i_r + 1}}\HTP_{k_1 l_1 g_1}\Star \dotsb \Star \HTP_{k_r l_r g_r} \hbar^{i_1 + \dotsb + i_r}\Bigr)\\\circ\Bigl(\sum_{i^+ = -1}^\infty \sum_{\substack{k^+, l^+, g^+ \ge 0\\k^+ + g^+ = i^+ + 1}} \hat{\OPQ}^+_{k^+ l^+ g^+}\hbar^{i^+}\Bigr)\end{multlined}
\\
&\quad =\sum_{i=-\infty}^\infty\Bigl(\sum_{r=0}^\infty\frac{1}{r!}\sum_{{\substack{k^+, l^+, g^+, k_1, l_1, g_1, \dotsc, k_r,l_r,g_r \ge 0 \\
k^+ + k_1 + \dotsb + k_r + g^+ + g_1 + \dotsb + g_r = i + r + 1 }}}(\HTP_{k_1 l_1 g_1}\Star \dotsb \Star \HTP_{k_r l_r g_r})\circ\hat{\OPQ}_{k^+l^+g^+}^+\Bigr)\hbar^i
 \\
&\quad =\sum_{i=-\infty}^\infty\Bigl(\sum_{r=0}^\infty\frac{1}{r!} \!\!\!\!\underbrace{\sum_{\substack{k^+, l^+, g^+\ge 0 \\ k_1, l_1, g_1, \dotsc, k_r,l_r,g_r \ge 0 \\
k^+ + k_1 + \dotsb + k_r + g^+ \\ + g_1 + \dotsb + g_r = i + r + 1 }}\sum_{\substack{h_1, \dotsc, h_r \ge 0 \\ h_1 + \dotsb + h_r = l^+}}(\HTP_{k_1 l_1 g_1},\dotsc,\HTP_{k_r l_r g_r})\circ_{h_1,\dotsc,h_r}\OPQ_{k^+l^+g^+}^+}_{\displaystyle =:(*)^+_{i,r}:\hat{\Sym}U^+ \rightarrow\hat{\Sym}U^-}\Bigr)\hbar^i
\end{align*}
and similarly
\begin{align*}
& \BVOp^-\circ\,e^{\MVMorF} \\
& \quad = \sum_{i=-\infty}^\infty \Bigl(\sum_{r=0}^\infty \frac{1}{r!}\!\!\!\!\underbrace{\sum_{\substack{k^-, l^-, g^-\ge 0 \\ k_1, l_1, g_1, \dotsc, k_r,l_r,g_r \ge 0 \\
k^- + k_1 + \dotsb + k_r + g^- \\ + g_1 + \dotsb + g_r = i + r + 1 }}\sum_{\substack{h_1, \dotsc, h_r \ge 0 \\ h_1 + \dotsb + h_r = k^-}}\OPQ_{k^- l^- g^-}^-\circ_{h_1,\dotsc,h_r}(\HTP_{k_1 l_1 g_1},\dotsc,\HTP_{k_r l_r g_r})}_{\displaystyle =: (*)^-_{i,r}: \hat{\Sym}U^+ \rightarrow\hat{\Sym}U^-}\Bigr)\hbar^i.
\end{align*}
We will consider $\pi_l \circ (*)^+_{i,r} \circ \iota_k$ for fixed $i\in \Z$ and $k$, $l\ge 0$. Because of the definition of $\circ_{h_1,\dotsc,h_r}$, only the terms with $l_1 + \dotsb + l_r = l$ and $k^+ + k_1 + \dotsb + k_r - h_1 - \dotsb - h_r = k^+ + k_1 + \dotsb + k_r - l^+= k$ survive in $\pi_l \circ (*)^+_{i,r} \circ \iota_k$. Denoting $g\coloneqq i - k + 1$, we have
\[\begin{aligned}
k^+ + k_1 + \dotsb + k_r + g^+ + g_1 + \dotsb + g_r &= i +r +1 \\
k^+ + k_1 + \dotsb + k_r  - l^+ &= k \\
l_1 + \dotsb + l_r &= l
\end{aligned}\!\Equiv
\begin{aligned}
g_1 + \dotsb + g_r + g^+ + l^+ -  r &= g \\
k^+ + k_1 + \dotsb + k_r  - l^+ &= k \\
l_1 + \dotsb + l_r &= l.
\end{aligned}\]
Therefore, it holds
\begin{align*}
\pi_l \circ (*)^+_{i,r} \circ \iota_k  = \!\!\sum_{\substack{k^+,l^+,g^+\ge 0\\ 
k_1,l_1,g_1,\dotsc,k_r,l_r,g_r\ge 0 \\
k^+ + k_1 + \dotsb + k_r - l^+ = k \\
l_1 + \dotsb + l_r = l  \\ g_1 + \dotsb + g_r + g^+ + l^+ - r = g }}\sum_{\substack{h_1,\dotsc,h_r\ge 0\\ h_1 + \dotsb + h_r = l^+}}(\HTP_{k_1 l_1 g_1},\dotsc,\HTP_{k_r l_r g_r})\circ_{h_1,\dotsc, h_r} \OPQ_{k^+ l^+ g^+}^+.
\end{align*}\ToDo[caption={DONE Finis the computation of the eponential},noline]{Finish the computation of the exponential!!}
%\begin{align*}
%\pi_l \circ (*)^-_{i,r} \circ \iota_k  = \!\! \sum_{\substack{k^-,l^-,g^-,k_1,l_1,g_1,\dotsc,k_r,l_r,g_r\ge 0 \\ g_1 + \dotsb + g_r + g^- + k^- - r = g \\
%l^- + l_1 + \dotsb + l_r - k^- = l \\
%k_1 + \dotsb + k_r = k}}\sum_{\substack{h_1,\dotsc,h_r\ge 0\\ h_1 + \dotsb + h_r = k^-}}\!\OPQ_{k^- l^- g^-}^-\circ_{h_1,\dotsc, h_r}(\HTP_{k_1 l_1 g_1},\dotsc,\HTP_{k_r l_r g_r}).
%\end{align*}
%\begin{align*}
%\left\{\begin{aligned} 
%&h_1,\dotsc,h_r\ge 0\\
%&k^+,l^+,g^+,k_1,l_1,g_1,\dotsc,k_r,l_r,g_r\ge 0 \\
%&h_1 + \dotsb + h_r = l^+ \\
%&k^+ + l_1 + \dotsb + l_r - l^+ = l \\
%&l_1 + \dotsb + l_r = l \\
%&g_1 + \dotsb + g_r + g^- + l^+ - r = g
%\end{aligned}\right\} = \bigsqcup_{\substack{r' = 0, \dotsc, r \\ k', l', g'}} \left\{\begin{aligned}
%&h_{r'+1} = \\
%&k^+, l^+, g^+, k_1, l_1, g_1, \dotsc, k_{r'}, l_{r'}, g_{r'}\ge 0 \\
%&h_1 + \dotsb + h_{r'}=l^+ \\
%&k^+ + l_1 + \dotsb + l_{r'} - l^+ = l \\
%&l_1 + \dotsc + l_{r'} = l'\end{aligned}\right\} \\
%&g_1 + \dotsb + g_{r'} + g^+ + l^+ - r' = g'
%\left\{\begin{aligned} &k_{r'+1}, l_{r'+1}, g_{r'+1}, \dotsc, k_r, l_r, g_r \ge 0 \\ 
%&k_{r'+1} + \dotsb + k_{r} = k'' \\
%&l_{r'+1} + \dotsb + l_r = l'' \\
%&g_{r'+1} + \dotsb + g_r - r'' = g''
%\end{aligned}\right\}
%\end{align*}
Collecting the terms with $h_i=0$ and using the graded commutativity of $\Star$, we obtain
\begin{align*}
\pi_l \circ (*)^+_{i,r} \circ \iota_k &= \begin{multlined}[t]\underbrace{\sum_{\substack{h_1, \dotsc, h_r \ge 1\\ k^+, l^+, g^+, k_1, l_1, g_1, \dotsc, k_r, l_r, g_r\ge 0 \\  h_1 + \dotsb + h_{r} = l^+ \\k_1 + \dotsb + k_r + k^+ - l^+ = k \\ l_1+ \dotsb + l_r = l \\ g_1 + \dotsb + g_r + g^+ + l^+ - r = g}} (\HTP_{k_1 l_1 g_1},\dotsc,\HTP_{k_r l_r g_r})\circ_{h_1,\dotsc,h_r}\OPQ_{k^+ l^+ g^+}^+}_{\displaystyle=:(\square^+)^r_{klg}}\\
+ \sum_{r'=0}^{r-1} \binom{r}{r'}\hspace{-.5cm}\sum_{\substack{h_1,\dotsc,h_{r'}\ge 1 \\
k^+,l^+,g^+ \ge 0\\
k_1,l_1,g_1,\dotsc,k_r,l_r,g_r\ge 0 \\
h_1 + \dotsb + h_{r'} = l^+ \\
k_1 + \dotsb + k_r + k^+ - l^+ = k \\
l_1 + \dotsb + l_r = l \\
g_1 + \dotsb + g_r + g^+ + l^+ - r = g }}\hspace{-.5cm}\begin{aligned}[t]
(\HTP_{k_1 l_1 g_1},\dotsc,\HTP_{k_{r'} l_{r'} g_{r'}})\circ_{h_1,\dotsc, h_{r'}} \OPQ^+_{k^+ l^+ g^+}&\\
\Star \HTP_{k_{r'+1} l_{r'+1} g_{r' + 1}}\Star\dotsb\Star\HTP_{k_r l_r g_r}&
\end{aligned}
\end{multlined} \\
%&\quad= \begin{multlined}[t](\square^+)_{klg}^r + \sum_{r'=0}^{r-1} \binom{r}{r'}  \sum_{\substack{k', l', g'}}  \Biggl( \sum_{\substack{h_1, \dotsc, h_{r'}\ge 1 \\ k^+, l^+, g^+ \ge 0 \\ k_1, l_1, g_1, \dotsc, k_{r'}, l_{r'}, g_{r'} \ge 0\\ h_1 + \dotsb + h_{r'} = l^+ \\ k_1 + \dotsb + k_{r'} + k^+ - l^+ = k' \\ l_1 + \dotsb + l_{r'} = l' \\ g_1 + \dotsb + g_{r'} + g^+ + l^+ - r' = g'}} (\HTP_{k_1 l_1 g_1},\dotsc,\HTP_{k_r l_r g_r})\circ_{h_1,\dotsc,h_r}\OPQ_{k^+ l^+ g^+}^+ \Biggr) \\
%\Star\Biggl(\underbrace{\sum_{\substack{k_{r'+1},l_{r'+1},g_{r'+1},\dotsc,k_r, l_r, g_r \ge 0 \\ k_{r'+1} + \dotsb + k_r = k - k' \\
%l_{r'+1} + \dotsb + l_r = l - l' \\
%g_{r'+1} + \dotsb + g_r - (r - r') = g - g' }} \HTP_{k_{r'+1} l_{r'+1} g_{r' + 1}}\Star\dotsb\Star\HTP_{k_r l_r g_r}}_{\displaystyle =:(\triangle)_{k-k',l-l',g-g'}^{r-r'}} \Biggr)\end{multlined}\\
&= (\square^+)^r_{klg} + \sum_{r'=0}^{r-1}\binom{r}{r'}\begin{aligned}[t]&\sum_{\substack{0\le k' \le k \\ 0\le l'\le l \\ 0\le g'\le g + r - r'}} \Biggl((\square^+)_{k'l'g'}^{r'} \\ &\Star\hspace{-1.21cm}\underbrace{\sum_{\substack{k_{r'+1},l_{r'+1},g_{r'+1},\dotsc,k_r,l_r,g_r\ge 0\\
k_{r'+1} + \dotsb + k_r = k-k' \\
l_{r'+1} + \dotsb + l_r = l-l' \\
g_{r'+1} + \dotsb + g_r - (r - r') = g-g'}}\HTP_{k_{r'+1} l_{r'+1} g_{r' + 1}}\Star\dotsb\Star\HTP_{k_r l_r g_r}}_{\displaystyle =: (\triangle)^{r-r'}_{k-k',l-l',g-g'}}\Biggr). \end{aligned}
\end{align*}
We used here that for fixed $r\ge 0$, $0\le r'\le r-1$, $k$, $l\ge 0$ and $g\in \Z$, we have
\begin{equation*}
\left\{\begin{aligned}
&h_1, \dotsc, h_{r'} \ge 1 \\
&k^+, l^+, g^+ \ge 0 \\
&k_1, l_1, g_1, \dotsc, k_r, l_r, g_r \ge 0 \\\hline
&h_1+\dotsb+h_{r'}=l^+\\
&k_1+\dotsb+k_r+k^+-l^+=k\\
&l_1+\dotsb+l_r=l\\
&g_1+\dotsb+g_r+g^++l^+-r=g
\end{aligned}\right\} = \!\!\!\!\!\!\!\begin{aligned}[t]\bigsqcup_{\substack{0\le k' \le k\\ 0\le l'\le l\\ -r'\le g' \le g+r-r'}}\left\{\begin{aligned}
&h_1, \dotsc, h_r'\ge 1\\
&k^+,l^+,g^+\ge 0\\
&k_1,l_1,g_1,\dotsc,k_{r'},l_{r'},g_{r'} \ge 0\\\hline
&h_1 + \dotsb + h_{r'} = l^+\\
&k_1 + \dotsb + k_{r'} + k^+ - l^+ = k' \\
&l_1 + \dotsb + l_{r'} = l' \\
&g_1 + \dotsb + g_{r'} + g^+ + l^+ - r' = g'
\end{aligned}\right\}\\
\times
\left\{\begin{aligned}
&k_{r'+1}, l_{r'+1}, g_{r'+1}, \dotsc, k_r, l_r, g_r \ge 0 \\\hline
&k_{r'+1} + \dotsb + k_r = k-k' \\
&l_{r'+1} + \dotsb + l_r = l-l' \\
&g_{r'+1} + \dotsb + g_r - (r-r') = g-g'
\end{aligned}\right\},
\end{aligned}
\end{equation*}
where the notation is the vertical version of $\{ \cdot \mid \cdot \}$. In fact, the summation starts from $g'=0$ because if $g'<0$, then $(\square^+)_{k'l'g'}^{r'} = 0$. Summing over $r\in \N_0$, we get
\begin{equation}\label{Eq:PositiveEq}\begin{aligned}
&\sum_{r=0}^\infty \frac{1}{r!} \pi_l \circ (*)_{i,r}^+ \circ \iota_k \\
&\quad = \sum_{r=0}^\infty \frac{1}{r!} (\square^+)_{klg}^r  + \sum_{r=0}^\infty \sum_{r'=0}^{r-1} \frac{1}{r'!} \frac{1}{(r-r')!}\!\!\!\!\!\sum_{\substack{0\le k' \le k \\ 0\le l'\le l \\ 0 \le g'\le g + r - r'}}\!\!\!\!(\square^+)^{r'}_{k'l'g'}\Star (\triangle)^{r-r'}_{k-k', l- l', g - g'} \\
&\quad= \sum_{r=0}^\infty \frac{1}{r!} (\square^+)_{klg}^r + \sum_{\substack{0\le k' \le k \\ 0\le l'\le l \\ g'\ge 0}}\Bigl(\sum_{r' = 0}^\infty \frac{1}{r'!} (\square^+)_{k'l'g'}^{r'}\Bigr)\Star\Bigl(\sum_{t=1}^\infty \frac{1}{t!}(\triangle)_{k-k',l-l',g-g'}^{t}\Bigr),
\end{aligned}\end{equation}
where we used the substitution $t=r-r'$ and the fact that if $g-g' < -t$, then $(\triangle)_{k-k',l-l',g-g'}^t = 0$. Similarly, we obtain
\begin{equation}\label{Eq:NegativeEq}\begin{aligned}
&\sum_{r=0}^\infty \frac{1}{r!} \pi_l \circ (*)_{i,r}^- \circ \iota_k \\
&\quad = \sum_{r=0}^\infty \frac{1}{r!} (\square^-)_{klg}^r + \sum_{\substack{0\le k' \le k \\ 0\le l'\le l \\ g'\ge 0}}\Bigl(\sum_{t=1}^\infty \frac{1}{t!}(\triangle)_{k-k',l-l',g-g'}^{t}\Bigr)\Star\Bigl(\sum_{r' = 0}^\infty \frac{1}{r'!} (\square^-)_{k'l'g'}^{r'}\Bigr).
\end{aligned}\end{equation}
Now, $e^{\MVMorF}\BVOp^+ = \BVOp^- e^{\MVMorF}$ is equivalent to 
\begin{equation}\label{Eq:SequenceOfEquations}
\sum_{r=0}^\infty \frac{1}{r!} \pi_l \circ (*)^+_{i,r} \circ \iota_k = \sum_{r=0}^\infty \frac{1}{r!}\pi_l \circ (*)^-_{i,r} \circ \iota_k\quad\text{for all }k, l \ge 0\text{ and }i\in \Z.
\end{equation}
By our previous computations, this follows from
\[ \sum_{r=0}^\infty \frac{1}{r!} (\square^+)_{klg}^r = \sum_{r=0}^\infty \frac{1}{r!} (\square^-)_{klg}^r\quad \text{for all }k, l, g\ge 0, \]
which are precisely equations \eqref{Eq:WeakIBLMor}. In order to prove the reverse implication, one needs to do induction on signatures $(k,l,g)$ as in (a). It is possible in the strict case by the following lemma.
\begin{SubClaim}[Induction in strict case I]
Suppose that $\HTP_{0lg} = \HTP_{k0g} = 0$ for all $k$, $l$, $g\ge 0$. Then for any $k$, $l\ge 0$ and $i\in\Z$, $g=i-k+1$, only the terms with $(k',l',g')\prec(k,l,g)$ contribute to the sums on the right-hand side of \eqref{Eq:PositiveEq} and \eqref{Eq:NegativeEq}. These sums vanish if $(k,l,g) = (0,0,0)$.
\end{SubClaim}
\begin{proof}
Recall the definition of $\prec$ on p.\,\pageref{Enum:OrderingOfSignatures}. First of all, $(\triangle)_{k-k',l-l',g-g'}^t \neq 0$ implies
\[ D\coloneqq (k+l+2g) - (k' + l' + 2 g') = (k-k') + (l-l') + 2(g-g') \ge t + t + 2(-t) = 0, \]
where $k-k'$, $l-l'\ge t$ holds due to strictness. Suppose that $D=0$. Then $g'\ge g$ must hold because $k-k'\ge 0$ and $l-l'\ge 0$. If $g'=g$, then $l=l'$ and $k=k'$ must hold; this is a contradiction with $k-k'\ge t \ge 1$. Therefore, it holds $g'>g$, and case (ii) of the definition of $\prec$ applies.

If $(k,l,g) = (0,0,0)$, then $(\triangle)^t_{0,0,-g'} = 0$ because $t\ge 1$ and $k-k'\ge t$.\renewcommand{\qed}{\hfill\textit{(Subclaim) }$\square$}
\end{proof}
\item We have
\[ e^\MVMorF = \sum_{k,l,g\ge 0} \langle e^\MVMorF \rangle_{klg} \hbar^{k+g-1}, \]
where
\begin{align}
\langle e^{\MVMorF}\rangle_{klg} &= \sum_{\substack{r\ge 0 \\ k_1, l_1, g_1, \dotsc, k_r, l_r, g_r \ge 0 \\ k_1 + \dotsb + k_r = k \\ l_1 + \dotsb + l_r = l \\ g_1 + \dotsb + g_r - r + 1 = g}} \frac{1}{r!} \HTP_{k_1 l_1 g_1}\Star\dotsb\Star\HTP_{k_r l_r g_r} \nonumber \\
& = \HTP_{klg} + \sum_{\substack{r\ge 2 \\ k_1, l_1, g_1, \dotsc, k_r, l_r, g_r \ge 0 \\ k_1 + \dotsb + k_r = k \\ l_1 + \dotsb + l_r = l \\ g_1 + \dotsb + g_r - r + 1 = g}} \frac{1}{r!} \HTP_{k_1 l_1 g_1}\Star\dotsb\Star\HTP_{k_r l_r g_r} \label{Eq:ComponentOfExpo}
\end{align}
for all $(k,l,g)\neq (0,0,1)$ with $k$, $l\ge 0$, $g\in\Z$. For $(k,l,g) = (0,0,1)$, one has to add the unit $\StarProdOne$ coming from the summand with $r=0$. We find that
\[\langle e^{\MVMorF^-}e^{\MVMorF^+}\rangle_{klg} = \hspace{-2cm}\sum_{\substack{r^-, r^+ \ge 0 \\ k_{1}^-, l_1^-, g_1^-, \dotsc, k_{r^-}^-, l_{r^-}^-, g_{r^-}^- \ge 0 \\ k_{1}^+, l_1^+, g_1^+, \dotsc, k_{r^+}^+, l_{r^+}^+, g_{r^+}^+ \ge 0 \\ k_1^- + \dotsb + k^-_{r^-} = l_1^+ + \dotsb + l_{r^+}^+ \\ k_1^+ + \dotsb + k_{r^+}^+ = k \\ l_1^- + \dotsb + l_{r^-}^- = l \\ g_1^+ + \dotsb + g_{r^+}^+ + g_1^- + \dotsb + g_r^- - r^+ - r^- + k_1^- + \dotsb + k_{r^-}^- + 1 = g}}\hspace{-2.4cm}\frac{1}{r^+! r^-!} (\HTP^-_{k_1^- l_1^- g_1^-} \Star \dotsb \Star \HTP^-_{k_{r^-}^- l_{r^-}^- g_{r^-}^-})\circ (\HTP^+_{k_1^+ l_1^+ g_1^+} \Star \dotsb \Star \HTP^+_{k_{r^+}^+ l_{r^+}^+ g_{r^+}^+})\]
for all $k$, $l\ge 0$, $g\in \Z$. The composition $\MVMorF^-\DiamComp\MVMorF^+ = \log(e^{\MVMorF^-}\circ e^{\MVMorF^+})$ is the unique $\MVMorF$ such that $e^{\MVMorF} = e^{\MVMorF^-}\circ e^{\MVMorF^+}$; this is equivalent to 
\begin{equation}\label{Eq:IndEqComp}
 \langle e^\MVMorF \rangle_{klg} = \langle e^{\MVMorF^-}e^{\MVMorF^+}\rangle_{klg}\quad\text{for all }k, l\ge 0, g\in\Z.
\end{equation}
In the strict case, this can be solved for $(\HTP_{klg})$ by induction on signatures.
\begin{SubClaim}[Induction in strict case II]
Suppose that $\HTP^+$ and $\HTP^-$ are strict. Then it holds $(k_i,l_i,g_i)\prec (k,l,g)$ for all $i=1$,~$\dotsc$, $r$ and $r\ge 2$ in \eqref{Eq:ComponentOfExpo} for all $k$, $l\ge 0$, $g\in\Z$.
\end{SubClaim}
\begin{proof}
We compute
\[ k + l + 2g - 2 = (k_1 + l_1 + 2 g_1 -2) + \dotsb + (k_r + l_r + 2 g_r -2), \]
where $k_i + l_i + 2g_i - 2 \ge 0$ by strictness. It follows that $k+l+2g \ge k_i + l_i + 2 g_i$ holds for all $i=1$,~$\dotsc$, $r$. If, e.g., $k + l + 2g = k_1 + l_1 + 2 g_1$, then $k_2 = l_2 = \dotsb = k_r = l_r = 1$, $g_2 = \dotsb = g_r = 0$, and hence $g = 1-r$. However, this can not happen as $r\ge 2$ and $g\ge 0$. Therefore, case (i) of $\prec$ always occurs.
\renewcommand{\qed}{\hfill\textit{(Subclaim) }$\square$}
\end{proof}
Using this, we can set $\HTP_{110} = \HTP^-_{110}\circ\HTP^+_{110}$ and $\HTP_{0lg}=\HTP_{k0g}\coloneqq 0$ for all $k$, $l$, $g\ge 0$ and solve~\eqref{Eq:IndEqComp} for $(\HTP_{klg})$ for all $k$, $l\ge 1$, $g\ge 0$ by induction over $(k,l,g)$. The solution \eqref{Eq:CompositionOfMorphisms} will solve~\eqref{Eq:IndEqComp} for all $k$, $l\ge 0$, $g\in \Z$ (the equations for $g<0$ consist of disconnected gluings and can be checked by splitting into connected components).\Correct[caption={Negative $g$},noline]{How is it with negative $g$'s? I never really checked.} Conditions on the filtration degree are easy to check as in \cite[Lemma~8.5]{Cieliebak2015}.
\qedhere
\end{ProofList}
\end{proof}

We expect that weak complete filtered $\IBLInfty$-algebras in $\MV$-formalism with weak morphisms form a subcategory of the category of complete filtered $\MV$-algebras (to see this, it remains to prove a version of (c) of Proposition~\ref{Prop:EqCharOfMVIBL} for weak morphisms). The identity morphism in this category is the continuous $\K((\hbar))$-linear extension of the trivial extension of the identity $\Id: \hat{U} \rightarrow \hat{U}$ to $\hat{\Sym}U((\hbar))\rightarrow\hat{\Sym}U((\hbar))$ (see (iii) of Remark~\ref{Rem:ExpLogStar}).

In the following, we will compare Definition~\ref{Def:ComplFiltrIBL} to the definition of filtered $\IBLInfty$-algebras from \cite[Section~8]{Cieliebak2015}.

A \emph{filtered $\IBLInfty$-algebra of bidegree $(d,\gamma)$ over $\K$ on a complete filtered graded vector space $W$ according to \cite{Cieliebak2015}} is a collection of homogenous $\K$-linear maps $\OPQ_{klg}:\hat{\Sym}_k U\rightarrow\hat{\Sym}_l U$ of finite filtration degrees for all $k$, $l$, $g\in \N_0$, where $U$ was defined in \eqref{Eq:DefOfU}, which satisfy, firstly, (1) and (2) of (a) of Proposition~\ref{Prop:EqCharOfMVIBL} with the strict inequality in (2) for all $(k,l,g)$ from the set 
\begin{equation}\label{Eq:UnstableSignatures}
\bigl\{(0,0,0), (1,0,0), (0,1,0), (2,0,0), (0,2,0), (0,0,1)\bigr\}
\end{equation}
and, secondly, the equations \eqref{Eq:ComponentsBlaBla}. A \emph{morphism of filtered $\IBLInfty$-algebras of bidegree $(d,\gamma)$ over~$\K$ on complete filtered graded vector spaces $W^+$ and $W^-$ according to \cite{Cieliebak2015}} 
is a collection of homogenous $\K$-linear maps $\HTP_{klg}: \hat{\Sym}_k U^+ \rightarrow \hat{\Sym}_l U^-$ of finite filtration degrees for all $k$, $l$, $g\in \N_0$ which satisfy, firstly, (1) and (2) of (b) of Proposition~\ref{Prop:EqCharOfMVIBL} with the strict inequality in (2) for $(k,l,g)\in\eqref{Eq:UnstableSignatures}$ and, secondly, the equations~\eqref{Eq:SequenceOfEquations}.

Recall \eqref{Eq:DefOfU} and notice that if we define the filtration $\Filtr_W^\lambda U \coloneqq (\Filtr^\lambda W)[1]$ for all $\lambda\in \R$ and denote the corresponding filtration degree by $\Norm{\cdot}_W$, then it holds $\Norm{\cdot} = \Norm{\cdot}_W + \gamma$ on~$U$, and for a map $\OPQ_{klg}:\hat{\Sym}_k U\rightarrow\hat{\Sym}_l U$, we have
\[ \Norm{\OPQ_{klg}}\ge -2\gamma(k+g-1)\quad\Equiv\quad\Norm{\OPQ_{klg}}_W \ge \gamma(2-2g-k-l). \]

Signatures \eqref{Eq:UnstableSignatures} together with $(1,1,0)$ correspond to unstable surfaces, i.e., those $(k,l,g)$ for which $\chi_{klg}=2-2g-k-l \ge 0$. Allowing morphisms $\MVMorF$ which have non-zero components of these signatures leads to the appearance of an infinite number of summands in $\langle \BVOp^- e^{\MVMorF} \rangle_{klg}$, $\langle e^{\MVMorF}\BVOp^+\rangle_{klg}$ and $\langle e^{\MVMorF^-}e^{\MVMorF^+}\rangle_{klg}$ (and in the Maurer-Cartan equation \eqref{Eq:TwistingEq} later); the strict inequalities for filtration degrees of these components seem to be the minimal condition to algebraically handle this situation. Again, the author does not see any technical reason for imposing strict filtration degree conditions for~$\OPQ_{klg}$. See Figure~\ref{Fig:Bubbling} for the graphical explanation of the \emph{bubbling}.

We observe the following differences between our approach to weak $\IBLInfty$-algebras and the original approach from \cite{Cieliebak2015}: 

\begin{enumerate}[label=(\roman*)]
\item \emph{Bubbling.} The definition of \cite{Cieliebak2015} is symmetric in inputs and outputs, whereas our definition is not (compare \eqref{Eq:UnstableSignatures} and (3) of Proposition~\ref{Prop:EqCharOfMVIBL}). We will illustrate this in Example~\ref{Ex:AsymOfMV} below.

\item \emph{General ring.} The theory of \cite{Cieliebak2015} is formulated over a filtered graded commutative ring~$R$, i.e., $W$ is a filtered $R$-module and the maps are $R$-linear. We can tweak our formalism to handle this situation if $R$ is a $\K$-algebra (e.g., the Novikov ring) by replacing $\K((\hbar))$ with $R((\hbar))$ in Definition~\ref{Def:ComplFiltrIBL}.

\item \emph{Completions.} The $\BV$-formalism of \cite{Cieliebak2015} uses the completion $\hat{\Sym}U$ with respect to the union filtration $\Filtr_\cup$ of the induced filtration and the filtration by weights (see Section~\ref{Sec:DetailsOnFiltr}); however, this is not necessarily a filtered bialgebra (see Example~\ref{Ex:CombinedOnSymetric}), Proposition~\ref{Prop:ConvPwrSer} might not apply, and it is not clear whether~$e^{\HTP}$ (or the multiplication with the exponential of the Maurer-Cartan element later) are well-defined operators on $\hat{\Sym}U((\hbar))$.
\end{enumerate}

\begin{figure}
\centering
{ \begingroup \allowdisplaybreaks
\def\dist{0.25} %distance between two surfaces
  \def\rad{0.5} % radius of bdd
  \def\ecc{0.1} % eccentricity of bdd
  \def\hght{1} % height of surfaces
  \def\dif{1.5} % distance of two circles
  \def\radO{\rad} % radius of bdd
  \def\eccO{\ecc} % eccentricity of bdd
  \def\hghtO{2*\hght+\dist} % height of surfaces
  \def\difO{\dif} % distance of two circles
  \def\gencanc{0.05} % legth of extra line in genus
  \def\genecc{20} % eccentricity of genus
  \def\genrad{0.45} % radius of genus 
\begin{subfigure}{.45\textwidth}
\centering
%auto-ignore
\begin{tikzpicture}
%%% First Object
%  \coordinate (P1) at (0,0);
%  \coordinate (P2) at ($(P1) + (2*\dif,0)$);
%  \coordinate (P3) at ($(P2) + (2*\dif,0)$);
%
%
%  
%  \coordinate (P7) at ($(P1)+(0,-\hght)$);
%  \\
%
%  
%  \draw (P1) arc (180:360:{\rad} and {\ecc});
%  \draw (P1) arc (180:0:{\rad} and {\ecc});
%  
%  \draw (P2) arc (180:360:{\rad} and {\ecc});
%  \draw (P2) arc (180:0:{\rad} and {\ecc});
%  
%  \draw (P3) arc (180:360:{\rad} and {\ecc});
%  \draw (P3) arc (180:0:{\rad} and {\ecc});
%  
%  \draw (P4) arc (180:360:{\rad} and {\ecc});
%  \draw (P4) arc (180:0:{\rad} and {\ecc});
%  
%  \draw (P5) arc (180:360:{\rad} and {\ecc});
%  \draw (P5) arc (180:0:{\rad} and {\ecc});
%  
%
%  
%  \draw (P7) arc (180:360:{\rad} and {\ecc});
%  \draw[dashed] (P7) arc (180:0:{\rad} and {\ecc});
%
%  \draw (P8) arc (180:360:{\rad} and {\ecc});
%  \draw[dashed] (P8) arc (180:0:{\rad} and {\ecc});
% 
%  %\draw ($(P1)+(2*\rad,0)$) to[out=-90,in=-90] (P2);
%  %\draw ($(P3)+(2*\rad,0)$) to[out=-90,in=-90] (P4);
%  \draw ($(P4)+(2*\rad,0)$) to[out=-90,in=-90] (P5);
%  
%  \draw (P1)--(P7);
%  \draw ($(P8)+(2*\rad,0)$) to[out=35,in=-130] ($(P6)+(2*\rad,0)$);
%  
%  \tikzset{decorate sep/.style 2 args={decorate,decoration={shape backgrounds,shape=circle,shape size=#1,shape sep=#2}}} 
% 
% \draw[decorate sep={0.3mm}{2mm},fill] ($(P7)+(2*\rad+\dif,0)$) -- ($(P8)+(-\dif,0)$);
%
% \draw[decorate sep={0.3mm}{2mm},fill] ($(P2)+(2*\rad+\dif,0)$) -- ($(P3)+(-\dif,0)$);
% 
%  \draw[decorate sep={0.3mm}{2mm},fill] ($(P5)+(2*\rad+\dif,0)$) -- ($(P6)+(-\dif,0)$);
%  
%   \draw ($(P1)+(2*\rad,0)$) to[out=-90,in=-90] ($(P1)+(2*\rad+\dist,0)$);
%   \draw ($(P2)+(-\dist,0)$) to[out=-90,in=-90] ($(P2)$);
%   \draw ($(P2)+(2*\rad,0)$) to[out=-90,in=-90] ($(P2)+(2*\rad+\dist,0)$);
%   \draw ($(P3)+(-\dist,0)$) to[out=-90,in=-90] ($(P3)$);
%   \draw ($(P3)+(2*\rad,0)$) to[out=-90,in=-90] ($(P3)+(2*\rad+\dist,0)$);
%   \draw ($(P4)+(-\dist,0)$) to[out=-90,in=-90] ($(P4)$);
%   \draw ($(P7)+(2*\rad,0)$) to[out=90,in=90] ($(P7)+(2*\rad+\dist,0)$);
%   \draw ($(P8)+(-\dist,0)$) to[out=90,in=90] ($(P8)$);
%   
%% Second body
%
%  \coordinate (P9) at ($(P1) + (0,\dist)$);
%  \coordinate (P10) at ($(P2) + (0,\dist)$);
%  \coordinate (P11) at ($(P1) + (0,\dist+\hght)$);
%  \coordinate (P12) at ($(P2) + (0,\dist+\hght)$);
%  
%  \draw (P9) arc (180:360:{\rad} and {\ecc});
%  \draw[dashed] (P9) arc (180:0:{\rad} and {\ecc});  
%  \draw (P10) arc (180:360:{\rad} and {\ecc});
%  \draw[dashed] (P10) arc (180:0:{\rad} and {\ecc});
%  \draw (P11) arc (180:360:{\rad} and {\ecc});
%  \draw (P11) arc (180:0:{\rad} and {\ecc});
%  \draw (P12) arc (180:360:{\rad} and {\ecc});
%  \draw (P12) arc (180:0:{\rad} and {\ecc});
%  
%  \draw (P9) -- (P11);
%  \draw ($(P10)+(2*\rad,0)$) -- ($(P12)+(2*\rad,0)$);
%
%% Third Body
%
%  \coordinate (P13) at ($(P3) + (0,\dist)$);
%  \coordinate (P14) at ($(P4)+(0,\dist)$); 
%  \coordinate (P15) at ($(P3) + (0,\dist+\hght)$);
%  \coordinate (P16) at ($(P4) + (0,\dist+\hght)$);
%
%  
%  \draw (P13) arc (180:360:{\rad} and {\ecc});
%  \draw[dashed] (P13) arc (180:0:{\rad} and {\ecc});
%  
%  \draw (P14) arc (180:360:{\rad} and {\ecc});
%  \draw[dashed] (P14) arc (180:0:{\rad} and {\ecc}); 
%  \draw (P15) arc (180:360:{\rad} and {\ecc});
%  \draw (P15) arc (180:0:{\rad} and {\ecc});
%  \draw (P16) arc (180:360:{\rad} and {\ecc});
%  \draw (P16) arc (180:0:{\rad} and {\ecc});
%
%  \draw (P13) -- (P15);
%  \draw ($(P14)+(2*\rad,0)$) -- ($(P16)+(2*\rad,0)$);
%  
% Caps   
  
  \coordinate (P4) at (0,0);
  \coordinate (P8) at (0,-\hght);
  \coordinate (P5) at ($(P4)+(\dif,0)$);
  \coordinate (P6) at ($(P5)+(2*\dif,0)$);
  
  \draw (P5) arc (180:360:{\rad} and {\ecc});
  \draw (P5) arc (180:0:{\rad} and {\ecc});
  
  \draw (P6) arc (180:360:{\rad} and {\ecc});
  \draw (P6) arc (180:0:{\rad} and {\ecc});  
  
  \coordinate (P17) at ($(P5)+(0,\dist)$);
  \draw (P17) to[out=90,in=180] ($(P17)+(\rad,.8*\hght)$) to[out=0,in=90] ($(P17)+(2*\rad,0)$);
  \draw (P17) arc (180:360:{\rad} and {\ecc});
  \draw[dashed] (P17) arc (180:0:{\rad} and {\ecc});
  
    \coordinate (P18) at ($(P6)+(0,\dist)$);
  \draw (P18) to[out=90,in=180] ($(P18)+(\rad,.8*\hght)$) to[out=0,in=90] ($(P18)+(2*\rad,0)$);
  \draw (P18) arc (180:360:{\rad} and {\ecc});
  \draw[dashed] (P18) arc (180:0:{\rad} and {\ecc}); 
  
  \draw ($(P8)+(2*\rad,0)$) to[out=30,in=-120] ($(P6)+(2*\rad,0)$);
  
  \draw ($(P4)+(2*\rad,0)$) to[out=-90,in=-90] ($(P5)$);
  
  \draw ($(P5)+(2*\rad,0)$) to[out=-90,in=-90] ($(P5)+(2*\rad+\dist,0)$);
  \draw ($(P6)+(-\dist,0)$) to[out=-90,in=-90] ($(P6)$); 

  \tikzset{decorate sep/.style 2 args={decorate,decoration={shape backgrounds,shape=circle,shape size=#1,shape sep=#2}}} 
 
 \draw[decorate sep={0.3mm}{2mm},fill] ($(P5)+(2*\rad+\dif,0)$) -- ($(P6)+(-\dif,0)$);  
  
  
  
% 
%  \coordinate (C1) at ($(P1)+(2*\rad,-\hght)$);
%  \coordinate (C2) at ($(P4)+(0,-\hght)$); 
% 
%  \draw (P1) to[out=-90,in=180] (C1);
%  \draw (C1) to[out=0,in=180] (C2);
%  \draw (C2) to[out=0,in=-90] ($(P4)+(2*\rad,0)$);
%  
%  \tikzset{decorate sep/.style 2 args={decorate,decoration={shape backgrounds,shape=circle,shape size=#1,shape sep=#2}}} 
% 
% \draw[decorate sep={0.3mm}{2mm},fill] ($(P2)+(2*\rad+\dif,0)$) -- ($(P3)+(-\dif,0)$);
%  
%
%% Second Object
%
%  \coordinate (P5) at ($(P4)+(0,\dist)$);
%  \coordinate (P6) at ($(P5) + (\dif,0)$);
%  \coordinate (P7) at ($(P6) + (2*\dif,0)$);
%  \coordinate (P8) at ($(P7) + (\dif,0)$);
%  
%  \draw (P5) arc (180:360:{\rad} and {\ecc});
%  \draw[dashed] (P5) arc (180:0:{\rad} and {\ecc});
%  
%  \draw (P6) arc (180:360:{\rad} and {\ecc});
%  \draw[dashed] (P6) arc (180:0:{\rad} and {\ecc});
%  
%  \draw (P7) arc (180:360:{\rad} and {\ecc});
%  \draw[dashed] (P7) arc (180:0:{\rad} and {\ecc});
%  
%  \draw (P8) arc (180:360:{\rad} and {\ecc});
%  \draw[dashed] (P8) arc (180:0:{\rad} and {\ecc});
%
% \draw ($(P5)+(2*\rad,0)$) to[out=90,in=90] (P6);
% \draw ($(P7)+(2*\rad,0)$) to[out=90,in=90] (P8);
% 
% \coordinate (C3) at ($(P5)+(2*\rad,\hght)$);
%  \coordinate (C4) at ($(P8)+(0,\hght)$); 
% 
%  \draw (P5) to[out=90,in=180] (C3);
%  \draw (C3) to[out=0,in=180] (C4);
%  \draw (C4) to[out=0,in=90] ($(P8)+(2*\rad,0)$);
%  
%  \draw[decorate sep={0.3mm}{2mm},fill] ($(P6)+(2*\rad+\dif,0)$) -- ($(P7)+(-\dif,0)$);
%
%% Caps
%
%  \coordinate (P9) at ($(P2)+(0,\dist)$);
%  \draw (P9) to[out=90,in=180] ($(P9)+(\rad,.8*\hght)$) to[out=0,in=90] ($(P9)+(2*\rad,0)$);
%  \draw (P9) arc (180:360:{\rad} and {\ecc});
%  \draw[dashed] (P9) arc (180:0:{\rad} and {\ecc});
%  
%  \coordinate (P10) at ($(P3)+(0,\dist)$);
%  \draw (P10) to[out=90,in=180] ($(P10)+(\rad,.8*\hght)$) to[out=0,in=90] ($(P10)+(2*\rad,0)$);
%  \draw (P10) arc (180:360:{\rad} and {\ecc});
%  \draw[dashed] (P10) arc (180:0:{\rad} and {\ecc});
%
%  \coordinate (P11) at ($(P6)+(0,-\dist)$);
%  \draw (P11) to[out=-90,in=180] ($(P11)+(\rad,-.8*\hght)$) to[out=0,in=-90] ($(P11)+(2*\rad,0)$);
%  \draw (P11) arc (180:360:{\rad} and {\ecc});
%  \draw (P11) arc (180:0:{\rad} and {\ecc});
%
%  \coordinate (P12) at ($(P7)+(0,-\dist)$);
%  \draw (P12) to[out=-90,in=180] ($(P12)+(\rad,-.8*\hght)$) to[out=0,in=-90] ($(P12)+(2*\rad,0)$);
%  \draw (P12) arc (180:360:{\rad} and {\ecc});
%  \draw (P12) arc (180:0:{\rad} and {\ecc});
% 
\end{tikzpicture}
\caption{Connected bubbling of $\HTP_{010}$ in $\BVOp^- e^{\MVMorF}$.}
\end{subfigure}
\begin{subfigure}{.45\textwidth}
\centering
%auto-ignore
\begin{tikzpicture}
%%% First Object
%  \coordinate (P1) at (0,0);
%  \coordinate (P2) at ($(P1) + (2*\dif,0)$);
%  \coordinate (P3) at ($(P2) + (2*\dif,0)$);
%
%
%  
%  \coordinate (P7) at ($(P1)+(0,-\hght)$);
%  \\
%
%  
%  \draw (P1) arc (180:360:{\rad} and {\ecc});
%  \draw (P1) arc (180:0:{\rad} and {\ecc});
%  
%  \draw (P2) arc (180:360:{\rad} and {\ecc});
%  \draw (P2) arc (180:0:{\rad} and {\ecc});
%  
%  \draw (P3) arc (180:360:{\rad} and {\ecc});
%  \draw (P3) arc (180:0:{\rad} and {\ecc});
%  
%  \draw (P4) arc (180:360:{\rad} and {\ecc});
%  \draw (P4) arc (180:0:{\rad} and {\ecc});
%  
%  \draw (P5) arc (180:360:{\rad} and {\ecc});
%  \draw (P5) arc (180:0:{\rad} and {\ecc});
%  
%
%  
%  \draw (P7) arc (180:360:{\rad} and {\ecc});
%  \draw[dashed] (P7) arc (180:0:{\rad} and {\ecc});
%
%  \draw (P8) arc (180:360:{\rad} and {\ecc});
%  \draw[dashed] (P8) arc (180:0:{\rad} and {\ecc});
% 
%  %\draw ($(P1)+(2*\rad,0)$) to[out=-90,in=-90] (P2);
%  %\draw ($(P3)+(2*\rad,0)$) to[out=-90,in=-90] (P4);
%  \draw ($(P4)+(2*\rad,0)$) to[out=-90,in=-90] (P5);
%  
%  \draw (P1)--(P7);
%  \draw ($(P8)+(2*\rad,0)$) to[out=35,in=-130] ($(P6)+(2*\rad,0)$);
%  
%  \tikzset{decorate sep/.style 2 args={decorate,decoration={shape backgrounds,shape=circle,shape size=#1,shape sep=#2}}} 
% 
% \draw[decorate sep={0.3mm}{2mm},fill] ($(P7)+(2*\rad+\dif,0)$) -- ($(P8)+(-\dif,0)$);
%
% \draw[decorate sep={0.3mm}{2mm},fill] ($(P2)+(2*\rad+\dif,0)$) -- ($(P3)+(-\dif,0)$);
% 
%  \draw[decorate sep={0.3mm}{2mm},fill] ($(P5)+(2*\rad+\dif,0)$) -- ($(P6)+(-\dif,0)$);
%  
%   \draw ($(P1)+(2*\rad,0)$) to[out=-90,in=-90] ($(P1)+(2*\rad+\dist,0)$);
%   \draw ($(P2)+(-\dist,0)$) to[out=-90,in=-90] ($(P2)$);
%   \draw ($(P2)+(2*\rad,0)$) to[out=-90,in=-90] ($(P2)+(2*\rad+\dist,0)$);
%   \draw ($(P3)+(-\dist,0)$) to[out=-90,in=-90] ($(P3)$);
%   \draw ($(P3)+(2*\rad,0)$) to[out=-90,in=-90] ($(P3)+(2*\rad+\dist,0)$);
%   \draw ($(P4)+(-\dist,0)$) to[out=-90,in=-90] ($(P4)$);
%   \draw ($(P7)+(2*\rad,0)$) to[out=90,in=90] ($(P7)+(2*\rad+\dist,0)$);
%   \draw ($(P8)+(-\dist,0)$) to[out=90,in=90] ($(P8)$);
%   
%% Second body
%
%  \coordinate (P9) at ($(P1) + (0,\dist)$);
%  \coordinate (P10) at ($(P2) + (0,\dist)$);
%  \coordinate (P11) at ($(P1) + (0,\dist+\hght)$);
%  \coordinate (P12) at ($(P2) + (0,\dist+\hght)$);
%  
%  \draw (P9) arc (180:360:{\rad} and {\ecc});
%  \draw[dashed] (P9) arc (180:0:{\rad} and {\ecc});  
%  \draw (P10) arc (180:360:{\rad} and {\ecc});
%  \draw[dashed] (P10) arc (180:0:{\rad} and {\ecc});
%  \draw (P11) arc (180:360:{\rad} and {\ecc});
%  \draw (P11) arc (180:0:{\rad} and {\ecc});
%  \draw (P12) arc (180:360:{\rad} and {\ecc});
%  \draw (P12) arc (180:0:{\rad} and {\ecc});
%  
%  \draw (P9) -- (P11);
%  \draw ($(P10)+(2*\rad,0)$) -- ($(P12)+(2*\rad,0)$);
%
%% Third Body
%
%  \coordinate (P13) at ($(P3) + (0,\dist)$);
%  \coordinate (P14) at ($(P4)+(0,\dist)$); 
%  \coordinate (P15) at ($(P3) + (0,\dist+\hght)$);
%  \coordinate (P16) at ($(P4) + (0,\dist+\hght)$);
%
%  
%  \draw (P13) arc (180:360:{\rad} and {\ecc});
%  \draw[dashed] (P13) arc (180:0:{\rad} and {\ecc});
%  
%  \draw (P14) arc (180:360:{\rad} and {\ecc});
%  \draw[dashed] (P14) arc (180:0:{\rad} and {\ecc}); 
%  \draw (P15) arc (180:360:{\rad} and {\ecc});
%  \draw (P15) arc (180:0:{\rad} and {\ecc});
%  \draw (P16) arc (180:360:{\rad} and {\ecc});
%  \draw (P16) arc (180:0:{\rad} and {\ecc});
%
%  \draw (P13) -- (P15);
%  \draw ($(P14)+(2*\rad,0)$) -- ($(P16)+(2*\rad,0)$);
%  
% Caps   
  
  \coordinate (P1) at (0,0);
  \coordinate (P2) at ($(P1)+(0,-\dist)$);
  \coordinate (P3) at ($(P1)+(2*\dif,0)$);
  \coordinate (P4) at ($(P3)+(0,-\dist)$);
  \coordinate (P5) at ($(P3)+(\dif,0)$);
  \coordinate (P6) at ($(P5)+(0,\hght)$);  
  
  \draw (P1) arc (180:360:{\rad} and {\ecc});
  \draw[dashed] (P1) arc (180:0:{\rad} and {\ecc});
  \draw (P2) arc (180:360:{\rad} and {\ecc});
  \draw (P2) arc (180:0:{\rad} and {\ecc});  
  \draw (P3) arc (180:360:{\rad} and {\ecc});
  \draw[dashed] (P3) arc (180:0:{\rad} and {\ecc});
  \draw (P4) arc (180:360:{\rad} and {\ecc});
  \draw (P4) arc (180:0:{\rad} and {\ecc});  

  \draw (P2) to[out=-90,in=180] ($(P2)+(\rad,-.8*\hght)$) to[out=0,in=-90] ($(P2)+(2*\rad,0)$);
  
  \draw (P4) to[out=-90,in=180] ($(P4)+(\rad,-.8*\hght)$) to[out=0,in=-90] ($(P4)+(2*\rad,0)$);
 
  \draw ($(P1)$) to[out=60,in=-150] ($(P6)$);
  
  \draw ($(P1)+(2*\rad,0)$) to[out=90,in=90] ($(P1)+(2*\rad+\dist,0)$);
  
  \draw ($(P3)+(-\dist,0)$) to[out=90,in=90] ($(P3)$);
  \draw ($(P3)+(2*\rad,0)$) to[out=90,in=90] ($(P5)$); 

  \tikzset{decorate sep/.style 2 args={decorate,decoration={shape backgrounds,shape=circle,shape size=#1,shape sep=#2}}} 
 
 \draw[decorate sep={0.3mm}{2mm},fill] ($(P1)+(2*\rad+\dif,0)$) -- ($(P3)+(-\dif,0)$);  
  
  
  
% 
%  \coordinate (C1) at ($(P1)+(2*\rad,-\hght)$);
%  \coordinate (C2) at ($(P4)+(0,-\hght)$); 
% 
%  \draw (P1) to[out=-90,in=180] (C1);
%  \draw (C1) to[out=0,in=180] (C2);
%  \draw (C2) to[out=0,in=-90] ($(P4)+(2*\rad,0)$);
%  
%  \tikzset{decorate sep/.style 2 args={decorate,decoration={shape backgrounds,shape=circle,shape size=#1,shape sep=#2}}} 
% 
% \draw[decorate sep={0.3mm}{2mm},fill] ($(P2)+(2*\rad+\dif,0)$) -- ($(P3)+(-\dif,0)$);
%  
%
%% Second Object
%
%  \coordinate (P5) at ($(P4)+(0,\dist)$);
%  \coordinate (P6) at ($(P5) + (\dif,0)$);
%  \coordinate (P7) at ($(P6) + (2*\dif,0)$);
%  \coordinate (P8) at ($(P7) + (\dif,0)$);
%  
%  \draw (P5) arc (180:360:{\rad} and {\ecc});
%  \draw[dashed] (P5) arc (180:0:{\rad} and {\ecc});
%  
%  \draw (P6) arc (180:360:{\rad} and {\ecc});
%  \draw[dashed] (P6) arc (180:0:{\rad} and {\ecc});
%  
%  \draw (P7) arc (180:360:{\rad} and {\ecc});
%  \draw[dashed] (P7) arc (180:0:{\rad} and {\ecc});
%  
%  \draw (P8) arc (180:360:{\rad} and {\ecc});
%  \draw[dashed] (P8) arc (180:0:{\rad} and {\ecc});
%
% \draw ($(P5)+(2*\rad,0)$) to[out=90,in=90] (P6);
% \draw ($(P7)+(2*\rad,0)$) to[out=90,in=90] (P8);
% 
% \coordinate (C3) at ($(P5)+(2*\rad,\hght)$);
%  \coordinate (C4) at ($(P8)+(0,\hght)$); 
% 
%  \draw (P5) to[out=90,in=180] (C3);
%  \draw (C3) to[out=0,in=180] (C4);
%  \draw (C4) to[out=0,in=90] ($(P8)+(2*\rad,0)$);
%  
%  \draw[decorate sep={0.3mm}{2mm},fill] ($(P6)+(2*\rad+\dif,0)$) -- ($(P7)+(-\dif,0)$);
%
%% Caps
%
%  \coordinate (P9) at ($(P2)+(0,\dist)$);
%  \draw (P9) to[out=90,in=180] ($(P9)+(\rad,.8*\hght)$) to[out=0,in=90] ($(P9)+(2*\rad,0)$);
%  \draw (P9) arc (180:360:{\rad} and {\ecc});
%  \draw[dashed] (P9) arc (180:0:{\rad} and {\ecc});
%  
%  \coordinate (P10) at ($(P3)+(0,\dist)$);
%  \draw (P10) to[out=90,in=180] ($(P10)+(\rad,.8*\hght)$) to[out=0,in=90] ($(P10)+(2*\rad,0)$);
%  \draw (P10) arc (180:360:{\rad} and {\ecc});
%  \draw[dashed] (P10) arc (180:0:{\rad} and {\ecc});
%
%  \coordinate (P11) at ($(P6)+(0,-\dist)$);
%  \draw (P11) to[out=-90,in=180] ($(P11)+(\rad,-.8*\hght)$) to[out=0,in=-90] ($(P11)+(2*\rad,0)$);
%  \draw (P11) arc (180:360:{\rad} and {\ecc});
%  \draw (P11) arc (180:0:{\rad} and {\ecc});
%
%  \coordinate (P12) at ($(P7)+(0,-\dist)$);
%  \draw (P12) to[out=-90,in=180] ($(P12)+(\rad,-.8*\hght)$) to[out=0,in=-90] ($(P12)+(2*\rad,0)$);
%  \draw (P12) arc (180:360:{\rad} and {\ecc});
%  \draw (P12) arc (180:0:{\rad} and {\ecc});
% 
\end{tikzpicture}
\caption{Connected bubbling of $\HTP_{100}$ in $e^{\MVMorF}\BVOp^+$.}
\end{subfigure}\\[.5cm]
\begin{subfigure}{\textwidth}
\centering
%auto-ignore
\begin{tikzpicture}
% First Object
  \coordinate (P1) at (0,0);
  \coordinate (P2) at ($(P1) + (\dif,0)$);
  \coordinate (P3) at ($(P2) + (2*\dif,0)$);
  \coordinate (P4) at ($(P3) + (\dif,0)$);
  
  \draw (P1) arc (180:360:{\rad} and {\ecc});
  \draw (P1) arc (180:0:{\rad} and {\ecc});
  
  \draw (P2) arc (180:360:{\rad} and {\ecc});
  \draw (P2) arc (180:0:{\rad} and {\ecc});
  
  \draw (P3) arc (180:360:{\rad} and {\ecc});
  \draw (P3) arc (180:0:{\rad} and {\ecc});
  
  \draw (P4) arc (180:360:{\rad} and {\ecc});
  \draw (P4) arc (180:0:{\rad} and {\ecc});

  \draw ($(P1)+(2*\rad,0)$) to[out=-90,in=-90] (P2);
  \draw ($(P3)+(2*\rad,0)$) to[out=-90,in=-90] (P4);  
 
  \coordinate (C1) at ($(P1)+(2*\rad,-\hght)$);
  \coordinate (C2) at ($(P4)+(0,-\hght)$); 
 
  \draw (P1) to[out=-90,in=180] (C1);
  \draw (C1) to[out=0,in=180] (C2);
  \draw (C2) to[out=0,in=-90] ($(P4)+(2*\rad,0)$);
  
  \tikzset{decorate sep/.style 2 args={decorate,decoration={shape backgrounds,shape=circle,shape size=#1,shape sep=#2}}} 
 
 \draw[decorate sep={0.3mm}{2mm},fill] ($(P2)+(2*\rad+\dif,0)$) -- ($(P3)+(-\dif,0)$);
  
  
   \draw ($(P2)+(2*\rad,0)$) to[out=-90,in=-90] ($(P2)+(2*\rad+\dist,0)$);
   \draw ($(P3)+(-\dist,0)$) to[out=-90,in=-90] ($(P3)$);

% Second Object

  \coordinate (P5) at ($(P4)+(0,\dist)$);
  \coordinate (P6) at ($(P5) + (\dif,0)$);
  \coordinate (P7) at ($(P6) + (2*\dif,0)$);
  \coordinate (P8) at ($(P7) + (\dif,0)$);
  
  \draw (P5) arc (180:360:{\rad} and {\ecc});
  \draw[dashed] (P5) arc (180:0:{\rad} and {\ecc});
  
  \draw (P6) arc (180:360:{\rad} and {\ecc});
  \draw[dashed] (P6) arc (180:0:{\rad} and {\ecc});
  
  \draw (P7) arc (180:360:{\rad} and {\ecc});
  \draw[dashed] (P7) arc (180:0:{\rad} and {\ecc});
  
  \draw (P8) arc (180:360:{\rad} and {\ecc});
  \draw[dashed] (P8) arc (180:0:{\rad} and {\ecc});

 \draw ($(P5)+(2*\rad,0)$) to[out=90,in=90] (P6);
 \draw ($(P7)+(2*\rad,0)$) to[out=90,in=90] (P8);
 
 \coordinate (C3) at ($(P5)+(2*\rad,\hght)$);
  \coordinate (C4) at ($(P8)+(0,\hght)$); 
 
  \draw (P5) to[out=90,in=180] (C3);
  \draw (C3) to[out=0,in=180] (C4);
  \draw (C4) to[out=0,in=90] ($(P8)+(2*\rad,0)$);
  
  \draw[decorate sep={0.3mm}{2mm},fill] ($(P6)+(2*\rad+\dif,0)$) -- ($(P7)+(-\dif,0)$);

   \draw ($(P6)+(2*\rad,0)$) to[out=90,in=90] ($(P6)+(2*\rad+\dist,0)$);
   \draw ($(P7)+(-\dist,0)$) to[out=90,in=90] ($(P7)$);

% Caps

  \coordinate (P9) at ($(P2)+(0,\dist)$);
  \draw (P9) to[out=90,in=180] ($(P9)+(\rad,.8*\hght)$) to[out=0,in=90] ($(P9)+(2*\rad,0)$);
  \draw (P9) arc (180:360:{\rad} and {\ecc});
  \draw[dashed] (P9) arc (180:0:{\rad} and {\ecc});
  
  \coordinate (P10) at ($(P3)+(0,\dist)$);
  \draw (P10) to[out=90,in=180] ($(P10)+(\rad,.8*\hght)$) to[out=0,in=90] ($(P10)+(2*\rad,0)$);
  \draw (P10) arc (180:360:{\rad} and {\ecc});
  \draw[dashed] (P10) arc (180:0:{\rad} and {\ecc});

  \coordinate (P11) at ($(P6)+(0,-\dist)$);
  \draw (P11) to[out=-90,in=180] ($(P11)+(\rad,-.8*\hght)$) to[out=0,in=-90] ($(P11)+(2*\rad,0)$);
  \draw (P11) arc (180:360:{\rad} and {\ecc});
  \draw (P11) arc (180:0:{\rad} and {\ecc});

  \coordinate (P12) at ($(P7)+(0,-\dist)$);
  \draw (P12) to[out=-90,in=180] ($(P12)+(\rad,-.8*\hght)$) to[out=0,in=-90] ($(P12)+(2*\rad,0)$);
  \draw (P12) arc (180:360:{\rad} and {\ecc});
  \draw (P12) arc (180:0:{\rad} and {\ecc});
%  \coordinate (P2) at (-0.5*\dif,\hght);
%  \coordinate (P3) at (0.5*\dif,\hght);
%  \coordinate (P4) at ($(P2)+(0,\dist)$);
%  \coordinate (P5) at ($(P4)+(-\dif,0)$);
%  \coordinate (P6) at ($(P5)+(0,-\dist)$);
%  \coordinate (P7) at ($(P6)+(0,-\hght)$);
%  \coordinate (P8) at ($(P3)+(0,\dist)$);
%  \coordinate (P9) at ($(P8)+(0,\hght)$);
%  \coordinate (P51) at ($(P5)+(-\dif,0)$);
%  \coordinate (P52) at ($(P51)+(-\dif,0)$);
%  \coordinate (P53) at ($(P52)+(-\dif,0)$);
%  \coordinate (P62) at ($(P53)+(0,-\dist)$);
%  \coordinate (P72) at ($(P62)+(0,-\hght)$);
%  \coordinate (PG1) at ($(P51)+(1.6*\rad,0.5*\hght)$);
%  \coordinate (PG2) at ($(P53)+(2*\rad,0.5*\hght)$);
 

%Pair of pants
  
%  \draw (P1) arc (180:360:{\rad} and {\ecc});
%  \draw[dashed] (P1) arc (180:0:{\rad} and {\ecc});
  
%   \draw (P3) arc (180:360:{\rad} and {\ecc});
%  \draw (P3) arc (180:0:{\rad} and {\ecc});
%  
%  \draw (P2) arc (180:360:{\rad} and {\ecc});
%  \draw (P2) arc (180:0:{\rad} and {\ecc});
%  
% \draw (P2) to[out=270,in=90] (P1);
% \draw ($(P3)+(2*\rad,0)$) to[out=270,in=90] ($(P1)+(2*\rad,0)$);
% \draw ($(P2)+(2*\rad,0)$) to[out=270,in=270] (P3); 
% 
% Maurer Cartan

% \draw (P4) arc (180:360:{\rad} and {\ecc});
% \draw[dashed] (P4) arc (180:0:{\rad} and {\ecc});
% 
% \draw (P5) arc (180:360:{\rad} and {\ecc});
% \draw[dashed] (P5) arc (180:0:{\rad} and {\ecc});
% 
% \draw (P53) arc (180:360:{\rad} and {\ecc});
% \draw[dashed] (P53) arc (180:0:{\rad} and {\ecc});
% 
% \draw ($(P51)+(2*\rad,0)$) to[out=90,in=90] (P5);
% \draw ($(P53)+(2*\rad,0)$) to[out=90,in=90] (P52);  
% \draw ($(P5)+(2*\rad,0)$) to[out=90,in=90] (P4);
% 
% \coordinate (P5m) at ($0.8*(P4)+(\rad,0)+0.2*(P53)+(0,1*\hght)$);
% \coordinate (P5mm) at ($0.2*(P4)+(\rad,0)+0.8*(P53)+(0,1*\hght)$); 
% 
% \draw (P53) to[out=90,in=180] (P5mm);
% \draw (P5m) to[out=0,in=90] ($(P4)+(2*\rad,0)$);
% \draw (P5mm) to[out=0,in=180] (P5m);
% 
%\tikzset{decorate sep/.style 2 args=
%{decorate,decoration={shape backgrounds,shape=circle,shape size=#1,shape sep=#2}}} 
% 
% \draw[decorate sep={0.3mm}{2mm},fill] ($0.5*(P62)+0.5*(P72) + (\dif,0)$) to ($0.5*(P6)+0.5*(P7)+(-\dif+2*\rad,0)$);

% Labels 
 
% \node at ($(P1)+(\rad,0.5*\hght)$) {$\OPQ_{210}$};
% \node at ($0.5*(P4)+(\rad,0)+0.5*(P5)+(0,0.5*\hght)$) {$\PMC_{lg}$};

% Cylinders 

%  \draw (P62) arc (180:360:{\rad} and {\ecc});
% \draw (P62) arc (180:0:{\rad} and {\ecc});  
% \draw (P72) arc (180:360:{\rad} and {\ecc});
% \draw[dashed] (P72) arc (180:0:{\rad} and {\ecc});
% \draw (P62) -- (P72);
% \draw ($(P62)+(2*\rad,0)$) -- ($(P72)+(2*\rad,0)$);
%
% \draw (P6) arc (180:360:{\rad} and {\ecc});
% \draw (P6) arc (180:0:{\rad} and {\ecc});  
% \draw (P7) arc (180:360:{\rad} and {\ecc});
% \draw[dashed] (P7) arc (180:0:{\rad} and {\ecc});
% \draw (P6) -- (P7);
% \draw ($(P6)+(2*\rad,0)$) -- ($(P7)+(2*\rad,0)$);
% 
% 
% \draw (P9) arc (180:360:{\rad} and {\ecc});
% \draw (P9) arc (180:0:{\rad} and {\ecc});  
% \draw (P8) arc (180:360:{\rad} and {\ecc});
% \draw[dashed] (P8) arc (180:0:{\rad} and {\ecc});
% \draw (P8) -- (P9);
% \draw ($(P8)+(2*\rad,0)$) -- ($(P9)+(2*\rad,0)$);
% 
%% Genus
%
% \draw (PG1) to[out=-\genecc,in=180+\genecc] coordinate[pos=\gencanc] (PG11) coordinate[pos=1-\gencanc] (PG12) ($(PG1) + (2*\genrad,0)$) ;
% \draw (PG11) to[out=\genecc,in=180-\genecc] (PG12);
% 
%\draw (PG2) to[out=-\genecc,in=180+\genecc] coordinate[pos=\gencanc] (PG21) coordinate[pos=1-\gencanc] (PG22) ($(PG2) + (2*\genrad,0)$) ;
% \draw (PG21) to[out=\genecc,in=180-\genecc] (PG22);
% 
% \draw[decorate sep={0.3mm}{2mm},fill] ($(PG2) + (3*\genrad,0)$) to ($(PG1)-(\genrad,0)$);

% 
% \draw (PG1) to[out=150,in=-60] ($(PG1) + (-\gencanc,\gencanc)$);
% \draw ($(PG1) + (2*\genrad,0)$) to[out=390,in=240] ($(PG1) + (2*\genrad,0) + (\gencanc,\gencanc)$);
% \draw (PG1) to[out=\genecc,in=180-\genecc] ($(PG1) + (2*\genrad,0)$);
 
\end{tikzpicture}
\caption{Connected bubbling involving $\HTP_{010}$ or $\HTP_{100}$ in the composition $e^{\MVMorF^-\DiamComp\MVMorF^+} = e^{\MVMorF^-}e^{\MVMorF^+}$.}
\end{subfigure}\\[.5cm]
\begin{subfigure}{.4\textwidth}
\centering
%auto-ignore
\begin{tikzpicture}
% First Object
  \coordinate (P1) at (0,0);
  \coordinate (P2) at ($(P1) + (\dif,0)$);
  \coordinate (P3) at ($(P2) + (0,\dist)$);
  \coordinate (P4) at ($(P3) + (\dif,0)$);
  
  \draw (P1) arc (180:360:{\rad} and {\ecc});
  \draw (P1) arc (180:0:{\rad} and {\ecc});
  
  \draw (P2) arc (180:360:{\rad} and {\ecc});
  \draw (P2) arc (180:0:{\rad} and {\ecc});
  
  \draw (P3) arc (180:360:{\rad} and {\ecc});
  \draw[dashed] (P3) arc (180:0:{\rad} and {\ecc});
  
  \draw (P4) arc (180:360:{\rad} and {\ecc});
  \draw[dashed] (P4) arc (180:0:{\rad} and {\ecc});

  \draw ($(P1)+(2*\rad,0)$) to[out=-90,in=-90] (P2);
  \draw ($(P3)+(2*\rad,0)$) to[out=90,in=90] (P4);  
  \draw (P1) to[out=-90,in=180] ($.5*(P1) + .5*(P2) + (\rad,-.9*\hght)$) to[out=0,in=-90] ($(P2)+(2*\rad,0)$);
  
  \draw (P3) to[out=90,in=180] ($.5*(P3) + .5*(P4) + (\rad,.9*\hght)$) to[out=0,in=90] ($(P4)+(2*\rad,0)$);
  
\end{tikzpicture}
\caption{Connected bubbling in the composition with no $\HTP_{010}$ and~$\HTP_{100}$.}
\end{subfigure}
\hspace{.5cm}
\begin{subfigure}{.4\textwidth}
\centering
%auto-ignore
\begin{tikzpicture}
% First Object
  \coordinate (P1) at (0,0);
  \coordinate (PG1) at ($(\rad+\gencanc,0)$);
   
   \draw (P1) to[out=90,in=180] ($(P1)+(2*\rad,\rad)$) to[out=0,in=90] ($(P1)+(4*\rad,0)$) to[out=-90,in=0] ($(P1)+(2*\rad,-\rad)$) to[out=180,in=-90] (P1);
  
  
 \draw (PG1) to[out=-\genecc,in=180+\genecc] coordinate[pos=\gencanc] (PG11) coordinate[pos=1-\gencanc] (PG12) ($(PG1) + (2*\genrad,0)$) ;
 \draw (PG11) to[out=\genecc,in=180-\genecc] (PG12);

%  \coordinate (PG1) at ($(P51)+(1.6*\rad,0.5*\hght)$);
% \coordinate (PG2) at ($(P53)+(2*\rad,0.5*\hght)$);
\end{tikzpicture}
\caption{Disconnected bubbling of $\HTP_{001}$ in the relation $\BVOp^- e^{\MVMorF}$, $e^{\MVMorF}\BVOp^+$ or in the composition.}
\end{subfigure}\\[.2cm]
\begin{subfigure}{.4\textwidth}
\centering
%auto-ignore
\begin{tikzpicture}
% First Object
  \coordinate (P1) at (0,0);
  \coordinate (P2) at ($(7*\rad,0)$);  
  \coordinate (PG1) at ($(.8*\rad+\gencanc,0)$);
  \coordinate (PG2) at ($(3.2*\rad+\gencanc,0)$);
   
  \draw (P1) to[out=90,in=180] ($(P1)+(3*\rad,\rad)$) to[out=0,in=90] ($(P1)+(6*\rad,0)$) to[out=-90,in=0] ($(P1)+(3*\rad,-\rad)$) to[out=180,in=-90] (P1);
  
 \draw (PG1) to[out=-\genecc,in=180+\genecc] coordinate[pos=\gencanc] (PG11) coordinate[pos=1-\gencanc] (PG12) ($(PG1) + (2*\genrad,0)$) ;
 \draw (PG11) to[out=\genecc,in=180-\genecc] (PG12);

 \draw (PG2) to[out=-\genecc,in=180+\genecc] coordinate[pos=\gencanc] (PG21) coordinate[pos=1-\gencanc] (PG22) ($(PG2) + (2*\genrad,0)$) ;
 \draw (PG21) to[out=\genecc,in=180-\genecc] (PG22);

  \draw (P2) to[out=90,in=180] ($(P2)+(\rad,\rad)$) to[out=0,in=90] ($(P2)+(2*\rad,0)$) to[out=-90,in=0] ($(P2)+(\rad,-\rad)$) to[out=180,in=-90] (P2);

%  \coordinate (PG1) at ($(P51)+(1.6*\rad,0.5*\hght)$);
% \coordinate (PG2) at ($(P53)+(2*\rad,0.5*\hght)$);
\end{tikzpicture}
\caption{A possible disconnected bubbling involving $\HTP_{000}$ in $\BVOp^- e^{\MVMorF}$, $e^{\MVMorF}\BVOp^+$ or in the composition.}
\end{subfigure}\\[.2cm]
\endgroup}
\caption[Bubbling in $\IBLInfty$-relations.]{
%The bubbling phenomenon amounts to appearance of infinitely many summands in relations of the connected surface calculus. It occurs in the relation for morphisms~\eqref{Eq:WeakIBLMor} if $\HTP_{100}$ or $\HTP_{010}$ is non-zero. It also occurs in the relation for composition (see \cite[Equation~(8.7)]{Cieliebak2015}) if $\HTP_{100}$, $\HTP_{010}$ or both $\HTP_{200}$ and $\HTP_{020}$ are non-zero. Finally, it occurs in the twisting with a Maurer-Cartan element $\PMC$ if $\PMC_{10}$ is non-zero. We illustrate these situations in Figure~\ref{Fig:Bubbling}.
%The bubbling is resolved algebraically by requiring the strict inequalities for the filtration degree. In fact, just the strict inequalities for $\HTP_{100}$, $\HTP_{010}$, $\HTP_{200}$ and $\HTP_{020}$ seem to be necessary to define the category of filtered $\IBLInfty$-algebras via connected surface calculus. The reasons why we include the strict conditions for $\HTP_{000}$ and $\HTP_{001}$ in Definition~\ref{Def:IBLConnected} are to keep compatibility with \cite{Cieliebak2015}, for Proposition~\ref{Prop:BVforIBL} below and also because we can, i.e., these conditions are preserved under composition. In \cite[Definition~8.1 and below]{Cieliebak2015}, in the definition of filtered $\IBLInfty$-structures, they require the strict inequality also for $\OPQ_{klg}$ with $(k,l,g)$ from \eqref{Eq:UnstableSignatures}. This we also can, but we do not see the necessity.
%Recall that \eqref{Eq:UnstableSignatures} correspond to unstable surfaces, i.e., those $(k,l,g)$ for which $\chi_{klg}=2-2g-k-l \ge 0$, without $(1,1,0)$.
%Notice also that $\HTP_{00g}$ are not involved in \eqref{Eq:WeakIBLMor} and that $\OPQ_{00g}$ are involved in \eqref{Eq:WeakIBLMor} for $k=l = 0$ but are not involved in \eqref{Eq:IBLFormula}. 
%Fixing $k$, $l$, $g$ and $r$ in~\eqref{Eq:WeakIBLMor}, there is only finitely many summands.
%
%On the right-hand side of~\eqref{Eq:WeakIBLMor}, all the non-trivial configurations appear until $r\le l$, and then only the ``bubbling'' of $\HTP_{100}$ as $r\to r+1$, i.e., connecting additional~$\HTP_{100}$ which does not change the signature $(k,l,g)$, occurs. On the left-hand side, the story is similar with $r\le k$ and~$\HTP_{010}$ instead. To handle these phenomenons, conditions $\Norm{\HTP_{100}}$, $\Norm{\HTP_{010}} > \gamma$ are necessary.
%
%First of all, only $\HTP_{klg}$ can bubble off, not $\OPQ_{klg}$. Therefore, the strict inequality 
%
%%In fact, because the components $(0,0,0)$ and $(0,0,1)$ of $\HTP$ \eqref{Eq:WeakIBLMor} at all.
%But it is easy to check that they are preserved by composition.
%We can also define the $\IBLInfty$-algebras using the surface calculus.
% and $\Norm{\HTP_{klg}}$ and all \eqref{Eq:UnstableSignatures}.
Bubbling in $\IBLInfty$-relations. If we glue any of the components above to a surface of signature $(k,l,g)$, the signature remains the same. Note that since~$g$ is defined via the Euler characteristic, adding a disconnected component without inputs and outputs decreases $g$ by one.}
\label{Fig:Bubbling}
\end{figure}

In the following case, our theory and the theory of \cite{Cieliebak2015} agree.

\begin{Proposition}[Equivalence of definitions in bounded case over $\K$]\label{Prop:BVforIBL}
Let $d\in \Z$ and $\gamma>0$. For $\zeta\in \R$, we denote by $\mathcal{W}_\zeta$ the class of complete filtered graded vector spaces $W$ for which there is an $\alpha>\zeta$ such that
\begin{equation}\label{Eq:FiltrCondU}
 \Filtr^{\alpha} W = W.
\end{equation}
For such $W$, we will consider weak $\IBLInfty$-algebras of bidegree $(d,\gamma)$ over $\K$ and their weak $\IBLInfty$-morphisms. We have the following:
\begin{ClaimList}
\item Filtered $\IBLInfty$-algebras and morphisms from \cite{Cieliebak2015} over $\mathcal{W}_{-\gamma}$ are also complete filtered $\MV$-algebras and morphisms from Definition~\ref{Def:ComplFiltrIBL}, respectively.
\item The two definitions agree over $\mathcal{W}_0$. If in addition $\gamma\ge 1$, then also the $\BV$-formalisms are identical.
\item The canonical $\dIBL$-structure on the (reduced) dual cyclic bar complex 
\[W = \CDBCyc V[2-n]\]
for a Poincar\'e duality algebra $V$ of degree $n$ from Section~\ref{Sec:Alg3} in Part~I satisfies (b).
\end{ClaimList}
%Let $\OPQ_{klg}: \hat{\Sym}_k(W[1]) \rightarrow \hat{\Sym}_l(W[1])$ for $k$, $l$, $g\in \N_0$ be a collection of homogenous $\K$-linear maps of finite filtration degrees satisfying (1) and (2) of Proposition~\ref{Prop:EqCharOfMVIBL} for $U=W[1]$. Then (4) is satisfied, and (3) is satisfied if and only if \eqref{Eq:UnstableSignatures} is. Therefore, the definitions of weak $\IBLInfty$-algebras over $\K$ and their morphisms according to Definition~\ref{Prop:EqCharOfMVIBL} and according to~\cite{Cieliebak2015} agree for $W$ satisfying \eqref{Eq:FiltrCondU}.
%
%The condition \eqref{Eq:FiltrCondU} is satisfied for $W = \CDBCyc V$ with the induced dual filtration from Part~I.
\end{Proposition}
\begin{proof}
For $W\in \mathcal{W}_\zeta$, there is an $\alpha>\zeta$ such that
\[ \Norm{\cdot} = \Norm{\cdot}_W + \gamma \ge  \alpha + \gamma \quad\text{on }U, \]
and hence
\begin{equation}\label{Eq:IneqWeight}
\Norm{\OPQ_{klg}(v)}_U \ge l (\alpha + \gamma) \quad\text{for all }v\in\hat{\Sym}_k U
\end{equation}
and any $k\ge 0$, $l\ge 1$, $g\ge 0$. It holds also for $l=0$ because the filtration on $\hat{S}_0 U \simeq \K$ is non-negative. We see that if $\zeta\ge -\gamma$, which is the case of both (a) and (b), then the sums $\sum_{l=0}^\infty \OPQ_{klg}$ converge automatically; i.e.,~(4) of (a) of Proposition~\ref{Prop:EqCharOfMVIBL} holds. It remains to study the relation of the strict filtration degree conditions
\begin{equation}\label{Eq:StrCondI}
\forall l, g\in \N_0:\quad  \Norm{\OPQ_{0lg}} = \Norm{\OPQ_{0lg}(1)}  > - 2\gamma(g-1)
\end{equation}
and
\begin{equation}\label{Eq:StrCondII}
\forall (k,l,g)\in \eqref{Eq:UnstableSignatures}:\quad\Norm{\OPQ_{klg}} > -2\gamma(k+g-1).
\end{equation}
Here, $\Norm{\OPQ_{0lg}} = \Norm{\OPQ_{0lg}(1)}$ holds because the trivial filtration on $\K$ is non-negative and it holds $\Norm{1} = 0$. For morphisms, the story is the same, and the rest is implied by Proposition~\ref{Prop:EqCharOfMVIBL}. 
\begin{ProofList}
\item Assuming \eqref{Eq:StrCondII}, we have to prove \eqref{Eq:StrCondI}. It is easy to see that if $\zeta \ge -\gamma$, then~\eqref{Eq:StrCondI} follows from~\eqref{Eq:IneqWeight} except for $(k,l,g) = (0,0,0)$, $(0,1,0)$, $(0,2,0)$ and $(0,0,1)$. These are precisely the signatures from \eqref{Eq:UnstableSignatures} with no input; in particular, they are implied by \eqref{Eq:StrCondII}.
\item Assuming \eqref{Eq:StrCondI}, we have to prove \eqref{Eq:StrCondII}. Clearly,~\eqref{Eq:StrCondI} implies the strict inequality for all~$(k,l,g)\in$\eqref{Eq:UnstableSignatures} with no input. It remains to check it for $(1,0,0)$ and $(2,0,0)$. Using $\zeta\ge-\gamma$, we obtain
\[ \Norm{\HTP_{100}(v)} \ge \Norm{\HTP_{100}} + \Norm{v} \ge 0 + (\gamma + \alpha) > 0\quad\text{for all }v\in\hat{\Sym}_1 V, \] 
and using $\zeta \ge 0$, we obtain
\[ \Norm{\HTP_{200}(v)} \ge \Norm{\HTP_{200}} + \Norm{v} \ge - 2 \gamma + 2(\gamma + \alpha)>0\quad\text{for all }v\in\hat{\Sym}_2 V. \]
Because $\im \HTP_{100}$, $\im \HTP_{200} \subset \K$, it must hold $\HTP_{100} = \HTP_{200} = 0$, and hence the strict filtration degree condition is automatically satisfied.

The equivalence of $\BV$-formalisms for $\gamma\ge 1$ follows from Lemma~\ref{Lem:BoundCondOnFiltr} because $\Filtr^{1} U = (\Filtr^{1-\gamma} W)[1] = W[1] = U$, where $\alpha>0 \ge 1-\gamma$.
\item For any $k\in \N_0$ and $k \le \lambda < k+1$, we have
\begin{align*}
\Filtr^\lambda_{\WeightMRM} \BCyc V &= \BCyc_1 V \oplus \dotsb \oplus \BCyc_k V, \\
\Filtr^\lambda \DBCyc V &= \{ \psi \in \DBCyc V \mid \Restr{\psi}{\Filtr^\lambda \BCyc V} = 0\} \\
& \simeq (\DBCyc V)_{k+1} \oplus (\DBCyc V)_{k+2} \oplus \dotsb,
\end{align*}
where $\Filtr_{\WeightMRM}$ is the increasing filtration by weights, $\Filtr$ the dual filtration and $(\DBCyc V)_{i}$  the graded dual to $\BCyc_i V$. We also see that $\Filtr^\lambda \DBCyc V = \DBCyc V$ for all $\lambda<1$. Since $\CDBCyc V$ is the completion of $\DBCyc V$ in the graded category, we have $\CDBCyc V\in W_0$. Finally, we know that $\gamma = 2$ for the canonical $\dIBL$-algebra.\Correct[caption={DONE Possible problems},noline]{There might be problem with using \eqref{Eq:IneqWeight} in the way I am using this! Maybe over $\K$ there are even more zero operations.}\qedhere
\end{ProofList}
%Clearly, all strict inequalities of \eqref{Eq:StrCondI} are satisfied automatically due to \eqref{Eq:IneqWeight} except for $(0,0,0)$, $(0,1,0)$ and $(0,0,1)$. However, these are implied by \eqref{Eq:StrCondII}. On the other hand, all strict inequalities of \eqref{Eq:StrCondII} are implied by \eqref{Eq:StrCondI} except for $(1,0,0)$ and $(2,0,0)$. Now, $\Norm{\HTP_{200}}_U > - 2 \gamma$ is satisfied automatically due to \eqref{Eq:IneqWeight}. As for $(1,0,0)$, we use the inequality
%\[ \Norm{\HTP_{klg}(v)}_U \ge \Norm{\HTP_{klg}}_U + \Norm{v}_U \ge -2\gamma(g-1)\quad\text{for all }v\in \hat{\Sym}_k U. \]
%It follows that $\Norm{\HTP_{100}(v)}_U\ge 2\gamma$ for all $v\in \hat{U}$, and hence $\HTP_{100}(v) = 0$ because we have the trivial filtration on $\K$ for which $\lambda\neq 0$ implies $\Norm{\lambda} = 0$ for all $\lambda\in \K$. Therefore, $\HTP_{100} = 0$, and the equivalence is proven.
%
\end{proof}
%
%\begin{equation}\label{Eq:OverK}
%\text{For }f: \K \rightarrow \K\text{ linear}: \quad f\neq 0\ \Implies\ \Abs{f} = 0\quad \&\quad \Norm{f} = 0.
%\end{equation}
%
%
% They are contained \eqref{} and hence implied by the surface calculus. On the other hand $\HTP(1)$ in  the $\MV$-formalism implies the inqualities for all pairs in \eqref{} except for $(1,0,0)$ and $(2,0,0)$. For $(2,0,0)$, the strict inequality $\Norm{\HTP_{200}}>-2\gamma$ is implied by $\Norm{\HTP_{200}}\ge 0$ from \eqref{}. For $\Norm{\HTP_{000}}>0$, we use that we work over $\K$. If we view $\K$ as a graded ring concentrated in degree $0$ equipped with the trivial filtration, then we have the following:
%
%Therefore, $\HTP_{000} = 0$, and hence $\Norm{\HTP_{000}}=\infty > 0$ trivially. For $\HTP_{100}$ it is implied by the filtration condition that it is zero.
%
%Using \eqref{Eq:OverK}, we even have
%Because $\Abs{\OPQ_{00g}} = 2d(1-g)- 1$ where $d\in \Z$, we have $\OPQ_{00g} = 0$ for all $g\ge 0$ over $\K$. Because $\Abs{\HTP_{00g}} = 2d(1-g)$ and $\Norm{\HTP_{00g}}\ge 2\gamma(1-g)$ where $\gamma>0$, we see that $\HTP_{00g} = 0$ unless $g=1$ or $d=0$ and $g\ge 1$.
%
%\begin{Remark}[On $\BV$-formalism for $\IBLInfty$-algebra]
%Let us point out the differences for the purpose of generalizing (e.g., the Novikov ring). We still assume the filtration condition ... The additional condition coming from BV formalism is $\Norm{\HTP_{001}}>$. There is one condition missing in the BV-formalism but required by connected surface calculuse it $\Norm{\HTP_{010}}>$. For this, we might have to impose additionally $\varepsilon \HTP = 0$ or something like this.
%\end{Remark}
%\begin{Remark}[Combined filtration]
%The last thing concerns why we do not use the combined union filtration on $U$ to guarantee convergence of $\sum_{l=0}^\infty \OPQ_{klg}$. It does not really fit in the filtered $\MV$-formalism because as illustrated in Example~\ref{}, this $\Sym U$ is not filtered bialgebra. Note also the fitration condition is satisfied, then the completion with respect to $\HTP_\cup$ and $\HTP$ are isomorphic.
%
%Note that the additional always encode $\OPQ_{klg}$ into $\BVOp$, right? By taking the combined filtration.  We can always define the $\BV$-formalism. However, we then can not do the $e^\HTP$ calculus, i.e., interpret them as maps of the complexes. 
%However, in this setting, we can not use the theory from here to take exponentials and multiply, etc. And it has to be done by hand.
%
%We have ambiguity in the definitions.
%
%We will call then filtered $\IBLInfty$algebras
%We distinguish 
%weak, intput-strict, output-strict and strict.
%\Correct[caption={Bad notation weak, strict, filtered},noline]{What is weak, strict and filtered? A filtered weak IBL-infinity algebra, A filtered strict IBL-infinity algebra, IBL-infinity algebra --- filtered by the trivial filtration and strict }
%\end{Remark}
\begin{Remark}[Generalization over algebra]\label{Rem:GenOverAlg}
It is easy to see that (a) of Proposition~\ref{Prop:BVforIBL} also holds when we work over a non-negatively filtered augmented unital graded $\K$-algebra~$R$. However, (b) should not generalize as there are no conditions in the filtered $\MV$-formalism implying the strict filtration degree condition for $(1,0,0)$ and $(2,0,0)$.
\end{Remark}

\begin{Example}[Asymmetry of $\MV$-formalism]\label{Ex:AsymOfMV}
%One might think that there is a mistake somewhere in switching sums in Proposition somewhere because how is it possible that the equations are well defined without these conditions?.
How is it possible that there are no strict filtration degree conditions on $(1,0,0)$ and $(2,0,0)$ in the filtered $\MV$-formalism even though these unstable surfaces can obviously bubble off (see Figure~\ref{Fig:Bubbling})?\footnote{Note that this question is relevant only when we work over a general $\K$-algebra $R$, so that $\HTP_{100}$ and $\HTP_{200}$ do not necessarily vanish.}

Heuristically, increasing the number of $(1,0,0)$'s glued to the outputs increases the number of times $\bar{\MVCoProd}$ has to be applied to the input to split it and feed it into the new $(1,0,0)$'s. The limit conilpotency property~\eqref{Eq:LimConilp} steps in, increases the filtration degree and provides convergence of the infinite sum. As for the bubbling of $(2,0,0)$, we look at (c) and (d) of Figure~\ref{Fig:Bubbling} and see that there is always an increasing  number of $(0,2,0)$'s or $(1,0,0)$'s in the adjacent components. Now, $(0,2,0)$ increases the filtration degree due to the strict filtration degree condition, and $(1,0,0)$ increases the filtration degree due to the limit conilpotency property as explained above. The convergence is again established.\Add[caption={Particular bubbling in composition},noline]{How is the bubbling of $\HTP_{200}^{\otimes k}\HTP_{100}^{\otimes k}\HTP_{030}^{\otimes k}$ handled in the composition? Maybe it is because there is a strict filtration degree condition on $\HTP_{0lg}$.}
%requires strict filtration degree conditions for $(0,1,0)$ and $(0,2,0)$ but not for $(1,0,0)$ and $(2,0,0)$, as the formalism from \cite{Cieliebak2015} does, and as it seems to be necessary due to the bubbling from Figure~\ref{Fig:Bubbling}.
%condition $(1,0,0)$ and $(2,0,0)$. As pointed out in Remark~\ref{} the bubbling of $(1,0,0)$ might appear in the equation .. and the bubbling of $(2,0,0)$ in the composition $\MVMorF \circ \MVMorF$.

We will now illustrate the bubbling of $\HTP_{100}$ and~$\HTP_{010}$ in the relations for morphisms with a concrete computation. Let $\HTP_{100}: \hat{\Sym}_1 U \rightarrow\hat{\Sym}_0 U$, $\HTP_{010}: \hat{\Sym}_0 U \rightarrow \hat{\Sym}_1 U$, $\OPQ_{1l0}: \hat{\Sym}_1 U \rightarrow \hat{\Sym}_l U$ and $\OPQ_{k10}: \hat{\Sym}_k U \rightarrow \hat{\Sym}_1 U$ for $k$, $l\ge 0$ be $\K$-linear maps such that the following sums converge to $\K((\hbar))$-linear operators $\hat{\Sym}U((\hbar))\rightarrow\hat{\Sym}U((\hbar))$:
\begin{align*}
\BVOp &\coloneqq \sum_{l=1}^\infty \hat{\OPQ}_{1l0}, & \MVMorF&\coloneqq \HTP_{100}, \\
\BVOp'&\coloneqq \sum_{k=1}^\infty \hat{\OPQ}_{k10}\hbar^{k-1}, & \MVMorF' &\coloneqq\HTP_{010}\hbar^{-1}.
\end{align*}
Assume that $\Norm{\HTP} = \Norm{\HTP_{100}}\ge 0$. Because $\HTP_{100}(1)=0$, convergence of the exponential
\[ e^{\MVMorF} = \sum_{r=0}^\infty \frac{1}{r!}\HTP_{100}^{\Star r}: \hat{\Sym}U((\hbar))\rightarrow\hat{\Sym}U((\hbar)) \]
relies purely on the limit conilpotency property (see the proof of Proposition~\ref{Prop:ConvPwrSer}). Assume that $\Norm{\HTP_{010}} = \Norm{\HTP_{010}(1)}>2\gamma$, i.e., $\Norm{\MVMorF'} = \Norm{\MVMorF'(1)} = \Norm{\HTP_{010}(1)} - 2 \gamma >0$. Clearly, convergence of the exponential
\begin{equation}\label{Eq:ExpII}
e^{\MVMorF'} = \sum_{r=0}^\infty \frac{1}{r!}\HTP_{010}^{\Star r}\hbar^{-r}: \hat{\Sym}U((\hbar))\rightarrow\hat{\Sym}U((\hbar))
\end{equation}
relies purely on the strict filtration degree condition. 

For all $v\in\Sym_1 U$, we have
\begin{equation}\label{Eq:BubbleI}
\begin{aligned}
 (e^{\MVMorF}\BVOp)(v) &= \sum_{l=1}^\infty e^{\MVMorF}\hat{\OPQ}_{1l0}(v) \\
 & = \sum_{l=1}^\infty \frac{1}{l!} \mu^{(l)}\HTP_{100}^{\otimes l}\bar{\MVCoProd}^{(l)}\OPQ_{1l0}(v)\\ 
 & = \sum_{l=1}^\infty \HTP_{100}^{\otimes l}\bigl(\OPQ_{1l0}(v)\bigr).
\end{aligned}
\end{equation}
In this simplest case, the limit conilpotency property takes the form $\bar{\MVCoProd}^{(k)}(\Sym_l U) = 0$ for $k>l$, which bounds the number of $\HTP_{100}$'s applied to individual summands.
\Correct[caption={DONE Combinatorial factor},noline]{Maybe there is no combinatorial factor! Checked and it is correct.}
Next, we have
\begin{equation}\label{Eq:BubbleII}
\begin{aligned}
(\BVOp' e^{\MVMorF'})(1) & = \sum_{r=0}^\infty \frac{1}{r!} \BVOp'\HTP_{010}(1)^r\hbar^{-r} \\
& = \sum_{r=0}^\infty \frac{1}{r!} \sum_{i=1}^\infty \hat{\OPQ}_{i10} \HTP_{010}(1)^r \\
& = \sum_{r=1}^\infty\sum_{i=1}^r \frac{1}{i!(r-i)!} \OPQ_{i10}(\HTP_{010}(1)^i)\HTP_{010}(1)^{r-i} \hbar^{-r + i - 1} \\
& = \Bigl(\sum_{i=1}^\infty\frac{1}{i!}\OPQ_{i10}(\HTP_{010}(1)^i)\Bigr)\Bigl(\underbrace{\sum_{t=0}^\infty\frac{1}{t!}\HTP_{010}(1)^t \hbar^{-t-1}}_{=e^{\HTP'(1)}\hbar^{-1}}\Bigr).
\end{aligned}
\end{equation}
Clearly, the condition $\Norm{\HTP_{010}(1)}>0$ is required for the convergence of the sum in the first bracket.
%In both \eqref{Eq:BubbleI} and \eqref{Eq:BubbleII}, we recognize the bubbling phenomenon; in the former case, it is resolved indirectly by imposing conditions on the possible input of $\HTP_{100}$'s (by requiring that $\sum_{l=1}^\infty\frac{1}{l!}\OPQ_{1l0}(v)$ converges), and in the latter case, it is resolved directly by imposing conditions on $\HTP_{010}$.
%This explains the absence of a strict filtration degree condition on~$\HTP_{100}$ in the formalism of filtered $\MV$-algebras and the need of a condition on the convergence of infinite sums of $\OPQ_{klg}$ over $l$.
\end{Example}



%If we have filtered $\IBLInfty$-algebra with. And their morphisms are can be interpreted as well-defined maps $e^\HTP: \hat{\Sym}U^+ \rightarrow \hat{\Sym}U^-$. We could define $\BVOp$ with respect to $\Filtr_\cup$, but then 

Let us now consider the twisting of a complete filtered $\IBLInfty$-algebra in $\MV$-formalism $(\OPQ_{klg})$ on $W$ with a Maurer-Cartan element $\PMC$. The \emph{Maurer-Cartan element} is, by definition, a morphism from the trivial $\IBLInfty$-algebra $0$ to $W$. Because $\hat{\Sym}0 = \hat{\Sym}_0 0 = \K$, only the $(0,l,g)$-components denote by $\PMC_{lg}: \K \rightarrow \hat{\Sym}_l U$ for $l$, $g\ge 0$ might be non-zero, and thus\Add[noline,caption={$l=0$ Maurer Cartan}]{Kai does not have $\PMC_{0g}$. Notice that having $\PMC_{0g}$ causes disconnected twisting!}
\[ \PMC = \sum_{l,g \ge 0} \PMC_{lg} \hbar^{g-1}:\ \hat{\Sym}0((\hbar))\longrightarrow\hat{\Sym}U((\hbar)).\]
Clearly, $e^\PMC(1) = e^{\PMC(1)}\in \hat{\Sym}U((\hbar))$. Consider the left multiplication $L_{e^{\PMC(1)}}: \hat{\Sym}U((\hbar)) \rightarrow \hat{\Sym}U((\hbar))$, and let $\BVOp$ be the $\BV$-operator for $(\OPQ_{klg})$. The $\BV$-operator for the twisted $\IBLInfty$-algebra $(\OPQ_{klg}^\PMC)$ is defined by
\begin{equation}\label{Eq:TwistingEq}
\BVOp^\PMC \coloneqq L_{e^{-\PMC(1)}} \circ \BVOp \circ L_{e^{\PMC(1)}}:\hat{\Sym}U((\hbar))\longrightarrow\hat{\Sym}U((\hbar)).
\end{equation}
The twisting always produces an input-strict $\IBLInfty$-algebra. Indeed, we have
\[ \BVOp \circ e^\PMC = 0\quad\Equiv\quad\BVOp(e^{\PMC(1)})=0\quad\Equiv\quad\BVOp^{\PMC}(1) = 0. \]
This is called the \emph{Maurer-Cartan equation}.

\begin{Proposition}[Twisted $\IBLInfty$-algebra]\label{Eq:TwistingProp}
Let $(W,\BVOp)$ be a complete filtered $\IBLInfty$-algebra in $\MV$-formalism, and let $\PMC$ be a Maurer-Cartan element. Then $\BVOp^\PMC$ defined by~\eqref{Eq:TwistingEq} is an input-strict complete filtered $\IBLInfty$-algebra in $\MV$-formalism of the same bidegree as $(W,\BVOp)$, and for its components $(\OPQ_{klg}^\PMC)$, it holds
\[ \OPQ^\PMC_{klg} = \sum_{\substack{r\ge 0\\ k', l', g', l_1, g_1, \dotsc, l_{r}, g_{r} \ge 0 \\ k' - h_1 - \dotsb - h_{r}  = k \\ l' + l_1 + \dotsb + l_{r} - h_1 - \dotsb - h_r = l \\ g' + h_1 + \dotsb + h_{r} - r + g_1 + \dotsb+ g_{r} = g}}\hspace{-1cm}\OPQ_{k'l'g'}\circ_{h_1,\dotsc,h_{r}}(\PMC_{l_1 g_1},\dotsc,\PMC_{l_{r},g_{r}})\quad\text{for all }k\ge 1, l, g\ge 0, \]
where $(\OPQ_{klg})$ and $(\PMC_{lg})$ are components of $\BVOp$ and $\PMC$, respectively.
\end{Proposition}
\begin{proof}
Using $L_{e^{\PMC(1)}} = e^{\PMC}\Star \Id$, we compute
\begin{align*}
&\BVOp \circ L_{e^{\PMC(1)}} \\
&\quad = \sum_{\substack{r\ge 0 \\ k', l', g', l_1, g_1, \dotsc, l_{r'}, g_{r'} \ge 0}}  \frac{1}{r!} \hat{\OPQ}_{k'l'g'}\circ(\PMC_{l_1 g_1} \Star \dotsb \Star \PMC_{l_r g_r}\Star \Id)\hbar^{g_1 + \dotsb + g_r - r + k' + g' - 1} \\
&\quad = \sum_{\substack{r\ge 0 \\ k, k', l', g', l_1, g_1, \dotsc, l_{r'}, g_{r'} \ge 0 \\ h_1, \dotsc, h_r \ge 0 \\ h_1 + \dotsb + h_r + k = k'}}  \frac{1}{r!} \OPQ_{k'l'g'}\circ_{h_1,\dotsc,h_r,k}(\PMC_{l_1 g_1}, \dotsb, \PMC_{l_r g_r}, \Id)\hbar^{g_1 + \dotsb + g_r - r + k' + g' - 1} \\
&\quad = \sum_{\substack{
r\ge 0 \\ 0 \le r' \le r \\ k, k', l', g', l_1, g_1, \dotsc, l_{r'}, g_{r'} \ge 0 \\ h_1, \dotsc, h_{r'} \ge 1 \\ h_1 + \dotsb + h_{r'} + k = k'}}\hspace{-.2cm}\begin{multlined}[t]\frac{1}{(r-r')!}\frac{1}{r'!} \PMC_{l_{r'+1} g_{r'+1}}\Star \dotsb \Star \PMC_{l_r g_r} \\ \Star \bigl(\OPQ_{k'l'g'}\circ_{h_1,\dotsc,h_{r'},k}(\PMC_{l_1 g_1},\dotsc,\PMC_{l_{r'} g_{r'}},\Id)\bigr) \hbar^{g_1 + \dotsb + g_r - r + k' + g' - 1}\end{multlined}\\
&\quad=\begin{multlined}[t]
\Bigl(\sum_{t\ge 0} \frac{1}{t!} \PMC_{l_1' g_1'}\Star\dotsb\Star\PMC_{l_{t}'g_t'} \hbar^{g_{1}' + \dotsb + g_t' - t}\Bigr) \\
\Star \Bigl(\sum_{k,l,g\ge 0}\hspace{-.5cm}\sum_{\substack{r'\ge 0\\
k,k',l',g',l_1,g_1,\dotsc,l_r,g_r\ge 0\\
h_1,\dotsc, h_{r'}\ge 1 \\
k' - h_1 - \dotsb - h_{r'} =  k\\
l' + l_1 + \dotsb + l_{r'} - h_1 - \dotsb - h_{r'} = l\\
g' + h_1 + \dotsb + h_{r'} - r' + g_1 + \dotsb + g_{r'} = g
}} \hspace{-.5cm} \frac{1}{r'!}\OPQ_{k'l'g'}\circ_{h_1,\dotsc,h_{r'},k}(\PMC_{l_1 g_1},\dotsc,\PMC_{l_{r'} g_{r'}},\Id) \hbar^{k + g -1}\Bigr)
\end{multlined}\\
&\quad=\begin{multlined}[t]
\Bigl(\sum_{t\ge 0} \frac{1}{t!} \bigl(\PMC_{l_1' g_1'}\Star\dotsb\Star\PMC_{l_{t}'g_t'}\Star\Id\bigr) \hbar^{g_{1}' + \dotsb + g_t' - t}\Bigr) \\
\circ \Bigl(\sum_{k,l,g\ge 0}\hspace{-.5cm}\sum_{\substack{r'\ge 0\\
k,k',l',g',l_1,g_1,\dotsc,l_r,g_r\ge 0\\
h_1,\dotsc, h_{r'}\ge 1 \\
k' - h_1 - \dotsb - h_{r'} =  k\\
l' + l_1 + \dotsb + l_{r'} - h_1 - \dotsb - h_{r'} = l\\
g' + h_1 + \dotsb + h_{r'} - r' + g_1 + \dotsb + g_{r'} = g
}} \hspace{-.5cm} \frac{1}{r'!}\OPQ_{k'l'g'}\circ_{h_1,\dotsc,h_{r'}}(\PMC_{l_1 g_1},\dotsc,\PMC_{l_{r'} g_{r'}})\Star\Id \hbar^{k + g -1}\Bigr)
\end{multlined}\\
&\quad=L_{e^{\PMC(1)}}\circ \Bigl(\sum_{k,l,g\ge 0}\hspace{-.5cm}\sum_{\substack{r'\ge 0\\
k,k',l',g',l_1,g_1,\dotsc,l_r,g_r\ge 0\\
h_1,\dotsc, h_{r'}\ge 1 \\
k' - h_1 - \dotsb - h_{r'} =  k\\
l' + l_1 + \dotsb + l_{r'} - h_1 - \dotsb - h_{r'} = l\\
g' + h_1 + \dotsb + h_{r'} - r' + g_1 + \dotsb + g_{r'} = g
}} \hspace{-.5cm} \frac{1}{r'!}\reallywidehat{\OPQ_{k'l'g'}\circ_{h_1,\dotsc,h_{r'}}(\PMC_{l_1 g_1},\dotsc,\PMC_{l_{r'} g_{r'}})}\hbar^{k + g -1}\Bigr).
\end{align*}
The claim follows.
\end{proof}

%Note that $\hat{\Sym}U((\hbar))$ is a completion of the tensor product and hence containt even infinite series in $h^{-n}$ if the filtration is OK.
% $\circ_{h_1,\dotsc,h_k}$ on multilinear maps from Section~\ref{Sec:Alg1}. 
%\begin{Remark}[Some questions]\label{Rem:OpQueMV}\begin{RemarkList}
% \item Do (b) and (c) of Proposition~\ref{Prop:EqCharOfMVIBL} hold for weak $\IBLInfty$-morphisms as well or is there some other version of \eqref{Eq:WeakIBLMor} and \eqref{Eq:CompositionOfMorphisms} possibly involving disconnected gluing?
% \item Are $e^\HTP$ and $L_{e^{\PMC(1)}}$ always well-defined operators on $\hat{\Sym}U((\hbar))$ if the completion with respect to the combined filtration $\Filtr_\cup$ is taken as in \cite{Cieliebak2015}?\qedhere
%\end{RemarkList}\end{Remark}


\section{BV-complexes for IBL-infinity-algebras}\label{Sec:BVCompl}

We plan to use the filtered $\MV$-formalism from the previous section to study chain-maps between the following chain complexes.

\begin{Definition}[Chain complexes for $\IBLInfty$-algebras]\label{Def:BVCompl}
Given a complete filtered $\IBLInfty$-algebra in $\MV$-formalism $(U,\BVOp)$, we call the chain complex $(\hat{\Sym}U((\hbar)),\BVOp)$ the \emph{$\BV$-complex}. If $(U,\BVOp)$ is input-strict, i.e., if $\BVOp(1) = 0$, then $(\hat{\Sym}U[[\hbar]],\BVOp)$ is a chain complex as well, and we call it the \emph{strict $\BV$-complex}. If $(U,\BVOp)$ is strict, we also have the chain complex $(\hat{\Sym}_1 U, \OPQ_{110})$, and we call it the \emph{$\IBLInfty$-chain complex} (c.f., Definition~\ref{Def:HomIBL} in Part~I).
\end{Definition}

Clearly, the definition works also for filtered $\IBLInfty$-algebras from \cite{Cieliebak2015} (we are just not sure about Proposition~\ref{Prop:ObservationsMor} below, and thus we rather stick to our formalism).

\begin{Remark}[$\BV$-bicomplex for surfaces]\label{Rem:BVBicomp}
Writing $\BVOp = \BVOp_1 + \BVOp_2 \hbar + \BVOp_3 \hbar^2 + \dotsb$ in the input-strict case, we have 
\[ 0 \overset{!}{=} \BVOp^2 = \BVOp_1^2 + \hbar(\BVOp_1 \BVOp_2 + \BVOp_2 \BVOp_1) + \hbar^2(\BVOp_2^2 + \BVOp_1 \BVOp_3 + \BVOp_3 \BVOp_1) + \hbar^3(\dotsb) + \dotsb. \]
We see that $\BVOp_1^2 = 0$ and $[\BVOp_1,\BVOp_2] = 0$, where $[\cdot,\cdot]$ is the graded commutator, always hold. Moreover, if $[\BVOp_1,\BVOp_3] = 0$, then also $\BVOp_2^2 = 0$. Now, $\Abs{\BVOp_1} = -1$ always holds, and if $d=-1$, then $\Abs{\hbar} = -2$ and $\Abs{\BVOp_2} = -1 - 2d = 1$. We see that if $[\BVOp_1,\BVOp_3]=0$ and $d=-1$, then $(\hat{\Sym}U, \BVOp_1, \BVOp_2)$ is a (homological) mixed complex. If $\BVOp_k = 0$ for $k \ge 3$, then the $[\hbar^{-1},\hbar]]$- and $[[\hbar]]$-versions of the total complex (see Remark~\ref{Rem:MixedCompl}) compute the $\BV$- and strict $\BV$-homology, respectively. Note that by \cite[Proposition~2.2]{Cieliebak2018b}, these versions are quasi-isomorphism invariants of mixed complexes.\footnote{A morphism of mixed complexes commutes with both the boundary operator and the differential. A quasi-isomorphism of homological mixed complexes is a morphism of mixed complexes which is a quasi-isomorphism with respect to the boundary operator.}

In the case of the canonical $\dIBL$-algebra on cyclic cochains of a Poincar\'e duality algebra of degree $n$, we have $d=n-3$, and so $d=-1$ corresponds to $n=2$.
\end{Remark}
\begin{Proposition}[Morphisms of $\BV$-complexes]\label{Prop:ObservationsMor}
In the category of complete filtered $\IBLInfty$-algebras in $\MV$-formalism, the following holds:
\begin{ClaimList}
 \item A weak morphism of weak $\IBLInfty$-algebras induces a chain map $e^\HTP$ of $\BV$-complexes.
 \item An input-strict morphism of input-strict $\IBLInfty$-algebras induces a chain map of both the $\BV$-complex and its strict version. 
 \item The $(1,1,0)$-part of a strict $\IBLInfty$-morphism of strict $\IBLInfty$-algebras induces a chain map of $\IBLInfty$-chain complexes.
 \item For a Maurer-Cartan element $\PMC$, the left multiplication $L_{e^{\PMC(1)}}: \hat{\Sym}U((\hbar))\rightarrow \hat{\Sym}U((\hbar))$ is an isomorphism of the $\BV$-complexes $(\hat{\Sym}U((\hbar)),\BVOp)\simeq (\hat{\Sym}U((\hbar)),\BVOp^\PMC)$.
\end{ClaimList}
\end{Proposition}
\begin{proof}
Clear.
\end{proof}

Notice that whereas $L_{e^{\PMC(1)}}$ is an isomorphism of chain complexes and it even has non-negative filtration degree, it might not be, and it should not be, an $\IBLInfty$-morphism (see question~(iii) of Remark~\ref{Rem:SomeQuestionsFilter}). This would clarify the author's confusion explained in the preamble.

\begin{Remark}[Some questions]\label{Rem:SomeQuestionsFilter}\begin{RemarkList}
\item What is the geometric meaning of 
\begin{equation}\label{Eq:HomGeom}
\H(\hat{\Sym}U((\hbar)),\BVOp) \simeq \H(\hat{\Sym}U((\hbar)),\BVOp^\PMC), \quad\H(\hat{\Sym}U[[\hbar]],\BVOp)\quad\text{and}\quad\H(\hat{\Sym}U[[\hbar]],\BVOp^\PMC),
\end{equation}
where $U=\DBCyc\HDR(M)[3-n]$ is the (reduced) dual cyclic bar complex of the de Rham cohomology of a closed oriented $n$-manifold $M$ with the filtration dual to the (homological, i.e., increasing) filtration by weights, $\BVOp$ represents the canonical $\dIBL$-structure for the intersection pairing on $\HDR(M)$ and $\PMC$ is the Chern-Simons Maurer-Cartan element? Are some of the homologies~\eqref{Eq:HomGeom} isomorphic? How about the bicomplex from Remark~\ref{Rem:BVBicomp} for surfaces?
\item Are there any implications between the following statements (a)--(c)? A strict $\IBLInfty$-morphism of strict $\IBLInfty$-algebras induces
\begin{enumerate}[label=(\alph*),leftmargin=1.5cm]
\item a quasi-isomorphism of $\IBLInfty$-complexes,
\item a quasi-isomorphism of strict $\BV$-complexes and
\item a quasi-isomorphism of $\BV$-complexes? 
\end{enumerate}
\item For a Maurer-Cartan element $\PMC$, what is
\[ \log L_{e^{\PMC(1)}}:\ \hat{\Sym}U((\hbar)) \longrightarrow \hat{\Sym}U((\hbar)), \]
and can one see that this map is, in general, not in the form of an $\IBLInfty$-morphism?
\qedhere
\end{RemarkList}
\end{Remark}

\Add[caption={Add formulas},noline]{Add here formulas for $\PMC = \HTP_* \MC$. It should be the composition $\HTP \DiamComp \MC$ but $\MC$ is weak morphism! But very simple weak morphism. Maybe the induction will work. Add the formula for $\HTP^\MC = \log(L_{e^{-\MC(1)}}\circ e^\HTP \circ L_{e^{\MC(1)}})$. See Page 99 in Diary I.}

\section{Composition of polynomials in convolution product}\label{Sec:CompConvA}
\renewcommand{\Star}{*}
We will work on a (non-filtered) $\N_0$-graded bialgebra $V$ over $\K$ and consider $\K$-linear maps $V\rightarrow V$. Nevertheless, the results extend in a straightforward way to weight-graded complete filtered bialgebras $\hat{V}$ and $\K$-linear maps of finite filtration degrees, and ultimately to the formal series $\hat{V}((\hbar))$ and $\K((\hbar))$-linear maps of finite filtration degrees.

The theory is based on the following observation.
\begin{figure}[t]
\centering
%auto-ignore
\begin{tikzpicture}

\def\distI{4cm}
\def\distII{6cm}
\def\hghtI{4cm}
\def\hghtII{.6cm}
\def\lenII{.5cm}
\def\angI{30}
\def\angII{30}
\def\angIII{10}
\def\pI{.6cm}
\def\pII{1.2cm}
\def\RectHor{.2cm}
\def\RectVer{0cm}

\coordinate[label={[xshift=.45cm,yshift=-.1cm]$\delta^{(n)}$}] (AI) at (0,0);
\coordinate[label={[xshift=.45cm,yshift=-.1cm]$\delta^{(n)}$}] (AII) at ($(AI) + (\distI,0)$);
\coordinate[label={[xshift=.45cm,yshift=-.1cm]$\delta^{(n)}$}] (AIII) at ($(AII) + (\distII,0)$);
\coordinate[label={[xshift=.5cm,yshift=-.5cm]$\mu^{(m)}$}] (AIV) at ($(AI) + (0,-\hghtI)$);
\coordinate[label={[xshift=.5cm,yshift=-.5cm]$\mu^{(m)}$}] (AV) at ($(AIV) + (\distI,0)$);
\coordinate[label={[xshift=.5cm,yshift=-.5cm]$\mu^{(m)}$}] (AVI) at ($(AV) + (\distII,0)$);

\coordinate (AICI) at ($(AI) + (-\pII,-\hghtII)$);
\coordinate (AICII) at ($(AI) + (-\pI,-\hghtII)$);
\coordinate (AICIII) at ($(AI) + (\pII,-\hghtII)$);

\coordinate (AIICI) at ($(AII) + (-\pII,-\hghtII)$);
\coordinate (AIICII) at ($(AII) + (-\pI,-\hghtII)$);
\coordinate (AIICIII) at ($(AII) + (\pII,-\hghtII)$);

\coordinate (AIIICI) at ($(AIII) + (-\pII,-\hghtII)$);
\coordinate (AIIICII) at ($(AIII) + (-\pI,-\hghtII)$);
\coordinate (AIIICIII) at ($(AIII) + (\pII,-\hghtII)$);

\coordinate (AIVCI) at ($(AIV) + (-\pII,\hghtII)$);
\coordinate (AIVCII) at ($(AIV) + (-\pI,\hghtII)$);
\coordinate (AIVCIII) at ($(AIV) + (\pII,\hghtII)$);

\coordinate (AVCI) at ($(AV) + (-\pII,\hghtII)$);
\coordinate (AVCII) at ($(AV) + (-\pI,\hghtII)$);
\coordinate (AVCIII) at ($(AV) + (\pII,\hghtII)$);

\coordinate (AVICI) at ($(AVI) + (-\pII,\hghtII)$);
\coordinate (AVICII) at ($(AVI) + (-\pI,\hghtII)$);
\coordinate (AVICIII) at ($(AVI) + (\pII,\hghtII)$);


\draw (AI)--++(90:\lenII);
\draw (AII)--++(90:\lenII);
\draw (AIII)--++(90:\lenII);
\draw (AIV)--++(-90:\lenII);
\draw (AV)--++(-90:\lenII);
\draw (AVI)--++(-90:\lenII);

\draw (AICI) -- (AIVCI); 
\draw (AICII) -- (AVCI); 
\path (AICIII) -- (AVICI) coordinate[pos=.4] (PVIIS) coordinate[pos=.9] (PVIIE);
\draw (AICIII) -- (PVIIS); \draw (PVIIE) -- (AVICI);
\draw[dashed] (PVIIS) -- (PVIIE); 

\path (AIICI) --  (AIVCII) coordinate[pos=.05] (PIS) coordinate[pos=.25] (PIE) {} coordinate[pos=.4] (PIIS) {} coordinate[pos=.6] (PIIE) {};
\draw (AIICI) -- (PIS); \draw (PIE) -- (PIIS); \draw (PIIE) -- (AIVCII);
\draw[dashed] (PIS) -- (PIE); \draw[dashed] (PIIS) -- (PIIE);
\path (AIICII) -- (AVCII) coordinate[pos=.15] (PIIIS) coordinate[pos=.4] (PIIIE);
\draw (AIICII) -- (PIIIS); \draw (PIIIE) -- (AVCII);
\draw[dashed] (PIIIS) -- (PIIIE);
\path (AIICIII) -- (AVICII) coordinate[pos=.1] (PVIS) coordinate[pos=.8] (PVIE);
\draw (AIICIII) -- (PVIS); \draw (PVIE) -- (AVICII);
\draw[dashed] (PVIS) -- (PVIE);

\path (AIIICI) -- (AIVCIII) coordinate[pos=.1] (PIVS) coordinate[pos=.9] (PIVE) ;
\draw (AIIICI) -- (PIVS); \draw (PIVE) -- (AIVCIII);
\draw[dashed] (PIVS) -- (PIVE);

\path (AIIICII) -- (AVCIII) coordinate[pos=.2] (PVS) coordinate[pos=.9] (PVE);
\draw (AIIICII) -- (PVS); \draw (PVE) -- (AVCIII);
\draw[dashed] (PVS) -- (PVE); 
\draw (AIIICIII) -- (AVICIII);

\draw (AI) -- (AICI);
\draw (AI) -- (AICII);
\draw (AI) -- (AICIII);

\draw (AII) -- (AIICI);
\draw (AII) -- (AIICII);
\draw (AII) -- (AIICIII);

\draw (AIII) -- (AIIICI);
\draw (AIII) -- (AIIICII);
\draw (AIII) -- (AIIICIII);

\draw (AIV) -- (AIVCI);
\draw (AIV) -- (AIVCII);
\draw (AIV) -- (AIVCIII);

\draw (AV) -- (AVCI);
\draw (AV) -- (AVCII);
\draw (AV) -- (AVCIII);

\draw (AVI) -- (AVICI);
\draw (AVI) -- (AVICII);
\draw (AVI) -- (AVICIII);

\tikzset{
DecorMyDots/.style 2 args={decorate,decoration={shape backgrounds,shape=circle,shape size=#1,shape sep=#2},fill},
DecorMyDots/.default={0.3mm}{2mm},
DecorMyBrace/.style={decorate, decoration={brace,amplitude=#1}},
DecorMyBrace/.default={7mm}
};
 
\draw[DecorMyDots] ($.3*(AII) + .7*(AIII) + .5*(0,\hghtII)$) -- ($.7*(AII) + .3*(AIII) + .5*(0,\hghtII)$);
\draw[DecorMyDots] ($.3*(AV) + .7*(AVI) - .5*(0,\hghtII)$) -- ($.7*(AV) + .3*(AVI) - .5*(0,\hghtII)$);
\draw[DecorMyBrace] ($(AI)+(-2*\lenII,\lenII)$) -- ($(AIII)+(2*\lenII,\lenII)$) node[midway,above,yshift=7mm]{$m$-times};
\draw[DecorMyBrace] ($(AVI)+(2*\lenII,-\lenII)$) -- ($(AIV)+(-2*\lenII,-\lenII)$) node[midway,below,yshift=-7mm]{$n$-times};
\draw[DecorMyDots] ($.7*(AICII) + .3*(AICIII) + .3*(0,\hghtII)$) -- ($.4*(AICII) + .6*(AICIII) + .3*(0,\hghtII)$);
\draw[DecorMyDots] ($.7*(AIICII) + .3*(AIICIII) + .3*(0,\hghtII)$) -- ($.4*(AIICII) + .6*(AIICIII) + .3*(0,\hghtII)$);
\draw[DecorMyDots] ($.7*(AIIICII) + .3*(AIIICIII) + .3*(0,\hghtII)$) -- ($.4*(AIIICII) + .6*(AIIICIII) + .3*(0,\hghtII)$);
\draw[DecorMyDots] ($.7*(AVICII) + .3*(AVICIII) + .3*(0,-\hghtII)$) -- ($.4*(AVICII) + .6*(AVICIII) + .3*(0,-\hghtII)$);
\draw[DecorMyDots] ($.7*(AVCII) + .3*(AVCIII) + .3*(0,-\hghtII)$) -- ($.4*(AVCII) + .6*(AVCIII) + .3*(0,-\hghtII)$);
\draw[DecorMyDots] ($.7*(AIVCII) + .3*(AIVCIII) + .3*(0,-\hghtII)$) -- ($.4*(AIVCII) + .6*(AIVCIII) + .3*(0,-\hghtII)$);

\draw[dotted] ($(AICI)+(-\RectHor,\RectVer)$) -- ($(AIIICIII)+(\RectHor,\RectVer)$) -- ($(AVICIII)+(\RectHor,-\RectVer)$) node[midway,right] {$\sigma_{m,n}$} -- ($(AIVCI)+(-\RectHor,-\RectVer)$) -- cycle;

\end{tikzpicture}
\caption[The ``heart'' of the iterated bialgebra compatibility condition.]{The ``heart and veins'' of the iterated bialgebra compatibility condition $\MVCoProd^{(n)}\MVProd^{(m)}$, and the definition of $\sigma_{m,n}$.}
\label{Fig:Spider}
\end{figure}
\begin{Lemma}[Iterated compatibility condition]\label{Lem:ItCompCond}
Let $(V, \MVProd, \MVCoProd, \MVUnit, \MVAug)$ be an $\N_0$-graded bialgebra. Write $V = \bigoplus_{i=0}^\infty V_i$,\footnote{We use the lower index because it should correspond to the weight in the weight-graded setting of $V = \Sym U$.} and for every $i\in \N_0$, let $\iota_i : V_i \rightarrow V$ and $\pi_i: V \rightarrow V_i$ be the canonical inclusion and projection, respectively. Let $m$, $n\in \N$, and let $\sigma_{m,n}\in \Perm_{m n}$ be the permutation of $mn$ elements depicted in Figure~\ref{Fig:Spider}. For $A\in \N_0^{m\times n}$, let
\[ F_{A}\coloneqq \bigotimes_{i=1}^m \bigotimes_{j=1}^n \pi_{A_{ij}}\circ \iota_{A_{ij}} : V^{\otimes m n } \longrightarrow V^{\otimes m n}. \]
Then for the iterated product $\MVProd^{(m)}: V^{\otimes m} \rightarrow V$ and the iterated coproduct $\MVCoProd^{(n)}: V\rightarrow V^{\otimes n}$, the following holds:
\begin{equation}\label{Eq:ItCompRel}
\MVCoProd^{(n)}\MVProd^{(m)} = \MVProd^{(m)\otimes n} \sigma_{m,n} \MVCoProd^{(n)\otimes m} =  \sum_{A\in \N_0^{m\times n}} \MVProd^{(m)\otimes n} \sigma_{m,n}F_A \MVCoProd^{(n)\otimes m}.
\end{equation}
For $n=m=2$, this reduces to the well-known compatibility relation
\begin{equation}\label{Eq:BasicCompCond}
\MVCoProd\circ \MVProd = (\MVProd\otimes\MVProd)\circ (\Id\otimes\tau\otimes\Id)\circ(\MVCoProd\otimes\MVCoProd).
\end{equation}
\end{Lemma}
\begin{proof}
Clearly, the second equality of \eqref{Eq:ItCompRel} holds because $\sum_{A\in\N_0^{m\times n}} F_A = \Id^{mn}$. As for the first equality, it holds for $m=n=2$ from the definition of a bialgebra; the rest is obtained by induction on $m$, $n$, distinguishing the two cases $\MVCoProd^{(n)}\MVProd^{(m)} = (\MVCoProd\otimes\Id^{\otimes n-2})\MVCoProd^{(n-1)}\MVProd^{(m)}$ and $\MVCoProd^{(n)}\MVProd^{(m)} = \MVCoProd^{(n)}\MVProd^{(m-1)}(\mu\otimes\Id^{\otimes n-2})$, respectively. In the first case, the induction step reads
\begin{align*}
\MVCoProd^{(n)}\MVProd^{(m)} & = (\MVCoProd\otimes\Id^{\otimes n-2})\MVCoProd^{(n-1)}\MVProd^{(m)} \\
& = (\MVCoProd\otimes\Id^{\otimes n-2})\MVProd^{(m)\otimes n-1}\sigma_{m,n-1}\MVCoProd^{(n-1)\otimes m} \\
& = \MVProd^{(m)\otimes n}(\sigma_{m,2} \MVCoProd^{\otimes m} \otimes \Id^{\otimes m(n-2)})\sigma_{m,n-1}\MVCoProd^{(n-1)\otimes m} \\
& = \MVProd^{(m)\otimes n} \sigma_{m,n} \delta^{(n)\otimes m}.
\end{align*}
On the third line, we used that $\MVCoProd\MVProd^{(m)}=(\MVProd^{(m)}\otimes\MVProd^{(m)})\sigma_{m,2}\MVCoProd^{\otimes m}$, which equals \eqref{Eq:BasicCompCond} for $m=2$, and for $m\ge 3$, it is proven by induction with the induction step
\begin{align*}
 \MVCoProd\MVProd^{(m)} & = \MVCoProd\MVProd^{(m-1)}(\MVProd\otimes\Id^{\otimes m-2}) \\
 & = (\MVProd^{(m-1)}\otimes\MVProd^{(m-1)})\sigma_{m-1,2}\MVCoProd^{\otimes m-1}(\MVProd\otimes\Id^{\otimes m-2}) \\
 & = (\MVProd^{(m-1)}\otimes\MVProd^{(m-1)})\sigma_{m-1,2}(\MVProd\otimes\MVProd \otimes\Id^{m-2})(\Id \otimes \tau\otimes \Id \otimes \Id^{m-2})\MVCoProd^{\otimes m} \\
 & = (\MVProd^{(m)}\otimes\MVProd^{(m)})\sigma_{m,2}\MVCoProd^{\otimes m}.
\end{align*}
The second case is analogous.
%Recall that $\Id = \sum_{n=0}^\infty \iota_n \circ \pi_n$
%\begin{align*}
% \CoProd^{(n)}\Prod^{(n)} &= (\CoProd\otimes\Id^{n-2})(\Prod^{(n-1)})^{\otimes n-1} \sigma_{n-1} (\CoProd^{(n-1)})^{\otimes n - 1}(\mu \otimes \Id^{n-2}) \\
% & = \bigl(\mu^{(n)}\otimes \mu^{(n)} \otimes (\mu^{(n-1)})^{\otimes n - 2}\bigr)\bigl(\sigma_{n-1}'\CoProd^{\otimes n-1} \sigma_{n-1} \Prod^{\otimes n-1} \sigma_{n-1}''\bigr)\bigl(\delta^{(n)} \otimes \delta^{\otimes n} \otimes (\delta^{n-1})^{\otimes n-2}\bigr) \\
% & = \bigl(\mu^{(n)}\otimes \mu^{(n)} \otimes (\mu^{(n-1)})^{\otimes n - 2}\bigr)\bigl(\Id\otimes \Id \otimes \bigr)\bigl(\delta^{(n)} \otimes \delta^{\otimes n} \otimes (\delta^{n-1})^{\otimes n-2}\bigr) 
%\end{align*}
\end{proof}
\begin{Definition}[Compositions of monomials in convolution product]\label{Def:ConComp}
Consider an $\N_0$-graded bialgebra $(V,\MVProd,\MVCoProd,\MVUnit,\MVAug)$. Given linear maps $\MVMorF_1$, $\dotsc$, $\MVMorF_r$, $\MVMorF_1', \dotsc, \MVMorF_{r'}': V \rightarrow V$ and a matrix $A\in \N_0^{r'\times r}$ for $r$, $r'\in \N$, we define the \emph{$A$-composition} $(\MVMorF_1,\dotsc,\MVMorF_r)\SquareComp_A(\MVMorF_1,\dotsc,\MVMorF_{r'}): V \rightarrow V$ by the formula
\begin{align*}
&(\MVMorF_1,\dotsc,\MVMorF_r)\SquareComp_A(\MVMorF_1,\dotsc,\MVMorF_{r'}) \\
&\qquad\coloneqq \MVProd^{(r)}\circ(\MVMorF_1 \otimes \dotsb \otimes \MVMorF_r)\circ \Prod^{(r')\otimes r} \circ  \sigma_{r',r}\circ F_A \circ \MVCoProd^{(r)\otimes r'} \circ (\MVMorF_1' \otimes \dotsb \otimes \MVMorF_{r'}')\circ\MVCoProd^{(r')},
\end{align*}
where $\sigma_{r',r}$ and $F_A$ were defined in Lemma~\ref{Lem:ItCompCond}.

We call a matrix $A\in\N_0^{r'\times r}$ \emph{connected} if for the matrix
\[ B \coloneqq \begin{pmatrix} 0 & A \\ A^T & 0 \end{pmatrix}, \]
the matrix product $B^{r+r'}$ has at least one row with all entries non-zero.\footnote{Such $A$ represents a connected weighted bipartite graph.} In this case, we say that the $A$-composition $(\MVMorF_1,\dotsc,\MVMorF_r)\SquareComp_A(\MVMorF_1,\dotsc,\MVMorF_{r'})$ is \emph{connected.}

Suppose that $V$ is connected, i.e., $V_0 = \langle 1\rangle$. Given linear maps $\MVMorF_1$, $\dotsc$, $\MVMorF_r$, $\MVMorF_1'$, $\dotsc$, $\MVMorF_{r'}': V\rightarrow V$ for $r$, $r'\in \N$, we define the \emph{connected composition} $(\MVMorF_1, \dotsc, \MVMorF_r)\circ_{\mathrm{con}}(\MVMorF_1',\dotsc,\MVMorF_{r'}'): V \rightarrow V$ by
\[ (\MVMorF_1, \dotsc, \MVMorF_r)\circ_{\mathrm{con}}(\MVMorF_1,\dotsc,\MVMorF_{r'}) = \sum_{\substack{A\in \N_0^{(r'+1)\times (r+1)}\\ A\  \mathrm{connected} \\ A_{r'+1,r+1} = 0 }} (\MVMorF_1,\dotsc,\MVMorF_r,\Id)\SquareComp_A (\MVMorF_1',\dotsc,\MVMorF_{r'}',\Id).\]
Given linear maps $\MVMorF$, $\MVMorF_1$, $\dotsc$, $\MVMorF_r: V \rightarrow V$ and parameters $h_1$, $\dotsc$, $h_r\in \N_0$ for $r\in \N$, we define the \emph{partial compositions} $\MVMorF\circ_{h_1, \dotsc, h_r}(\MVMorF_1, \dotsc, \MVMorF_r)$, $(\MVMorF_1, \dotsc, \MVMorF_r)\circ_{h_1, \dotsc, h_r}\MVMorF: V\rightarrow V$ by
\begin{align*}
&\MVMorF\circ_{h_1, \dotsc, h_r}(\MVMorF_1, \dotsc, \MVMorF_r) \\
&\qquad\coloneqq\begin{multlined}[t]\MVProd(\MVMorF\otimes \Id)(\MVProd\otimes\Id)(\Id\otimes \tau)\bigl(\bigl[(\MVProd^{(r)}\otimes \MVProd^{(r)})(F_{h_1,\dotsc,h_r} \otimes \Id^{\otimes r})\sigma_r\MVCoProd^{\otimes r}\bigr] \otimes \Id \bigr)\\(\MVMorF_1\otimes \dotsb \otimes \MVMorF_r\otimes \Id)\MVCoProd^{(r+1)},\end{multlined}\\
&(\MVMorF_1, \dotsc, \MVMorF_r)\circ_{h_1, \dotsc, h_r}\MVMorF\\
&\qquad\coloneqq\begin{multlined}[t]\MVProd^{(r+1)}(\MVMorF_1\otimes\dotsb\otimes\MVMorF_r\otimes\Id) \bigl(\bigl[\MVProd^{\otimes r}\sigma_r^{-1}(F_{h_1,\dotsc,h_r}\otimes\Id^{\otimes r})(\MVCoProd^{(r)}\otimes\MVCoProd^{(r)})\bigr]\otimes\Id\bigr)\\(\Id\otimes\tau)(\MVCoProd\otimes\Id)(\MVMorF\otimes\Id)\MVCoProd,\end{multlined}
\end{align*}
where we set $F_{h_1,\dotsc, h_r} \coloneqq F_{(h_1,\dotsc,h_r)} = F_{(h_1,\dotsc,h_r)^T}=\iota_{h_1}\pi_{h_1} \otimes \dotsb \otimes \iota_{h_r}\pi_{h_r}$ and $\sigma_r \coloneqq \sigma_{r,2}$, and we omit writing the composition $\circ$. For $r=1$, we have $\MVProd^{(1)} = \MVCoProd^{(1)}\coloneqq \Id$ by definition, and both equations above reduce to 
\begin{align*}
\MVMorF \circ_h \MVMorF_1 &= \MVProd(\MVMorF\otimes \Id)(\MVProd\otimes\Id)(\Id\otimes\tau)(F_h\otimes\Id^{\otimes 2})(\MVCoProd\otimes \Id)(\MVMorF_1 \otimes \Id)\MVCoProd \\
\Bigl(\!\!&= \MVProd(\MVMorF\otimes \Id)(\MVProd\otimes\Id)(F_h\otimes\Id^{\otimes 2})(\Id\otimes\tau)(\MVCoProd\otimes \Id)(\MVMorF_1 \otimes \Id)\MVCoProd\Bigr).
\end{align*}
(c.f., Definition~\ref{Def:CircS} in Section~\ref{Sec:Alg1} in Part~I for the case of $V = \Sym(C[1])$)
\end{Definition}

\begin{Proposition}[Partial compositions]\label{Prop:PartCompAComp}
Let $\MVMorF$, $\MVMorF_1$, $\dotsc$, $\MVMorF_r : V \rightarrow V$ be linear maps on a connected $\N_0$-graded bialgebra $V$, and let $h_1$,~$\dotsc$, $h_r \in \N_0$ for $r\in \N$. Then the following formulas hold:
\begin{align*}
\MVMorF\circ_{h_1, \dotsc, h_r}(\MVMorF_1, \dotsc, \MVMorF_r) &= \sum_{\substack{A \in \N_0^{(r+1)\times 2} \\ A = \left(\begin{smallmatrix}
h_1 & \bullet \\
\vphantom{\int\limits^x}\smash{\vdots} &  \smash{\vdots} \\
h_r & \bullet \\ 
\bullet & 0
\end{smallmatrix}\right)}} (\MVMorF,\Id)\SquareComp_{A}(\MVMorF_1,\dotsc,\MVMorF_r,\Id), \\
(\MVMorF_1, \dotsc, \MVMorF_r)\circ_{h_1, \dotsc, h_r}\MVMorF &= \sum_{\substack{A \in \N_0^{2\times(r+1)} \\ A = \left(\begin{smallmatrix}
h_1 & \dotsb & h_r & \bullet \\
\bullet & \dotsb & \bullet & 0
\end{smallmatrix}\right)
}} (\MVMorF_1,\dotsc,\MVMorF_r,\Id)\SquareComp_A(\MVMorF,\Id).
\end{align*}
\end{Proposition}

\begin{proof}
We compute
\begin{align*}
& \sum_{\substack{A \in \N_0^{(r+1)\times 2} \\ A = \left(\begin{smallmatrix}
h_1 & \bullet \\
\vphantom{\int\limits^x}\smash{\vdots} & \smash{\vdots} \\
h_r & \bullet \\ 
\bullet & 0
\end{smallmatrix}\right)}} (\MVMorF\Star\Id)\SquareComp_A (\MVMorF_1\Star\dotsb\Star\MVMorF_r \Star \Id) \\
&\quad =\sum_{\substack{A \in \N_0^{(r+1)\times 2} \\ A = \left(\begin{smallmatrix}
h_1 & \bullet \\
\vphantom{\int\limits^x}\smash{\vdots} & \smash{\vdots} \\
h_r & \bullet \\ 
\bullet & 0
\end{smallmatrix}\right)}} \MVProd(\MVMorF \otimes \Id)\Prod^{(r+1)\otimes 2}\sigma_{r+1,2}F_A\MVCoProd^{\otimes r+1}(\MVMorF_1 \otimes \dotsb \otimes \MVMorF_{r}\otimes\Id)\MVCoProd^{(r+1)} \\
&\quad = \begin{multlined}[t]\MVProd(\MVMorF \otimes \Id)\Prod^{(r+1)\otimes 2}\sigma_{r+1}(\iota_{h_1}\pi_{h_1}\otimes \Id \otimes\dotsb \otimes\iota_{h_r}\pi_{h_r}\otimes \Id\otimes  \Id \otimes \iota_0\pi_0)\MVCoProd^{\otimes r+1} \\ (\MVMorF_1 \otimes \dotsb \otimes \MVMorF_{r}\otimes\Id)\MVCoProd^{(r+1)}\end{multlined}\\
&\quad = \begin{multlined}[t]\MVProd(\MVMorF \otimes \Id)\Prod^{(r+1)\otimes 2}(F_{h_1,\dotsc,h_r}\otimes \Id\otimes  \Id^{\otimes r} \otimes \iota_0\pi_0)\sigma_{r+1} \MVCoProd^{\otimes r+1} \\ (\MVMorF_1 \otimes \dotsb \otimes \MVMorF_{r}\otimes\Id)\MVCoProd^{(r+1)}\end{multlined} \\
&\quad =\begin{multlined}[t]\MVProd(\MVMorF\otimes \Id)(\MVProd\otimes\Id)(\Id\otimes \tau)\bigl(\bigl[\MVProd^{(r)\otimes 2}(F_{h_1,\dotsc,h_r} \otimes \Id^{\otimes r})\sigma_r\MVCoProd^{\otimes r}\bigr] \otimes \Id \bigr)\\(\MVMorF_1\otimes \dotsb \otimes \MVMorF_r\otimes \Id)\MVCoProd^{(r+1)}\end{multlined} \\
&\quad=\MVMorF\circ_{h_1, \dotsc, h_r}(\MVMorF_1, \dotsc, \MVMorF_r).
\end{align*}
On the line before the last line we used that for a connected bialgebra $V$, it holds
\[ (\Id\otimes\iota_0\pi_0)\MVCoProd(v)=v\otimes 1\quad\text{and}\quad (\iota_0\pi_0\otimes\Id)\MVCoProd(v) = 1 \otimes v\quad\text{for all }v\in V.\]
The case of $(\MVMorF_{1},\dotsc,\MVMorF_r)\circ_{h_1,\dotsc,h_r}\MVMorF$ is treated analogously, using that $\sigma_{m,n} = \sigma_{n,m}^{-1}$ (this is visible by looking at the highlighted square in Figure~\ref{Fig:Spider}, which is symmetric under the rotation by $180^\circ$).
\end{proof}

We see that the operations $\circ_{h_1,\dotsc,h_r}$ and $\circ_{\mathrm{con}}$, which are the cornerstone of the surface calculus for $\Sym U$ in Section~\ref{Sec:FilteredMV}, originate naturally from $\SquareComp_A$. Clearly, we can now replace~$\Sym U$ by any connected weight-graded bialgebra~$V$ and develop a surface calculus for $\BVInfty$-algebras over~$V$.

The following formulas were stated in Remark~\ref{Rem:Compositions} in Section~\ref{Sec:Alg1} in Part~I and come from \cite{Cieliebak2015}, where they were used in a slightly different form (and without a proof).
%
%\begin{align*}
%&\MVMorF\circ_{h_1, \dotsc, h_r}(\MVMorF_1, \dotsc, \MVMorF_r) \\
%&\qquad = \sum_{\substack{A\in \N_0^{r} \times \N_0^2 \\ A_{\bullet 1} = (h_1,\dotsc,h_r)^T}}\MVProd(\MVMorF\otimes\Id)\MVProd^{(r) \otimes 2} \sigma_{r,2} F_{A} \MVCoProd^{\otimes r}(\MVMorF_1\otimes\dotsb\otimes\MVMorF_r)\MVCoProd^{(r)} \\
%&\qquad = \mu(\MVMorF \otimes \Id)\MVProd^{(r)\otimes 2}\sigma_{r,2}(\iota_{h_1}\pi_{h_1}\otimes\Id \otimes \dotsb\otimes\iota_{h_r}\pi_{h_r}\otimes \Id)\MVCoProd^{\otimes r}(\MVMorF_1\otimes \dotsb \otimes \MVMorF_r)\MVCoProd^{(r)} \\
%&\qquad = \mu(\MVMorF \otimes \Id)\MVProd^{(r)\otimes 2}(\iota_{h_1}\pi_{h_1}\otimes\dotsb\otimes \iota_{h_r}\pi_{h_r}\otimes \Id^{\otimes r})\sigma_r\MVCoProd^{\otimes r}(\MVMorF_1\otimes \dotsb \otimes \MVMorF_r)\MVCoProd^{(r)}
%\end{align*}
%
%\begin{align*}
%& (\MVMorF_1, \dotsc, \MVMorF_r)\circ_{h_1, \dotsc, h_r}\MVMorF \\
%&\qquad = \sum_{\substack{A\in \N_0^{2} \times \N_0^{r} \\ A_{1 \bullet} = (h_1,\dotsc,h_r)}}\mu^{(r)}(\MVMorF_1\otimes\dotsb\otimes\MVMorF_r)\mu^{\otimes r}\sigma_{2,r}F_A\MVCoProd^{(r)\otimes 2}(\MVMorF\otimes\Id)\MVCoProd \\
%&\qquad = \mu^{(r)}(\MVMorF_1\otimes\dotsb\otimes\MVMorF_r)\mu^{\otimes r} \sigma_{2,r}(\iota_{h_1}\pi_{h_1}\otimes\dotsb\otimes \iota_{h_r}\pi_{h_r}\otimes \Id^{\otimes r})\MVCoProd^{(r)\otimes 2}(\MVMorF\otimes\Id)\MVCoProd\\
%&\qquad = \mu^{(r)}(\MVMorF_1\otimes\dotsb\otimes\MVMorF_r)\mu^{\otimes r} \sigma_{r}^{-1}(\iota_{h_1}\pi_{h_1}\otimes\dotsb\otimes \iota_{h_r}\pi_{h_r}\otimes \Id^{\otimes r})\MVCoProd^{(r)\otimes 2}(\MVMorF\otimes\Id)\MVCoProd
%\end{align*}
%
%
%$\mu^{(1)}= \MVCoProd^{(1)}= \Id$  
%We set $F_{h_1,\dotsc,h_r} \coloneqq F_{(h_1,\dotsc,h_r)} = F_{(h_1,\dotsc,h_r)^T}$
%
%
%
%\begin{Proposition}
%In the setting above, it holds
%\begin{align*}
%\MVMorF\circ_{h_1, \dotsc, h_r}(\MVMorF_1, \dotsc, \MVMorF_r) &= \sum_{A} (\MVMorF\Star\Id) \circ_A (\MVMorF_1\Star\dotsb\Star\MVMorF_r) \\
%(\MVMorF_1, \dotsc, \MVMorF_r)\circ_{h_1, \dotsc, h_r}\MVMorF &= \sum_{A} (\MVMorF_1\Star\dotsb\Star\MVMorF_r)\circ_A(\MVMorF\Star\Id)
%\end{align*}
%\end{Proposition}

\begin{Proposition}[Formulas involving partial compositions] \label{Prop:PartCompositions}
For $r\in \N$, let $\MVMorF$, $\MVMorF_1$, $\dotsc$, $\MVMorF_r : V \rightarrow V$ be linear maps on a connected $\N_0$-graded bialgebra $V$. Then the following formulas hold:
\begin{align}\label{Eq:FormI}
\hat{\MVMorF} \circ \hat{\MVMorF}_1 &= \sum_{h\ge 0} \widehat{\MVMorF\circ_h \MVMorF_1}, \\ 
   \hat{\MVMorF} \circ (\MVMorF_1 \Star \dotsb \Star \MVMorF_r) &= \sum_{k\ge 0}\sum_{\substack{h_1, \dotsc, h_r \ge 0 \\ h_1 + \dotsb + h_r = k}} (\MVMorF \iota_k \pi_k)\circ_{h_1,\dotsc, h_r}(\MVMorF_1,\dotsc, \MVMorF_r), \label{Eq:FormII} \\
   (\MVMorF_1 \Star \dotsb \Star \MVMorF_r) \circ \hat{\MVMorF} &= \sum_{l\ge 0} \sum_{\substack{h_1,\dotsc,h_r \ge 0 \\ h_1 + \dotsb + h_r = l}} (\MVMorF_1,\dotsc,\MVMorF_r) \circ_{h_1,\dotsc,h_r} (\iota_l\pi_l\MVMorF),\label{Eq:FormIII} \\
   \MVMorF\circ_{h_1,\dotsc,h_{r-1},0}(\MVMorF_1,\dotsc, \MVMorF_r) &= \MVMorF\circ_{h_1,\dotsc, h_{r-1}}(\MVMorF_1,\dotsc, \MVMorF_{r-1}) \Star \MVMorF_r, \label{Eq:FormIV} \\ \label{Eq:FormV}
  (\MVMorF_1,\dotsc,\MVMorF_r)\circ_{0,h_2,\dotsc,h_r}\MVMorF &= \MVMorF_1 \Star (\MVMorF_2,\dotsc,\MVMorF_r)\circ_{h_2,\dotsc,h_r} \MVMorF.
\end{align}
(Recall that $\hat{\MVMorF} \coloneqq \MVMorF\Star \Id: V \rightarrow V$.)
\end{Proposition}

\begin{proof}
As for \eqref{Eq:FormI}, we compute
\begin{align*}
\hat{\MVMorF} \circ \hat{\MVMorF}_1 &= \MVProd(\MVMorF\otimes \Id)\MVCoProd\MVProd(\MVMorF_1\otimes\Id)\MVCoProd \\
&= \MVProd(\MVMorF\otimes \Id)(\MVProd\otimes\MVProd)(\Id\otimes\tau\otimes\Id)(\MVCoProd\otimes\MVCoProd)(\MVMorF_1\otimes\Id)\MVCoProd \\
&= 
\MVProd(\MVProd\otimes\Id)(\MVMorF\otimes \Id^{\otimes 2})(\MVProd\otimes\Id^{\otimes 2})(\Id\otimes\tau\otimes \Id)(\MVCoProd\otimes\Id^{\otimes 2}) (\MVMorF_1\otimes\Id^{\otimes 2})(\MVCoProd\otimes\Id)\MVCoProd \\
& = 
 \MVProd\bigl(\bigl[\underbrace{\MVProd(\MVMorF\otimes\Id)(\MVProd\otimes\Id)(\Id\otimes\tau)(\MVCoProd\otimes\Id)(\MVMorF_1\otimes\Id)\MVCoProd}_{=\sum_{h\ge 0}\MVMorF\circ_h \MVMorF_1}\bigr]\otimes\Id\bigr)\MVCoProd \\
& = \sum_{h\ge 0} \reallywidehat{\MVMorF \circ_h \MVMorF_1}.
\end{align*}
On the second line, we used the compatibility condition; on the third line, we used associativity and coassociativity.

In order to see \eqref{Eq:FormII}, the easiest is to use Proposition~\ref{Prop:PartCompAComp}:
\begin{align*}
\hat{\MVMorF}(\MVMorF_1 \Star \dotsb \Star \MVMorF_r)&= \sum_{k\ge 0}\sum_{\substack{h_1,\dotsc,h_r\ge 0 \\ h_1 + \dotsb + h_r = k}}\sum_{\substack{A\in \N_0^{r \times 2} \\ A=\left(\begin{smallmatrix}
h_1 & \bullet \\ 
\vphantom{\int\limits^x}\smash{\vdots} & \smash{\vdots} \\
h_r & \bullet
\end{smallmatrix}\right)}}(\MVMorF \iota_k \pi_k,\Id)\SquareComp_{A}(\MVMorF_1,\dotsc,\MVMorF_r) \\
& = \sum_{k\ge 0}\sum_{\substack{h_1,\dotsc,h_r\ge 0 \\ h_1 + \dotsb + h_r = k}}\sum_{\substack{A'\in \N_0^{(r+1)\times 2} \\ A'=\left(\begin{smallmatrix}
h_1 & \bullet \\
\vphantom{\int\limits^x}\smash{\vdots} & \smash{\vdots} \\
h_r & \bullet \\ 
\bullet & 0
\end{smallmatrix}\right)}}(\MVMorF \iota_k \pi_k,\Id)\SquareComp_{A'}(\MVMorF_1,\dotsc,\MVMorF_r,\Id) \\
& = \sum_{k\ge 0}\sum_{\substack{h_1,\dotsc,h_r\ge 0 \\ h_1 + \dotsb + h_r = k}} (\MVMorF\iota_k\pi_k)\circ_{h_1,\dotsc,h_r}(\MVMorF_1,\dotsc,\MVMorF_r).
\end{align*}
On the second line, the bottom-left $\bullet$, i.e., how many ``veins'' of $\Id$ go into $\MVMorF \iota_k \pi_k$, is forced to be $0$ because $\MVMorF \iota_k \pi_k$ has $k$ inputs and $h_1 + \dotsb + h_r = k$. Equation~\eqref{Eq:FormIII} is proven analogously.

As for \eqref{Eq:FormIV}, we prefer to manipulate the expression with bialgebra operations:
%
%
%
% $h_1 + \dotsb + h_r = k$
%
%we start with the case $r=2$. We compute
%\begin{align*}
% &\hat{\MVMorF}\circ (\MVMorF_1\Star \MVMorF_2)\\
% &\ =\MVProd(\MVMorF\otimes \Id)\MVCoProd\MVProd(\MVMorF_1\otimes \MVMorF_2)\MVCoProd \\
% &\ = \MVProd(\MVMorF\otimes\Id)(\MVProd\otimes\MVProd)(\Id\otimes\tau\otimes\Id)(\MVCoProd\otimes\MVCoProd)(\MVMorF_1\otimes \MVMorF_2)\MVCoProd \\
% &\ = \sum_{\substack{h_1, h_2 \ge 0}} \MVProd(\MVMorF\otimes\Id)\MVProd^{\otimes 2}(F_{h_1,h_2}\otimes\Id^{\otimes 2})\sigma_2\MVCoProd^{\otimes 2}(\MVMorF_1\otimes \MVMorF_2)\MVCoProd^{(2)} \\
% &\ = \sum_{\substack{h_1, h_2 \ge 0 \\ h_1 + h_2 = k'}}\MVProd(\MVMorF\otimes\Id)(\MVProd\otimes\Id)(\Id\otimes \tau)\bigl(\bigl[(\MVProd\otimes\MVProd)(F_{h_1,h_2}\otimes\Id^{\otimes 2})\sigma_2\MVCoProd^{\otimes 2}\bigr]\otimes\Id\bigr)(\MVMorF_1\otimes \MVMorF_2\otimes \Id)\MVCoProd^{(3)}\\
% &\ = \sum_{\substack{h_1, h_2\ge 0 \\ h_1 + h_2 = k'}} \MVMorF\circ_{h_1,h_2}(\MVMorF_1,\MVMorF_2). 
%\end{align*}
%Next, for every $r\ge 1$, $h_1$, $\dotsc$, $h_r\ge 0$ and a linear $\MVMorF_{r+1}: E_{k_{r+1}}\rightarrow E_{l_{r+1}}$, $k_{r+1}$, $l_{r+1} \ge 0$, the following holds: 
%\[ \begin{aligned}
%& \MVMorF\circ_{h_1,\dotsc,h_r}(\MVMorF_1,\dotsc, \MVMorF_r \Star \MVMorF_{r+1}) =  \\
%& \quad = \begin{multlined}[t] 
%\MVProd(\MVMorF\otimes \Id)(\MVProd\otimes\Id)(\Id\otimes\tau)\bigl(\bigl[(\MVProd^{(r)}\otimes\MVProd^{(r)})(F_{h_1,\dotsc,h_r}\otimes \Id^{\otimes r})\sigma_r \MVCoProd^{\otimes r}\bigr]\otimes \Id\bigr)\\(\Id^{\otimes r-1}\otimes \MVProd \otimes \Id)(\MVMorF_1\otimes \dotsb\otimes \MVMorF_r \otimes \MVMorF_{r+1}\otimes \Id)(\Id^{\otimes r-1}\otimes \MVCoProd \otimes \Id)\MVCoProd^{(r)}
%\end{multlined} \\ 
%&\quad=\begin{multlined}[t]
%\MVProd(\MVMorF\otimes \Id)(\MVProd\otimes\Id)(\Id\otimes\tau)\bigl(\bigl[(\MVProd^{(r)}\otimes\MVProd^{(r)})(F_{h_1,\dotsc,h_r}\otimes \Id^{\otimes r})\sigma_r \MVCoProd^{\otimes r}\\ (\Id^{\otimes r-1}\otimes \MVProd) \bigr]\otimes \Id\bigr) (\MVMorF_1\otimes \dotsb\otimes \MVMorF_r \otimes \MVMorF_{r+1}\otimes \Id)\MVCoProd^{(r+1)}
%\end{multlined}\\
%&\quad =: (*).
%\end{aligned}
%\]
%The auxiliary computations
%\[ \begin{aligned} \MVCoProd^{\otimes r}(\Id^{\otimes r-1}\otimes \MVProd) &= \MVCoProd^{\otimes r-1} \otimes (\MVCoProd \MVProd) \\ &= \MVCoProd^{\otimes r-1} \otimes \bigl((\MVProd \otimes \MVProd)(\Id \otimes \tau \otimes \Id)(\MVCoProd\otimes\MVCoProd)\bigr) \\ 
%&= \bigl(\Id^{\otimes 2r-2} \otimes (\MVProd\otimes \MVProd)(\Id \otimes \tau \otimes \Id)\bigr) \MVCoProd^{\otimes r+1}, \\
%\sigma_r \bigl(\Id^{\otimes 2r-2} \otimes (\MVProd\otimes \MVProd)(\Id \otimes \tau \otimes \Id)\bigr) &= (\Id^{\otimes r-1}\otimes\MVProd\otimes\Id^{\otimes r-1}\otimes \MVProd)\sigma_{r+1}, \\
% \MVMorF_{h_1,\dotsc,h_r}(\Id^{\otimes r-1} \otimes \MVProd) &= \sum_{\substack{i, j\ge 0 \\ i + j = h_r}} (\Id^{\otimes r-1}\otimes \MVProd)F_{h_1, \dotsc, h_{r-1}, i, j}, 
%\end{aligned} \]
%in this order, show that
%\[ \begin{aligned}
%(*) & = \begin{multlined}[t] \smash{\sum_{\substack{i,j\ge 0 \\ i + j = h_r}}}\vphantom{\sum}\MVProd(\MVMorF\otimes \Id)(\MVProd \otimes \Id)(\Id \otimes \tau)\bigl(\bigl[(\MVProd^{(r+1)}\otimes\MVProd^{(r+1)}) (F_{h_1,\dotsc,h_{r-1},i,j} \\ \otimes \Id^{\otimes r+1})
%\sigma_{r+1}\MVCoProd^{\otimes r+1}\bigr]\otimes \Id \bigr) (\MVMorF_1\otimes \dotsb\otimes \MVMorF_{r+1}\otimes \Id)\MVCoProd^{(r+1)} \end{multlined} \\
%& = \sum_{\substack{i,j \ge 0 \\ i+j= h_r}} \MVMorF\circ_{h_1, \dotsc, h_{r-1}, i, j}(\MVMorF_1,\dotsc,\MVMorF_{r+1}).
%\end{aligned}\]
%This together with \eqref{Eq:Comput1} and associativity of $\Star$, which is easy to check, implies~\eqref{Eq:CompFormula}.
\begin{align*}
&\MVMorF\circ_{h_1,\dotsc,h_{r-1},0}(\MVMorF_1,\dotsc,\MVMorF_r) \\
&\quad=\begin{multlined}[t]\MVProd(\MVMorF\otimes\Id)(\MVProd\otimes\Id)(\Id\otimes\tau)\bigl(\bigl[\MVProd^{(r)\otimes 2}(F_{h_1,\dotsc,h_{r-1},0}\otimes\Id^{\otimes r})\sigma_r\MVCoProd^{\otimes r}\bigr]\otimes\Id\bigr)\\(\MVMorF_1\otimes\dotsb\otimes\MVMorF_{r}\otimes\Id)\MVCoProd^{(r+1)}\end{multlined}\\
&\quad=\begin{multlined}[t]\MVProd(\MVMorF\otimes\Id)(\MVProd\otimes\Id)(\Id\otimes\tau)\bigl(\bigl[(\MVProd^{(r-1)}\otimes\MVProd^{(r)})(F_{h_1,\dotsc,h_{r-1}}\otimes\Id^{\otimes r})(\sigma_{r-1}\otimes\Id)\\ (\MVCoProd^{\otimes r-1}\otimes\Id)\bigr]\otimes\Id\bigr)(\MVMorF_1\otimes\dotsb\otimes\MVMorF_{r}\otimes\Id)\MVCoProd^{(r+1)}\end{multlined}\\
&\quad=\begin{multlined}[t]\MVProd(\MVMorF\otimes\Id)(\MVProd\otimes\Id)(\Id\otimes\tau)(\Id\otimes\MVProd\otimes\Id)\bigl(\bigr[\MVProd^{(r-1)\otimes 2}(F_{h_1,\dotsc,h_{r-1}}\otimes\Id^{\otimes r-1})\sigma_{r-1}\\\MVCoProd^{\otimes r-1}\bigr]\otimes \Id^{\otimes 2}\bigr)(\MVMorF_1\otimes\dotsb\otimes\MVMorF_{r}\otimes\Id)\MVCoProd^{(r+1)}\end{multlined}\\
&\quad=\begin{multlined}[t]\MVProd(\MVMorF\otimes\Id)(\MVProd\otimes\Id)(\Id\otimes\MVProd)(\Id\otimes\tau\otimes\Id)\bigl(\bigr[\MVProd^{(r-1)\otimes 2}(F_{h_1,\dotsc,h_{r-1}}\otimes\Id^{\otimes r-1})\sigma_{r-1}\\\MVCoProd^{\otimes r-1}\bigr]\otimes \Id^{\otimes 2}\bigr)(\MVMorF_1\otimes\dotsb\otimes\MVMorF_{r-1}\otimes\Id\otimes\MVMorF_{r})\MVCoProd^{(r+1)}\end{multlined}\\
&\quad=\begin{multlined}[t]\MVProd\bigl(\bigl[\MVProd(\MVMorF\otimes\Id)(\MVProd\otimes\Id)(\Id\otimes\tau)\bigl(\bigl[\MVProd^{(r-1)\otimes 2}(F_{h_1,\dotsc,h_{r-1}}\otimes\Id^{\otimes r-1})\sigma_{r-1} \\ \MVCoProd^{\otimes r-1}\bigr]\otimes\Id\bigr)(\MVMorF_1\otimes\dotsb\otimes\MVMorF_{r-1}\otimes\Id)\MVCoProd^{(r)}\bigr]\otimes\MVMorF_r\bigr)\MVCoProd\end{multlined}\\
&\quad=\MVProd\bigl(\MVMorF\circ_{h_1,\dotsc,h_{r-1}}(\MVMorF_1,\dotsc,\MVMorF_{r-1})\otimes\MVMorF_r\bigr)\MVCoProd \\
&\quad= \MVMorF\circ_{h_1,\dotsc,h_{r-1}}(\MVMorF_1,\dotsc,\MVMorF_{r-1})\Star \MVMorF_r
\end{align*}
Equation~\eqref{Eq:FormV} is proven analogously.
%The last equation is similar
%\[\begin{aligned}
% &\MVMorF \circ_{h_1,\dotsc,h_{r-1},0}(\MVMorF_1,\dotsc,\MVMorF_{r}) = \\
% &\quad =\begin{multlined}[t]\MVProd (\MVMorF\otimes \Id)(\MVProd\otimes \Id)(\Id\otimes \tau)(\Id\otimes \MVProd \otimes \Id)\bigl([(\MVProd^{(r-1)}\otimes\MVProd^{(r-1)})(F_{h_1,\dotsc,h_{r-1}}\otimes\Id^{\otimes r-1}) \\ (\sigma_{r-1}\otimes \Id)(\MVCoProd^{\otimes r-1} \otimes \Id)] \otimes \Id \bigr)(\MVMorF_1\otimes \dotsb\otimes \MVMorF_r \otimes \Id)\MVCoProd^{(r+1)} \end{multlined} \\
%&\quad = \begin{multlined}[t] \MVProd(\MVMorF\otimes\Id)(\MVProd\otimes\Id)\underbrace{(\Id\otimes\tau)(\Id \otimes \MVProd \otimes \Id)(\Id^{\otimes 2}\otimes \tau)}_{(\Id \otimes \MVProd)(\Id \otimes\tau \otimes\Id)}\bigl([(\MVProd^{(r-1)}\otimes\MVProd^{(r-1)})  (F_{h_1,\dotsc,h_{r-1}}\\ \otimes\Id^{\otimes r-1}) (\sigma_{r-1}\otimes \Id)(\MVCoProd^{\otimes r-1} \otimes \Id)] \otimes \Id \bigr)(\MVMorF_1\otimes \dotsb\otimes \MVMorF_{r-1} \otimes \Id \otimes \MVMorF_r)\MVCoProd^{(r+1)}\end{multlined} \\
%& \quad = \MVProd(\MVMorF\circ_{h_1,\dotsc,h_{r-1}}(\MVMorF_1,\dotsc,\MVMorF_{r-1}) \otimes \MVMorF_r)\MVCoProd
%\end{aligned}\]
\end{proof}
\end{document}
%\clearpage
%\section{Iterated compatibility condition}
%
%\clearpage
%HERE BULLSHIT STARTS
%\clearpage 
%
%
%\begin{Example}[$\IBLInfty$-algebras as filtered $\MV$-algebras]\label{Ex:IBLasMV}
%\begin{ExampleList}
%\item We first deal with the category of strict (non-filtered) $\IBLInfty$-algebras. We consider the algebra of power series $R=\K[[\hbar]]$ with the $(\hbar)$-adic filtration and the symmetric bialgebra $(\Sym(U),\mu,\delta,\eta,\varepsilon)$ with the shuffle coproduct and with the filtration by weights $\Filtr^\lambda_{\mathrm{w}} \Sym(U) = \bigoplus_{k\ge\lambda}\Sym_k(U)$. We have $\hat{\Sym}^{\mathrm{w}}(U) = \prod_{k=0}^\infty \Sym_k(U)$, and the limit conilpotency condition \eqref{Eq:LimConilp} for $\bar{\delta}$ is easy to check.
%
%The $\BV$-operator corresponding to an $\IBLInfty$-algebra is given by 
%\begin{equation}\label{Eq:StrictBV}
%\BVOp = \BVOp_{1} + \BVOp_{2}\hbar + \BVOp_{3}\hbar^2 + \dotsb\quad\text{with}\quad \BVOp_i = \sum_{\substack{k+g=i \\ k\ge 1, g \ge 0}} \underbrace{\sum_{l=1}^\infty\hat{\OPQ}_{klg}}_{\displaystyle=: Q_{kg}},
%\end{equation}
%where $\OPQ_{klg}:\hat{\Sym}^{\mathrm{w}}(U)\rightarrow\hat{\Sym}^{\mathrm{w}}(U)$ are trivial extensions of $\OPQ_{klg}: \Sym_k(U) \rightarrow \Sym_l(U)$ (see \ref{}). Because $\Norm{\OPQ_{klg}}_{\mathrm{w}} \ge l - k$, the operator $Q_{kg}: \hat{\Sym}^{\mathrm{w}}(U)\rightarrow\hat{\Sym}^{\mathrm{w}}(U)$ is well-defined, and clearly $\Norm{Q_{kg}}_{\mathrm{w}} \ge 1 - k$ and $\Norm{\BVOp_i}_{\mathrm{w}}\ge 1-i$. Since $\Norm{\hbar^{i-1}} = i-1$, we see that $\Norm{\BVOp}\ge 0$.\ToDo[noline]{Shall $\Norm{\BVOp} \ge 0$ (does not necessarily mean that it preserves the filtration but is implied by!!!) be included in the definition? Maybe preservation of filtration shall be added.} We also have $\BVOp(1)=0$ and hence $\Norm{\BVOp(1)} = \infty$ because there are no operations with no input.
%
%A strict $\IBLInfty$-morphism can be written as
%\begin{equation}\label{Eq:StrictMor}
%\MVMorF = \MVMorF_1 + \MVMorF_2 \hbar + \MVMorF_3 \hbar^2 + \dotsb \quad\text{with}\quad \MVMorF_i = \sum_{\substack{k+g = i \\ k\ge 1, g \ge 0}}\underbrace{\sum_{l=1}^\infty f_{klg}}_{\displaystyle =: F_{kg}}. 
%\end{equation}
%Similarly as for $\BVOp$, we argue that $\Norm{\MVMorF} \ge 0$ and $\Norm{\MVMorF(1)} = \infty$.
%
%\item We now deal with the category of (filtered) weak $\IBLInfty$-algebras. Let $U$ be a filtered graded vector space; the filtration is denoted by $\Filtr$. We assume that~$\Filtr$ is bounded from above in degree $\ge 1$; i.e., we require that
%\begin{equation}\label{Eq:CondFiltr}
%\Filtr^1 U = U.
%\end{equation}
%The reason for this assumption is discussed in Remark~\ref{} below. We consider the induced filtration $\Filtr$ on $V\coloneqq \Sym(U)$ and take the standard bialgebra structure as in (a); it clearly preserves the induced filtration. Condition~\eqref{Eq:CondFiltr} implies that
%\[ \Norm{v}\ge\Norm{v}_{\mathrm{w}}\quad\text{for all }v\in\Sym(U). \]
%In particular, the limit conilpotency condition for $\delta$ is satisfied. Let $R=\K((\hbar))$ be the field of Laurent series with the filtration 
%\[ \Filtr^\lambda R = \Span_{\K}\{\hbar^i \mid i \ge\lambda\}.\]
%Recall that $\K((\hbar))$ is the localization of $\K[[\hbar]]$ at the powers of $\hbar$, i.e., the space of formal power series $\sum_{i=-\infty}^\infty a_i \hbar^{i}$ with $a_i\in \K$ and $a_i = 0$ for all but finitely many $i<0$. Also, $\K((\hbar))$ is complete with respect to $\Filtr$.
%
%In the weak setting, we allow $k=0$ or $l=0$, and we have
%\begin{align*}
%\BVOp &= \hbar^{-1}\BVOp_0 + \BVOp_1 + \BVOp_2 \hbar + \BVOp_3 \hbar^2 + \dotsb, \\
%\MVMorF  &= \hbar^{-1}\MVMorF_0 + \MVMorF_1 + \MVMorF_2 \hbar + \MVMorF_3 \hbar^2 + \dotsb.
%\end{align*}
%Note that the $\hbar^{-1}$-term consists of $(0,l,0)$ for $l\in\N_0$. We assume that the maps $\OPQ_{klg}$, $f_{klg}: \hat{\Sym}_k U \rightarrow \hat{\Sym}_l U$ satisfy 
%\begin{equation}\label{Eq:MapFiltrCond}
%\Norm{f_{klg}}, \Norm{\OPQ_{klg}} \ge 1-g-k\quad\text{for all }k, l, g\ge 0.
%\end{equation}
%Because $\Norm{\OPQ_{klg}(v)}\ge \Norm{\OPQ_{klg}(v)}_{\mathrm{w}} = (l-k) + \Norm{v}_{\mathrm{w}}$ for all $v\in \Sym(U)$, the operations $Q_{kg} = \sum_{l=0}^\infty \OPQ_{klg}: \hat{\Sym}(U) \rightarrow \hat{\Sym}(U)$ are well defined and satisfy $\Norm{Q_{kg}}\ge 1-g-k$. Therefore, it holds $\Norm{\BVOp}\ge 0$. Similar reasoning shows that $\MVMorF$ is well-defined and satisfies $\Norm{\MVMorF}\ge 0$. Next, we compute
%\[ \MVMorF(1) = \Bigl(\sum_{l=0}^\infty f_{0l0}(1) \Bigr)\hbar^{-1} +  \Bigl(\sum_{l=0}^\infty f_{0l1}(1)\Bigr) + \hbar\Bigl(\sum_{l=0}^\infty f_{0l2}(1)\Bigr) + \dotsb. \]
%Because $\Norm{\hbar}=1$ and $\Norm{\MVMorF_{0lg}(1)}\ge\Norm{\MVMorF_{0lg}(1)}_{\mathrm{w}} = l$, the only terms which can spoil $\Norm{\MVMorF(1)}>0$ are those for $(k,l,g)=(0,0,0)$, $(0,1,0)$, $(0,0,1)$. We therefore have to assume 
%\begin{equation}\label{Eq:MapFiltrCondII}
%\Norm{f_{000}(1)}>1 \quad \text{and}\quad \Norm{f_{010}(1)}>1.
%\end{equation}
%\Correct[noline]{Why is there no condition on $(1,0,0)$. This can also bubble off in the relation for morphisms! It is perhaps diminished by $l+1$ of the precomposed operation. However, what is the reason to impose degree conditions on $\OPQ_{klg}$?}
%The condition $\Norm{f_{001}(1)}>0$ is satisfied automatically because of \eqref{Eq:CondFiltr}. Notice that \eqref{Eq:MapFiltrCondII} concerns only ``bubbles'' in the surface calculus. All in all, we see that $\IBLInfty$-algebras with operations with no inputs or outputs together with conditions \eqref{Eq:CondFiltr}, \eqref{Eq:MapFiltrCond}, \eqref{Eq:MapFiltrCondII} fit in the formalism of $\MV$-algebras.
%
%In the formalism of \cite{Cieliebak2015}, they use the filtration $\Filtr^\lambda \K((\hbar)) = \Span_{\K}\{\hbar^i \mid i \ge \frac{\lambda}{2\gamma}\}$ so that $\Norm{\hbar} = 2\gamma$. Then \eqref{Eq:MapFiltrCond} has to be replaced with $\Norm{f_{klg}}\ge 2\gamma(1-g-k)$. Moreover, one assumes that $\Filtr$ comes from another filtration $\Filtr'$ of $U$ such that ${\Filtr'}^\lambda = \Filtr^{\lambda + \gamma}$ for all $\lambda\in \R$, so that $\Norm{\cdot} = \Norm{\cdot}' + \gamma$. It follows that $\Norm{f_{klg}} = \Norm{f_{klg}}' + \gamma(l-k)$. The condition \eqref{Eq:MapFiltrCond} is then implied by $\Norm{\MVMorF_{klg}}' \ge \gamma (2-2g-k-l) = \gamma \chi_{klg}$. Next, \eqref{Eq:CondFiltr} reduces to ${\Filtr'}^{1-\gamma} U = U$. This is especially nice if $\gamma = 1$ because it implies $\Norm{\cdot}'\ge 0$ and morally corresponds to the degree shift $U = U'[1]$ (recall that $\Ext C = \Sym(C[1])$). Let us note that a discussion on shifting the filtration degree appears already in \cite[Remark~8.2]{Cieliebak2015}.
%
%Finally, we remark that if we work over $\K$, then \eqref{Eq:MapFiltrCondII} implies  $f_{001} = f_{000} = 0$ since $\Sym_0 U = \K$ is filtered by the trivial filtration (c.f., (ii) of Remark~\ref{Rem:FilteredMV}). In applications in symplectic geometry, however, the Novikov ring with an internal filtration is used, and hence these operations might be non-trivial.
%\qedhere
%\end{ExampleList}
%\end{Example}
%
%Part (b) of Example~\ref{Ex:IBLasMV} perfectly applies to the dual cyclic bar complex, reduced or non-reduced, with the filtration by weights. Therefore, we have established the following.
%
%\begin{Corollary}
%The category of weak $\IBLInfty$-algebras with non-negative filtrations is a subcategory of the category of filtered $\MV$-algebras.
%\end{Corollary}
%
%In \cite{Cieliebak2015}, they use the combined filtration $\Filtr_\cup = \Filtr\cup\Filtr_{\mathrm{w}}$ on $\Sym(U)$. Due to $\Norm{\cdot}_\cup \ge \Norm{\cdot}_{\mathrm{w}}$, it would be a natural candidate for (b) of Example~\ref{Ex:IBLasMV}. However, the following remark shows that it is, in fact, not a good candidate.
%
%
%$\chi_{klg} = 2 - 2g - k -l \le 0$ and the unstable surfaces are those with 
%It is not necessary to impose filtration degree condition on $\BVOp$ because we do not have to exponentiate it. However, we need $\OPQ_{klg}$ at least continuous 
%
%
%(0,0,0), (1,0,0), (0,1,0), (2,0,0), (0,2,0), (1,1,0), (0,0,1)
%
%\Add[inline,caption={Filtrations}]{
%1) Are every two $\IBLInfty$-algebras for which there is a morphism isomorphic as weak $\IBLInfty$-algebras? Explain what is happening.
%2) Explain the issue with other unstable surfaces. It is, in fact, not necessary to impose strict filtration conditions on all unstable surfaces. Only spheres can bubble. How is it with $(1,0,0)$?
%3) Add links to the computational proofs of the relations for $\Star$ etc.
%4) Do we need to impose filtration degree on $\BVOp$? (no need to exponentiate)}
%
%Let us denote 
%
%\begin{equation}\label{Eq:FiltrCond}\tag{FC}
%\reallywidehat{\reallywidehat{\tilde{\Sym}} \otimes \reallywidehat{\tilde{\Sym}}}\simeq \reallywidehat{\tilde{\Sym}\tilde{\otimes}\tilde{\Sym}}
%\end{equation}
%
%\[ \Filtr^\lambda (\reallywidehat{\tilde{\Sym}} \otimes \reallywidehat{\tilde{\Sym}})^d = \bigoplus_{\substack{d_1 + d_2 = d\\\lambda_1+\lambda_2=\lambda}} \Filtr^{\lambda_1} \hat{\tilde{\Sym}}^{d_1}\otimes\Filtr^{\lambda_2}\hat{\tilde{\Sym}}^{d_2}\]
%
%\[ \Filtr^\lambda(\tilde{\Sym}\tilde{\otimes}\tilde{\Sym})^d = \]
%
%\[ 1+1, a+b, a\otimes b, a{\otimes}b, {a}\otimes{b}, a\COtimes b \]
%
%Let us argue whyy 
%
%Consider $U = \Ten W$ with the weight-filtration, and let 
%\[ \sum_{k=1}^\infty (w\smallotimes \dotsb\smallotimes w) \otimes (\underbrace{1\cdot\dotsb\cdot 1}_{k\text{-times}}) \in \reallywidehat{\Sym(U) \otimes \Sym(U)} \]
%
%
%Clearly, $w\smallotimes\dotsb\smallotimes w\in\Filtr^k\Sym_1(U)\subset \Filtr_{\mathrm{cup}}^k \Sym(U)$ and $1\cdot\dotsb\cdot 1\in \Filtr^0 \Sym_k(U)\subset \Filtr_{\mathrm{cup}}^k \Sym(U)$
%
%
%$(\lambda,k)$ $\Filtr^\lambda \Sym_k(U)$. 
%
%
%\section{Bullshit}
%We will compare the situation to the case of $\AInfty$-algebras from~\cite{FOOI}.
%
%
%filtered commutative 
%
%\begin{Remark}
%In the correct quasi-free resolution, also $\mu$ should be resolved.
%\end{Remark}
%
%
%
%
%
%
%In this Appendix, we discuss some algebraic details and generalizations. In particular, we are 
%
%
%
%We will use the framework of $\MV$-algebras from \cite{Markl2015}. 
%
%
%Here str. reg. means strict regular $\IBLInfty$-algebra 
%Notice that we can define  weak $\IBLInfty$-algebras without filtrations, but then we can consider just strict morphism which does not make much sense. We describe what we want to say about curved $\IBLInfty$- the same as about curved $\AInfty$-. 
%
%We also want to discuss the unstable curves.
%
%We also want to say that $\IBL$ is 
%
%quasi-free resolution of a Koszul properad.
%
%There might be recent work of B.~Vallete dealing with curved properads. He also promises formulas for the homotopy transfer using leveled graphs. 
%
%
%We can take again infinite sums 
%
%\begin{Lemma}[Equivalent characterization using series]
%
%\end{Lemma}
%
%
%
%
%define topology of infinite sums.
%
%The natural structure on the completion is the filtration, hence, the completion is a functor on the category of graded vector spaces filtered by decreasing filtrations.
%
%It is not a topological vector space because the multiplication $+: C\otimes C \rightarrow C$ is not continuous.\footnote{Take $0\neq v_1\in V$ and $v_i\in V^{\otimes i}$ then $\|v_1 + v_i \| = 1$ but $\|v_1 \otimes v_i\| = i+1$.}
%
%We defined the completion as the space of series with $=$ and a filtration. Here we relate it to the completion as a limit in the category (inverse limit). We will call this space the space of filtered power series
%
%
%
%Let us note that the completion functor is exact, i.e., it preserves short exact sequences.
%
%
%\begin{Lemma}
%Those guys are isomorphic as filtered vector spaces.
%\end{Lemma}
%\begin{proof}
%Standardly, we would have
%\[ \Norm{x_i} \ge \min \{\Norm{x_i^k} \mid k\in \N_0\} \le \Norm{x_i^l} \quad\text{for all }l\in \N_0. \]
%However, because the filtration of $\Ext C$ is weight-graded, the first inequality is in fact an equality, and hence $\Norm{x_i^l}\ge \Norm{x_i}$ for all $l\in \N_0$. This implies that $y_j$ converge, i.e., that $y_j\in \hat{\Ext}_j C$ and that $\Norm{y_j} \ge \inf\{\Norm{x_i^j} \mid \} = \lambda''_j\ge \in \R$. It remains to prove that $\lambda_j' \to \infty$. Suppose that there is a $K>0$ such that there is an infinite sequence $\lambda_{j_n}'< K $.  However, this is a constradiction because there is just finitely many $x_i^j$ below every $\lambda$-level. Let us consider $\lambda_j'$ slightly below but uniformly. Therefore, $\sum y_j \in B$. This map is injective and surjective. Clearly, it also preserves the filtration degree.
%\end{proof}
%
%Can the Fukaya's theory be expressed in the MV-framework?
%
%
%\begin{Lemma}[Resummation]
%Let $\hat{C}$ be a completed filtered vector space. Let $\sum_{i=1}^\infty x_i \in \hat{C}$, and let $r: \N \times \N \rightarrow \N$ be any bijection. Then
%\begin{equation}\label{Eq:Resummation}
%\sum_{i=1}^\infty x_i = \sum_{i=1}^\infty \sum_{j=1}^\infty x_{r(i,j)}. \end{equation}
%\end{Lemma}
%\begin{proof}
%Given $K>0$, let $i_0'\in\N$ be such that $\Norm{x_i}\ge K$ for all $i\ge i_0$. Let $i_0\in \N$ be such that $i_0\ge i_0'$ and $\{1,\dotsc, i_0'\}\subset r(1,\N)\cup\dotsb\cup r(i_0,\N)$. Now, in the sum
%\[ \Delta_{i_0}\coloneqq \sum_{i=1}^{i_0} x_i - \sum_{i=1}^{i_0} \sum_{j=1}^\infty x_{r(i,j)}, \] 
%only elements $x_i$ with $i>i_0$ remain, and hence $\|\Delta_{i_0}\|\ge K$. Therefore, both sides of~\eqref{Eq:Resummation} converge to the same element in $\hat{C}$.
%\end{proof}
%\begin{Remark}
%If $\sum_{i=1}^\infty x_i = \sum_{i=1}^\infty y_i$, does there exist $\sum_{i=1}^\infty z_i$ and resummations $r_1$, $r_2 : \N\times\N\rightarrow \N$ such that $x_i = \sum_{j=1}^\infty z_{r_1(i,j)}$ and $y_i = \sum_{j=1}^\infty z_{r_2(i,j)}$ for all $i\in \N$?
%\end{Remark}
%
%Now, let $\Ext C$ be a weight-graded vector space.
%\[ \widehat{\Ext C}^{\text{ind}}  = \bigl\{ \sum_{i=1}^\infty x_i \mid x_i \in \Filtr^{\lambda_i}\hat{\Ext}_k C\bigr \} \]
%With the topology of infinite sums is canonically 
%
%
%\begin{Proposition}
%The space of power series 
%They are not isomorphic as filtered vector spaces but they are 
%\end{Proposition}
%
%The description 
%
%
%The special filtration condition for the unstable operations of a weak ibl-infinity algebra are not needed at all.
%
%They are needed just for morphisms.
%
%The filtration is not needed to define a weak ibl-infinity algebra in terms of connecting surfaces.
%
%
%In this appendix, we discuss some details of the $\BV$-formalism of Remark~\ref{}. Deformation theory fits in the framework of weak (or curved) algebras, and it is necessary to have a complete filtration. We compare the situation for $\IBLInfty$-algebras to the situation for $\AInfty$-algebras from \cite{FOOOI}. We also use the bialgebra calculus and prove some relations from Remark~\ref{}.
%
%Let $C$ be a graded vector space equipped with a decreasing graded filtration $\Filtr^\lambda C$. The exterior algebra $\Ext C = \bigoplus_{k\ge 0}\Ext_k C$ is a weight-graded vector space. We consider the following filtrations: 
%\begin{itemize}
%\item the induced filtration $\Filtr^\lambda\Ext C \coloneqq \bigoplus_{k=0}^\infty \Filtr^\lambda \Ext_k C$,
%\item the weight filtration $\Filtr_{\text{w}}^\lambda\Ext C = \bigoplus_{k\ge\lambda}\Ext_k C$,
%\item and the combined filtration 
%\[ \Filtr^\lambda_{\text{comb}} \Ext C \coloneqq \Filtr^\lambda\Ext C \cup\Filtr^\lambda_{\text{w}}\Ext C = \Filtr_\lambda \Ext_0 C \oplus \dotsb \oplus \Filtr^\lambda\Ext_{\ceil{\lambda}-1} C \oplus \Ext_{\ceil{\lambda}} C \oplus \dotsb. \]
%\end{itemize}
%These filtrations are weight-graded in the sense that
%\[ \Filtr^\lambda\Ext C = \bigoplus_{\substack{k\ge 0 \\ d\in \Z}} (\Filtr^\lambda\Ext C)^d_k\quad\text{where}\quad(\Filtr^\lambda\Ext C)^d_k = \Filtr^\lambda\Ext C \cap(\Ext_k C)^d. \]
%We have the following completions of $\Ext C$:
%\begin{enumerate}
% \item The weight-graded completion with respect to $\Filtr_{\text{w}}^\lambda$ equals $\Ext C$.
% \item The weight-graded completions with respect to $\Filtr^\lambda$ and $\Filtr^\lambda_{\text{comb}}$ are the same and equal to 
% \[ \tilde{\Ext}C \coloneqq \bigoplus_{k\ge 0} \hat{\Ext}_k C. \]
% \item The graded completions with respect to $\Filtr^\lambda$, $\Filtr_{\text{w}}^\lambda$ and $\Filtr_{\text{comb}}^\lambda$ are pairwise different which is illustrated in the following figure:
% \missingfigure
% Here, we identify the completion with a power series   
% 
% 
% We denote the by $\hat{\Ext}C$
%\item We consider the weight completion of the weight-graded completionr
%i.e., the space
%\[  \WGComplExt C = \bigoplus_{d\in\Z} 
%\prod_{k\ge 0} (\ComplExt_k C)^d \]
%\end{enumerate}
%For all $k$ the sum has to be convergent.
%
%
%We also have which is also a weight-graded filtration. We can therefore take graded or weight-graded completions in the sense that we complete the corresponding direct summand (we forget vector space completions not to escape the category of graded vector spaces). We also have the combined filtration 
%
%which is again a weight graded filtration.
%
%
%
%The weight-graded completion with respect to $\Filtr_{\text{w}}^\lambda\Ext C$ is trivial, and the weight-graded completions with respect to $\Filtr^\lambda\Ext C$ and $\Filtr^\lambda_{\text{comb}}\Ext C$ are both equal to
%\[ \WGComplExt C \coloneqq \bigoplus_{k\ge 0} \ComplExt_k C. \]
%The graded completions are, in general, pairwise different; the situation is depicted in Figure~\ref{}.
%
%We denote by $\WGComplExt C = \bigoplus_{d\in\Z} (\WGComplExt C)^d $ 
%
%
%
%
%
%Since the weight-graded completion with respect $\Filtr_{\text{w}}$ is the identity, we have the following interesting cases:
%\begin{itemize}
%\item $\hat{\Ext}^{\text{ind}} C$: the graded completion of $\Ext C$ with respect to the induced filtration $\Filtr^\lambda\Ext C$.
%\item $\hat{\Ext}^{\text{w}} C$: the graded completion of $\Ext C$ with respect to the weight filtration $\Filtr^\lambda_{\text{w}} \Ext C$.
%\item $\hat{\Ext}^{\text{comb}} C$: the graded completion of $\Ext C$ with respect to the combined filtration 
%\[ \Filtr^\lambda_{\text{comb}} \coloneqq \Filtr^\lambda_{\text{ind}}\cup\Filtr^\lambda_{\text{w}}. \]
%\end{itemize}
%
%Let $V$ be a graded vector space with a graded decreasing filtration $\Filtr^\lambda V$. Then $\Ten V$ has the induced
%
%
% Then $\Ext V$ has the induced weight-graded filtration. Hence we can pick graded completion or weight graded completion (or weight completion but this would get us out of the category of graded vector spaces). The filtration by weights is also a graded filtration (but not weight graded). The union is also a graded filtration and there is only one way how to take its completion. But we could have taken sequences of different completions. Is it all the same in the end?
%
%
%\begin{itemize}
%\item graded completion of combined graded filtration
%\item weight graded completion and subsequently the combined graded filtration. 
%\item  
%\end{itemize}
%
%
%\begin{Example}[Product does not extend to completion for the combined filtration]
%Let $x_i \in \Filtr_{-i} C$ and $y_i\in \Filtr_{i} C$ for all $i\in\N$. Then $x_i\in \Filtr_{\text{comb}}^1\Ext C$ and $y_i\in\Filtr_{\text{comb}}^i\Ext C$, and hence $x_i \otimes y_i \in \Filtr_{\text{comb}}^{i+1}(\Ext C \otimes \Ext C)$. It follows that the sum of $x_i\otimes y_i$ converges, i.e., that $\sum_{i=1}^\infty x_i\otimes y_i \in \Ext C \COtimes \Ext C$. However, $x_i \odot y_i \in \Filtr^0 \Ext_2 C \subset \Filtr_{\text{comb}}^2 \Ext C$ for all $i\in \N$, and we can surely choose $(C,\Filtr)$ and $x_i$, $y_i$ such that we can not do better. Hence, the sum $\sum_{i=1}^\infty x_i\odot y_i$, which is the candidate for $\hat{\mu}(\sum_{i=1}^\infty x_i\otimes y_i)$, does not converge.
%
%This problem does not occur when the decreasing filtration is bounded from above.
%\end{Example}
%
%$(\tilde{\mu},\tilde{\Delta},\tilde{\Ext}C)$ which can be used to do the calculus with $f: \hat{\Ext}_k C \rightarrow \hat{\Ext}_l C$ of finite filtration degree.
%We have the completed bialgebra $(\breve{\mu},\breve{\Delta},\breve{\Ext}C)$ which 
%I can not make $\odot$ of maps because I do not know what the filtration degree is!
%
%\clearpage
%
%
%QUESTIONS:
%\begin{enumerate}
% \item We twist just strict filtered $\IBLInfty$-algebras. Why not weak?
% \item When we twist a strict algebra, we obtain weak operations as well. Are they all $0$ for the de Rham complex? They have zero inputs. In general, they do not vanish due to the Maurer-Cartan relation. This is the case, however, for the $\AInfty$. In fact, the MC equation is equivalent to vanishing of the twisted $a_0^m$. The relations are namely more complicated and contain sums!
% \item If we do $\AInfty$, then   
%\end{enumerate}
%
%Recall that 
%
%Recall the stabiliy condition for a pointed riemannian surface:it is stable if $\chi(\sigma\backslash \eta) < 0$ where 
%\[ \chi = 2 - 2g - k - l\]
%euler characteristic for closed surface with $k+l$ points removed. Then $\chi < 0$ corresponds to the special triples from \cite{Cieliebak2015} plus the boundary operator $(1,1,0)$. Indeed all of these may cause troubles in the 
%
%Why isn't it possible to twist weak filtered $\IBLInfty$-structures? Why is the whole theory only for strict?
%
%
%We consider the graded vector space
%$ \ComplExt\coloneqq \prod_{k=0}^\infty \ComplExt_k C $
%
%
%
%
%
%
%
%The formula 
%\[ e^{-\PMC} \circ \BVOp \circ e^{\PMC} \]
%
%which is a bookkeaper for the twisted operations
%\[ \]
%is an actual composition of maps. One just has to check the convergencies. Notice that $e^{\PMC}$ is an isomorphism 
%
%\[ \sum_{j\in\Z} \hbar^{j} \sum \Delta \]
%
%\clearpage
%
%Consider the components $e^\PMC_j:\ComplExt V \rightarrow \ComplExt V$ of $e^\PMC = \sum_{j\in\Z}\hbar^j e^\PMC_j$.
%For $f: \Ext_0 V \rightarrow \Ext_l V$, it holds
%\[ \hat{f}(x) = \mu(f\otimes \Id)\Delta(x) = f(1)\cdot x\quad\text{for all }x\in \Ext C, \]
%and this property extends to the completion. Therefore, we have
%\[ e^\PMC_j(x) = e^{\PMC}_j(1)\cdot x\qquad\text{for all }x\in \ComplExt V [[\hbar]]. \]
%Recall from \ref{} that $e^\PMC = \sum_{j=-\infty}^\infty\hbar^j (e^\PMC)_j$. 
%
%\[ \Norm{e^\PMC_j} \ge \inf\{\Norm{\PMC_{l_1 g_1}\odot\dotsb\odot\PMC_{l_r g_r}}\mid g_1 + \dotsb + g_r - r = j\}\] 
% 
%
%Notice that $e^\PMC \circ \Delta^\PMC = \Delta \circ e^\PMC$. The image is somewhere else!!
%
%
%Let us consider $\AInfty$. If $\PMC=\PMC_{10}$, then
%\[ e^\PMC = \sum_{j=0}^\infty \frac{1}{\hbar^j j!} \PMC_{10}^{\odot j} = e^{\frac{1}{\hbar}\PMC_{10}} \] 
%Here $\Norm{e^{\PMC}_{-j}}= \Norm{\PMC_{10}^{\odot j}} \ge j \gamma$. The multiplication is well-defined on $\hbar^{-1}$ where it is an isomorphism. 
%
%There are always negative powers even if genus is $0$. But if genus is $0$ there are only negative powers. If genus is non-zero there are both powers and in arbitrary $j$ because it can be multiplied.
%
%It is never sort of an isomorphism!!
%
%Let us compare it to the situation with $\AInfty$ from FOOO. There one has $m\in V$ and the symmetric exponential 
%\[ \exp(m) = \dotsb . \]
%And the multiplication is an isomorphism of the completion $\ComplExt V$. Therefore, it is a weak isomorphism of strict filtered $\AInfty$-algebras in the sense that it induces a isomorphism (algebraic isomorphism which is continuous and his inverse too) of the completed coalgebras.
%
%
%
%The consider 
%\clearpage
%
%
%$\OPQ_{210}\circ_1 \PMC_{10}$ vanishes if the $\AInfty$-algebra is a cyclic $\DGA$ but not in general. Might this be non-zero even if the manifold is formal?
%
%
%\clearpage
%
%
%
%
%This appendix is a supplement to Section \ref{Section:TheoryA}. We show here that our Definitions \ref{Def:IBLInfty}, \ref{Def:MaurerCartan} and \ref{Def:TwistedOperations} are equivalent to the corresponding definitions in \cite{Cieliebak2015}.  
%
%\begin{Definition}[Extensions]\label{Def:Extensions}
%Given a linear map $f: E_k C \rightarrow E_l C$, $k$, $l\ge 0$, we define the (``non-trivial'') \emph{extension} $\hat{f}: EC \rightarrow EC$ by
%\[ \hat{f}\coloneqq \mu\circ(f\otimes \Id)\circ \Delta. \]
%
%More generally, given $r\ge 1$ and linear maps $f_i : E_{k_i} C \rightarrow E_{l_i} C$, $k_i$, $l_i \ge 0$ for $i=1$,~$\dotsc$, $r$, we define their \emph{symmetric product} $f_1 \odot \dotsb \odot f_r: EC \rightarrow EC$ by
%\[ f_1 \odot \dotsb \odot f_r \coloneqq \mu^{(r)}(f_1 \otimes \dotsb \otimes f_r)\Delta^{(r)}, \]
%\end{Definition}
%
%
%
%\begin{Remark}
%It is easy to check that the binary operation $\odot$ is associative and that it holds $f_1 \odot \dotsb \odot f_r = (\dots(f_1 \odot \dotsb ) \odot f_{r-1}) \odot f_r$. It is also easy to see that $\odot$ is graded commutative: $f_1 \odot f_2 = (-1)^{\Abs{f_1}\Abs{f_2}} f_2 \odot f_1$. It is also easy to check that $f\circ_{h_{\sigma^{-1}_1}\dots h_{\sigma^{-1}_r}}(f_{\sigma^{-1}_1},\dotsc, f_{\sigma^{-1}_r}) = \varepsilon(\sigma,f) f\circ_{h_1,\dotsc,h_r}(f_1,\dotsc,f_r)$
%\end{Remark}
%
%Consider the filtration on $C$, the induced filtration on $E_k C$, the graded completion $\hat{E}_k C$ and the induced filtration on $\hat{E}_k C$ from Definition \ref{Def:Completions}. The space $EC$ inherits the direct sum filtration $\mathcal{F}_\lambda EC = \bigoplus_{k \ge 0} \mathcal{F}_\lambda E_k C$, which makes it a filtered weight-graded vector space; i.e., we have the filtration degree $\lambda$, the weight $k$ and the degree $d$.
%
%We also consider the filtration $\mathcal{F}'_\lambda EC \coloneqq \bigoplus_{i\ge \lambda} E_i C$ induced from the grading by $k$ and the filtration combined with weights $\mathcal{F}''_\lambda EC \coloneqq \mathcal{F}_\lambda EC \cup \mathcal{F}'_\lambda EC$ introduced in~\cite{Cieliebak2015}.
%
%\begin{Definition}[Two completions of $EC$]
%We denote by $\tilde{E}C$ the weight-graded completion of $EC$, i.e.,
%\[ \tilde{E}C = \bigoplus_{\substack{k\ge 0 \\ d\in \Z}} \widehat{(E_k C)^d} = \bigoplus_{k\ge 0} \hat{E}_k C. \]
%We denote by $\hat{E}C$ the graded completion of $EC$ with respect to the filtration combined with weights (see above); it is generated by the series
%\[  \sum_{i=0}^\infty c_i \quad \text{with } c_i\in \mathcal{F}_{\lambda_i}(E_{k_i} C)^d\text{ such that }\max(\lambda_i,k_i) \to \infty. \] 
%\end{Definition}
%
%\begin{Remark} \label{Remark:TwoCompletions}
%On $\tilde{E}C$, we can also consider the filtration combined with weights. The canonical map $EC\rightarrow \tilde{E}C$ preserves these filtrations and induces an isomorphism of the graded modules as follows. Denoting $E_k\coloneqq E_k C$, we have for every $\lambda\in \R$
%\[ \begin{aligned}
%&\tilde{E}C / \mathcal{F}''_{\lambda+\varepsilon}\tilde{E}C  =\\
%&\quad = \bigl(\hat{E}_0\bigr/\mathcal{F}_{\lambda+\varepsilon}\hat{E}_0) \oplus \dotsb \oplus \bigl(\hat{E}_{\ceil{\lambda}}/\mathcal{F}_{\lambda+\varepsilon} \hat{E}_{\ceil{\lambda}} \bigr) \oplus \bigl( \hat{E}_{\ceil{\lambda+\varepsilon}} / \mathcal{F}_{\lambda+\varepsilon} \hat{E}_{\ceil{\lambda+\varepsilon}}\bigr)  \\
% &\quad \simeq \begin{multlined}[t]\bigl(E_0\bigr/\mathcal{F}_{\lambda+\varepsilon}E_0) \oplus \dotsb \oplus \bigl(E_{\ceil{\lambda}}/\mathcal{F}_{\lambda} E_{\ceil{\lambda}} \bigr) \oplus \bigl( E_{\ceil{\lambda+\varepsilon}} / \mathcal{F}_{\lambda+\varepsilon} E_{\ceil{\lambda+\varepsilon}}\bigr) \\ \hfil = EC / \mathcal{F}''_{\lambda}EC. \end{multlined} 
%\end{aligned}\]
%From this it follows that $EC\rightarrow \tilde{E}C$ induces an isomorphism of the graded completions; i.e., a completion of $\tilde{E}C$ recovers $\hat{E}C$.
%\end{Remark}
%
%Given filtered vector spaces $C_1$ and $C_2$, the filtration of $C_1 \otimes C_2$ is defined by 
%\[\mathcal{F}_\lambda(C_1\otimes C_2) = \sum_{\lambda_1 + \lambda_2 = \lambda} \mathcal{F}_{\lambda_1}C_1 \otimes \mathcal{F}_{\lambda_2}C_2. \]
%We denote by $C_1 \COtimes C_2$ the completion. 
%It is easy see that
%\[ \begin{aligned}
%\Delta(\mathcal{F}_\lambda E_k C) & \subset \bigoplus_{k_1 + k_2 = k} \mathcal{F}_\lambda(E_{k_1}C \otimes E_{k_2} C) \\
%E_k C &\supset \mu\bigl(\bigoplus_{k_1 + k_2 = k} \mathcal{F}_\lambda(E_{k_1}C \otimes \mathcal{F}_{\lambda} E_{k_2} C)\bigr)
%\end{aligned} \]
%for every $\lambda\in \Z$, $k\in \N_0$. Moreover, $\Delta$ and $\mu$ preserve $k$ and $d$. Therefore, they extend to $\mu: EC \COtimes EC  \rightarrow \tilde{E}C$ and $\Delta : \tilde{E}C \rightarrow EC \COtimes EC$, where the weight-graded completions are taken. We get the following lemma.
%
%\begin{Lemma}
%Definitions \ref{Def:CircS} and \ref{Def:Extensions} extend to linear maps $\hat{E}_k C \rightarrow \hat{E}_l C$ of finite filtration degrees if we replace $E_k C$ by $\hat{E}_k C$, $EC$ by $\tilde{E}C$ (or $\hat{E}C$) and $\otimes$ by $\COtimes$. Lemma~\ref{Lem:Compositions} still holds.
%\end{Lemma}
%\begin{proof}
%The advantage of $f_1: \hat{E}_{k_1} C \rightarrow \hat{E}_{l_1} C$, $f_2: \hat{E}_{k_2}C\rightarrow \hat{E}_{l_2}C$ having finite filtration degree is that $f_1 \otimes f_2: \hat{E}_{k_1}C\otimes\hat{E}_{k_2}C \rightarrow \hat{E}_{l_1} C \otimes\hat{E}_{l_2}C$ extends to $f_1 \otimes f_2: \hat{E}_{k_1}C\COtimes\hat{E}_{k_2}C \rightarrow \hat{E}_{l_1} C\COtimes\hat{E}_{l_2}C$, or to $f_1 \otimes f_2: \tilde{E}C\COtimes\tilde{E}C \rightarrow \tilde{E}C\COtimes\tilde{E}C$ for their trivial extensions.
%
%Given two filtered vector spaces $C_1$ and $C_2$, it is a known fact that the canonical map $C_1 \otimes C_1 \rightarrow \hat{C}_1 \otimes \hat{C}_1$ induces an isomorphism $C_1 \COtimes C_2 \simeq \hat{C}_1 \COtimes \hat{C}_2$. Therefore, we have $\mu: \tilde{E}C\COtimes\tilde{E}C \rightarrow \tilde{E}C$ and $\Delta: \tilde{E}C \rightarrow \tilde{E}C\COtimes\tilde{E}C$, and we can now, for example, compose $\mu \circ (f_1\otimes f_2) : \tilde{E}C \rightarrow \tilde{E}C$ or $\mu\circ \Delta$. It is clear that Definitions \ref{Def:CircS} and \ref{Def:Extensions} can be reformulated and the proof of Lemma~\ref{Lem:Compositions} can be redone in this setting.
%
%Having a map $f: \tilde{E}C \rightarrow \tilde{E}C$ of finite filtration degree with respect to the filtration combined with weights --- e.g., a (trivial) extension of an $f: E_k C \rightarrow E_l C$ with finite filtration degree --- we can extend it to the completion, and hence to $f: \hat{E}C\rightarrow \hat{E}C$ by Remark \ref{Remark:TwoCompletions}.
%\end{proof}
%
%\begin{Definition}[Definitions in $\mathrm{BV}$-formalism]\label{Def:BVForm}
%Let $(C, (\OPQ_{klg})_{k,l\ge 1; g\ge 0})$ be a strict filtered $\IBLInfty$-algebra of bidegree $(d,\gamma)$, $\PMC=(\PMC_{lg})_{l\ge 1; g\ge 0}$ a Maurer-Cartan element and $(\OPQ_{klg}^\PMC)_{k,l\ge 1; g\ge 0}$ the twisted operations. Let $\hbar$ be a formal symbol with $\Abs{\hbar} = 2d$, and let $\hat{E}C[[\hbar]]$ be the space of Laurent series in $\hbar$ with coefficients in~$\hat{E}C$. We define the operators $\hat{\OPQ}$, $\hat{\OPQ}^\PMC$, $\PMC$, $e^{\PMC} : \hat{E}C[[\hbar]] \rightarrow \frac{1}{\hbar}\hat{E}C[[\hbar]]$ by
%\[\begin{aligned}
% \hat{\OPQ}&\coloneqq \sum_{\substack{k,l\ge 1 \\ g\ge 0}} \hat{\OPQ}_{klg} \hbar^{k+g-1}, \quad \hat{\OPQ}^\PMC = \sum_{\substack{k,l\ge 1 \\ g\ge 0}} \hat{\OPQ}_{klg}^{\PMC} \hbar^{k+g-1}, \quad \PMC\coloneqq \sum_{\substack{l\ge 1 \\ g \ge 0}} \hat{\PMC}_{lg} \hbar^{g-1},  \\
%e^{\PMC}&\coloneqq \sum_{r=0}^\infty \frac{1}{r!} \sum_{\substack{l_1, \dotsc, l_r\ge 1 \\ g_1,\dotsc, g_r\ge 0}}  (\PMC_{l_1 g_1}\odot \dotsb\odot \PMC_{l_r g_r}) \hbar^{g-r}.
%\end{aligned}\] 
%\end{Definition}
%
%\noindent Using conditions 1) \& 2) of Definitions \ref{Def:IBLInfty} and \ref{Def:MaurerCartan} it is straightforward to check that $\hat{\OPQ}$, $\hat{\OPQ}^\PMC$,  $\PMC$ and $e^\PMC$ are well-defined and homogenous of degree $-1$ (see \cite[Section 8]{Cieliebak2015}).
%
%    , and let $(\OPQ^\PMC_{klg})_{k,l\ge 1; g\ge 0}$ be the twisted operations. 
%
% Definitions \ref{Def:IBLInfty}, \ref{Def:MaurerCartan} and \ref{Def:TwistedOperations}
%
%\begin{Proposition}
%The $\IBLInfty$-relations \eqref{Eq:IBLInfRel} are equivalent to $\hat{p}\circ \hat{p} = 0$, the Maurer-Cartan equation \eqref{Eq:MaurerCartanEquation} is equivalent to $\hat{p} \circ e^n = 0$, and the twisted operations \eqref{Eq:TwistedOperations} satisfy $\hat{p}^n = e^{-n}\circ \hat{p} \circ e^{n}$
%\end{Proposition}
%
%
%
%Given $c\in \hat{E}C$, it holds $\Norm{\hat{\OPQ}_{klg}(c)} \to \infty$ as $l\to \infty$ because $\OPQ_{klg}$ increases the weight by $l-k$. Therefore, $Q_{kg}\coloneqq \sum_{l\ge 1} \hat{\OPQ}_{klg}: \hat{E}C \rightarrow \hat{E}C$ is well-defined, and we can write 
%\[ \hat{\OPQ} = \sum_{p=0}^\infty \Bigl(\sum_{\substack{k\ge 1, g\ge 0 \\ k + g - 1 = p}} Q_{kg}\Bigr)\hbar^p = \Delta_1 + \Delta_2 \hbar + \dotsb . \]
%This obviously defines a map $\hat{E}C[[\hbar]] \rightarrow \hat{E}C[[\hbar]]$.  
%
%
%
%
%$R[[h]]$ formal power series, $R((h))$ formal Laurent series
%
%consider $\sum_{i = 0}^\infty
% c_i \otimes d_i$ with $0\neq c_i \in C_{-i}$ and $0\neq d_i \in C_i$ Write $C=\bigoplus_{i\in \Z} C_i$ Then $c_i \in E_1 C = \mathcal{F}'_1 C \subset \mathcal{F}''_1 EC$, $d_i \in \mathcal{F}_i C \subset \mathcal{F}''_i C$, and hence $c_i \otimes d_i \in \mathcal{F}''_{i+1}(EC\otimes EC)$. Hence $\sum_{i=0}^\infty c_i \otimes d_i \in \hat{E}C\COtimes\hat{E}C$. But $\mu(c_i\otimes d_i) = c_i d_i \in E_2 C$ of degree $0$, and hence the sum does not converge. 
% 
%\begin{Example}
% Let $c_i$, $d_i \in C$ be such that $\Norm{c_i} = - i$ and $\Norm{d_i}= i$ for all $i\in \N$. Then $\Norm{c_i \otimes d_i} \ge 0$.
%\end{Example} 
% 
% 
%Now, $\iota_k: E_k \rightarrow EC$ preserves the filtration and $\pi_l: E_l \rightarrow E C$ preserves the filtration for $\lambda > l$; therefore, they extend to $\iota_k: \hat{E}_k \rightarrow \hat{E}C$, $\pi_l: \hat{E}_l \rightarrow \hat{E}C$. NEED FILTRATION WITH UNBOUNDED NEGATIVE DEGREES.
%
%
%An important property of the extension $\hat{f}$ of $f: E_{k'} \rightarrow E_{l'}$ is that $\hat{f}(E_k) \subset E_{k+l'-k'}$ for $k\ge k'$ and $\hat{f}(E_{k'})=0$ for $k<k'$.
%
%\section{RL's}
%
%Let $a_i$ be such that $A=\sum_{i=0}^\infty a_i$ exists. Consider the following conditions:
%\begin{description}
%\item[RL1] $\forall \sigma: \N \rightarrow \N$: $A=\sum_i a_{\sigma(i)}$
%\item[RL2] $\forall \sigma: \N^2 \rightarrow \N$: $A=\sum_i \sum_j a_{\sigma(i,j)}$
%\item[...]
%\item[RLn] $\forall \sigma: \N^n \rightarrow \N$: $A=\sum_{i_1}\dotsb\sum_{i_n} a_{\sigma(i_1,\dotsc,i_n)}$
%\end{description}
%If $a_i$ is bsolutely convergen, then all RL'S hold by Fubini and change of variables formula. Moreover, $a_i$ is absolutely convergent iff RL1 holds by the cotraposition of Riemann resummation. Does RL1 follow from RL2?
%\end{document}
