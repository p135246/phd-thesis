%auto-ignore
\providecommand{\MainFolder}{..}
\documentclass[\MainFolder/Text.tex]{subfiles}

\begin{document}
In this chapter, we use the $\BV$-formalism, which comes originally from quantum field theory, to formulate the theory of $\dIBL$-algebras on cyclic cochains of an odd symplectic vector space and its twisting with a Maurer-Cartan element.
The ``fields'' are cyclic Hochschild chains and the $\BV$-operator comes from the algebraic string bracket and cobracket.
We hope that this point of view will help to understand relations between string topology, Chern-Simons theory, symplectic field theory and string field theory (let's say ``S(F)T'').

In Section~\ref{BV:Summary}, we summarize relevant $\IBLInfty$-theory from \cite{Cieliebak2015}.

In Section~\ref{Sec:BVAction}, we define the canonical string $\BV$-operator on the space of ``observables'' and the total action; it consists of the free part and of the interaction part which corresponds to the Maurer-Cartan element.
We argue that the quantum master equation is satisfied for the canonical and the Chern-Simons Maurer-Cartan element and that the string $\BV$-operator twisted by the action corresponds to the twisted $\dIBL$-algebra (Proposition~\ref{Prop:BVActForCanMC}).
In general, we show that the action satisfies the quantum master equation if and only if its interaction part encodes a Maurer-Cartan element and that the twistings in both formalisms agree (Proposition~\ref{Prop:BVActForAnyMCElement}).

In Section~\ref{Sec:HPL}, we sketch how to apply ideas from \cite{Doubek2018} for quantum $\LInfty$-algebras to $\IBLInfty$-algebras.
We ask whether their formulas for the effective action and the path integral which come from the Homological Perturbation Lemma (and Wick's Theorem) agree with the formulas from \cite{Cieliebak2015} with summations over ribbon graphs (Question~\ref{Q:EqForm}).
\end{document}
