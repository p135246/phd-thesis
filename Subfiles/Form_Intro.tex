%auto-ignore
\providecommand{\MainFolder}{..}
\documentclass[\MainFolder/Text.tex]{subfiles}

\begin{document}

In this chapter, we study the following question:
\begin{Question}\label{Q:Form}
Let $M$ be a connected oriented closed $n$-manifold which is formal in the sense of rational homotopy theory. Let $\MC$ be the canonical and $\PMC$ the formal pushforward Maurer-Cartan element for $\dIBL(\CycC(\HDR(M)))$. Are $\dIBL^\MC(\CycC(\HDR(M)))$ and $\dIBL^\PMC(\CycC(\HDR(M)))$ homotopy equivalent as $\IBLInfty$-algebras?
\end{Question}
We remind that Theorem~\ref{IntroThm:B} in Part~I states that for a geometrically formal manifold~$M$ with $\HDR^1(M) = 0$, one can pick a special Hodge propagator $\PrpgStd$ such that $\PMC = \MC$ at least for $n\neq 2$.\footnote{We did not prove that the homotopy class of $\dIBL^\PMC(\CycC(\HDR(M)))$ does not depend on $\PrpgStd$, but we expect so.} Question~\ref{Q:Form} asks for a generalization of Theorem~\ref{IntroThm:B}. In the upcoming sections, we will propose a strategy to answer Question~\ref{Q:Form} in the case of $\HDR^1(M) = 0$, when the expected answer is ``yes''. Our main tool is the theory of Poincar\'e duality models from~\cite{Lambrechts2007}. Nevertheless, we also study $\DGA$'s of Hodge type and their small subalgebras from~\cite{Van2019}, which, as we think, naturally fit in the picture.

In Section~\ref{SubSec:CycStr}, we consider $\DGA$'s and define the notions of an orientation and cyclic structure (Definition~\ref{Def:CycStr}), non-degeneracy and Poincar\'e duality (Definition~\ref{Def:PoincDual}), degenerate subspace and non-degenerate quotient (Definition~\ref{Def:NonDegQ}), Hodge decomposition and Hodge pair (Definition~\ref{Def:HodgeDecomp}) and Hodge and small subalgebra from~\cite{Van2019} (Definition~\ref{Def:SmallSubalg}). We prove that in some cases orientations and cyclic structures are in one-to-one correspondence (Proposition~\ref{Prop:OrAndCyc}) and that being of Hodge type is equivalent to acyclicity of the degenerate subspace (Proposition~\ref{Prop:HodgeAcyc}). We describe the small subalgebra of~\cite{Van2019} in terms of Kontsevich-Soibelman--like evaluations of rooted binary trees with internal edges labeled with the identity or the standard Hodge homotopy (Proposition~\ref{Prop:SmallDescription}). We study what happens when we take small subalgebras and non-degenerate quotients iteratively (Proposition~\ref{Prop:PropPropertiessd}). We conjecture that the small subalgebra is an image of the Sullivan minimal model (Conjecture~\ref{Conj:HodgeSullMin}), and if this is the case, we prove that the non-degenerate quotient of the small subalgebra is independent of the Hodge decomposition up to a pairing preserving isomorphism (Conjecture~\ref{Conj:UnieqSmal}). We finish with some open questions (Questions~\ref{Q:QuestOnPoinc}). We also prove several lemmas (Lemmas~\ref{Lem:Pom}, \ref{Lem:PomLemma}, \ref{Lem:AutomaticInjectivity} and~\ref{Eq:LemSmallSub}) and make several remarks (Remarks~\ref{Rem:Eq}, \ref{Rem:VolForms}, \ref{Rem:NonDegPD}, \ref{Rem:RemarkHarm} and~\ref{Rem:OnHodgeSubalg}) which might be useful for future investigations.

In Section~\ref{SubSec:PoincModel}, we define differential Poincar\'e duality algebras\footnote{The same as cyclic $\DGA$'s for unital commutative $\DGA$'s up to a degree shift and the correspondence of orientations and cyclic structures} and Poincar\'e $\DGA$'s (Definition~\ref{Def:PDGA}), i.e., $\DGA$'s whose homology is a Poincar\'e duality algebra. We consider Poincar\'e duality models (Definition~\ref{Def:PDModel}). We argue that a Poincar\'e $\DGA$ is formal if and only if it is formal as a $\DGA$ (Proposition~\ref{Prop:PoincModelOfFormal}). We check that the proofs of the existence and uniqueness of Poincar\'e duality models from \cite{Lambrechts2007} still work when quasi-isomorphisms preserving orientation are requested (Propositions~\ref{Prop:ExOfLambrStan} and~\ref{Prop:LambrechtUnique}). We discuss minimal models (Remark~\ref{Rem:Models}) and conjecture that small Poincar\'e duality models can be thought of as minimal models for $\PDGA$'s (Corollary~\ref{Cor:SmalPoinc}). We finish with some open questions (Questions~\ref{Q:QuestionsPonc}).


In Section~\ref{Section:FuncIBL}, we formulate ``functoriality'' of the canonical $\dIBL^\MC$ construction on differential Poincar\'e duality algebras up to $\IBLInfty$-homotopy (Conjectures~\ref{Conj:Functorialityo} and~\ref{Conj:ExteofDFS}). We define $\IBLInfty$-formality of a $\PDGA$ (Definition~\ref{Def:IBLFormality}). We conjecture that $\IBLInfty$-formality is implied by $\DGA$-formality (Conjecture~\ref{Con:DGAIBLForm}). We discuss another relevant notions of formality (Remark~\ref{Rem:Intfor}). For the de Rham complex of a smooth manifold, we conjecture that the $\IBLInfty$-algebra on cyclic cochains of de Rham cohomology twisted by the Chern-Simons Maurer-Cartan element is homotopy equivalent to the $\dIBL$-algebra on cyclic cochains of the non-degenerate quotient of the small subalgebra of the de Rham complex twisted by the canonical Maurer-Cartan element (Conjecture~\ref{Conj:SmallSimplCon}). Assuming that the conjectures hold, we obtain a positive answer to Question~\ref{Q:Form} when $\HDR^1(M)=0$ (Corollary~\ref{Cor:FormCorollary}).
\end{document}
