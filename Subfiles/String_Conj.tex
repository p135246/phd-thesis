%auto-ignore
\providecommand{\MainFolder}{..}
\documentclass[\MainFolder/Text.tex]{subfiles}

\begin{document} 
\section{Conjectured relation to string topology}\label{Sec:StringTopology}
\Add[inline,caption={DONE Simplify string top.
and cyc. hom. comparison}]{Add somewhere that the map from $(\CDBCyc \HDR(M),b^*)$ to $(C(\StringSpace M),\Bdd)$ with no degree shifts does the job (after removing all degree shifts from definitions).}


Given a smooth connected oriented $n$-dimensional manifold $M$, we consider the equivariant homology of the free loop space $\Loop M \coloneqq \{\gamma: \Sph{1} \rightarrow M \text{ continuous}\}$ with respect to the reparametrization action of $\Sph{1}$. It is defined as the singular homology of the Borel construction  
$$ \LoopBorel M \coloneqq \EG\Sph{1} \times_{\Sph{1}} \Loop M \coloneqq (\EG\Sph{1}\times \Loop M)/\Sph{1}, $$
where $\EG\Sph{1} = \Sph{\infty} \rightarrow \BG \Sph{1} = \CP^\infty$ is a model for the universal bundle for $\Sph{1}$, and we quotient out the diagonal action. We denote this homology by
$$ \StringH(\Loop M) \coloneqq \H(\LoopBorel M). $$
Recall that $(\EG\Sph{1}\times \Loop M)/\Sph{1}$ is the homotopically correct quotient replacing $\Loop M/\Sph{1}$, which we prefer to use because the diagonal action of $\Sph{1}$ on $\EG\Sph{1}$ is free, and hence $\EG\Sph{1}\times \Loop M \rightarrow (\EG\Sph{1}\times \Loop M)/\Sph{1}$ is a principal $\Sph{1}$-bundle (in contrast to the pathological map $\Loop M \rightarrow \Loop M /\Sph{1}$). Recall also that $\EG \Sph{1}$ is contractible.


The ``geometric versions'' of the homologies were defined in \cite{Sullivan1999} as the degree shifts
$$ \GeomH(\Loop M) \coloneqq \H(\Loop M)[n]\quad\text{and}\quad \GeomStringH(\Loop M)\coloneqq \StringH(\Loop M)[n]. $$
%If $M$ is fixed, we write just $\GeomH$ and $\GeomStringH$. However, we will use this notation only temporary because it collides with our previous notation for the homology of an $\IBLInfty$-algebra and the harmonic forms, respectively.

There is the \emph{loop product} $\LoopPr: \GeomH(\Loop M)^{\otimes 2} \rightarrow \GeomH(\Loop M)$ of degree $0$ which makes $\GeomH(\Loop M)$ into a graded commutative associative algebra. There is also the \emph{loop coproduct} $\LoopCoPr: \ConstRedGeomH(\Loop M) \rightarrow \ConstRedGeomH(\Loop M)^{\otimes 2}$ of degree $1-2n$ which is graded cocommutative and coassociative and is a derivation of $\LoopPr$. The geometric construction of~$\LoopPr$ and~$\LoopCoPr$ on transverse smooth chains in $\Loop M$ was described in~\cite{Sullivan1999} and~\cite{Basu2011}, respectively. Here, the symbol $\ConstRedGeomH(\Loop M)$ stands for the degree shifted relative homology
$$  \ConstRedGeomH(\Loop M)\coloneqq\H(\Loop M, M)[n] $$
with respect to constant loops $M \hookrightarrow \Loop M$. The geometric construction of~$\LoopCoPr$ does not work on the whole $\GeomH(\Loop M)$ because of the phenomenon of ``vanishing of small loops'' depicted in \cite[Figure 4, p.\,13]{Cieliebak2007}.
%Note that $\StringOp_2$ restricts to $\ConstRedGeomStringH^{\otimes 2}(\Loop M) \rightarrow \ConstRedGeomStringH(\Loop M)$.
\Add[caption={DONE Reduced loop product}]{Is it really true? What about the example of torus.}

The projection $\EG\Sph{1} \times \Loop M \rightarrow \LoopBorel M$ is an $\Sph{1}$-principal bundle and thus induces a Gysin sequence. This sequence written using the geometric versions reads
\begin{equation}\label{Eq:Gysin}
\begin{tikzcd}
\dots\arrow{r}& \GeomH_i \arrow{r}{\Erase} & \GeomStringH_i \arrow{r}{\cap c} & \GeomStringH_{i-2} \arrow{r}{\Mark} & \GeomH_{i-1} \arrow{r} & \dots,
\end{tikzcd}
\end{equation}
where the map $\Mark$ adds a marked point in each string in a family in all possible positions, the map $\Erase$ erases the marked point of each string in a family, $c\in \StringCoH^{2}(\Loop M)$ is the Euler class of the circle bundle and $\cap$ the cap product.


The \emph{string bracket} $\tilde{\StringOp}_2: \GeomStringH(\Loop M)^{\otimes 2}\rightarrow \GeomStringH(\Loop M)$ and the \emph{string cobracket} $\tilde{\StringCoOp}_2: \ConstRedGeomStringH(\Loop M) \rightarrow \ConstRedGeomStringH(\Loop M)^{\otimes 2}$ are defined by
$$ \tilde{\StringOp}_2 \coloneqq \Erase \circ \LoopPr \circ \Mark^{\otimes 2}\quad\text{and}\quad\tilde{\StringCoOp}_2 \coloneqq \Erase^{\otimes 2} \circ \nu \circ \Mark. $$
Here, the symbol $\ConstRedGeomStringH(\Loop M)$ stands for the degree shifted relative $\Sph{1}$-equivariant homology
$$ \ConstRedGeomStringH(\Loop M) \coloneqq \underbrace{\StringH(\EG \Sph{1} \times_{\Sph{1}} \Loop M, \EG \Sph{1} \times_{\Sph{1}} M)}_{\displaystyle \eqqcolon \ConstRedStringH(\Loop M)}[n]. $$
Because $\Abs{\Mark} = 1$ and $\Abs{\Erase} = 0$, we have for all $\xi \in \ConstRedGeomStringH(\Loop M)$ and $\xi_1$, $\xi_2 \in \GeomStringH$ the relations\Correct[caption={DONE Typo}]{$\xi$ in the exponent should be $\Abs{\xi_1}$}
\begin{equation}\label{Eq:StringOpCoOp}
\begin{aligned}
\tilde{\StringOp}_2(\xi_1,\xi_2) &= (-1)^{\Abs{\xi_1}} \Erase(\Mark(\xi_1)\LoopPr\Mark(\xi_2)), \\
\tilde{\StringCoOp}_2(\xi) & = \sum \Erase(\nu^{1}) \otimes \Erase(\nu^2),
\end{aligned}
\end{equation}
where we write $\nu(\Mark(\xi)) = \sum \nu^1 \otimes \nu^2$. The operations $\tilde{\StringOp}_2$ and $\tilde{\StringCoOp}_2$ have degrees~$2$ and $2-2n$ with respect to the grading on $\GeomStringH(\Loop M)$, respectively. In fact, we will consider $\tilde{\StringOp}_2$ and $\tilde{\StringCoOp}_2$ given by \eqref{Eq:StringOpCoOp} as operations on the even degree shift $\StringH(\Loop M)[2-n] = \GeomStringH(\Loop M)[2-2n]$, which have degrees $2(2-n)$ and $0$, respectively. The symbols $\StringOp_2$ and $\StringCoOp_2$ will denote their degree shifts to $\StringH(\Loop M)[3-n]$, which have degrees of an $\IBL$-algebra from Definition~\ref{Def:IBLInfty}.
%We state the following well-known result (see~\cite{Sullivan2002}).
%
%\begin{Proposition}[$\IBL$-structure on string topology]\label{Prop:StringLieBiAlg}
%For an oriented manifold~$M$ of dimension~$n$, the triple 
%$$ \IBL(\ConstRedStringH_*(\Loop M)[2-n]) \coloneqq (\ConstRedStringH_*(\Loop M)[2-n], \StringOp_2, \StringCoOp_2) $$ is an involutive Lie bialgebra.%\footnote{To recall, this means that either $\tilde{\StringOp}_2: C^{\otimes 2} \rightarrow C$, $\tilde{\StringCoOp}_2: C \rightarrow C^{\otimes 2}$ are graded antisymmetric and satisfy \eqref{Eq:ClassicIBL}, or $\StringOp_2: C[1]^{\otimes 2} \rightarrow C[1]$, $\StringCoOp_2: C[1] \rightarrow C[1]^{\otimes 2}$ are symmetric and satisfy \eqref{Eq:IndIBL} (plus the corresponding degree conditions).}
%%\footnote{According to Proposition~\ref{Prop:ClasModIBL}, we can either prove the relations \eqref{Eq:ClassicIBL} for $\StringOp_2: \RedGeomStringH^{\otimes 2} \rightarrow \RedGeomStringH$, $\StringCoOp_2: \RedGeomStringH \rightarrow \RedGeomStringH^{\otimes 2}$ or the relations \eqref{Eq:IndIBL} for the degree shifts $\StringOp_2 : \RedGeomStringH[1]^{\otimes 2} \rightarrow \RedGeomStringH[1]$, $\StringCoOp: \RedGeomStringH[1] \rightarrow \RedGeomStringH[1]^{\otimes 2}$.}
%\end{Proposition}

In work in progress \cite{Cieliebak2018b}, they consider the map 
$$I_{\lambda,*}: \CycH_{-\bullet- 1}(\DR(M)) \longrightarrow \StringCoH^\bullet(\Loop M; \R)$$ defined on the chain level as a cyclic version of Chen's iterated integrals $I_\lambda$. Recall that $\CycH_{-\bullet-1}(\DR) = \H_\bullet(\BCyc \DR, \Hd)$ \Correct[caption={DONE Homology exchangend with cohomology}]{DONE The homology of $\OPQ_{110}$ should be the cyclic cohomology},
where $\Hd: \B \DR = \bigoplus_{k\ge 1} \DR[1]^{\otimes k} \rightarrow \B \DR $ is the Hochschild differential of the de Rham dga $(\DR,m_1,m_2)$, see Section~\ref{Sec:Alg2}. They prove in \cite{Cieliebak2018b} that if $M$ is simply-connected, then the map~$I_{\lambda,*}$ induces an isomorphism $\RedCycH_{-\bullet-1}(\DR(M))) \simeq \RedStringCoH^\bullet(\Loop M)$, where 
$$ \RedStringCoH(\Loop M) \coloneqq \StringCoH(\EG \Sph{1} \times_{\Sph{1}} \Loop M, \EG \Sph{1} \times_{\Sph{1}} \{x_0\}) $$
is the \emph{reduced $\Sph{1}$-equivariant cohomology} with respect to a base point $x_0 \in M$ (the constant loop at $x_0$). Dualizing their map, we obtain the isomorphism 
\begin{equation}\label{Eq:StringIsom}
\RedCycCoH^{-\bullet-1}(\DR(M))\simeq \RedStringH_\bullet(\Loop M; \R).
\end{equation}

%From now on, the symbols \underline{$\HIBL$ and $\Harm$} will mean the homology of an $\IBLInfty$-algebra and the harmonic forms, respectively.

Suppose from now on that $M$ is closed. Pick a Riemannian metric and an admissible Hodge propagator $\Prpg \in \DR^{n-1}(\Bl_\Diag(M\times M))$. We will assume that $\Prpg$ is special, i.e., that it satisfes (P1)--(P5) from Section~\ref{Section:Proof1}, so that the Chern-Simons Maurer-Cartan element~$\PMC$ is strictly reduced, and hence the twisted reduced $\IBLInfty$-algebra $\dIBL^\PMC\bigl(\RedCycC(\Harm)\bigr)$ and the induced $\IBL$-algebra $\IBL(\HIBL^{\PMC,\mathrm{red}}(\CycC(\Harm)))$ are well-defined.
%\footnote{Because the $\AInfty$-algebra $\Harm(M)_\PMC$ is homologically unital and augmented, we believe that it is possible to define its reduced cyclic cohomology $\RedCycCoH^\bullet(\Harm(M)_\PMC)$ even if $\Harm(M)_\PMC$ was not strictly unital and strictly augmented. Maybe it is even possible to define the twisted $\IBL$-structure on the reduced homology if $\PMC$ is not strictly reduced.}
Recall that $\HIBL^\PMC_\bullet(\CycC(\Harm)) = \CycH_{n-3-\bullet}(\Harm_\PMC)$, where $\Harm_\PMC$ is the $\AInfty$-algebra on $\Harm$ twisted by $\PMC_{10}$. From~\cite{Cieliebak2018}, we have
\begin{equation}\label{Eq:IBLIsom}
\CycCoH(\Harm(M)_\PMC) \simeq \CycCoH(\DR(M)).
\end{equation}
Now, \eqref{Eq:StringIsom} and \eqref{Eq:IBLIsom} give the following version of the string topology conjecture.

%\begin{Remark}[Intuition for \eqref{Eq:IBLIsom} from the $\IBLInfty$-viewpoint]\label{Rem:IBLViewpoint}
%Suppose that $\DR$ is finite dimensional. Then we have the canonical $\dIBL$-structure $\dIBL(\CDBCyc \DR{} [2-n])$ equipped with the canonical Maurer-Cartan element $\MC$ given by \eqref{Eq:CanonMC}. The fact (B) in the Overview gives the $\IBLInfty$-quasi-isomorphism $\HTP: \dIBL(\CDBCyc \DR{}[2-n]) \rightarrow \dIBL(\CDBCyc \Harm{}[2-n])$ extending the dual to the inclusion $\iota_\Harm: \Harm{}\rightarrow \DR{}$, such that the formal pushforward Maurer-Cartan element~$\PMC$ becomes a genuine pushforward of $\MC$ along $\HTP$, i.e.~$\PMC = \HTP_* \MC$ (see Appendix~\ref{Sec:Appendix}). In this case, \cite[Proposition 9.6]{Cieliebak2015} asserts that we can twist $\HTP$ to obtain an $\IBLInfty$-quasi-isomorphism $\HTP^\MC: \dIBL^\MC(\CDBCyc \DR{}[2-n]) \rightarrow \dIBL^\PMC(\CDBCyc \Harm{}[2-n])$.
%
%Proposition~\ref{Prop:} implies that $\DR{}_\MC = (\DR{},m_1,m_2)$ is the de Rham algebra, and hence $\HIBL^\MC_*(\CDBCyc \DR{}[2-n]) = r(\CycCoH^*(\DR{}))[3-n]$. We also have $\HIBL^\PMC_*(\CDBCyc \Harm [2-n]) = r\bigl(\CycH_*(\Harm_\PMC) \bigr)[3-n]$, and hence \eqref{Eq:IBLIsom} is equivalent to $$ \HIBL^\PMC(\CDBCyc\Harm{}[2-n]) \simeq \HIBL^\MC(\CDBCyc \DR{}[2-n]), $$
%which is induced by $\HTP_{110}^\MC : \CDBCyc \DR{}[3-n] \rightarrow \CDBCyc \Harm{}[3-n]$.
%
%In the reality, $\HTP$ is a Fr\'echet $\IBLInfty$-quasi-isomorphism of the Fr\'echet $\dIBL$ algebras $\dIBLFr(\CDBCyc \DR{}_{\infty}[2-n])\rightarrow \dIBLFr(\CDBCyc \Harm{}[2-n]) = \dIBL(\CDBCyc \Harm{}[2-n])$ (see Remark~\ref{}), and one has to show, among other things, that the inclusion $\CDBCyc \DR{}_{\infty} \subset \CDBCyc \DR{}$ is a quasi-isomorphism for cyclic cohomology.  
%\end{Remark}

\begin{Conjecture}[String topology conjecture for simply-connected manifold]\label{Conj:StringTopology}
Let $M$ be an oriented closed manifold of dimension $n$. There is a chain map 
$$(C^{\mathrm{sing}}(\Loop_{\Sph{1}}M; \R),\Bdd)\longrightarrow (\CDBCyc\Harm(M),\OPQ_{110}^\PMC), $$
where $C^{\mathrm{sing}}$ denotes the (smooth) singular chain complex and $\Bdd$ the standard boundary operator, which, if $M$ is simply-connected, satisfies the following:
\begin{itemize}
\item It induces an isomorphism $\RedStringH(\Loop M; \R)[2-n] \simeq \HIBL^{\PMC,\mathrm{red}}\bigl(\CycC(\Harm(M))\bigr)$.
\item It intertwines $\StringOp_2$ on $\StringH(\Loop M; \R)$ and $\OPQ_{210}$.
\item The pullback of $\OPQ_{120}^\PMC$ to $\RedStringH(\Loop M; \R)$ is compatible with $\StringCoOp_2$ on $\ConstRedStringH(\Loop M; \R)$ under the morphism induced by the inclusion $(\Loop M, x_0) \rightarrow (\Loop M,M)$.
\end{itemize}
%This holds for any choice of the Hodge propagator $\Prpg$.
\end{Conjecture}

\begin{Remark}[On string topology conjecture]\phantomsection
\begin{RemarkList}
\item The conjecture can be interpreted as follows. There is an $\IBL$-structure on $\RedStringH(\Loop M;\R)$ compatible with Chas-Sullivan operations, and the $\IBLInfty$-algebra $\dIBL^\PMC(\RedCycC(\Harm(M)))$ is its chain model.
\item The loop coproduct $\tau$ is geometrically defined only on $\ConstRedStringH(\Loop M)$; the conjecture thus provides an extension of $\StringCoOp_2$ to $\RedStringH(\Loop M)$. In~\cite{Basu2011}, it is shown that the geometric definition of $\tau$ can be extended to $\H(\Loop M)$ for manifolds with zero Euler characteristic, i.e., $\chi(M) = 0$. This extension depends on the choice of a non-vanishing vector field on~$M$. By homotopy invariance (see (v) below), our extension of $\StringCoOp_2$ should not depend on the admissible Hodge propagator $\Prpg$.
\item The loop product $\LoopPr$ is geometrically defined on $\H(\Loop M)$; however, it does not always induce an associative product on $\RedH(\Loop M) = \H(\Loop M, x_0)$. Indeed, the examples of $\T^2$ (see \cite{Basu2011}) and $\Sph{3}$ (see \cite{Sullivan1999}) show that $\H(x_0;\R) \subset \H(\Loop M;\R)$ is not an ideal with respect to $\LoopPr$. By \cite{Tamanoi2010}, this does not happen when $\chi(M) \neq 0$, and hence, in this case,~$\LoopPr$ restricts to $\H(\Loop M, x_0; \R)$.
\item The computation for $\Sph{n}$ with $n\ge 2$ and the computation for $\CP^n$ in Section~\ref{Section:Computation} support the conjecture. The computation for $\Sph{1}$ in Section~\ref{Section:HomSphere} provides a counterexample for non-simply-connected $M$. Surfaces of genus~$g\ge 1$ should be considered.

\item  We expect that if $M_1$ and $M_2$ are homotopy equivalent, then the $\IBLInfty$-algebras $\dIBL^{\PMC_1}(\CycC(\HDR(M_1)))$ and $\dIBL^{\PMC_2}(\CycC(\HDR(M_2)))$ are $\IBLInfty$-homotopy equivalent.
\qedhere
\end{RemarkList}
\end{Remark}

\end{document}
