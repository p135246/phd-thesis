%auto-ignore
\providecommand{\MainFolder}{..}
\documentclass[\MainFolder/Text.tex]{subfiles}

\begin{document}

\section{Basics of IBL-infinity-algebras
%for the \texorpdfstring{$\IBLInfty$-theory}{IBL-infinity-theory}
}
\allowdisplaybreaks
\label{Sec:Alg1}

%For the following definition, Definition~\ref{Def:Grading}, \ref{Def:SymAlgebra}, \ref{Def:Filtrations} and \ref{Def:Completion} are relevant.
\Correct[caption={DONE Bad picture},noline]{There migh be wrong dimensions on one side of the maurer cartan element. Compare to the section about filtrations. REMARK THAT THE DEFINITION NEEDS THE WEIGHT STRUCTURE OF THE BIALGEBRA. No Problem.}
\Add[caption={DONE New algebraic setting in Appendix},noline]{There is a new algebraic description of the iterated compatibility condition in the appendix.}
\begin{Definition}[Exterior algebra]\label{Def:ExtAlg}
Given a graded vector space $C$ over $\R$, we define the \emph{exterior algebra} over $C$ by
\[ \Ext C \coloneqq \Sym(C[1]). \]
The weight-$k$ component is denoted by $\Ext_k C$ and the weight-reduced part by~$\RExt C$. If $C$ is in addition filtered, then $\Ext_k C$ is filtered by the induced filtration and its completion is denoted by $\hat{\Ext}_k C$.
\end{Definition}


We have the concatenation product $\Prod : \Ext C \otimes \Ext C \rightarrow \Ext C$ and the shuffle coproduct $\CoProd: \Ext C \rightarrow \Ext C\otimes \Ext C$ defined~by
\begin{align*}
&\Prod(c_{11}\dots c_{1k} \otimes c_{21} \dots c_{2k'}) \coloneqq c_{11} \dots c_{1k} c_{21} \dots c_{2k'}\quad\text{and}\\[\jot]
&\CoProd(c_1 \dots c_k) \coloneqq \sum_{\substack{k_1,\,k_2 \ge 0\\ k_1 + k_2 = k}} \sum_{\sigma\in \Perm_{k_1, k_2}} \varepsilon(\sigma,c) c_{\sigma^{-1}_1} \dots c_{\sigma_{k_1}^{-1}}\otimes c_{\sigma_{k_1+1}^{-1}}\dots c_{\sigma_{k_1 + k_2}^{-1}}
\end{align*}
for all homogenous $c_{ij}$, $c_i \in C[1]$ and $k$, $k'\ge 0$, respectively, where $\Perm_{k_1, k_2}\subset \Perm_{k_1+k_2}$ denotes the set of shuffle permutations with blocks of lengths $k_1$ and $k_2$. These operations satisfy the relations of an \emph{associative bialgebra} (see \cite{Loday2012}):
\begin{equation}\label{Eq:Bialgebra}
\text{Assoc.~bialg.}\quad \left\{
\begin{aligned}
\Prod\circ (\Id\otimes \Prod) &= \Prod\circ (\Prod \otimes \Id), \\ 
(\Id\otimes \CoProd)\circ\CoProd &= (\CoProd\otimes \Id)\circ \CoProd, \\
\CoProd \circ \Prod &= (\Prod\otimes \Prod) \circ (\Id \otimes \tau\otimes \Id) \circ (\CoProd\otimes \CoProd).
\end{aligned} \right.
\end{equation}
Here $\tau: C_1 \otimes C_2 \rightarrow C_2 \otimes C_1$, $c_1\otimes c_2 \mapsto (-1)^{\Abs{c_1}\Abs{c_2}}c_2\otimes c_1$ denotes the \emph{twist map}.


We will use the bialgebra calculus ($\coloneqq$\,relations \eqref{Eq:Bialgebra}) to write down explicit formulas for the operations $\circ_{h_1, \dotsc, h_r}$ which were briefly introduced in \cite{Cieliebak2015}; these operations take symmetric maps $f_1$, $\dotsc$, $f_r$ and connect $h_1$, $\dotsc$, $h_r$ of their outputs to the inputs of a symmetric map $f$ in all possible ways, so that the result, which we denote by $f\circ_{h_1, \dotsc, h_r}(f_1, \dotsc,f_r)$, becomes a symmetric map again.

%In accordance with~\cite{Cieliebak2015}, one should think of a map $f: \Ext_k C \rightarrow \Ext_l C$ as of a Riemannian surface with $k$ ingoing and $l$ outcoming boundary components (\eqqcolon\,ends). The partial composition of two maps then corresponds to gluing of Riemannian surfaces at a given number of boundary components. This will be discussed once more Remark~\ref{Rem:SignConv}.


\begin{Definition}[Partial compositions] \label{Def:CircS}
Let $C$ be a graded vector space. For $i$, $j\ge 0$, we denote by 
\begin{align*} \pi_i : \Ext C \longrightarrow \Ext_i C,& \quad \iota_i : \Ext_i C \longrightarrow \Ext C, \\
 \Id_i : \Ext_i C \longrightarrow \Ext_i C,& \quad \begin{aligned}[t]\tau_{i,j}: \Ext_i C\otimes \Ext_j C &\longrightarrow \Ext_j C \otimes \Ext_i C \end{aligned}
 \end{align*}
the components of the canonical projection $\pi$, the canonical inclusion $\iota$, the identity $\Id$ and the twist map $\tau$, respectively. We also set 
\[ \CoProd_{i,j} \coloneqq (\pi_i \otimes \pi_j)\circ \CoProd\circ \iota_{i+j} \quad\text{and}\quad \Prod_{i,j}\coloneqq \pi_{i+j}\circ \Prod\circ (\iota_i \otimes \iota_j). \]
For $k'$, $k_1$, $l'$, $l_1\ge 0$, let $f: \Ext_{k'}C \rightarrow \Ext_{l'} C$ and $f_1: \Ext_{k_1} C \rightarrow \Ext_{l_1} C$ be linear maps, and let $0 \le h \le \min(k', l_1)$. We set  
\[k \coloneqq k' + k_1 - h \quad\text{and}\quad l \coloneqq l' + l_1 - h \]
and define the \emph{composition of $f$ and $f_1$ at $h$ common outputs} to be the linear map $f \circ_h f_1: \Ext_k C \rightarrow \Ext_l C$ given by 
\begin{equation}\label{Eq:CompositionSimple}
f \circ_h f_1 \coloneqq \begin{multlined}[t]\Prod_{l', l_1 - h} \circ (f\otimes \Id_{l_1-h})\circ (\Prod_{h, k'-h}\otimes \Id_{l_1-h})\circ (\Id_{h} \otimes \tau_{\rule{0pt}{7pt}l_1-h,k'-h}) \\[\jot] \circ (\CoProd_{h,l_1-h}\otimes \Id_{k'-h}) \circ (f_1 \otimes \Id_{k'-h} ) \circ \CoProd_{k_1, k'-h}. \end{multlined}
\end{equation}
More generally, we define the composition of $f: E_{k'}\rightarrow E_{l'}$ with $r\ge 1$ linear maps $f_{i}: E_{k_i} \rightarrow E_{l_i}$ with $k_i$, $l_i \ge 0$ for $i=1$,~$\dotsc$, $r$ at $0 \le h_i \le l_i$ common outputs such that $h\coloneqq h_1 + \dotsb + h_r \le k'$ as follows. We set 
\[ k\coloneqq k' + k_1 + \dotsb + k_r-h\quad\text{and}\quad l\coloneqq l' +  l_1 + \dotsb + l_r  - h \]
and define $f\circ_{h_1, \dotsc, h_r}(f_1, \dotsc, f_r): \Ext_k C \rightarrow \Ext_l C$ by\ToDo[caption={DONE Add the other composition},noline]{Add the definition of $(f_1,\dotsc,f_r)\circ_{h_1,\dotsc,h_r} f$. IT IS DONE IN THE APPENDIX.}
\begin{equation} \label{Eq:CompositionFormula}
\begin{aligned}
&f\circ_{h_1, \dotsc, h_r}(f_1, \dotsc, f_r)\\
&\qquad \coloneqq \begin{multlined}[t]
\Prod\circ (f\otimes \Id)\circ(\Prod\otimes\Id)\circ(\Id\otimes \tau) \\[\jot] \circ \big(\big[(\Prod^{(r)}\otimes \Prod^{(r)})\circ (F_{h_1,\dotsc,h_r} \otimes \Id^{\otimes r})\circ \sigma_r\circ \CoProd^{\otimes r}\big] \otimes \Id \big) \\[\jot] \circ (f_1\otimes \dotsb \otimes f_r\otimes \Id)\circ\CoProd^{(r+1)},
\end{multlined}\end{aligned}
\end{equation}
where we have:
\begin{itemize}
\item The operation $\Prod^{(r)}$ is the ``product with $r$ inputs'' and the operation $\CoProd^{(r)}$ the ``coproduct with $r$ outputs''; they are defined~by
\begin{align*}
\Prod^{(r)} &\coloneqq \Prod(\Id \otimes \Prod)\dotsb(\Id^{\otimes r-2} \otimes \Prod), & \Prod^{(1)}&\coloneqq \Id,\\
\CoProd^{(r)} &\coloneqq (\Id^{\otimes r-2} \otimes \CoProd)\dotsb(\Id \otimes \CoProd)\CoProd, &\CoProd^{(1)}&\coloneqq \Id.
\end{align*}

\item $F_{h_1,\dotsc, h_r} \coloneqq (\iota_{h_1}\pi_{h_1}) \otimes \dotsb \otimes (\iota_{h_r}\pi_{h_r})$.%is the ``filter''.
\item The permutation $\sigma_r \in \Perm_{2r}$ is given by  \Add[caption={DONE Better definition!!!}]{This needs a better definition of $\sigma_r$!!!}
\[\sigma_r: (1,2,3,4\dotsc, 2r-1, 2r) \longmapsto (1,r+1,2,r+2, \dotsc, r, 2r).\]
\item The symbols $f$ and $f_i$ inside the formula denote the \emph{trivial extensions} of $f$ and $f_i$, respectively; we extend a linear map $f: E_{k'} C \rightarrow E_{l'} C$ trivially to $f: \Ext C \rightarrow \Ext C$ by defining $f(\Ext_i C)=0$ for $i\neq k'$.
\end{itemize}
\end{Definition}

\begin{Remark}[On partial compositions]\phantomsection\label{Rem:Compositions}
\begin{RemarkList}
\item Defining $f\circ_{h_1, \dotsc, h_r}(f_1, \dotsc, f_r): \Ext_k C \rightarrow \Ext_l C$ using~\eqref{Eq:CompositionFormula} makes sense because the right hand side is a trivial extension of its component $\Ext_k C \rightarrow \Ext_l C$. In fact, all $\Prod$, $\CoProd$, $\pi$, $\iota$ in \eqref{Eq:CompositionFormula} can be replaced with $\Prod_{i,j}$, $\CoProd_{i,j}$, $\pi_i$, $\iota_i$ for unique $i$, $j$, so that trivial extensions do not have to be used at all. In this way, it can be seen that \eqref{Eq:CompositionSimple} is indeed a special case of~\eqref{Eq:CompositionFormula}. 
\item If $h = k' = l_1$, then $f \circ_{h} f_1 = f\circ f_1$.
\item It holds $f \circ_0 f_1 = (-1)^{\Abs{f}\Abs{f_1}} f_1 \circ_0 f$ and 
\[ f\circ_{h_1,\dotsc,h_r}(f_1,\dotsc,f_r) = \varepsilon(\sigma,f) f\circ_{h_{\sigma_1^{-1}},\dotsc,h_{\sigma_r^{-1}}}(f_{\sigma_1^{-1}},\dotsc,f_{\sigma_r^{-1}}). \]
\item Consider the (``non-trivial'') extension $\hat{f}\coloneqq \Prod(f\otimes \Id)\CoProd: \Ext C \rightarrow \Ext C$ and the symmetric product $f_1 \odot \dotsb \odot f_r \coloneqq \Prod^{(r)}(f_1 \otimes \dotsb \otimes f_r)\CoProd^{(r)}: \Ext C \rightarrow \Ext C$. The following formulas from~\cite{Cieliebak2015} hold:
\begin{equation} \label{Eq:Mix}
\begin{aligned}
f\circ_{h_1,\dotsc,h_{r-1},0}(f_1,\dotsc, f_r) &= f\circ_{h_1,\dotsc, h_{r-1}}(f_1,\dotsc, f_{r-1}) \odot f_r, \\
\hat{f} \circ \hat{f}_1 &= \sum_{h = 0}^{\min(k',l_1)} \widehat{f\circ_h f_1}, \\ 
   \hat{f} \circ (f_1 \odot \dotsb \odot f_r) &= \sum_{\substack{h_1, \dotsc, h_r \ge 0 \\ h_1 + \dotsb + h_r = k'}} f\circ_{h_1,\dotsc, h_r}(f_1,\dotsc, f_r).
\end{aligned}
\end{equation}
We also have the ``weak associativity''
\begin{equation}\label{Eq:WeakAssoc}
\qquad\qquad \mathclap{\sum_{\substack{0 \le h_2 \le \min(f_3^-, f_2^+) \\
0 \le h_1 \le \min(f_1^+,f_2^- + f_3^- - h_2) \\
h_1 + h_2 = h}}}\quad\qquad f_1 \circ_{h_1} (f_2 \circ_{h_2} f_3) = \qquad\quad \mathclap{\sum_{\substack{0 \le h_1 \le \min(f_1^+,f_2^-) \\ 0 \le h_2 \le \min(f_1^+ + f_2^+ - h_1, f_3^-) \\ h_1 + h_2 = h}}}\qquad\quad (f_1 \circ_{h_1} f_2) \circ_{h_2} f_3
\end{equation}
for every $0\le h \le \min(k_1 + k_2 + k_3, l_1 + l_2 + l_3)$, where $f^+$ denotes the number of inputs and $f^-$ the number of outputs of $f$. The weak associativity of $\circ_h$ can be proven using the associativity of $\,\hat{\cdot}$ and the second relation of \eqref{Eq:Mix}.
\item We refer to Section~\ref{Sec:CompConvA} of Appendix~\ref{App:IBLMV} for a thorough treatment of partial compositions. We show there that $\circ_{h_1,\dotsc,h_r}$ can be defined on maps on any connected weight-graded bialgebra using natural bilinear operations $\SquareComp_A$ on polynomials in the convolution product. In Proposition~\ref{Prop:PartCompositions} in the appendix, we prove the relations above.\qedhere
\end{RemarkList}
\end{Remark}

If $C$ is filtered by a decreasing filtration, then the bialgebra operations extend continuously to 
\begin{align*}
\Prod: \hat{\Ext}_{k_1} C \COtimes \hat{\Ext}_{k_2} C &\longrightarrow \hat{\Ext}_{k_1+k_2} C \quad \text{and}\\ 
\CoProd: \hat{\Ext}_k C &\longrightarrow \bigoplus_{\substack{l_1, l_2 \ge 0 \\ l_1 + l_2 = k}} \hat{\Ext}_{l_1}C\COtimes\hat{\Ext}_{l_2}C
\end{align*}
for all $k_1$, $k_2$, $k\in \N_0$ because they preserve the filtration degree (see~\cite{Fresse} for a similar construction). Next, if $f_1: \hat{\Ext}_{k_1}C \rightarrow \hat{\Ext}_{l_1}C$ and $f_2: \hat{\Ext}_{k_2}C\rightarrow \hat{\Ext}_{l_2}C$ have finite filtration degrees, then $f_1 \otimes f_2: \hat{\Ext}_{k_1} C \otimes \hat{\Ext}_{k_2} C \rightarrow \hat{\Ext}_{l_1}C\otimes\hat{\Ext}_{l_2}C$ has finite filtration degree too, and hence it extends continuously to $f_1 \otimes f_2: \hat{\Ext}_{k_1} C\COtimes \hat{\Ext}_{k_2} C \rightarrow \hat{\Ext}_{l_1}C\COtimes\hat{\Ext}_{l_2}C$.
Using these facts, we can canonically extend Definition~\ref{Def:CircS} to maps $f: \hat{\Ext}_{k'} C \rightarrow \hat{\Ext}_{l'} C$ and $f_i: \hat{\Ext}_{k_i}C \rightarrow \hat{\Ext}_{l_i}C$ of finite filtration degrees. The resulting map $f\circ_{h_1,\dotsc,h_r}(f_1,\dotsc,f_r): \hat{\Ext}_k C \rightarrow \hat{\Ext}_l C$ will have finite filtration degree too. Moreover, the formulas in Remark~\ref{Rem:Compositions} will still hold.%The details will be given in~\cite{MyPhD}.


We will now rephrase the definitions of an $\IBLInfty$-algebra, a Maurer-Cartan element and twisted operations from~\cite{Cieliebak2015} in terms of~$\circ_{h_1, \dotsc, h_r}$.
% !!!! It can also be seen graphically from Figure \ref{Fig:Surfaces}.!!!!
% That the relations are satisfied.
%The following is a combination of Definitions 8.1 and 2.3 and Lemma 2.5 in \cite{Cieliebak2015}: 
\begin{Def}[$\IBLInfty$-algebra] \label{Def:IBLInfty} Let $C$ be a graded vector space equipped with a decreasing filtration, and let $d\in \Z$ and $\gamma\ge 0$ be fixed constants. An \emph{$\IBLInfty$-algebra of bidegree $(d,\gamma)$} on~$C$ is a collection of linear maps $\OPQ_{klg}: \hat{\Ext}_k C \rightarrow \hat{\Ext}_l C$ for all $k,l\ge 1$, $g\ge 0$ which are homogenous, of finite filtration degree and satisfy the following conditions: 
%\marginnote{Discuss $\circ_s$ once more. Make a remark how $\circ_s$ is defined? I actually do not need it anywhere but $q_{120}^m$ and Appendix. Does filtered $\IBLInfty$-algebra with the grading filtration gives this $\IBLInfty$? What is the filtration degree?}
% The filtration degree of the canonical dIBL with respect to the filtration by weights is 2. The pushforward Maurer-Cartan element satisfies the filtration condition strictly. All maps p_klg, f_klg constructed by summation over ribbon graphs satisfy the filtration condition with =.
\begin{enumerate}[label=\arabic*)]
\item $\Abs{\OPQ_{klg}} = - 2d(k+g-1) - 1$.
\item $\Norm{\OPQ_{klg}} \ge \gamma \chi_{klg}$,
where $\chi_{klg}\coloneqq2-2g-k-l$. %(cf. $e$ in \eqref{Eq:EulerFormula})
\item The \emph{$\IBLInfty$-relations} hold: for all $k,l\ge 1$, $g\ge 0$, we have
\begin{equation} \label{Eq:IBLInfRel}
\sum_{h=1}^{g+1} \sum_{\substack{k_1, k_2, l_1, l_2 \ge 1 \\ g_1, g_2 \ge 0 \\k_1 + k_2 = k + h \\ l_1 + l_2 = l+ h\\ g_1 + g_2 = g+ 1 -h }} \OPQ_{k_2 l_2 g_2} \circ_h \OPQ_{k_1 l_1 g_1} = 0.
\end{equation}
%where the operator $\circ_s$ connects exactly $s$ outputs of $\OPQ_{k_1 l_1 g_1}$ to exactly $s$ inputs of $\OPQ_{k_2 l_2 g_2}$ in all possible ways producing a map $\widehat{E}_k C \rightarrow \widehat{E}_l C$.
\end{enumerate}
We denote a given $\IBLInfty$-algebra structure on $C$ by $\IBLInfty(C)$; i.e., we write $\IBLInfty(C)=(C,(\OPQ_{klg}))$.
%We can also write $\IBLInfty(C) = (\OPQ_{klg})$, where the notation $(\OPQ_{klg})$ denotes the collection of $\OPQ_{klg}$ for all $k$, $l\ge 1$, $g\ge 0$.
%If the filtration is trivial, i.e., $\Filtr_\lambda C = 0$ for all $\lambda\in \R$, and hence $\hat{\Ext}_k C = \Ext_k C$ for all $k\ge 1$ and the condition 2) holds trivially, we call $\IBLInfty(C)$ simply an \emph{$\IBLInfty$-algebra} of degree $d$ on $C$.

If $\OPQ_{klg} \equiv 0$ for all $(k,l,g)\neq (1,1,0)$, $(2,1,0)$, $(1,2,0)$, then we call $\IBLInfty(C)$ a \emph{$\dIBL$-algebra} and denote it by $\dIBL(C)$. If in addition $\OPQ_{110} \equiv 0$, then we have an \emph{$\IBL$-algebra} $\IBL(C)$.

If the operations on the completed exterior powers~$\hat{\Ext}_k C$ arise as continuous extensions of operations $\OPQ_{klg}: \Ext_k C \rightarrow \Ext_l C$, then we call the $\IBLInfty$-algebra \emph{completion-free} and denote $C$ together with the operations $\OPQ_{klg}: \Ext_k C \rightarrow \Ext_l C$ by $\ShortIBLInfty(C)$.
\end{Def}
%We note that filtered algebras are defined in Definition~\ref{Def:FiltAlg} in the appendix and that they are compared to the definition above in Remark~\ref{Rem:FiltrStr}.

The acronym $\IBL$ stands for an \emph{involutive Lie bialgebra.} It follows namely from the $\IBLInfty$-relations \eqref{Eq:IBLInfRel} that for $\IBL(C) = (C,\OPQ_{210},\OPQ_{120})$ the following holds: 
\begin{equation*}
%\label{Eq:IndIBL}
\raisebox{2ex}{$\text{Lie bialg.}\;\left\{\rule{0pt}{5ex}\right.$}\;
\begin{aligned}   
   0&= \OPQ_{210}\circ_1 \OPQ_{210} &&\leftarrow\text{Jacobi id.}\\
   0 &= \OPQ_{120} \circ_1 \OPQ_{120} &&\leftarrow\text{co-Jacobi id.}\\
   0 &= \OPQ_{120}\circ_1 \OPQ_{210} + \OPQ_{210}\circ_1 \OPQ_{120}&&\leftarrow\text{Drinfeld id.} \\
   0 &= \OPQ_{210} \circ_2 \OPQ_{120} &&\leftarrow\text{Involutivity}
\end{aligned}
\end{equation*}
The acronym $\dIBL$ stands for a \emph{differential involutive Lie bialgebra} --- an involutive Lie bialgebra together with a differential (a boundary operator in our case) such that the bracket and cobracket are chain maps.
 
\begin{Proposition}[Odd degree shift of an $\IBL$-algebra]\label{Prop:ClasModIBL}
Let $(C,\OPQ_{210}, \OPQ_{120})$ be an $\IBL$-algebra of degree $d$ from Definition~\ref{Def:IBLInfty}, and let $\tilde{\OPQ}_{210} : C^{\otimes 2} \rightarrow C$ and $\tilde{\OPQ}_{120}: C \rightarrow C^{\otimes 2}$ be the linear maps defined by
\begin{equation}\label{Eq:ClasModIBL}
\begin{aligned}
\SuspU \tilde{\OPQ}_{210}(x_1 \otimes x_2) &\coloneqq \OPQ_{210}(\pi(\SuspU^2 x_1 \otimes x_2)) \quad\text{and} \\
\SuspU^2 \tilde{\OPQ}_{120}(x) &\coloneqq \iota(\OPQ_{120}(\SuspU x))
\end{aligned}
\end{equation}
for all $x_1$, $x_2$, $x\in C$, where $\iota: \Sym_2(C[1]) \rightarrow C[1]^{\otimes 2}$ is the section of $\pi: C[1]^{\otimes 2} \rightarrow \Sym_2(C[1])$ from Definition~\ref{Def:SymAlgebra} and~$\SuspU$ is a formal symbol of degree $\Abs{\SuspU} = -1$. Then the degrees satisfy
\[ \Abs{\tilde{\OPQ}_{210}} = \Abs{\OPQ_{210}} - 1 = -2d - 2\quad\text{and}\quad\Abs{\tilde{\OPQ}_{120}} = \Abs{\OPQ_{120}} + 1 = 0, \]
the operations $\tilde{\OPQ}_{210}$ and $\tilde{\OPQ}_{120}$ are graded antisymmetric, i.e., we have
\[ \tilde{\OPQ}_{210} \circ\tau = - \tilde{\OPQ}_{210}\quad\text{and}\quad\tau \circ \tilde{\OPQ}_{120} = - \tilde{\OPQ}_{120} \]
for the twist map $\tau$, and the relations
\begin{align*}
%\label{Eq:ClassicIBL}
0&=\tilde{\OPQ}_{210}\circ (\tilde{\OPQ}_{210}\otimes \Id)\circ (\Id^{\otimes 3}+ t_3 + t_3^2), \\
0&=(\Id^{\otimes 3}+t_3 + t_3^2)\circ (\tilde{\OPQ}_{120}\otimes\Id)\circ \tilde{\OPQ}_{120}, \\
0&= x_1 \cdot \tilde{\OPQ}_{120}(x_2) - (-1)^{ x_1 x_2} x_2 \cdot \tilde{\OPQ}_{120}(x_1) - \tilde{\OPQ}_{120}(\tilde{\OPQ}_{210}(x_1,x_2)), \\
0& = \tilde{\OPQ}_{210} \circ \tilde{\OPQ}_{120},
\end{align*}
hold for all $x_1$, $x_2\in C$. Here, $t_3 \in \Perm_3$ denotes the cyclic permutation with $t_3(1) = 2$ acting on $C^{\otimes 3}$, and we define
\[ x\cdot (y_1 \otimes y_2) \coloneqq \tilde{\OPQ}_{210}(x,y_1)\otimes y_2 + (-1)^{ x y_1} y_1 \otimes \tilde{\OPQ}_{210}(x,y_2) \]
for all $x$, $y_1$, $y_2 \in C$.
%On the other hand, starting with~$\tilde{\OPQ}_{210}$ and $\tilde{\OPQ}_{120}$ satisfying the conditions above and defining $\OPQ_{210}: \Ext_2 C \rightarrow \Ext_1 C$ and $\OPQ_{120}: \Ext_1 C \rightarrow \Ext_2 C$ by \eqref{Eq:ClasModIBL}, we get an $\IBL$-algebra $(C,\OPQ_{210},\OPQ_{120})$ according to Definition~\ref{Def:IBLInfty}.
\end{Proposition}

\begin{proof}
The proof is a lengthy but straightforward computation.
\end{proof}

Consider the \emph{sign-action of $\Perm_k$} on $C^{\otimes k}$ given by $\sigma \mapsto \bar{\sigma}$, where 
\[ \bar{\sigma}(c_1\otimes\dotsb\otimes c_k)\coloneqq(-1)^\sigma\varepsilon(\sigma,c)c_{\sigma_1^{-1}}\otimes\dotsb\otimes c_{\sigma_k^{-1}} \]
for all $c_1$, $\dotsc$, $c_k\in C$ and $\sigma\in \Perm_k$. We define 
\[ \Lambda C \coloneqq \bigoplus_{k=0}^\infty \Lambda_k C\quad\text{with}\quad \Lambda_k C \coloneqq C^{\otimes k}/\sim, \]
where $c\sim \bar{\sigma}(c)$ for all $c\in C^{\otimes k}$ and $\sigma\in \Perm_k$. It is easy to see that the degree-shift map $\theta^{\otimes k}: c_1\otimes \dotsb \otimes c_k \in C^{\otimes k} \mapsto \varepsilon(c,\theta)(\theta c_1)\otimes\dotsb\otimes(\theta c_k)\in C[1]^{\otimes k}$ is equivariant with respect to the sign-action of $\Perm_k$ on $C^{\otimes k}$ and the standard action of $\Perm_k$ on $C[1]^{\otimes k}$ for all $k$, and thus it induces an isomorphism of vector spaces $\Lambda C$ and $\Ext C$. We use the following notation:
\[\begin{tikzcd}
 \OPQ_{klg}: \hat{\Ext}_k C \arrow{r} \arrow{d}{\theta^{\otimes k}} & \hat{\Ext}_l C \arrow{d}{\theta^{\otimes l}} \\
 \tilde{\OPQ}_{klg}: \hat{\Lambda}_k C \arrow{r} & \hat{\Lambda}_l C.
\end{tikzcd}\]
In fact, $\OPQ_{klg}$ and $\tilde{\OPQ}_{klg}$ are related precisely by the degree shift \eqref{Eq:DegreeShiftConvII}. 

\begin{figure}[t]
\centering
\begin{subfigure}{\textwidth}
\centering
%auto-ignore
\begin{tikzpicture}
\tikzset{decorate sep/.style 2 args=
{decorate,decoration={shape backgrounds,shape=circle,shape size=#1,shape sep=#2}}}

 \def\ecc{0.1} % eccentricity of bdd
    \def\gencanc{0.05} % legth of extra line in genus
  \def\genecc{20} % eccentricity of genus
  \def\distI{0.25}
  \def\radI{0.5}
  \def\radIII{0.2}
  \def\eccI{0.1}  
  \def\gencancI{0.05} % legth of extra line in genus
  \def\geneccI{20} % eccentricity of genus
  \def\genradI{0.45} % radius of genus
  \def\mezI{0.4} %Mezera mezi sousednimi nozkami s elipsou
  \def\mezII{3.8} % Mezera mezi dvema MC
  \def\mezIII{0.3} % Mezera na MC mala
  \def\mezIV{0.8} % Mezera mezi cylindry
  \def\mezV{0.1} % canc pro tri teck
  \def\triI{0.2mm}
  \def\triII{0.6mm}
  \def\vysI{1} % Vyska cylindru a velke operace
  \def\vysII{0.2} % Dolni vyska MC
  \def\vysIII{1} % Horni vyska MC
  \def\distIII{3*\radI} % Distance of MC genus
  
% TOP 
  
  \coordinate (A1) at (0,0);
  \coordinate (A2) at ($(A1)+(2*\radI,0)$);
  \coordinate (A3) at ($(A2)+(\mezI,0)$);
  \coordinate (A4) at ($(A3)+(2*\radI,0)$);
  \coordinate (A5) at ($(A4)+(\mezIII,0)$);
  \coordinate (A6) at ($(A5)+(2*\radI,0)$);
  \coordinate (A7) at ($(A6)+(\mezI,0)$);
  \coordinate (A8) at ($(A7)+(2*\radI,0)$);
  \coordinate (A9) at ($(A8)+(\mezI,0)$);
  
  
  \coordinate (B1) at ($(A1)+(0,-\vysI)$);
  \coordinate (B2) at ($(B1)+(2*\radI,0)$);
  \coordinate (B3) at ($(B2)+(\mezI,0)$);
  \coordinate (B4) at ($(B3)+(2*\radI,0)$);
  \coordinate (B5) at ($(B4)+(\mezIII,0)$);
  \coordinate (B6) at ($(B5)+(2*\radI,0)$);
  \coordinate (B7) at ($(B6)+(\mezI,0)$);
  \coordinate (B8) at ($(B7)+(2*\radI,0)$);
  \coordinate (B9) at ($(B8)+(\mezI,0)$);
  
  \coordinate (C1) at ($(B5)+(0,-\distI)$);
  \coordinate (C2) at ($(C1)+(2*\radI,0)$);
  \coordinate (C3) at ($(C2)+(\mezI,0)$);
  \coordinate (C4) at ($(C3)+(2*\radI,0)$);
  \coordinate (C5) at ($(C4)+(\mezIII,0)$);
  \coordinate (C6) at ($(C5)+(2*\radI,0)$);
  \coordinate (C7) at ($(C6)+(\mezI,0)$);
  \coordinate (C8) at ($(C7)+(2*\radI,0)$);
  \coordinate (C9) at ($(C8)+(\mezI,0)$);
  
  
  \coordinate (D1) at ($(C1)+(0,-\vysI)$);
  \coordinate (D2) at ($(D1)+(2*\radI,0)$);
  \coordinate (D3) at ($(D2)+(\mezI,0)$);
  \coordinate (D4) at ($(D3)+(2*\radI,0)$);
  \coordinate (D5) at ($(D4)+(\mezIII,0)$);
  \coordinate (D6) at ($(D5)+(2*\radI,0)$);
  \coordinate (D7) at ($(D6)+(\mezI,0)$);
  \coordinate (D8) at ($(D7)+(2*\radI,0)$);
  \coordinate (D9) at ($(D8)+(\mezI,0)$);
  
\draw (A1) arc (180:360:{\radI} and {\ecc});
\draw (A1) arc (180:0:{\radI} and {\ecc});
\draw (A3) arc (180:360:{\radI} and {\ecc});
\draw (A3) arc (180:0:{\radI} and {\ecc});
\draw (A5) arc (180:360:{\radI} and {\ecc});
\draw (A5) arc (180:0:{\radI} and {\ecc});
\draw (A7) arc (180:360:{\radI} and {\ecc});
\draw (A7) arc (180:0:{\radI} and {\ecc});

\draw (C1) arc (180:360:{\radI} and {\ecc});
\draw (C1) arc (180:0:{\radI} and {\ecc});
\draw (C3) arc (180:360:{\radI} and {\ecc});
\draw (C3) arc (180:0:{\radI} and {\ecc});
\draw (C5) arc (180:360:{\radI} and {\ecc});
\draw (C5) arc (180:0:{\radI} and {\ecc});
\draw (C7) arc (180:360:{\radI} and {\ecc});
\draw (C7) arc (180:0:{\radI} and {\ecc});

\draw (D1) arc (180:360:{\radI} and {\ecc});
\draw[dashed] (D1) arc (180:0:{\radI} and {\ecc});
\draw (D3) arc (180:360:{\radI} and {\ecc});
\draw[dashed] (D3) arc (180:0:{\radI} and {\ecc});
\draw (D5) arc (180:360:{\radI} and {\ecc});
\draw[dashed] (D5) arc (180:0:{\radI} and {\ecc});
\draw (D7) arc (180:360:{\radI} and {\ecc});
\draw[dashed] (D7) arc (180:0:{\radI} and {\ecc});

\draw (B1) arc (180:360:{\radI} and {\ecc});
\draw[dashed] (B1) arc (180:0:{\radI} and {\ecc});
\draw (B3) arc (180:360:{\radI} and {\ecc});
\draw[dashed] (B3) arc (180:0:{\radI} and {\ecc});
\draw (B5) arc (180:360:{\radI} and {\ecc});
\draw[dashed] (B5) arc (180:0:{\radI} and {\ecc});
\draw (B7) arc (180:360:{\radI} and {\ecc});
\draw[dashed] (B7) arc (180:0:{\radI} and {\ecc});

\draw (A1)--(B1);
\draw (A8)--(B8);
\draw (C1)--(D1);
\draw (C8)--(D8);

\draw (A4) to[out=-90,in=-90] (A5);
\draw (C4) to[out=-90,in=-90] (C5);
\draw (B4) to[out=90,in=90] (B5);
\draw (D4) to[out=90,in=90] (D5);

\draw[decorate sep={\triI}{\triII},fill] ($(A2)+(\mezV,0)$) to ($(A3)+(-\mezV,0)$);

\draw[decorate sep={\triI}{\triII},fill] ($(A6)+(\mezV,0)$) to ($(A7)+(-\mezV,0)$);

\draw[decorate sep={\triI}{\triII},fill] ($(B2)+(\mezV,0)$) to ($(B3)+(-\mezV,0)$);

\draw[decorate sep={\triI}{\triII},fill] ($(B6)+(\mezV,0)$) to ($(B7)+(-\mezV,0)$); 

\draw[decorate sep={\triI}{\triII},fill] ($(C2)+(\mezV,0)$) to ($(C3)+(-\mezV,0)$);

\draw[decorate sep={\triI}{\triII},fill] ($(C6)+(\mezV,0)$) to ($(C7)+(-\mezV,0)$);

\draw[decorate sep={\triI}{\triII},fill] ($(D2)+(\mezV,0)$) to ($(D3)+(-\mezV,0)$);

\draw[decorate sep={\triI}{\triII},fill] ($(D6)+(\mezV,0)$) to ($(D7)+(-\mezV,0)$);

\coordinate (U1) at ($(B1)-(0,\distI)$);
\coordinate (U2) at ($(U1)+(2*\radI,0)$);
\coordinate (U3) at ($(U2)+(\mezI,0)$);
\coordinate (U4) at ($(U3)+(2*\radI,0)$);
\coordinate (U5) at ($(U1)-(0,\vysI)$);
\coordinate (U6) at ($(U5)+(2*\radI,0)$);
\coordinate (U7) at ($(U6)+(\mezI,0)$);
\coordinate (U8) at ($(U7)+(2*\radI,0)$);

\coordinate (V1) at ($(A8)+(\mezIII,0)$);
\coordinate (V2) at ($(V1)+(2*\radI,0)$);
\coordinate (V3) at ($(V2)+(\mezI,0)$);
\coordinate (V4) at ($(V3)+(2*\radI,0)$);
\coordinate (V5) at ($(V1)-(0,\vysI)$);
\coordinate (V6) at ($(V5)+(2*\radI,0)$);
\coordinate (V7) at ($(V6)+(\mezI,0)$);
\coordinate (V8) at ($(V7)+(2*\radI,0)$);

\draw (U5) arc (180:360:{\radI} and {\ecc});
\draw[dashed] (U5) arc (180:0:{\radI} and {\ecc});
\draw (U7) arc (180:360:{\radI} and {\ecc});
\draw[dashed] (U7) arc (180:0:{\radI} and {\ecc});
\draw (V5) arc (180:360:{\radI} and {\ecc});
\draw[dashed] (V5) arc (180:0:{\radI} and {\ecc});
\draw (V7) arc (180:360:{\radI} and {\ecc});
\draw[dashed] (V7) arc (180:0:{\radI} and {\ecc});

\draw (U1) arc (180:360:{\radI} and {\ecc});
\draw (U1) arc (180:0:{\radI} and {\ecc});
\draw (U3) arc (180:360:{\radI} and {\ecc});
\draw (U3) arc (180:0:{\radI} and {\ecc});
\draw (V1) arc (180:360:{\radI} and {\ecc});
\draw (V1) arc (180:0:{\radI} and {\ecc});
\draw (V3) arc (180:360:{\radI} and {\ecc});
\draw (V3) arc (180:0:{\radI} and {\ecc});

\draw (U1)--(U5);
\draw (U4)--(U8);
\draw (U2)--(U6);
\draw (U3)--(U7);

\draw (V1)--(V5);
\draw (V4)--(V8);
\draw (V2)--(V6);
\draw (V3)--(V7);

\draw[decorate sep={\triI}{\triII},fill] ($(U2)+(\mezV,0)$) to ($(U3)+(-\mezV,0)$);
\draw[decorate sep={\triI}{\triII},fill] ($(U6)+(\mezV,0)$) to ($(U7)+(-\mezV,0)$);
\draw[decorate sep={\triI}{\triII},fill] ($(V2)+(\mezV,0)$) to ($(V3)+(-\mezV,0)$);
\draw[decorate sep={\triI}{\triII},fill] ($(V6)+(\mezV,0)$) to ($(V7)+(-\mezV,0)$);

\coordinate (G1) at ($(B1)+(\radI,0.5*\vysI)$);
  \draw (G1) to[out=-\geneccI,in=180+\geneccI] coordinate[pos=\gencancI] (G11) coordinate[pos=1-\gencancI] (G12) ($(G1) + (2*\genradI,0)$) ;
\draw (G11) to[out=\genecc,in=180-\genecc] (G12);

\coordinate (G2) at ($(G11)+(\distIII,0)$);
  \draw (G2) to[out=-\geneccI,in=180+\geneccI] coordinate[pos=\gencancI] (G21) coordinate[pos=1-\gencancI] (G22) ($(G2) + (2*\genradI,0)$) ;
\draw (G21) to[out=\genecc,in=180-\genecc] (G22);

\coordinate (GG2) at ($(D8)+(-2*\genradI-\radI,0.5*\vysI)$);
  \draw (GG2) to[out=-\geneccI,in=180+\geneccI] coordinate[pos=\gencancI] (GG21) coordinate[pos=1-\gencancI] (GG22) ($(GG2) + (2*\genradI,0)$) ;
\draw (GG21) to[out=\genecc,in=180-\genecc] (GG22);

\coordinate (GG1) at ($(GG2)+(-\distIII,0)$);
  \draw (GG1) to[out=-\geneccI,in=180+\geneccI] coordinate[pos=\gencancI] (GG11) coordinate[pos=1-\gencancI] (GG12) ($(GG1) + (2*\genradI,0)$) ;
\draw (GG11) to[out=\genecc,in=180-\genecc] (GG12);

\draw[decorate sep={\triI}{\triII},fill] ($(G12)+(\mezV+\gencancI,0)$) to ($(G21)+(-\mezV-\gencancI,0)$);
\draw[decorate sep={\triI}{\triII},fill] ($(GG12)+(\mezV+\gencancI,0)$) to ($(GG21)+(-\mezV-\gencancI,0)$);

\node at ($(G22)+(2*\radI,0)$) {$\mathfrak{q}_{k_1 l_1 g_1}$};
\node at ($(GG11)+(-2*\radI,0)$) {$\mathfrak{q}_{k_2 l_2 g_2}$};
\end{tikzpicture}
\caption{The term $\OPQ_{k_2 l_2 g_2} \circ_h \OPQ_{k_1 l_1 g_1}$ in the $\IBLInfty$-equation \eqref{Eq:IBLInfRel}.}
\end{subfigure}
\par\bigskip
\begin{subfigure}{\textwidth}
\centering
%auto-ignore
\begin{tikzpicture}

\tikzset{decorate sep/.style 2 args=
{decorate,decoration={shape backgrounds,shape=circle,shape size=#1,shape sep=#2}}} 

\tikzset{decorate sepp/.style 2 args=
{decorate,decoration={shape backgrounds,shape=-,shape size=#1,shape sep=#2}}} 

%\def\dist{0.25} %distance between two surfaces
%  \def\rad{0.5} % radius of bdd
%  \def\ecc{0.1} % eccentricity of bdd
%  \def\hght{1} % height of surfaces
%  \def\dif{1.5} % distance of two circles
%  \def\radO{\rad} % radius of bdd
%  \def\eccO{\ecc} % eccentricity of bdd
%  \def\hghtO{2*\hght+\dist} % height of surfaces
%  \def\difO{\dif} % distance of two circles
%  \def\gencanc{0.05} % legth of extra line in genus
%  \def\genecc{20} % eccentricity of genus
%  \def\genrad{0.45} % radius of genus


 \def\ecc{0.1} % eccentricity of bdd
    \def\gencanc{0.05} % legth of extra line in genus
  \def\genecc{20} % eccentricity of genus
  \def\distI{0.25}
  \def\radI{0.5}
  \def\radIII{0.2}
  \def\eccI{0.1}  
  \def\gencancI{0.05} % legth of extra line in genus
  \def\geneccI{20} % eccentricity of genus
  \def\genradI{0.45} % radius of genus
  \def\mezI{0.4} %Mezera mezi sousednimi nozkami s elipsou
  \def\mezII{3.8} % Mezera mezi dvema MC
  \def\mezIII{0.3} % Mezera na MC mala
  \def\mezIV{0.8} % Mezera mezi cylindry
  \def\mezV{0.1} % canc pro tri teck
  \def\triI{0.2mm}
  \def\triII{0.6mm}
  \def\vysI{1} % Vyska cylindru a velke operace
  \def\vysII{0.2} % Dolni vyska MC
  \def\vysIII{1} % Horni vyska MC
  \def\distIII{3*\radI} % Distance of MC genus
% Big piece 
% ===============
  \coordinate (B1) at (0,0);
  \coordinate (B2) at ($(B1) + (2*\radI,0)$);
  \coordinate (B3) at ($(B2) + (\mezI,0)$);
  \coordinate (B4) at ($(B3) + (2*\radI,0)$); 
  \coordinate (B5) at ($(B4) + (\mezII,0)$); 
  \coordinate (B6) at ($(B5) + (2*\radI,0)$);
  \coordinate (B7) at ($(B6) + (\mezI,0)$);
  \coordinate (B8) at ($(B7) + (2*\radI,0)$);

  \coordinate (B9) at ($(B1)+(0,-\vysI)$);
  \coordinate (B10) at ($(B9)+(2*\radI,0)$);
  \coordinate (B11) at ($(B10)+(2*\mezI+\mezII + 4*\radI,0)$);
  \coordinate (B12) at ($(B11)+(2*\radI,0)$);
  
%  \coordinate (BE1) at ($(B10)+(\mezV,0)$);
%  \coordinate (BE2) at ($(B11)+(-\mezV-2*\radI,0)$);
%  
%  \coordinate (BE3) at ($(B2)+(\mezV,0)$);
%  \coordinate (BE4) at ($(B3)+(-\mezV,0)$);  
%  \coordinate (BE5) at ($(B6)+(\mezV,0)$);
%  \coordinate (BE6) at ($(B7)+(-\mezV,0)$);
%  \draw (B2) to[out=-90, in=-90] (BE3);
% \draw (BE4) to[out=-90, in=-90] (B3);
% \draw (B6) to[out=-90, in=-90] (BE5);
% \draw (BE6) to[out=-90, in=-90] (B7);         
%   \coordinate (BE7) at ($(B4)+(\mezV,0)$);
%   \draw (B4) to[out=-90, in=-90] (BE7);
%    \coordinate (BE8) at ($(B5)+(-\mezV,0)$);
%   \draw (BE8) to[out=-90, in=-90] (B5);
%%   \draw (BE1) arc (180:360:{\radI} and {\ecc});
%%  \draw[dashed] (BE1) arc (180:0:{\radI} and {\ecc});  
%% \draw (BE2) arc (180:360:{\radI} and {\ecc});
%%  \draw[dashed] (BE2) arc (180:0:{\radI} and {\ecc});    
%
%  \draw (B10) to[out=90, in=90] (BE1);  
%  \draw ($(BE2)+(2*\radI,0)$) to[out=90, in=90] (B11);
  
    \draw (B1) arc (180:360:{\radI} and {\ecc});
  \draw(B1) arc (180:0:{\radI} and {\ecc});
  
  \draw (B3) arc (180:360:{\radI} and {\ecc});
  \draw (B3) arc (180:0:{\radI} and {\ecc});  
  
  \draw (B5) arc (180:360:{\radI} and {\ecc});
  \draw (B5) arc (180:0:{\radI} and {\ecc});    
  
  \draw (B7) arc (180:360:{\radI} and {\ecc});
  \draw (B7) arc (180:0:{\radI} and {\ecc});    
  
  \draw (B9) arc (180:360:{\radI} and {\ecc});
  \draw[dashed] (B9) arc (180:0:{\radI} and {\ecc});
  
  \draw (B11) arc (180:360:{\radI} and {\ecc});
  \draw[dashed] (B11) arc (180:0:{\radI} and {\ecc});

  
 


  \draw (B1)--(B9);
  \draw (B8)--(B12);


  \coordinate (BG1) at ($(B5)+(-1.5*\radI,-0.5*\vysI)$);
  \draw (BG1) to[out=-\geneccI,in=180+\geneccI] coordinate[pos=\gencancI] (BG11) coordinate[pos=1-\gencancI] (BG12) ($(BG1) + (2*\genradI,0)$) ;
\draw (BG11) to[out=\genecc,in=180-\genecc] (BG12);  

\coordinate (BG2) at ($(BG1) + (\distIII,0)$);
  \draw (BG2) to[out=-\geneccI,in=180+\geneccI] coordinate[pos=\gencancI] (BG21) coordinate[pos=1-\gencancI] (BG22) ($(BG2) + (2*\genradI,0)$) ;
\draw (BG21) to[out=\genecc,in=180-\genecc] (BG22);  

\draw[decorate sep={\triI}{\triII},fill] ($(BG12)+(\mezV+\gencancI,0)$) to ($(BG21)+(-\mezV-\gencancI,0)$);



\draw[decorate sep={\triI}{\triII},fill] ($(B2)+(\mezV,0)$) to ($(B3)+(-\mezV,0)$);
\draw[decorate sep={\triI}{\triII},fill] ($(B6)+(\mezV,0)$) to ($(B7)+(-\mezV,0)$);
%
%\draw[decorate sep={\triI}{\triII},fill] ($(B10)+(\mezV,0)$) to ($(B4)+(0,-\vysI)$);
%

\coordinate (BX1) at ($(B10)+(\mezI,0)$);
\draw (BX1) arc (180:360:{\radI} and {\ecc});
\draw[dashed] (BX1) arc (180:0:{\radI} and {\ecc});

\coordinate (BX2) at ($(B11)+(-\mezI-2*\radI,0)$);
\draw (BX2) arc (180:360:{\radI} and {\ecc});
\draw[dashed] (BX2) arc (180:0:{\radI} and {\ecc});

\draw[decorate sep={\triI}{\triII},fill] ($(B10)+(\mezV,0)$) to ($(BX1)+(-\mezV,0)$);
\draw[decorate sep={\triI}{\triII},fill] ($(BX2)+(2*\radI+\mezV,0)$) to ($(B11)+(-\mezV,0)$);

\def\shift{0.3*\mezIII}

\draw ($(BX1)+(2*\radI,0)$) to[out=90, in=180] ($(BX1)+(2*\radI+0.5*\mezIII,\shift)$);

\draw ($(B4)$) to[out=-90, in=180] ($(B4)+(0.5*\mezIII,-\shift)$);

\draw ($(B5)$) to[out=-90, in=0] ($(B5)+(-0.5*\mezIII,-\shift)$);

\draw ($(BX2)$) to[out=90, in=0] ($(BX2)+(-0.5*\mezIII,\shift)$);

\draw[densely dotted] ($(B4)+(0.5*\mezIII,-\shift)$)--($(B5)+(-0.5*\mezIII,-\shift)$);

\draw[densely dotted] ($(BX1)+(2*\radI+0.5*\mezIII,\shift)$)--($(BX2)+(-0.5*\mezIII,\shift)$);
% Cylinders
%==================

\coordinate (C01) at ($(B4)+(\mezIII,0)$);
\coordinate (C02) at ($(C01)+(2*\radI,0)$);
\coordinate (C03) at ($(C02)+(\mezI,0)$);
\coordinate (C04) at ($(C03)+(2*\radI,0)$);
\coordinate (C05) at ($(C01)+(0,-\vysI)$);
\coordinate (C06) at ($(C02)+(0,-\vysI)$);
\coordinate (C07) at ($(C03)+(0,-\vysI)$);
\coordinate (C08) at ($(C04)+(0,-\vysI)$);

\draw (C01) arc (180:360:{\radI} and {\ecc});
\draw (C01) arc (180:0:{\radI} and {\ecc});
\draw (C03) arc (180:360:{\radI} and {\ecc});
\draw (C03) arc (180:0:{\radI} and {\ecc});
\draw (C05) arc (180:360:{\radI} and {\ecc});
\draw[dashed] (C05) arc (180:0:{\radI} and {\ecc});
\draw (C07) arc (180:360:{\radI} and {\ecc});
\draw[dashed] (C07) arc (180:0:{\radI} and {\ecc});

\draw (C01)--(C05);
\draw (C02)--(C06);
\draw (C03)--(C07);
\draw (C04)--(C08);

\coordinate (C1) at ($(B8)+(\mezIII,0)$);
\coordinate (C2) at ($(C1)+(2*\radI,0)$);
\coordinate (C3) at ($(C2)+(\mezI,0)$);
\coordinate (C4) at ($(C3)+(2*\radI,0)$);
\coordinate (C5) at ($(C1)+(0,-\vysI)$);
\coordinate (C6) at ($(C2)+(0,-\vysI)$);
\coordinate (C7) at ($(C3)+(0,-\vysI)$);
\coordinate (C8) at ($(C4)+(0,-\vysI)$);

\draw (C1) arc (180:360:{\radI} and {\ecc});
\draw (C1) arc (180:0:{\radI} and {\ecc});
\draw (C3) arc (180:360:{\radI} and {\ecc});
\draw (C3) arc (180:0:{\radI} and {\ecc});
\draw (C5) arc (180:360:{\radI} and {\ecc});
\draw[dashed] (C5) arc (180:0:{\radI} and {\ecc});
\draw (C7) arc (180:360:{\radI} and {\ecc});
\draw[dashed] (C7) arc (180:0:{\radI} and {\ecc});

\draw (C1)--(C5);
\draw (C2)--(C6);
\draw (C3)--(C7);
\draw (C4)--(C8);

\draw[decorate sep={\triI}{\triII},fill] ($(C2)+(\mezV,-0.5*\vysI)$) to ($(C3)+(-\mezV,-0.5*\vysI)$);
\draw[decorate sep={\triI}{\triII},fill] ($(C02)+(\mezV,-0.5*\vysI)$) to ($(C03)+(-\mezV,-0.5*\vysI)$);

%\draw[decorate sep={\triI}{\triII},fill] ($(C08)+(\mezIII,0)$) to ($(B11)+(-\mezV,0)$);
%\draw[decorate sep={\triI}{\triII},fill] ($(C04)+(\mezIII,0)$) to ($(B5)+(-\mezV,0)$);

%\draw[dashed] ($(C04)+(2*\mezV,0)$) to ($(B5)+(-2*\mezV,0)$);
%\draw[dashed] ($(C08)+(2*\mezV,0)$) to ($(B11)+(-2*\mezV,0)$);
%\draw[dashed] ($(B10)+(2*\mezV,0)$) to ($(C05)+(-2*\mezV,0)$);
%
%\draw[dashed] ($(C06)+(\mezV,0)$) to ($(C07)+(-\mezV,0)$);
%\draw[dashed] ($(C02)+(\mezV,0)$) to ($(C03)+(-\mezV,0)$);
%\draw[dashed] ($(B4)+(\mezV,0)$) to ($(C01)+(-\mezV,0)$);

% Maurer I
% ===============
\coordinate (M1) at ($(B1)+(0,\distI)$);
\coordinate (M2) at ($(M1)+(2*\radI,0)$);
\coordinate (M3) at ($(M2)+(\mezI,0)$);
\coordinate (M4) at ($(M3)+(2*\radI,0)$);
\coordinate (M5) at ($(M4)+(\mezIII,0)$);
\coordinate (M6) at ($(M5)+(2*\radI,0)$);
\coordinate (M7) at ($(M6)+(\mezI,0)$);
\coordinate (M8) at ($(M7)+(2*\radI,0)$);


\draw (M1) arc (180:360:{\radI} and {\ecc});
\draw[dashed] (M1) arc (180:0:{\radI} and {\ecc});
\draw (M3) arc (180:360:{\radI} and {\ecc});
\draw[dashed] (M3) arc (180:0:{\radI} and {\ecc});
\draw (M5) arc (180:360:{\radI} and {\ecc});
  \draw[dashed] (M5) arc (180:0:{\radI} and {\ecc});
\draw (M7) arc (180:360:{\radI} and {\ecc});
  \draw[dashed] (M7) arc (180:0:{\radI} and {\ecc});

%\coordinate (MD1) at ($(M2)+(\mezV,0)$);
%\draw (M2) to[out=90,in=90] (MD1);
%\coordinate (MD2) at ($(M3)+(-\mezV,0)$);
%\draw (MD2) to[out=90,in=90] (M3);   
%\coordinate (MD3) at ($(M6)+(\mezV,0)$);
%\draw (MD3) to[out=90,in=90] (M6);
%\coordinate (MD4) at ($(M7)+(-\mezV,0)$);
%\draw (MD4) to[out=90,in=90] (M7);

\draw[decorate sep={\triI}{\triII},fill] ($(M2)+(\mezV,0)$) to ($(M3)+(-\mezV,0)$);
\draw[decorate sep={\triI}{\triII},fill] ($(M6)+(\mezV,0)$) to ($(M7)+(-\mezV,0)$);


\draw (M4) to[out=90, in=90] (M5);

\coordinate (MT1) at ($(M2)+(\radI,\vysIII)$);
\coordinate (MT2) at ($(M7)+(-\radI,\vysIII)$);
\draw (M1) to[out=90,in=180] (MT1);
\draw (MT2) to[out=0,in=90] (M8);
\draw (MT1)--(MT2);

\coordinate (MG1) at ($(M1)+(2*\radI,0.5*\vysI)$);
  \draw (MG1) to[out=-\geneccI,in=180+\geneccI] coordinate[pos=\gencancI] (MG11) coordinate[pos=1-\gencancI] (MG12) ($(MG1) + (2*\genradI,0)$) ;
\draw (MG11) to[out=\genecc,in=180-\genecc] (MG12);

\coordinate (MG2) at ($(MG1)+(\distIII,0)$);
  \draw (MG2) to[out=-\geneccI,in=180+\geneccI] coordinate[pos=\gencancI] (MG21) coordinate[pos=1-\gencancI] (MG22) ($(MG2) + (2*\genradI,0)$) ;
\draw (MG21) to[out=\genecc,in=180-\genecc] (MG22);   

\draw[decorate sep={\triI}{\triII},fill] ($(MG12)+(\mezV+\gencancI,0)$) to ($(MG21)+(-\mezV-\gencancI,0)$);

% Maurer II
% =============

\coordinate (M21) at ($(B5)+(0,\distI)$);
\coordinate (M22) at ($(M21)+(2*\radI,0)$);
\coordinate (M23) at ($(M22)+(\mezI,0)$);
\coordinate (M24) at ($(M23)+(2*\radI,0)$);
\coordinate (M25) at ($(M24)+(\mezIII,0)$);
\coordinate (M26) at ($(M25)+(2*\radI,0)$);
\coordinate (M27) at ($(M26)+(\mezI,0)$);
\coordinate (M28) at ($(M27)+(2*\radI,0)$);


\draw (M21) arc (180:360:{\radI} and {\ecc});
\draw[dashed] (M21) arc (180:0:{\radI} and {\ecc});
\draw (M23) arc (180:360:{\radI} and {\ecc});
\draw[dashed] (M23) arc (180:0:{\radI} and {\ecc});
\draw (M25) arc (180:360:{\radI} and {\ecc});
  \draw[dashed] (M25) arc (180:0:{\radI} and {\ecc});
\draw (M27) arc (180:360:{\radI} and {\ecc});
  \draw[dashed] (M27) arc (180:0:{\radI} and {\ecc});

%\coordinate (M2D1) at ($(M22)+(\mezV,0)$);
%\draw (M22) to[out=90,in=90] (M2D1);
%\coordinate (M2D2) at ($(M23)+(-\mezV,0)$);
%\draw (M2D2) to[out=90,in=90] (M23);   
%\coordinate (M2D3) at ($(M26)+(\mezV,0)$);
%\draw (M2D3) to[out=90,in=90] (M26);
%\coordinate (M2D4) at ($(M27)+(-\mezV,0)$);
%\draw (M2D4) to[out=90,in=90] (M27);


\draw[decorate sep={\triI}{\triII},fill] ($(M22)+(\mezV,0)$) to ($(M23)+(-\mezV,0)$);
\draw[decorate sep={\triI}{\triII},fill] ($(M26)+(\mezV,0)$) to ($(M27)+(-\mezV,0)$);

\draw (M24) to[out=90, in=90] (M25);

\coordinate (M2T1) at ($(M22)+(0,\vysIII)$);
\coordinate (M2T2) at ($(M27)+(0,\vysIII)$);
\draw (M21) to[out=90,in=180] (M2T1);
\draw (M2T2) to[out=0,in=90] (M28);
\draw (M2T1)--(M2T2);

\coordinate (M2G1) at ($(M21)+(2*\radI,0.5*\vysI)$);
  \draw (M2G1) to[out=-\geneccI,in=180+\geneccI] coordinate[pos=\gencancI] (M2G11) coordinate[pos=1-\gencancI] (M2G12) ($(M2G1) + (2*\genradI,0)$) ;
\draw (M2G11) to[out=\genecc,in=180-\genecc] (M2G12);

\coordinate (M2G2) at ($(M2G1)+(\distIII,0)$);
  \draw (M2G2) to[out=-\geneccI,in=180+\geneccI] coordinate[pos=\gencancI] (M2G21) coordinate[pos=1-\gencancI] (M2G22) ($(M2G2) + (2*\genradI,0)$) ;
\draw (M2G21) to[out=\genecc,in=180-\genecc] (M2G22);   

\draw[decorate sep={\triI}{\triII},fill] ($(M2G12)+(\mezV+\gencancI,0)$) to ($(M2G21)+(-\mezV-\gencancI,0)$);


\draw[decorate sep={\triI}{\triII},fill] ($(M8)+(\mezV,0.5*\vysI)$) to ($(M21)+(-\mezV,0.5*\vysI)$);

% LABELS

 
\node at ($(M2G22)+(1.5*\radI,0)$) {$\mathfrak{n}_{l_r g_r}$};
\node at ($(MG22)+(1.5*\radI,0)$) {$\mathfrak{n}_{l_1 g_1}$};
\node at ($(B2)+(1*\radI,-0.5*\vysI)$) {$\mathfrak{q}_{k'l' g'}$};

\end{tikzpicture}
\caption{The term $\OPQ_{k' l' g'}\circ_{h_1,\dotsc,h_r} (\PMC_{l_1 g_1},\dotsc,\PMC_{l_r g_r})$ in the Maurer-Cartan equation \eqref{Eq:MaurerCartanEquation}. We remark that the contour of the surface corresponding to $\OPQ_{k'l'g'}$ starts on the left and continues to the right along the dotted line behind the two trivial cylinders.}
\end{subfigure}
\par\bigskip
\begin{subfigure}{\textwidth}
\centering
%auto-ignore

\begin{tikzpicture}

\tikzset{decorate sep/.style 2 args=
{decorate,decoration={shape backgrounds,shape=circle,shape size=#1,shape sep=#2}}} 

\tikzset{decorate sepp/.style 2 args=
{decorate,decoration={shape backgrounds,shape=-,shape size=#1,shape sep=#2}}} 

%\def\dist{0.25} %distance between two surfaces
%  \def\rad{0.5} % radius of bdd
%  \def\ecc{0.1} % eccentricity of bdd
%  \def\hght{1} % height of surfaces
%  \def\dif{1.5} % distance of two circles
%  \def\radO{\rad} % radius of bdd
%  \def\eccO{\ecc} % eccentricity of bdd
%  \def\hghtO{2*\hght+\dist} % height of surfaces
%  \def\difO{\dif} % distance of two circles
%  \def\gencanc{0.05} % legth of extra line in genus
%  \def\genecc{20} % eccentricity of genus
%  \def\genrad{0.45} % radius of genus


 \def\ecc{0.1} % eccentricity of bdd
    \def\gencanc{0.05} % legth of extra line in genus
  \def\genecc{20} % eccentricity of genus
  \def\distI{0.25}
  \def\radI{0.375}
  \def\radIII{0.2}
  \def\eccI{0.1}  
  \def\gencancI{0.05} % legth of extra line in genus
  \def\geneccI{20} % eccentricity of genus
  \def\genradI{0.35} % radius of genus
  \def\mezI{0.4} %Mezera mezi sousednimi nozkami s elipsou
  \def\mezII{3.2} % Mezera mezi dvema MC
  \def\mezIII{0.3} % Mezera na MC mala
  \def\mezIV{0.8} % Mezera mezi cylindry
  \def\mezV{0.1} % canc pro tri teck
  \def\triI{0.2mm}
  \def\triII{0.6mm}
  \def\vysI{1} % Vyska cylindru a velke operace
  \def\vysII{0.2} % Dolni vyska MC
  \def\vysIII{1} % Horni vyska MC
  \def\distIII{3*\radI} % Distance of MC genus
% Big piece 
% ===============
  \coordinate (B1) at (0,0);
  \coordinate (B2) at ($(B1) + (2*\radI,0)$);
  \coordinate (B3) at ($(B2) + (\mezI,0)$);
  \coordinate (B4) at ($(B3) + (2*\radI,0)$); 
  \coordinate (B5) at ($(B4) + (\mezII,0)$); 
  \coordinate (B6) at ($(B5) + (2*\radI,0)$);
  \coordinate (B7) at ($(B6) + (\mezI,0)$);
  \coordinate (B8) at ($(B7) + (2*\radI,0)$);

  \coordinate (B9) at ($(B1)+(0,-\vysI)$);
  \coordinate (B10) at ($(B9)+(2*\radI,0)$);
  \coordinate (B11) at ($(B10)+(2*\mezI+\mezII + 4*\radI,0)$);
  \coordinate (B12) at ($(B11)+(2*\radI,0)$);
  
%  \coordinate (BE1) at ($(B10)+(\mezV,0)$);
%  \coordinate (BE2) at ($(B11)+(-\mezV-2*\radI,0)$);
%  
%  \coordinate (BE3) at ($(B2)+(\mezV,0)$);
%  \coordinate (BE4) at ($(B3)+(-\mezV,0)$);  
%  \coordinate (BE5) at ($(B6)+(\mezV,0)$);
%  \coordinate (BE6) at ($(B7)+(-\mezV,0)$);
%  \draw (B2) to[out=-90, in=-90] (BE3);
% \draw (BE4) to[out=-90, in=-90] (B3);
% \draw (B6) to[out=-90, in=-90] (BE5);
% \draw (BE6) to[out=-90, in=-90] (B7);         
%   \coordinate (BE7) at ($(B4)+(\mezV,0)$);
%   \draw (B4) to[out=-90, in=-90] (BE7);
%    \coordinate (BE8) at ($(B5)+(-\mezV,0)$);
%   \draw (BE8) to[out=-90, in=-90] (B5);
%%   \draw (BE1) arc (180:360:{\radI} and {\ecc});
%%  \draw[dashed] (BE1) arc (180:0:{\radI} and {\ecc});  
%% \draw (BE2) arc (180:360:{\radI} and {\ecc});
%%  \draw[dashed] (BE2) arc (180:0:{\radI} and {\ecc});    
%
%  \draw (B10) to[out=90, in=90] (BE1);  
%  \draw ($(BE2)+(2*\radI,0)$) to[out=90, in=90] (B11);
  
    \draw (B1) arc (180:360:{\radI} and {\ecc});
  \draw(B1) arc (180:0:{\radI} and {\ecc});
  
  \draw (B3) arc (180:360:{\radI} and {\ecc});
  \draw (B3) arc (180:0:{\radI} and {\ecc});  
  
  \draw (B5) arc (180:360:{\radI} and {\ecc});
  \draw (B5) arc (180:0:{\radI} and {\ecc});    
  
  \draw (B7) arc (180:360:{\radI} and {\ecc});
  \draw (B7) arc (180:0:{\radI} and {\ecc});    
  
  \draw (B9) arc (180:360:{\radI} and {\ecc});
  \draw[dashed] (B9) arc (180:0:{\radI} and {\ecc});
  
  \draw (B11) arc (180:360:{\radI} and {\ecc});
  \draw[dashed] (B11) arc (180:0:{\radI} and {\ecc});

  
  \draw (B8)--(B12);


  \coordinate (BG1) at ($(B5)+(-1.5*\radI,-0.5*\vysI)$);
  \draw (BG1) to[out=-\geneccI,in=180+\geneccI] coordinate[pos=\gencancI] (BG11) coordinate[pos=1-\gencancI] (BG12) ($(BG1) + (2*\genradI,0)$) ;
\draw (BG11) to[out=\genecc,in=180-\genecc] (BG12);  

\coordinate (BG2) at ($(BG1) + (\distIII,0)$);
  \draw (BG2) to[out=-\geneccI,in=180+\geneccI] coordinate[pos=\gencancI] (BG21) coordinate[pos=1-\gencancI] (BG22) ($(BG2) + (2*\genradI,0)$) ;
\draw (BG21) to[out=\genecc,in=180-\genecc] (BG22);  

\draw[decorate sep={\triI}{\triII},fill] ($(BG12)+(\mezV+\gencancI,0)$) to ($(BG21)+(-\mezV-\gencancI,0)$);



\draw[decorate sep={\triI}{\triII},fill] ($(B2)+(\mezV,0)$) to ($(B3)+(-\mezV,0)$);
\draw[decorate sep={\triI}{\triII},fill] ($(B6)+(\mezV,0)$) to ($(B7)+(-\mezV,0)$);
%
%\draw[decorate sep={\triI}{\triII},fill] ($(B10)+(\mezV,0)$) to ($(B4)+(0,-\vysI)$);
%

\coordinate (BX1) at ($(B10)+(\mezI,0)$);
\draw (BX1) arc (180:360:{\radI} and {\ecc});
\draw[dashed] (BX1) arc (180:0:{\radI} and {\ecc});

\coordinate (BX2) at ($(B11)+(-\mezI-2*\radI,0)$);
\draw (BX2) arc (180:360:{\radI} and {\ecc});
\draw[dashed] (BX2) arc (180:0:{\radI} and {\ecc});

\draw[decorate sep={\triI}{\triII},fill] ($(B10)+(\mezV,0)$) to ($(BX1)+(-\mezV,0)$);
\draw[decorate sep={\triI}{\triII},fill] ($(BX2)+(2*\radI+\mezV,0)$) to ($(B11)+(-\mezV,0)$);

\def\shift{0.3*\mezIII}

\draw ($(BX1)+(2*\radI,0)$) to[out=90, in=180] ($(BX1)+(2*\radI+0.5*\mezIII,\shift)$);

\draw ($(B4)$) to[out=-90, in=180] ($(B4)+(0.5*\mezIII,-\shift)$);

\draw ($(B5)$) to[out=-90, in=0] ($(B5)+(-0.5*\mezIII,-\shift)$);

\draw ($(BX2)$) to[out=90, in=0] ($(BX2)+(-0.5*\mezIII,\shift)$);

\draw[densely dotted] ($(B4)+(0.5*\mezIII,-\shift)$)--($(B5)+(-0.5*\mezIII,-\shift)$);

\draw[densely dotted] ($(BX1)+(2*\radI+0.5*\mezIII,\shift)$)--($(BX2)+(-0.5*\mezIII,\shift)$);
% Cylinders
%==================

\coordinate (C01) at ($(B4)+(\mezIII,0)$);
\coordinate (C02) at ($(C01)+(2*\radI,0)$);
\coordinate (C03) at ($(C02)+(\mezI,0)$);
\coordinate (C04) at ($(C03)+(2*\radI,0)$);
\coordinate (C05) at ($(C01)+(0,-\vysI)$);
\coordinate (C06) at ($(C02)+(0,-\vysI)$);
\coordinate (C07) at ($(C03)+(0,-\vysI)$);
\coordinate (C08) at ($(C04)+(0,-\vysI)$);

\draw (C01) arc (180:360:{\radI} and {\ecc});
\draw (C01) arc (180:0:{\radI} and {\ecc});
\draw (C03) arc (180:360:{\radI} and {\ecc});
\draw (C03) arc (180:0:{\radI} and {\ecc});
\draw (C05) arc (180:360:{\radI} and {\ecc});
\draw[dashed] (C05) arc (180:0:{\radI} and {\ecc});
\draw (C07) arc (180:360:{\radI} and {\ecc});
\draw[dashed] (C07) arc (180:0:{\radI} and {\ecc});

\draw (C01)--(C05);
\draw (C02)--(C06);
\draw (C03)--(C07);
\draw (C04)--(C08);

\coordinate (C1) at ($(B8)+(\mezIII,0)$);
\coordinate (C2) at ($(C1)+(2*\radI,0)$);
\coordinate (C3) at ($(C2)+(\mezI,0)$);
\coordinate (C4) at ($(C3)+(2*\radI,0)$);
\coordinate (C5) at ($(C1)+(0,-\vysI)$);
\coordinate (C6) at ($(C2)+(0,-\vysI)$);
\coordinate (C7) at ($(C3)+(0,-\vysI)$);
\coordinate (C8) at ($(C4)+(0,-\vysI)$);

\draw (C1) arc (180:360:{\radI} and {\ecc});
\draw (C1) arc (180:0:{\radI} and {\ecc});
\draw (C3) arc (180:360:{\radI} and {\ecc});
\draw (C3) arc (180:0:{\radI} and {\ecc});
\draw (C5) arc (180:360:{\radI} and {\ecc});
\draw[dashed] (C5) arc (180:0:{\radI} and {\ecc});
\draw (C7) arc (180:360:{\radI} and {\ecc});
\draw[dashed] (C7) arc (180:0:{\radI} and {\ecc});

\draw (C1)--(C5);
\draw (C2)--(C6);
\draw (C3)--(C7);
\draw (C4)--(C8);

\draw[decorate sep={\triI}{\triII},fill] ($(C2)+(\mezV,-0.5*\vysI)$) to ($(C3)+(-\mezV,-0.5*\vysI)$);
\draw[decorate sep={\triI}{\triII},fill] ($(C02)+(\mezV,-0.5*\vysI)$) to ($(C03)+(-\mezV,-0.5*\vysI)$);

%\draw[decorate sep={\triI}{\triII},fill] ($(C08)+(\mezIII,0)$) to ($(B11)+(-\mezV,0)$);
%\draw[decorate sep={\triI}{\triII},fill] ($(C04)+(\mezIII,0)$) to ($(B5)+(-\mezV,0)$);

%\draw[dashed] ($(C04)+(2*\mezV,0)$) to ($(B5)+(-2*\mezV,0)$);
%\draw[dashed] ($(C08)+(2*\mezV,0)$) to ($(B11)+(-2*\mezV,0)$);
%\draw[dashed] ($(B10)+(2*\mezV,0)$) to ($(C05)+(-2*\mezV,0)$);
%
%\draw[dashed] ($(C06)+(\mezV,0)$) to ($(C07)+(-\mezV,0)$);
%\draw[dashed] ($(C02)+(\mezV,0)$) to ($(C03)+(-\mezV,0)$);
%\draw[dashed] ($(B4)+(\mezV,0)$) to ($(C01)+(-\mezV,0)$);

% Maurer I
% ===============
\coordinate (M1) at ($(B1)+(0,\distI)$);
\coordinate (M2) at ($(M1)+(2*\radI,0)$);
\coordinate (M3) at ($(M2)+(\mezI,0)$);
\coordinate (M4) at ($(M3)+(2*\radI,0)$);
\coordinate (M5) at ($(M4)+(\mezIII,0)$);
\coordinate (M6) at ($(M5)+(2*\radI,0)$);
\coordinate (M7) at ($(M6)+(\mezI,0)$);
\coordinate (M8) at ($(M7)+(2*\radI,0)$);


\draw (M1) arc (180:360:{\radI} and {\ecc});
\draw[dashed] (M1) arc (180:0:{\radI} and {\ecc});
\draw (M3) arc (180:360:{\radI} and {\ecc});
\draw[dashed] (M3) arc (180:0:{\radI} and {\ecc});
\draw (M5) arc (180:360:{\radI} and {\ecc});
  \draw[dashed] (M5) arc (180:0:{\radI} and {\ecc});
\draw (M7) arc (180:360:{\radI} and {\ecc});
  \draw[dashed] (M7) arc (180:0:{\radI} and {\ecc});

%\coordinate (MD1) at ($(M2)+(\mezV,0)$);
%\draw (M2) to[out=90,in=90] (MD1);
%\coordinate (MD2) at ($(M3)+(-\mezV,0)$);
%\draw (MD2) to[out=90,in=90] (M3);   
%\coordinate (MD3) at ($(M6)+(\mezV,0)$);
%\draw (MD3) to[out=90,in=90] (M6);
%\coordinate (MD4) at ($(M7)+(-\mezV,0)$);
%\draw (MD4) to[out=90,in=90] (M7);

\draw[decorate sep={\triI}{\triII},fill] ($(M2)+(\mezV,0)$) to ($(M3)+(-\mezV,0)$);
\draw[decorate sep={\triI}{\triII},fill] ($(M6)+(\mezV,0)$) to ($(M7)+(-\mezV,0)$);


\draw (M4) to[out=90, in=90] (M5);

\coordinate (MT1) at ($(M2)+(\radI,\vysIII)$);
\coordinate (MT2) at ($(M7)+(-\radI,\vysIII)$);
\draw (M1) to[out=90,in=180] (MT1);
\draw (MT2) to[out=0,in=90] (M8);
\draw (MT1)--(MT2);

\coordinate (MG1) at ($(M1)+(2*\radI,0.5*\vysI)$);
  \draw (MG1) to[out=-\geneccI,in=180+\geneccI] coordinate[pos=\gencancI] (MG11) coordinate[pos=1-\gencancI] (MG12) ($(MG1) + (2*\genradI,0)$) ;
\draw (MG11) to[out=\genecc,in=180-\genecc] (MG12);

\coordinate (MG2) at ($(MG1)+(\distIII,0)$);
  \draw (MG2) to[out=-\geneccI,in=180+\geneccI] coordinate[pos=\gencancI] (MG21) coordinate[pos=1-\gencancI] (MG22) ($(MG2) + (2*\genradI,0)$) ;
\draw (MG21) to[out=\genecc,in=180-\genecc] (MG22);   

\draw[decorate sep={\triI}{\triII},fill] ($(MG12)+(\mezV+\gencancI,0)$) to ($(MG21)+(-\mezV-\gencancI,0)$);

% Maurer II
% =============

\coordinate (M21) at ($(B5)+(0,\distI)$);
\coordinate (M22) at ($(M21)+(2*\radI,0)$);
\coordinate (M23) at ($(M22)+(\mezI,0)$);
\coordinate (M24) at ($(M23)+(2*\radI,0)$);
\coordinate (M25) at ($(M24)+(\mezIII,0)$);
\coordinate (M26) at ($(M25)+(2*\radI,0)$);
\coordinate (M27) at ($(M26)+(\mezI,0)$);
\coordinate (M28) at ($(M27)+(2*\radI,0)$);


\draw (M21) arc (180:360:{\radI} and {\ecc});
\draw[dashed] (M21) arc (180:0:{\radI} and {\ecc});
\draw (M23) arc (180:360:{\radI} and {\ecc});
\draw[dashed] (M23) arc (180:0:{\radI} and {\ecc});
\draw (M25) arc (180:360:{\radI} and {\ecc});
  \draw[dashed] (M25) arc (180:0:{\radI} and {\ecc});
\draw (M27) arc (180:360:{\radI} and {\ecc});
  \draw[dashed] (M27) arc (180:0:{\radI} and {\ecc});

%\coordinate (M2D1) at ($(M22)+(\mezV,0)$);
%\draw (M22) to[out=90,in=90] (M2D1);
%\coordinate (M2D2) at ($(M23)+(-\mezV,0)$);
%\draw (M2D2) to[out=90,in=90] (M23);   
%\coordinate (M2D3) at ($(M26)+(\mezV,0)$);
%\draw (M2D3) to[out=90,in=90] (M26);
%\coordinate (M2D4) at ($(M27)+(-\mezV,0)$);
%\draw (M2D4) to[out=90,in=90] (M27);


\draw[decorate sep={\triI}{\triII},fill] ($(M22)+(\mezV,0)$) to ($(M23)+(-\mezV,0)$);
\draw[decorate sep={\triI}{\triII},fill] ($(M26)+(\mezV,0)$) to ($(M27)+(-\mezV,0)$);

\draw (M24) to[out=90, in=90] (M25);

\coordinate (M2T1) at ($(M22)+(0,\vysIII)$);
\coordinate (M2T2) at ($(M27)+(0,\vysIII)$);
\draw (M21) to[out=90,in=180] (M2T1);
\draw (M2T2) to[out=0,in=90] (M28);
\draw (M2T1)--(M2T2);

\coordinate (M2G1) at ($(M21)+(2*\radI,0.5*\vysI)$);
  \draw (M2G1) to[out=-\geneccI,in=180+\geneccI] coordinate[pos=\gencancI] (M2G11) coordinate[pos=1-\gencancI] (M2G12) ($(M2G1) + (2*\genradI,0)$) ;
\draw (M2G11) to[out=\genecc,in=180-\genecc] (M2G12);

\coordinate (M2G2) at ($(M2G1)+(\distIII,0)$);
  \draw (M2G2) to[out=-\geneccI,in=180+\geneccI] coordinate[pos=\gencancI] (M2G21) coordinate[pos=1-\gencancI] (M2G22) ($(M2G2) + (2*\genradI,0)$) ;
\draw (M2G21) to[out=\genecc,in=180-\genecc] (M2G22);   

\draw[decorate sep={\triI}{\triII},fill] ($(M2G12)+(\mezV+\gencancI,0)$) to ($(M2G21)+(-\mezV-\gencancI,0)$);


\draw[decorate sep={\triI}{\triII},fill] ($(M8)+(\mezV,0.5*\vysI)$) to ($(M21)+(-\mezV,0.5*\vysI)$);

% LABELS

 
\node at ($(M2G22)+(1.7*\radI,0)$) {$\mathfrak{n}_{l_r g_r}$};
\node at ($(MG22)+(1.7*\radI,0)$) {$\mathfrak{n}_{l_1 g_1}$};
\node at ($(B2)+(-2*\radI,-0.5*\vysI)$) {$\mathfrak{q}_{k'l'g'}$};

% Additional cylinders

\coordinate (D1) at ($(B1)+(-\mezIII-\mezI-4*\radI,0)$);
\coordinate (D2) at ($(D1)+(2*\radI,0)$);
\coordinate (D3) at ($(D2)+(\mezI,0)$);
\coordinate (D4) at ($(D3)+(2*\radI,0)$);
\coordinate (D5) at ($(D1)+(0,-\vysI)$);
\coordinate (D6) at ($(D5)+(2*\radI,0)$);
\coordinate (D7) at ($(D6)+(\mezI,0)$);
\coordinate (D8) at ($(D7)+(2*\radI,0)$);


\draw (D1) arc (180:360:{\radI} and {\ecc});
\draw (D1) arc (180:0:{\radI} and {\ecc});
\draw (D3) arc (180:360:{\radI} and {\ecc});
\draw (D3) arc (180:0:{\radI} and {\ecc});
\draw (D5) arc (180:360:{\radI} and {\ecc});
\draw[dashed] (D5) arc (180:0:{\radI} and {\ecc});
\draw (D7) arc (180:360:{\radI} and {\ecc});
\draw[dashed] (D7) arc (180:0:{\radI} and {\ecc});
\draw (D1)--(D5);
\draw (D4) to[out=-90,in=-90] (B1);
\draw (D8) to[out=90,in=90] (B9);

\draw[decorate sep={\triI}{\triII},fill] ($(D6)+(\mezV,0)$) to ($(D7)+(-\mezV,0)$);
\draw[decorate sep={\triI}{\triII},fill] ($(D2)+(\mezV,0)$) to ($(D3)+(-\mezV,0)$);

\coordinate (T1) at ($(D1)+(0,\distI+\vysI)$); 
\coordinate (T2) at ($(T1)+(2*\radI,0)$);
\coordinate (T3) at ($(T2)+(\mezI,0)$);
\coordinate (T4) at ($(T3)+(2*\radI,0)$);
\coordinate (T5) at ($(T1)+(0,-\vysI)$);
\coordinate (T6) at ($(T2)+(0,-\vysI)$);
\coordinate (T7) at ($(T3)+(0,-\vysI)$);
\coordinate (T8) at ($(T4)+(0,-\vysI)$);

\draw (T1)--(T5);
\draw (T2)--(T6);
\draw (T3)--(T7);
\draw (T4)--(T8);

\draw (T1) arc (180:360:{\radI} and {\ecc});
\draw (T1) arc (180:0:{\radI} and {\ecc});
\draw (T3) arc (180:360:{\radI} and {\ecc});
\draw (T3) arc (180:0:{\radI} and {\ecc});
\draw (T5) arc (180:360:{\radI} and {\ecc});
\draw[dashed] (T5) arc (180:0:{\radI} and {\ecc});
\draw (T7) arc (180:360:{\radI} and {\ecc});
\draw[dashed] (T7) arc (180:0:{\radI} and {\ecc});

\draw[decorate sep={\triI}{\triII},fill] ($(T2)+(\mezV,0)$) to ($(T3)+(-\mezV,0)$);
\draw[decorate sep={\triI}{\triII},fill] ($(T6)+(\mezV,0)$) to ($(T7)+(-\mezV,0)$);

\end{tikzpicture}
\caption{The term $\OPQ_{k' l' g'}\circ_{l_1,\dotsc, l_r} (\PMC_{l_1 g_1},\dotsc,\PMC_{l_r g_r})$ in the twisted operation \eqref{Eq:TwistedOperations}. The remark to Figure (b) applies too.}
\end{subfigure}
\caption[$\IBLInfty$-relations, Maurer-Cartan equation and twisting graphically.]{Graphical representation of compositions appearing in Definitions \ref{Def:IBLInfty}, \ref{Def:MaurerCartan} and \ref{Def:TwistedOperations} as gluing of connected Riemannian surfaces. The figure is to be read from the top  to the bottom, the empty cylinder represents the identity, and the resulting surface must be connected. We emphasize that the gluing is not associative (c.f., weak associativity \eqref{Eq:WeakAssoc}).}
\label{Fig:Surfaces}
\end{figure}
%
%\noindent The following is a combination of Definition 9.1 and Lemma 2.9 in \cite{Cieliebak2015}:
%
\begin{Def}[Maurer-Cartan element] \label{Def:MaurerCartan}
A \emph{Maurer-Cartan element} for an $\IBLInfty$-algebra $\IBLInfty(C)$ from Definition~\ref{Def:IBLInfty} is a collection $\PMC \coloneqq (\PMC_{lg})_{l\ge 1, g\ge 0}$ of elements $\PMC_{lg}\in \hat{\Ext}_l C$ which are homogenous, of finite filtration degree and satisfy the following conditions:
\begin{enumerate}[label=\arabic*)]
\item $\Abs{\PMC_{lg}} = - 2d(g-1)$.
\item $\Norm{\PMC_{lg}}\ge\gamma \chi_{0lg}$ with $>$ for $(l,g)=(1,0)$, $(2,0)$ (see Definition~\ref{Def:IBLInfty} for~$\chi_{klg}$).
\item The \emph{Maurer-Cartan equation} holds: for all $l\ge 1$, $g\ge 0$, we have
\begin{equation} \label{Eq:MaurerCartanEquation}
\begin{aligned}
\sum_{r\ge 1}\frac{1}{r!}  \sum_{\substack{l', k', l_1, \dotsc, l_r\ge 1 \\ g', g_1, \dotsc, g_r \ge 0 \\ h_1, \dotsc, h_r \ge 1 \\ 
l_1 + \dotsb + l_r + l' - k'= l \\ g_1 + \dotsb + g_r + g' +  k' = g + r \\ h_1 + \dotsb + h_r - k' =0 } } \OPQ_{k' l' g'}\circ_{h_1,\dotsc,h_r} (\PMC_{l_1 g_1},\dotsc,\PMC_{l_r g_r}) = 0,
\end{aligned} 
\end{equation}
where we view $\PMC_{lg}$ as a linear map $\PMC_{lg}: \hat{\Ext}_0 C = \R \rightarrow \hat{\Ext}_l C$ with $\PMC_{lg}(1) = \PMC_{lg}$.
%where $\odot$ is the symmetric product of symmetric maps, and $\circ_{s_1,\dotsc,s_r}$ denotes the part of the composition with exactly $s_i$ outputs of $\PMC_{l_i g_i}$ connected to exactly $s_i$ inputs of $\OPQ_{klg}$ in all possible ways so that the result is symmetric.

\end{enumerate}
\end{Def}

% \noindent The following comes from Proposition 9.3 and its proof \cite{Cieliebak2015}:

\begin{Def}[Twisted operations] \label{Def:TwistedOperations}
In the setting of Definition~\ref{Def:MaurerCartan}, the \emph{twisted operations} $\OPQ_{klg}^\PMC: \hat{\Ext}_k C\rightarrow \hat{\Ext}_l C$ for $k,l\ge 1$, $g\ge 0$ are defined by
\begin{equation}\label{Eq:TwistedOperations}
 \OPQ_{klg}^\PMC =\sum_{r\ge 0} \frac{1}{r!} \sum_{\substack{k', l', l_1, \dotsc, l_r \ge 1 \\ g', g_1, \dotsc, g_r \ge 0\\ h_1, \dotsc, h_r \ge 1 \\
%k' \ge k,\,l'\le l \\
l_1 + \dotsb + l_r + l' - k' = l-k \\ g_1 + \dotsb +g_r + g' + k' = g + r + k \\ h_1 + \dotsb + h_r - k' = -k}} \OPQ_{k' l' g'}\circ_{h_1,\dotsc, h_r} (\PMC_{l_1 g_1},\dotsc,\PMC_{l_r g_r}).
\end{equation}
In \cite[Proposition~9.3]{Cieliebak2015}, they prove that $(\OPQ_{klg}^\PMC)_{k,l\ge 1, g\ge 0}$ is again an $\IBLInfty$-algebra of bidegree $(d,\gamma)$ on $C$ --- \emph{the twisted $\IBLInfty$-algebra}.  We denote it by $\IBLInfty^\PMC(C)$.
\end{Def}



Let $(\OPQ_{klg})$ be an $\IBLInfty$-algebra on $C$. The boundary operator $\OPQ_{110}: C[1] \rightarrow C[1]$ induces the boundary operator $\Bdd_k : \Ext_k C \rightarrow \Ext_k C$ for every $k\in \N$ (see \eqref{Eq:BddExt}). Because of the finite filtration degree, $\Bdd_k$ continuously extends to $\Bdd_k: \hat{\Ext}_k C \rightarrow \hat{\Ext}_k C$.
%, where we used that $\hat{\Ext}_k \hat{C} \simeq \hat{\Ext}_k C$ (see Remark~\ref{Rem:ComplTens}).
The following is easy to see using \eqref{Eq:CompositionSimple}:
\[\begin{aligned}
 \OPQ_{klg} \circ_1 \OPQ_{110} &= \OPQ_{klg} \circ \Bdd_k, \\
 \OPQ_{110} \circ_1 \OPQ_{klg} &= \Bdd_l \circ \OPQ_{klg}.
\end{aligned}\]
Because $\OPQ_{klg}$ are odd ($\coloneqq$\,have odd degree), we have
\[ \begin{aligned}
  [\Bdd,\OPQ_{klg}] &\coloneqq \Bdd_l \circ \OPQ_{klg} - (-1)^{\Abs{\Bdd}\Abs{\OPQ_{klg}}} \OPQ_{klg}\circ \Bdd_k \\
   &= \Bdd_l \circ \OPQ_{klg} + \OPQ_{klg}\circ \Bdd_k \\
   &= \OPQ_{110}\circ_1 \OPQ_{klg} + \OPQ_{klg}\circ_1 \OPQ_{110}.
  \end{aligned}\]
With this notation, the $\IBLInfty$-relations \eqref{Eq:IBLInfRel} for 
$\OPQ_{210}: \hat{\Ext}_2 C \rightarrow \hat{\Ext}_1 C$ and $\OPQ_{120}: \hat{\Ext}_1 C \rightarrow \hat{\Ext}_2 C$ become $[\Bdd,\OPQ_{210}] = 0$ and $[\Bdd,\OPQ_{120}] = 0$, respectively. If moreover the canonical maps $\Ext_k \H(\hat{C},\tilde{\OPQ}_{110}) \rightarrow \H(\hat{\Ext}_k C, \Bdd_k)$ induce the isomorphisms $\hat{\Ext}_k \H(\hat{C},\tilde{\OPQ}_{110})  \simeq \H(\hat{\Ext}_k C, \Bdd_k)$ for $k=1$, $2$, e.g., when Proposition~\ref{Prop:Kuenneth} holds, then we obtain the maps
\[ \OPQ_{210}: \hat{\Ext}_2\H(\hat{C},\tilde{\OPQ}_{110})  \rightarrow \hat{\Ext}_1\H(\hat{C},\tilde{\OPQ}_{110}) \quad \text{and}\quad \OPQ_{120}: \hat{\Ext}_1\H(\hat{C},\tilde{\OPQ}_{110})  \rightarrow \hat{\Ext}_2\H(\hat{C},\tilde{\OPQ}_{110}), \]
and $(\H(\hat{C},\tilde{\OPQ}_{110}),\OPQ_{210},\OPQ_{120})$ becomes an $\IBL$-algebra according to Definition~\ref{Def:IBLInfty} --- the \emph{induced $\IBL$-algebra on homology}.

\begin{Definition}[Homology]\label{Def:HomIBL}
We define the homology of an $\IBLInfty$-algebra $\IBLInfty(C)$ by
\[ \HIBL(C)[1] \coloneqq \H(\hat{C}[1], \OPQ_{110}). \]
It is a graded vector space with the induced filtration. If $\PMC$ is a Maurer-Cartan element for $\IBLInfty(C)$, we denote by $\HIBL^\PMC(C)$ the homology of $\IBLInfty^\PMC(C)$.
\end{Definition}

%These relations should be understood like the claim that $\OPQ_{210}\circ_1 \OPQ_{210}: \hat{\Ext}_3 C \rightarrow \hat{\Ext}_1 C$ induces the zero map $\HIBL_3(C) \rightarrow \HIBL_1(C)$, for example.
%It is an exercise to check that the first three relations of \eqref{Eq:IndIBL} are ``algebraically'' equivalent to \eqref{Eq:Bialgebra} (i.e., ignoring the completions and using just \eqref{Eq:CompositionSimple}) under the replacement $\Prod\mapsto \OPQ_{210}$, $\CoProd\mapsto \OPQ_{120}$ (notice that $\OPQ_{210}$, $\OPQ_{120}$ are odd whereas $\Prod$, $\CoProd$ even).
\begin{Remark}[Weak $\IBLInfty$-algebras and $\mathrm{BV}$-formalism]\phantomsection\label{Rem:BVForm}
\begin{RemarkList}
\item A possible generalization of the $\IBLInfty$-theory is to allow $k=0$ and $l=0$.
Such structures would be called \emph{weak $\IBLInfty$-algebras} while the structures from this section \emph{strict $\IBLInfty$-algebras.}
In fact, one does not need filtrations and completions to deal with the category of strict $\IBLInfty$-algebras unless deformations (twisting) are considered.
On the other hand, one needs filtrations and completions for the definition of a morphism of weak $\IBLInfty$-algebras already.
We refer to Appendix~\ref{App:IBLMV} for more details. 
\item Let $\CExt C[[\hbar]]$, resp.~$\CExt C((\hbar))$ be the spaces of formal power, resp.~Laurent series in the variable~$\hbar$ of degree $\Abs{\hbar} = 2d$ with coefficients in $\Ext C$ completed with respect to a suitable completion.
%\footnote{Note that $\Ext C$ has two filtrations $\F^1_\lambda C$ and $\F^2_\lambda C$: the filtration induced from $C$ and the filtration by weights, respectively. In \cite{Cieliebak2015}, they take the combined filtration $\F^1_\lambda C + \F^2_\lambda C$. We will discuss some variations in~\cite{MyPhD}.}
Operations of an $\IBLInfty$-algebra on $C$ can be encoded in a degree~$-1$ operator $\BVOp: \CExt C[[\hbar]] \rightarrow \CExt C[[\hbar]]$ called the \emph{$\mathrm{BV}$-operator,} while the data of a Maurer-Cartan element~$(\PMC_{lg})$ give rise to an element $e^{\PMC}\in \CExt C((\hbar))$. The prescriptions are
\[ \BVOp \coloneqq \sum_{i\ge 0}\BVOp_{i+1} \hbar^{i}\quad\text{and}\quad e^{\PMC} \coloneqq \sum_{j\in \Z} (e^{\PMC})_j \hbar^{j}, \]
where $\BVOp_i$ and $(e^\PMC)_j$ for $i\ge 1$, $j\in \Z$ are defined by 
\begin{align*}
\BVOp_i & \coloneqq \sum_{\substack{k\ge 1, g\ge 0 \\k+g=i}} \sum_{l\ge 1} \hat{\OPQ}_{klg}\quad\text{and} \\
(e^{\PMC})_j &\coloneqq \sum_{r=0}^\infty \frac{1}{r!} \sum_{\substack{g_1, \dotsc, g_r \ge 0 \\ g_1 + \dotsb +g_r - r= j }} \sum_{l_1, \dotsc, l_r\ge 1} \PMC_{l_1 g_1} \odot \dotsb \odot \PMC_{l_r g_r}.
\end{align*}
It can be shown that the $\IBLInfty$-relations~\eqref{Eq:IBLInfRel} and the Maurer-Cartan equation~\eqref{Eq:MaurerCartanEquation} are equivalent to  
\begin{equation}\label{Eq:BVEquat}
 \BVOp\circ \BVOp = 0\quad\text{and}\quad \BVOp(e^\PMC) = 0,
\end{equation}
respectively, and that the $\BVInfty$-operator $\BVOp^\PMC$ for the twisted $\IBLInfty$-structure $(\OPQ_{klg}^\PMC)$ satisfies
\begin{equation} \label{Eq:TwistBV}
\BVOp^\PMC(\bullet)= e^{-\PMC}\BVOp(e^\PMC\bullet),
\end{equation}
where we multiply with $e^{-\PMC}$ and $e^{\PMC}$, respectively.
These facts were shown in~\cite{Cieliebak2015} using~\eqref{Eq:Mix}.
We refer to Appendix~\ref{App:IBLMV} to the precise formulation of the $\BV$-formalism using a filtered version of the $\MV$-formalism from \cite{Markl2015}.\qedhere
%\footnote{One has to check that the compositions \eqref{Eq:BVEquat} and \eqref{Eq:TwistBV} are well-defined and pick a suitable completion $\CExt C$ so that all the constructions work.} 
%Some technical details, e.g., which completion $\CExt C$ we take so that $\BVOp$, $e^\PMC$ and $\Prod$ are well-defined, and why the compositions in \eqref{Eq:BVEquat}, \eqref{Eq:TwistBV} make sense, will be discussed in~\cite{MyPhD}. %For more about $\IBLInfty$-algebras as $\BVInfty$-algebras see~\cite{Markl2015a}.
\end{RemarkList}
\end{Remark}

In our applications in string topology, a canonical $\dIBL$-algebra $\dIBL(C)$ with a natural Maurer-Cartan element $\PMC$ coming from the Chern-Simons theory is given, and we want to study $\dIBL^\PMC(C)$, which will be a chain model of string topology.
We are also interested in the homology $\HIBL^\PMC(C)$, the $\IBL$-structure $\IBL(\HIBL^\PMC(C))$ and possible higher operations on $\HIBL^\PMC(C)$ induced by $\OPQ_{klg}^\PMC$; however, these higher maps are not chain maps in general.
The following proposition summarizes some observations in this situation:

\begin{Proposition}[Twist of a $\dIBL$-algebra]\label{Prop:dIBL}
Let $\dIBL(C) = (C,\OPQ_{110},\OPQ_{210},\OPQ_{120})$ be a $\dIBL$-algebra, and let $\PMC = (\PMC_{lg})$ be a Maurer-Cartan element. The Maurer-Cartan equation~\eqref{Eq:MaurerCartanEquation} reduces to the following:
\[ 
%\raisebox{-3ex}{\parbox{2em}{\centering MC-eq.}\quad $\Biggl\{$ } 
\begin{multlined}[b] 0 = \OPQ_{110}\circ_1 \PMC_{lg} + \OPQ_{120} \circ_1 \PMC_{l-1,g} +  \OPQ_{210}\circ_2 \PMC_{l+1,g-1} \\[\jot]+  \frac{1}{2}\sum_{\substack{l_1, l_2\ge 1 \\ g_1, g_2 \ge 0 \\ l_1 + l_2 = l + 1 \\ g_1 + g_2 = g}} \OPQ_{210}\circ_{1,1}(\PMC_{l_1 g_1}, \PMC_{l_2 g_2}) \end{multlined}\quad \forall l\ge1 , g\ge 0.
\]
In particular, the ``lowest'' equation is given by\footnote{In \cite[Definition 2.4.]{Cieliebak2015}, they define a partial ordering on the signatures $(k,l,g)$.}
\begin{equation} \label{Eq:MCEq}
(l,g) = (1,0): \qquad \OPQ_{110}(\PMC_{10}) + \frac{1}{2}\OPQ_{210}(\PMC_{10}, \PMC_{10}) = 0.
\end{equation}
This can be visualized as
{\begingroup \def\dist{0.25} %distance between two surfaces
  \def\rad{0.5} % radius of bdd
  \def\ecc{0.1} % eccentricity of bdd
  \def\hght{1} % height of surfaces
  \def\dif{1.5} % distance of two circles
  \def\radO{\rad} % radius of bdd
  \def\eccO{\ecc} % eccentricity of bdd
  \def\hghtO{2*\hght+\dist} % height of surfaces
  \def\difO{\dif} % distance of two circles
  \def\gencanc{0.05} % legth of extra line in genus
  \def\genecc{20} % eccentricity of genus
  \def\genrad{0.45} % radius of genus
\[0 =\quad \vcenterline{%auto-ignore
\begin{tikzpicture}
  \coordinate (P8) at (0,0);
  \coordinate (P9) at ($(P8)+(0,\hghtO)$);
  
  \coordinate (P10) at ($(P8)+(0,\hght)$);
  \coordinate (P11) at ($(P10)+(0,\dist)$);
  %\draw[dashed] (P11) arc (180:0:{\radO} and {\eccO});
  \draw (P10) arc (180:0:{\radO} and {\eccO});

  \draw (P10) arc (180:360:{\radO} and {\eccO});
  %\draw (P11) arc (180:360:{\radO} and {\eccO});

  %\draw (P9) arc (180:360:{\radO} and {\eccO});
  %\draw (P9) arc (180:0:{\radO} and {\eccO});  
  \draw (P8) arc (180:360:{\radO} and {\eccO});
  \draw[dashed] (P8) arc (180:0:{\radO} and {\eccO});
  \draw (P8) -- (P10);
  %\draw (P9) -- (P11);
  \draw ($(P8)+(2*\radO,0)$) -- ($(P10)+(2*\radO,0)$);
  %\draw ($(P9)+(2*\radO,0)$) -- ($(P11)+(2*\radO,0)$);
  
  \node at ($(P8)+(\radO,0.21*\hghtO-.7*\dist)$) {$\OPQ_{110}$};
  
   \node at ($(P11)+(\rad,0.5*\hght)$) {$\PMC_{10}$};
 
% Cylinders 
 
 
 \draw (P11) arc (180:360:{\rad} and {\ecc});
 \draw[dashed] (P11) arc (180:0:{\rad} and {\ecc});
 
 \draw (P11) to[out=90,in=180] ($(P11)+(\rad,\hght)$) to[out=0,in=90] ($(P11)+(2*\rad,0)$);
  
\end{tikzpicture}}\; + \frac{1}{2} \quad \vcenterline{%auto-ignore
\begin{tikzpicture}
  % Sphere
  \coordinate (P1) at (0,0);
  \coordinate (P2) at (-0.5*\dif,\hght);
  \coordinate (P3) at (0.5*\dif,\hght);
  \coordinate (P4) at ($(P2)+(0,\dist)$);
  \coordinate (P8) at ($(P3)+(0,\dist)$);
  \coordinate (P9) at ($(P8)+(0,\hght)$);

%Pair of pants
  
  \draw (P1) arc (180:360:{\rad} and {\ecc});
  \draw[dashed] (P1) arc (180:0:{\rad} and {\ecc});
  
   \draw (P3) arc (180:360:{\rad} and {\ecc});
  \draw (P3) arc (180:0:{\rad} and {\ecc});
  
  \draw (P2) arc (180:360:{\rad} and {\ecc});
  \draw (P2) arc (180:0:{\rad} and {\ecc});
  
 \draw (P2) to[out=270,in=90] (P1);
 \draw ($(P3)+(2*\rad,0)$) to[out=270,in=90] ($(P1)+(2*\rad,0)$);
 \draw ($(P2)+(2*\rad,0)$) to[out=270,in=270] (P3); 
 
% Maurer Cartan

 \draw (P4) arc (180:360:{\rad} and {\ecc});
 \draw[dashed] (P4) arc (180:0:{\rad} and {\ecc});
 
 \draw (P4) to[out=90,in=180] ($(P4)+(\rad,\hght)$) to[out=0,in=90] ($(P4)+(2*\rad,0)$);

% Labels 
 
 \node at ($(P1)+(\rad,0.5*\hght)$) {$\OPQ_{210}$};
 \node at ($(P4)+(\rad,0.5*\hght)$) {$\PMC_{10}$};
 \node at ($(P8)+(\rad,0.5*\hght)$) {$\PMC_{10}$};
 
% Cylinders 
 
 
 \draw (P8) arc (180:360:{\rad} and {\ecc});
 \draw[dashed] (P8) arc (180:0:{\rad} and {\ecc});
 
 \draw (P8) to[out=90,in=180] ($(P8)+(\rad,\hght)$) to[out=0,in=90] ($(P8)+(2*\rad,0)$);
  
\end{tikzpicture}}. \]
\endgroup}

The twisted $\IBLInfty$-algebra $\dIBL^\PMC(C)$ consists of the operations $\OPQ_{110}^\PMC$, $\OPQ_{210}^\PMC$ and~$\OPQ_{120}^\PMC$, which we call the \emph{basic operations}, and of the operations $\OPQ_{1lg}^\PMC$ for the pairs $(l,g)\in \N \times \N_0 \backslash \{(1,0),(2,0)\}$, which we call the \emph{higher operations}. These operations are given by 
\[ \begin{aligned}
\OPQ_{110}^\PMC &= \OPQ_{110} + \OPQ_{210}\circ_1 \PMC_{10},\\
\OPQ_{210}^\PMC &= \OPQ_{210}, \\
\OPQ_{120}^\PMC & = \OPQ_{120} + \OPQ_{210}\circ_1 \PMC_{20},\\
\OPQ_{1lg}^\PMC & = \OPQ_{210}\circ_1 \PMC_{lg}.
\end{aligned}\]
This can be visualized as
{ \begingroup \allowdisplaybreaks
\def\dist{0.25} %distance between two surfaces
  \def\rad{0.5} % radius of bdd
  \def\ecc{0.1} % eccentricity of bdd
  \def\hght{1} % height of surfaces
  \def\dif{1.5} % distance of two circles
  \def\radO{\rad} % radius of bdd
  \def\eccO{\ecc} % eccentricity of bdd
  \def\hghtO{2*\hght+\dist} % height of surfaces
  \def\difO{\dif} % distance of two circles
  \def\gencanc{0.05} % legth of extra line in genus
  \def\genecc{20} % eccentricity of genus
  \def\genrad{0.45} % radius of genus
\begin{align*}
\OPQ_{110}^\PMC & =\quad\vcenterline{%auto-ignore
\begin{tikzpicture}
  \coordinate (P8) at (0,0);
  \coordinate (P9) at ($(P8)+(0,\hghtO)$);
  
  \coordinate (P10) at ($(P8)+(0,\hght)$);
  \coordinate (P11) at ($(P10)+(0,\dist)$);
  \draw[dashed] (P11) arc (180:0:{\radO} and {\eccO});
  \draw (P10) arc (180:0:{\radO} and {\eccO});

  \draw (P10) arc (180:360:{\radO} and {\eccO});
  \draw (P11) arc (180:360:{\radO} and {\eccO});

  \draw (P9) arc (180:360:{\radO} and {\eccO});
  \draw (P9) arc (180:0:{\radO} and {\eccO});  
  \draw (P8) arc (180:360:{\radO} and {\eccO});
  \draw[dashed] (P8) arc (180:0:{\radO} and {\eccO});
  \draw (P8) -- (P10);
  \draw (P9) -- (P11);
  \draw ($(P8)+(2*\radO,0)$) -- ($(P10)+(2*\radO,0)$);
  \draw ($(P9)+(2*\radO,0)$) -- ($(P11)+(2*\radO,0)$);
  
  \node at ($(P8)+(\radO,0.21*\hghtO-.7*\dist)$) {$\OPQ_{110}$};
\end{tikzpicture}}\quad +\quad \vcenterline{%auto-ignore
\begin{tikzpicture}
  % Sphere
  \coordinate (P1) at (0,0);
  \coordinate (P2) at (-0.5*\dif,\hght);
  \coordinate (P3) at (0.5*\dif,\hght);
  \coordinate (P4) at ($(P2)+(0,\dist)$);
  \coordinate (P8) at ($(P3)+(0,\dist)$);
  \coordinate (P9) at ($(P8)+(0,\hght)$);

%Pair of pants
  
  \draw (P1) arc (180:360:{\rad} and {\ecc});
  \draw[dashed] (P1) arc (180:0:{\rad} and {\ecc});
  
   \draw (P3) arc (180:360:{\rad} and {\ecc});
  \draw (P3) arc (180:0:{\rad} and {\ecc});
  
  \draw (P2) arc (180:360:{\rad} and {\ecc});
  \draw (P2) arc (180:0:{\rad} and {\ecc});
  
 \draw (P2) to[out=270,in=90] (P1);
 \draw ($(P3)+(2*\rad,0)$) to[out=270,in=90] ($(P1)+(2*\rad,0)$);
 \draw ($(P2)+(2*\rad,0)$) to[out=270,in=270] (P3); 
 
% Maurer Cartan

 \draw (P4) arc (180:360:{\rad} and {\ecc});
 \draw[dashed] (P4) arc (180:0:{\rad} and {\ecc});
 
 \draw (P4) to[out=90,in=180] ($(P4)+(\rad,\hght)$) to[out=0,in=90] ($(P4)+(2*\rad,0)$);

% Labels 
 
 \node at ($(P1)+(\rad,0.5*\hght)$) {$\OPQ_{210}$};
 \node at ($(P4)+(\rad,0.5*\hght)$) {$\PMC_{10}$};

% Cylinders 
 
 
 \draw (P9) arc (180:360:{\rad} and {\ecc});
 \draw (P9) arc (180:0:{\rad} and {\ecc});  
 \draw (P8) arc (180:360:{\rad} and {\ecc});
 \draw[dashed] (P8) arc (180:0:{\rad} and {\ecc});
 \draw (P8) -- (P9);
 \draw ($(P8)+(2*\rad,0)$) -- ($(P9)+(2*\rad,0)$);
  
\end{tikzpicture}}, \\[1ex] 
\OPQ_{210}^\PMC &=\quad \vcenterline{%auto-ignore
\begin{tikzpicture}
  \coordinate (P1) at (0,0);
  \coordinate (P2) at (-0.5*\dif,\hght);
  \coordinate (P3) at (0.5*\dif,\hght);
  \coordinate (P4) at ($(P2)+(0,\dist)$);
  \coordinate (P8) at ($(P3)+(0,\dist)$);
  \coordinate (P9) at ($(P8)+(0,\hght)$);

%Pair of pants
  
  \draw (P1) arc (180:360:{\rad} and {\ecc});
  \draw[dashed] (P1) arc (180:0:{\rad} and {\ecc});
  
   \draw (P3) arc (180:360:{\rad} and {\ecc});
  \draw (P3) arc (180:0:{\rad} and {\ecc});
  
  \draw (P2) arc (180:360:{\rad} and {\ecc});
  \draw (P2) arc (180:0:{\rad} and {\ecc});
  
 \draw (P2) to[out=270,in=90] (P1);
 \draw ($(P3)+(2*\rad,0)$) to[out=270,in=90] ($(P1)+(2*\rad,0)$);
 \draw ($(P2)+(2*\rad,0)$) to[out=270,in=270] (P3); 
 
% Maurer Cartan


% Labels 
 
 \node at ($(P1)+(\rad,0.5*\hght)$) {$\OPQ_{210}$};

% Cylinders 
 
 
 \draw (P9) arc (180:360:{\rad} and {\ecc});
 \draw (P9) arc (180:0:{\rad} and {\ecc});  
 \draw (P8) arc (180:360:{\rad} and {\ecc});
 \draw[dashed] (P8) arc (180:0:{\rad} and {\ecc});
 \draw (P8) -- (P9);
 \draw ($(P8)+(2*\rad,0)$) -- ($(P9)+(2*\rad,0)$);
 
 
  \draw (P4) arc (180:360:{\rad} and {\ecc});
 \draw[dashed] (P4) arc (180:0:{\rad} and {\ecc});  
 \coordinate (P12) at ($(P4)+(0,\hght)$); 
 \draw (P12) arc (180:360:{\rad} and {\ecc});
 \draw (P12) arc (180:0:{\rad} and {\ecc});
 \draw (P4) -- (P12);
 \draw ($(P4)+(2*\rad,0)$) -- ($(P12)+(2*\rad,0)$);
\end{tikzpicture}}, \\[1ex] 
\OPQ_{120}^\PMC &=\quad \vcenterline{%auto-ignore
\begin{tikzpicture}


  \coordinate (P1) at (0,\hght);
  \coordinate (P2) at (-0.5*\difO,0);
  \coordinate (P3) at (0.5*\difO,0);
  \coordinate (P4) at ($(P1)+(0,\dist)$);
  \coordinate (P5) at ($(P4)+(0,\hght)$);
%Pair of pants 
 
  \draw (P5) arc (180:360:{\radO} and {\eccO});
  \draw (P5) arc (180:0:{\radO} and {\eccO});
  \draw (P4) arc (180:360:{\radO} and {\eccO});
  \draw[dashed] (P4) arc (180:0:{\radO} and {\eccO});
  \draw (P4) -- (P5);
  \draw ($(P4)+(2*\rad0,0)$) -- ($(P5)+(2*\rad0,0)$);
  
  \draw (P1) arc (180:360:{\radO} and {\eccO});
  \draw (P1) arc (180:0:{\radO} and {\eccO});
  
   \draw (P3) arc (180:360:{\radO} and {\eccO});
  \draw[dashed] (P3) arc (180:0:{\radO} and {\eccO});
  
  \draw (P2) arc (180:360:{\radO} and {\eccO});
  \draw[dashed] (P2) arc (180:0:{\radO} and {\eccO});
  
 \draw (P2) to[out=90,in=270] (P1);
 \draw ($(P3)+(2*\radO,0)$) to[out=90,in=270] ($(P1)+(2*\radO,0)$);
 \draw ($(P2)+(2*\radO,0)$) to[out=90,in=90] (P3); 
 
% Labels 
 
 \node at ($0.5*(P2)+(\radO,0)+0.5*(P3)+(0,0.21*\hghtO-.7*\dist)$) {$\OPQ_{120}$};

  
\end{tikzpicture}}+\quad\vcenterline{%auto-ignore
\begin{tikzpicture}
  % Sphere
  \coordinate (P1) at (0,0);
  \coordinate (P2) at (-0.5*\dif,\hght);
  \coordinate (P3) at (0.5*\dif,\hght);
  \coordinate (P4) at ($(P2)+(0,\dist)$);
  \coordinate (P5) at ($(P4)+(-\dif,0)$);
  \coordinate (P6) at ($(P5)+(0,-\dist)$);
  \coordinate (P7) at ($(P6)+(0,-\hght)$);
  \coordinate (P8) at ($(P3)+(0,\dist)$);
  \coordinate (P9) at ($(P8)+(0,\hght)$);

%Pair of pants
  
  \draw (P1) arc (180:360:{\rad} and {\ecc});
  \draw[dashed] (P1) arc (180:0:{\rad} and {\ecc});
  
   \draw (P3) arc (180:360:{\rad} and {\ecc});
  \draw (P3) arc (180:0:{\rad} and {\ecc});
  
  \draw (P2) arc (180:360:{\rad} and {\ecc});
  \draw (P2) arc (180:0:{\rad} and {\ecc});
  
 \draw (P2) to[out=270,in=90] (P1);
 \draw ($(P3)+(2*\rad,0)$) to[out=270,in=90] ($(P1)+(2*\rad,0)$);
 \draw ($(P2)+(2*\rad,0)$) to[out=270,in=270] (P3); 
 
% Maurer Cartan

 \draw (P4) arc (180:360:{\rad} and {\ecc});
 \draw[dashed] (P4) arc (180:0:{\rad} and {\ecc});
 
 \draw (P5) arc (180:360:{\rad} and {\ecc});
 \draw[dashed] (P5) arc (180:0:{\rad} and {\ecc});
 
 \draw ($(P5)+(2*\rad,0)$) to[out=90,in=90] (P4);
 
 \draw (P5) to[out=90,in=180] ($0.5*(P4)+(\rad,0)+0.5*(P5)+(0,\hght)$) to[out=0,in=90] ($(P4)+(2*\rad,0)$);

% Labels 
 
 \node at ($(P1)+(\rad,0.5*\hght)$) {$\OPQ_{210}$};
 \node at ($0.5*(P4)+(\rad,0)+0.5*(P5)+(0,0.5*\hght)$) {$\PMC_{20}$};

% Cylinders 

 \draw (P6) arc (180:360:{\rad} and {\ecc});
 \draw (P6) arc (180:0:{\rad} and {\ecc});  
 \draw (P7) arc (180:360:{\rad} and {\ecc});
 \draw[dashed] (P7) arc (180:0:{\rad} and {\ecc});
 \draw (P6) -- (P7);
 \draw ($(P6)+(2*\rad,0)$) -- ($(P7)+(2*\rad,0)$);
 
 
 \draw (P9) arc (180:360:{\rad} and {\ecc});
 \draw (P9) arc (180:0:{\rad} and {\ecc});  
 \draw (P8) arc (180:360:{\rad} and {\ecc});
 \draw[dashed] (P8) arc (180:0:{\rad} and {\ecc});
 \draw (P8) -- (P9);
 \draw ($(P8)+(2*\rad,0)$) -- ($(P9)+(2*\rad,0)$);
  
\end{tikzpicture}}, \\[1ex]
\OPQ^{\PMC}_{1lg} & =\quad \vcenterline{%auto-ignore
\begin{tikzpicture}
  % Sphere
  \coordinate (P1) at (0,0);
  \coordinate (P2) at (-0.5*\dif,\hght);
  \coordinate (P3) at (0.5*\dif,\hght);
  \coordinate (P4) at ($(P2)+(0,\dist)$);
  \coordinate (P5) at ($(P4)+(-\dif,0)$);
  \coordinate (P6) at ($(P5)+(0,-\dist)$);
  \coordinate (P7) at ($(P6)+(0,-\hght)$);
  \coordinate (P8) at ($(P3)+(0,\dist)$);
  \coordinate (P9) at ($(P8)+(0,\hght)$);
  \coordinate (P51) at ($(P5)+(-\dif,0)$);
  \coordinate (P52) at ($(P51)+(-\dif,0)$);
  \coordinate (P53) at ($(P52)+(-\dif,0)$);
  \coordinate (P62) at ($(P53)+(0,-\dist)$);
  \coordinate (P72) at ($(P62)+(0,-\hght)$);
  \coordinate (PG1) at ($(P51)+(1.6*\rad,0.5*\hght)$);
  \coordinate (PG2) at ($(P53)+(2*\rad,0.5*\hght)$);
 

%Pair of pants
  
  \draw (P1) arc (180:360:{\rad} and {\ecc});
  \draw[dashed] (P1) arc (180:0:{\rad} and {\ecc});
  
   \draw (P3) arc (180:360:{\rad} and {\ecc});
  \draw (P3) arc (180:0:{\rad} and {\ecc});
  
  \draw (P2) arc (180:360:{\rad} and {\ecc});
  \draw (P2) arc (180:0:{\rad} and {\ecc});
  
 \draw (P2) to[out=270,in=90] (P1);
 \draw ($(P3)+(2*\rad,0)$) to[out=270,in=90] ($(P1)+(2*\rad,0)$);
 \draw ($(P2)+(2*\rad,0)$) to[out=270,in=270] (P3); 
 
% Maurer Cartan

 \draw (P4) arc (180:360:{\rad} and {\ecc});
 \draw[dashed] (P4) arc (180:0:{\rad} and {\ecc});
 
 \draw (P5) arc (180:360:{\rad} and {\ecc});
 \draw[dashed] (P5) arc (180:0:{\rad} and {\ecc});
 
 \draw (P53) arc (180:360:{\rad} and {\ecc});
 \draw[dashed] (P53) arc (180:0:{\rad} and {\ecc});
 
 \draw ($(P51)+(2*\rad,0)$) to[out=90,in=90] (P5);
 \draw ($(P53)+(2*\rad,0)$) to[out=90,in=90] (P52);  
 \draw ($(P5)+(2*\rad,0)$) to[out=90,in=90] (P4);
 
 \coordinate (P5m) at ($0.8*(P4)+(\rad,0)+0.2*(P53)+(0,1*\hght)$);
 \coordinate (P5mm) at ($0.2*(P4)+(\rad,0)+0.8*(P53)+(0,1*\hght)$); 
 
 \draw (P53) to[out=90,in=180] (P5mm);
 \draw (P5m) to[out=0,in=90] ($(P4)+(2*\rad,0)$);
 \draw (P5mm) to[out=0,in=180] (P5m);
 
\tikzset{decorate sep/.style 2 args=
{decorate,decoration={shape backgrounds,shape=circle,shape size=#1,shape sep=#2}}} 
 
 \draw[decorate sep={0.3mm}{2mm},fill] ($0.5*(P62)+0.5*(P72) + (\dif,0)$) to ($0.5*(P6)+0.5*(P7)+(-\dif+2*\rad,0)$);

% Labels 
 
 \node at ($(P1)+(\rad,0.5*\hght)$) {$\OPQ_{210}$};
 \node at ($0.5*(P4)+(\rad,0)+0.5*(P5)+(0,0.5*\hght)$) {$\PMC_{lg}$};

% Cylinders 

  \draw (P62) arc (180:360:{\rad} and {\ecc});
 \draw (P62) arc (180:0:{\rad} and {\ecc});  
 \draw (P72) arc (180:360:{\rad} and {\ecc});
 \draw[dashed] (P72) arc (180:0:{\rad} and {\ecc});
 \draw (P62) -- (P72);
 \draw ($(P62)+(2*\rad,0)$) -- ($(P72)+(2*\rad,0)$);

 \draw (P6) arc (180:360:{\rad} and {\ecc});
 \draw (P6) arc (180:0:{\rad} and {\ecc});  
 \draw (P7) arc (180:360:{\rad} and {\ecc});
 \draw[dashed] (P7) arc (180:0:{\rad} and {\ecc});
 \draw (P6) -- (P7);
 \draw ($(P6)+(2*\rad,0)$) -- ($(P7)+(2*\rad,0)$);
 
 
 \draw (P9) arc (180:360:{\rad} and {\ecc});
 \draw (P9) arc (180:0:{\rad} and {\ecc});  
 \draw (P8) arc (180:360:{\rad} and {\ecc});
 \draw[dashed] (P8) arc (180:0:{\rad} and {\ecc});
 \draw (P8) -- (P9);
 \draw ($(P8)+(2*\rad,0)$) -- ($(P9)+(2*\rad,0)$);
 
% Genus

 \draw (PG1) to[out=-\genecc,in=180+\genecc] coordinate[pos=\gencanc] (PG11) coordinate[pos=1-\gencanc] (PG12) ($(PG1) + (2*\genrad,0)$) ;
 \draw (PG11) to[out=\genecc,in=180-\genecc] (PG12);
 
\draw (PG2) to[out=-\genecc,in=180+\genecc] coordinate[pos=\gencanc] (PG21) coordinate[pos=1-\gencanc] (PG22) ($(PG2) + (2*\genrad,0)$) ;
 \draw (PG21) to[out=\genecc,in=180-\genecc] (PG22);
 
 \draw[decorate sep={0.3mm}{2mm},fill] ($(PG2) + (3*\genrad,0)$) to ($(PG1)-(\genrad,0)$);

% 
% \draw (PG1) to[out=150,in=-60] ($(PG1) + (-\gencanc,\gencanc)$);
% \draw ($(PG1) + (2*\genrad,0)$) to[out=390,in=240] ($(PG1) + (2*\genrad,0) + (\gencanc,\gencanc)$);
% \draw (PG1) to[out=\genecc,in=180-\genecc] ($(PG1) + (2*\genrad,0)$);
 
\end{tikzpicture}}.
\end{align*}
\endgroup}
The $\IBLInfty$-relations satisfied by $(\OPQ_{klg}^\PMC)$ read for all $l\ge 1$, $g\ge 0$ as follows:
\begin{equation}\label{Eq:IBLInftydIBL}
\begin{aligned}
(3,1,0):\quad 0& = \OPQ_{210}^\PMC \circ_1 \OPQ_{210}^\PMC, \\[\jot]
(2,l,g):\quad 0&=\OPQ^\PMC_{1lg}\circ_1 \OPQ^\PMC_{210} + \OPQ^\PMC_{210}\circ_1\OPQ^\PMC_{1lg}, \\[\jot]
(1,l,g):\quad 0&= \begin{multlined}[t] \sum_{\substack{l_1, l_2 \ge 1 \\ g_1, g_2 \ge 0 \\ l_1 + l_2 = l+1 \\ g_1 + g_2 = g}} \OPQ^\PMC_{1l_1 g_1}\circ_1 \OPQ^\PMC_{1 l_2 g_2}+\OPQ^\PMC_{210} \circ_2 \OPQ^\PMC_{1, l+1, g-1}.\end{multlined}
\end{aligned}
\end{equation}
We call the relations for $(k,l,g) = (1,1,0)$, $(2,1,0)$, $(1,2,0)$, $(3,1,0)$, $(1,3,0)$, $(2,2,0)$, $(1,1,1)$ \emph{basic relations} because they contain all compositions of basic operations. In the order above, they read:
\[\begin{aligned} 
 0 & =\OPQ^\PMC_{110} \circ_1 \OPQ^\PMC_{110}, && \\
0 &=\OPQ^\PMC_{110}\circ_1\OPQ^\PMC_{210} + \OPQ^\PMC_{210}\circ_1 \OPQ^\PMC_{110}, && \\
0 &= \OPQ^\PMC_{110}\circ_1 \OPQ^\PMC_{120} + \OPQ^\PMC_{120}\circ_1 \OPQ^\PMC_{110}, && \\
0 &= \OPQ^\PMC_{210} \circ_1 \OPQ^\PMC_{210}, && \leftarrow\text{Jacobi identity} \\
0 &=\OPQ^\PMC_{120} \circ_1 \OPQ^\PMC_{120} + \OPQ^\PMC_{110}\circ_1 \OPQ^\PMC_{130} + \OPQ^\PMC_{130}\circ_1 \OPQ^\PMC_{110}, && \leftarrow\text{co-Jacobi id. up to htpy.} \\
0 & = \OPQ^\PMC_{120}\circ_1 \OPQ^\PMC_{210} + \OPQ^\PMC_{210}\circ_1 \OPQ^\PMC_{120}, && \leftarrow\text{Drinfeld identity} \\
0 & = \OPQ^\PMC_{210}\circ_2 \OPQ^\PMC_{120} + \OPQ^\PMC_{111}\circ_1 \OPQ^\PMC_{110} + \OPQ^\PMC_{110}\circ_1 \OPQ^\PMC_{111}. && \leftarrow\text{Involutivity up to htpy.}
\end{aligned}\]
The last four equations can be visualized as
{ \begingroup \allowdisplaybreaks
\def\dist{0.25} %distance between two surfaces
  \def\rad{0.4} % radius of bdd
  \def\ecc{0.1} % eccentricity of bdd
  \def\hght{1} % height of surfaces
  \def\dif{1.1} % distance of two circles
  \def\difbig{1.5*\dif} % distance of two circles
  \def\radO{\rad} % radius of bdd
  \def\eccO{\ecc} % eccentricity of bdd
  \def\hghtO{2*\hght+\dist} % height of surfaces
  \def\difO{\dif} % distance of two circles
  \def\gencanc{0.05} % legth of extra line in genus
  \def\genecc{20} % eccentricity of genus
  \def\genrad{0.3} % radius of genus
\begin{align*}
0 & =\quad\vcenterline{%auto-ignore
\begin{tikzpicture}
  % Sphere
  
  \coordinate (P1) at (0,0);
  \coordinate (P2) at (-0.5*\difbig,\hght);
  \coordinate (P3) at ($(0.5*\difbig,\hght)$);

  \coordinate (P4) at ($(P2)+(0,\dist)$);
  \coordinate (P5) at ($(P4)+(-0.5*\dif,\hght)$);
  \coordinate (P6) at ($(P4)+(0.5*\dif,\hght)$);


  \coordinate (P7) at ($(P6)+(0,-\hght)$);
  \coordinate (P8) at ($(P3)+(0,\dist)$);
  \coordinate (P9) at ($(P8)+(0,\hght)$);

%Pair of pants
  
  \draw (P1) arc (180:360:{\rad} and {\ecc});
  \draw[dashed] (P1) arc (180:0:{\rad} and {\ecc});
  
   \draw (P3) arc (180:360:{\rad} and {\ecc});
  \draw (P3) arc (180:0:{\rad} and {\ecc});
  
  \draw (P2) arc (180:360:{\rad} and {\ecc});
  \draw (P2) arc (180:0:{\rad} and {\ecc});
  
 \draw (P2) to[out=270,in=90] (P1);
 \draw ($(P3)+(2*\rad,0)$) to[out=270,in=90] ($(P1)+(2*\rad,0)$);
 \draw ($(P2)+(2*\rad,0)$) to[out=270,in=270] (P3); 
 
% Maurer Cartan


 
  \draw (P4) arc (180:360:{\rad} and {\ecc});
  \draw[dashed] (P4) arc (180:0:{\rad} and {\ecc});
  
   \draw (P6) arc (180:360:{\rad} and {\ecc});
  \draw (P6) arc (180:0:{\rad} and {\ecc});
  
  \draw (P5) arc (180:360:{\rad} and {\ecc});
  \draw (P5) arc (180:0:{\rad} and {\ecc});
  
 \draw (P5) to[out=270,in=90] (P4);
 \draw ($(P6)+(2*\rad,0)$) to[out=270,in=90] ($(P4)+(2*\rad,0)$);
 \draw ($(P5)+(2*\rad,0)$) to[out=270,in=270] (P6); 

% \draw (P4) arc (180:360:{\rad} and {\ecc});
% \draw[dashed] (P4) arc (180:0:{\rad} and {\ecc});
% 
% \draw (P5) arc (180:360:{\rad} and {\ecc});
% \draw[dashed] (P5) arc (180:0:{\rad} and {\ecc});
% 
% \draw ($(P5)+(2*\rad,0)$) to[out=90,in=90] (P4);
% 
% \draw (P5) to[out=90,in=180] ($0.5*(P4)+(\rad,0)+0.5*(P5)+(0,\hght)$) to[out=0,in=90] ($(P4)+(2*\rad,0)$);

% Labels 
 
 \node at ($(P1)+(\rad,0.42*\hght)$) {$\OPQ_{210}^\PMC$};
 \node at ($(P4)+(\rad,0.5*\hght)$) {$\OPQ_{210}^\PMC$};
% \node at ($0.5*(P4)+(\rad,0)+0.5*(P5)+(0,0.5*\hght)$) {$\PMC_{20}$};

% Cylinders 

% \draw (P6) arc (180:360:{\rad} and {\ecc});
% \draw (P6) arc (180:0:{\rad} and {\ecc});  
% \draw (P7) arc (180:360:{\rad} and {\ecc});
% \draw[dashed] (P7) arc (180:0:{\rad} and {\ecc});
% \draw (P6) -- (P7);
% \draw ($(P6)+(2*\rad,0)$) -- ($(P7)+(2*\rad,0)$);
% 
 
 \draw (P9) arc (180:360:{\rad} and {\ecc});
 \draw (P9) arc (180:0:{\rad} and {\ecc});  
 \draw (P8) arc (180:360:{\rad} and {\ecc});
 \draw[dashed] (P8) arc (180:0:{\rad} and {\ecc});
 \draw (P8) -- (P9);
 \draw ($(P8)+(2*\rad,0)$) -- ($(P9)+(2*\rad,0)$);
  
\end{tikzpicture}}, \\[1ex] 
0&=\quad \vcenterline{%auto-ignore
\begin{tikzpicture}
  \coordinate (P1) at (0,\hght);
  \coordinate (P2) at (-0.5*\difO,0);
  \coordinate (P3) at (0.5*\difO,0);
  \coordinate (P4) at ($(P1) + (-0.5*\difbig,\dist+\hght)$);
  \coordinate (P5) at ($(P4)+(-0.5*\difbig,-\hght)$);
  \coordinate (P6) at ($(P4) + (0.5*\difbig,-\hght)$);
%Pair of pants 
 
%  \draw (P5) arc (180:360:{\radO} and {\eccO});
%  \draw (P5) arc (180:0:{\radO} and {\eccO});
%  \draw (P4) arc (180:360:{\radO} and {\eccO});
%  \draw[dashed] (P4) arc (180:0:{\radO} and {\eccO});
%  \draw (P4) -- (P5);
%  \draw ($(P4)+(2*\rad0,0)$) -- ($(P5)+(2*\rad0,0)$);
  
  \draw (P1) arc (180:360:{\radO} and {\eccO});
  \draw (P1) arc (180:0:{\radO} and {\eccO});
  
   \draw (P3) arc (180:360:{\radO} and {\eccO});
  \draw[dashed] (P3) arc (180:0:{\radO} and {\eccO});
  
  \draw (P2) arc (180:360:{\radO} and {\eccO});
  \draw[dashed] (P2) arc (180:0:{\radO} and {\eccO});
  
 \draw (P2) to[out=90,in=270] (P1);
 \draw ($(P3)+(2*\radO,0)$) to[out=90,in=270] ($(P1)+(2*\radO,0)$);
 \draw ($(P2)+(2*\radO,0)$) to[out=90,in=90] (P3); 
 
 
 
  \draw (P4) arc (180:360:{\radO} and {\eccO});
  \draw (P4) arc (180:0:{\radO} and {\eccO});
  
   \draw (P6) arc (180:360:{\radO} and {\eccO});
  \draw[dashed] (P6) arc (180:0:{\radO} and {\eccO});
  
  \draw (P5) arc (180:360:{\radO} and {\eccO});
  \draw[dashed] (P5) arc (180:0:{\radO} and {\eccO});
  
 \draw (P5) to[out=90,in=270] (P4);
 \draw ($(P6)+(2*\radO,0)$) to[out=90,in=270] ($(P4)+(2*\radO,0)$);
 \draw ($(P5)+(2*\radO,0)$) to[out=90,in=90] (P6); 
 
% Labels 
 
 \node at ($0.5*(P2)+(\radO,0)+0.5*(P3)+(0,0.21*\hghtO-.7*\dist)$) {$\OPQ_{120}^\PMC$};
 
  \node at ($0.5*(P5)+(\radO,0)+0.5*(P6)+(0,0.21*\hghtO-.7*\dist)$) {$\OPQ_{120}^\PMC$};
  

\coordinate (P7) at ($(P5) + (0,-\dist)$);
\coordinate (P8) at ($(P7) + (0,-\hght)$);
\draw (P7)--(P8);
\draw ($(P7)+(2*\rad,0)$)--($(P8)+(2*\rad,0)$);

  \draw (P7) arc (180:360:{\radO} and {\eccO});
  \draw (P7) arc (180:0:{\radO} and {\eccO});
  
   \draw (P8) arc (180:360:{\radO} and {\eccO});
  \draw[dashed] (P8) arc (180:0:{\radO} and {\eccO});


\end{tikzpicture}}+
\vcenterline{%auto-ignore
\begin{tikzpicture}
  % Sphere
  \coordinate (P1) at (0,0);
  \coordinate (P2) at ($(P1)+(\dif,0)$);
  \coordinate (P3) at ($(P1)+(2*\dif,0)$);
  \coordinate (P4) at ($(P1) + (\dif,\hght)$);

%Pair of pants
  
  \draw (P1) arc (180:360:{\rad} and {\ecc});
  \draw[dashed] (P1) arc (180:0:{\rad} and {\ecc});
  
   \draw (P2) arc (180:360:{\rad} and {\ecc});
  \draw[dashed] (P2) arc (180:0:{\rad} and {\ecc});
  
  \draw (P3) arc (180:360:{\rad} and {\ecc});
  \draw[dashed] (P3) arc (180:0:{\rad} and {\ecc});
  
  \draw (P4) arc (180:360:{\rad} and {\ecc});
  \draw (P4) arc (180:0:{\rad} and {\ecc});
  
 \draw ($(P1) + (2*\rad,0)$) to[out=90,in=90] (P2);
 \draw ($(P2) + (2*\rad,0)$) to[out=90,in=90] (P3); 
 \draw (P1) to[out=90,in=-90] (P4);
 \draw ($(P3)+(2*\rad,0)$) to[out=90,in=-90] ($(P4)+(2*\rad,0)$);

 \node at ($(P1)+(\dif + \rad,0.5*\hght)$) {$\OPQ_{130}^\PMC$};


\coordinate (P5) at ($(P4)+(0,\dist)$);
\coordinate (P6) at ($(P5)+(0,\hght)$);
\draw (P5) arc (180:360:{\rad} and {\ecc});
\draw[dashed] (P5) arc (180:0:{\rad} and {\ecc});
\draw (P6) arc (180:360:{\rad} and {\ecc});
\draw (P6) arc (180:0:{\rad} and {\ecc}); 
\draw (P5) -- (P6);
\draw ($(P5) + (2*\rad,0)$) -- ($(P6) + (2*\rad,0)$);

 \node at ($(P5)+(\rad,0.5*\hght)$) {$\OPQ_{110}^\PMC$};
  
\end{tikzpicture}}
+\quad\vcenterline{%auto-ignore
\begin{tikzpicture}
  % Sphere
  \coordinate (P1) at ($(0,\hght+\dist)$);
  \coordinate (P2) at ($(P1)+(\dif,0)$);
  \coordinate (P3) at ($(P1)+(2*\dif,0)$);
  \coordinate (P4) at ($(P1) + (\dif,\hght)$);

%Pair of pants
  
  \draw (P1) arc (180:360:{\rad} and {\ecc});
  \draw[dashed] (P1) arc (180:0:{\rad} and {\ecc});
  
   \draw (P2) arc (180:360:{\rad} and {\ecc});
  \draw[dashed] (P2) arc (180:0:{\rad} and {\ecc});
  
  \draw (P3) arc (180:360:{\rad} and {\ecc});
  \draw[dashed] (P3) arc (180:0:{\rad} and {\ecc});
  
  \draw (P4) arc (180:360:{\rad} and {\ecc});
  \draw (P4) arc (180:0:{\rad} and {\ecc});
  
 \draw ($(P1) + (2*\rad,0)$) to[out=90,in=90] (P2);
 \draw ($(P2) + (2*\rad,0)$) to[out=90,in=90] (P3); 
 \draw (P1) to[out=90,in=-90] (P4);
 \draw ($(P3)+(2*\rad,0)$) to[out=90,in=-90] ($(P4)+(2*\rad,0)$);

 \node at ($(P1)+(\dif + \rad,0.5*\hght)$) {$\OPQ_{130}^\PMC$};


\coordinate (P5) at (\dif,0);
\coordinate (P6) at ($(P5)+(0,\hght)$);
\draw (P5) arc (180:360:{\rad} and {\ecc});
\draw[dashed] (P5) arc (180:0:{\rad} and {\ecc});
\draw (P6) arc (180:360:{\rad} and {\ecc});
\draw (P6) arc (180:0:{\rad} and {\ecc}); 
\draw (P5) -- (P6);
\draw ($(P5) + (2*\rad,0)$) -- ($(P6) + (2*\rad,0)$);

 
 
\coordinate (P6) at ($(P1) + (0,-\dist)$);
\coordinate (P7) at ($(P6) + (0,-\hght)$);
\draw (P6) arc (180:360:{\rad} and {\ecc});
\draw (P6) arc (180:0:{\rad} and {\ecc});
\draw (P7) arc (180:360:{\rad} and {\ecc});
\draw[dashed] (P7) arc (180:0:{\rad} and {\ecc}); 
\draw (P6) -- (P7);
\draw ($(P6) + (2*\rad,0)$) -- ($(P7) + (2*\rad,0)$);


\coordinate (P8) at ($(P3) + (0,-\dist)$);
\coordinate (P9) at ($(P8) + (0,-\hght)$);
\draw (P8) arc (180:360:{\rad} and {\ecc});
\draw (P8) arc (180:0:{\rad} and {\ecc});
\draw (P9) arc (180:360:{\rad} and {\ecc});
\draw[dashed] (P9) arc (180:0:{\rad} and {\ecc}); 
\draw (P8) -- (P9);
\draw ($(P8) + (2*\rad,0)$) -- ($(P9) + (2*\rad,0)$);

\node at ($(P7)+(\rad,0.5*\hght)$) {$\OPQ_{110}^\PMC$};

\end{tikzpicture}}, \\[1ex] 
0&=\quad \vcenterline{%auto-ignore
\begin{tikzpicture}
  
  \coordinate (P1) at (0,0);
  \coordinate (P2) at ($(P1) + (-0.5*\dif,\hght)$);  
  \coordinate (P3) at ($(P1) + (0.5*\dif,\hght)$);
  
  \coordinate (P4) at (0,-\dist);
  \coordinate (P5) at ($(P4) + (-0.5*\dif,-\hght)$);  
  \coordinate (P6) at ($(P4) + (0.5*\dif,-\hght)$);
  
  \draw (P1) to[out=90, in=-90] (P2);
  \draw ($(P1)+(2*\rad,0)$) to[out=90, in=-90] ($(P3)+(2*\rad,0)$);  
  \draw ($(P2)+(2*\rad,0)$) to[out=-90,in=-90] (P3);
 
   \draw (P4) to[out=-90, in=90] (P5);
  \draw ($(P4)+(2*\rad,0)$) to[out=-90, in=90] ($(P6)+(2*\rad,0)$);  
  \draw ($(P5)+(2*\rad,0)$) to[out=90,in=90] (P6); 
  
  \draw (P1) arc (180:360:{\rad} and {\ecc});
  \draw[dashed] (P1) arc (180:0:{\rad} and {\ecc});
  \draw (P5) arc (180:360:{\rad} and {\ecc});
  \draw[dashed] (P5) arc (180:0:{\rad} and {\ecc});
  \draw (P6) arc (180:360:{\rad} and {\ecc});
  \draw[dashed] (P6) arc (180:0:{\rad} and {\ecc});
 
  \draw (P2) arc (180:360:{\rad} and {\ecc});
  \draw (P2) arc (180:0:{\rad} and {\ecc});
  \draw (P3) arc (180:360:{\rad} and {\ecc});
  \draw (P3) arc (180:0:{\rad} and {\ecc});
  \draw (P4) arc (180:360:{\rad} and {\ecc});
  \draw (P4) arc (180:0:{\rad} and {\ecc}); 
 
 \node at ($(P1)+(\rad,0.5*\hght)$) {$\OPQ_{210}^\PMC$};
 \node at ($0.5*(P5)+(\rad,0)+0.5*(P6)+(0,0.5*\hght)$) {$\OPQ_{120}^\PMC$};

\end{tikzpicture}}+\quad\vcenterline{%auto-ignore
\begin{tikzpicture}

 \coordinate (P1) at (0,0);
 \coordinate (P2) at ($(P1)+(0.5*\dif,\hght)$);
 \coordinate (P3) at ($(P1) + (\dif,0)$);
 \coordinate (P4) at ($(P3) + (0,-\dist)$);
 \coordinate (P5) at ($(P4) + (0.5*\dif,-\hght)$);
 \coordinate (P6) at ($(P4) + (\dif,0)$);
 
 \draw (P1) to[out=90,in=-90] (P2);
 \draw ($(P1)+(2*\rad,0)$) to[out=90,in=90] (P3);
 \draw ($(P2)+(2*\rad,0)$) to[out=-90,in=90] ($(P3)+(2*\rad,0)$); 
 \draw (P4) to[out=-90,in=90] (P5);
 \draw ($(P4)+(2*\rad,0)$) to[out=-90,in=-90] (P6);
 \draw ($(P5)+(2*\rad,0)$) to[out=90,in=-90] ($(P6)+(2*\rad,0)$);
 
 
  \draw (P1) arc (180:360:{\rad} and {\ecc});
  \draw[dashed] (P1) arc (180:0:{\rad} and {\ecc});
  \draw (P3) arc (180:360:{\rad} and {\ecc});
  \draw[dashed] (P3) arc (180:0:{\rad} and {\ecc});
  \draw (P5) arc (180:360:{\rad} and {\ecc});
  \draw[dashed] (P5) arc (180:0:{\rad} and {\ecc});
 
 
  \draw (P2) arc (180:360:{\rad} and {\ecc});
  \draw (P2) arc (180:0:{\rad} and {\ecc});
  \draw (P4) arc (180:360:{\rad} and {\ecc});
  \draw (P4) arc (180:0:{\rad} and {\ecc});
  \draw (P6) arc (180:360:{\rad} and {\ecc});
  \draw (P6) arc (180:0:{\rad} and {\ecc});
  
    \node at ($(P2)+(\rad,-0.5*\hght)$) {$\OPQ_{120}^\PMC$};
 \node at ($(P5)+(\rad,0.5*\hght)$) {$\OPQ_{210}^\PMC$};
 
 \coordinate (P7) at ($(P1)+(0,-\dist)$);
 \coordinate (P8) at ($(P7)+(0,-\hght)$);
 \coordinate (P9) at ($(P6)+(0,\dist)$);
 \coordinate (P10) at ($(P9)+(0,\hght)$);
 
 \draw (P7)--(P8);
 \draw ($(P7)+(2*\rad,0)$)--($(P8)+(2*\rad,0)$);
 \draw (P9)--(P10);
 \draw ($(P9)+(2*\rad,0)$)--($(P10)+(2*\rad,0)$);
 
 
\draw (P8) arc (180:360:{\rad} and {\ecc});
\draw[dashed] (P8) arc (180:0:{\rad} and {\ecc});
\draw (P9) arc (180:360:{\rad} and {\ecc});
\draw[dashed] (P9) arc (180:0:{\rad} and {\ecc});
\draw (P7) arc (180:360:{\rad} and {\ecc});
\draw (P7) arc (180:0:{\rad} and {\ecc});
\draw (P10) arc (180:360:{\rad} and {\ecc});
\draw (P10) arc (180:0:{\rad} and {\ecc}); 
 
\end{tikzpicture}},\\[1ex]
0& =\quad \vcenterline{%auto-ignore
\begin{tikzpicture}

\coordinate (P1) at (0,0);
\coordinate (P2) at (-0.5*\dif,-\hght);
\coordinate (P3) at (0.5*\dif,-\hght);

\coordinate (P4) at ($(P2)+(0,-\dist)$);
\coordinate (P5) at ($(P3)+(0,-\dist)$);
\coordinate (P6) at ($(P4)+(0.5*\dif,-\hght)$);

\draw (P1) to[out=-90,in=90] (P2);
\draw ($(P1)+(2*\rad,0)$) to[out=-90,in=90] ($(P3)+(2*\rad,0)$);
\draw ($(P2)+(2*\rad,0)$) to[out=90,in=90] (P3);

\draw ($(P4)+(2*\rad,0)$) to[out=-90,in=-90] (P5);
\draw (P4) to[out=-90,in=90] (P6);
\draw ($(P5)+(2*\rad,0)$) to[out=-90,in=90] ($(P6)+(2*\rad,0)$);

 
\draw (P1) arc (180:360:{\rad} and {\ecc});
\draw (P1) arc (180:0:{\rad} and {\ecc});
\draw (P2) arc (180:360:{\rad} and {\ecc});
\draw[dashed] (P2) arc (180:0:{\rad} and {\ecc});
\draw (P3) arc (180:360:{\rad} and {\ecc});
\draw[dashed] (P3) arc (180:0:{\rad} and {\ecc});
\draw (P4) arc (180:360:{\rad} and {\ecc});
\draw (P4) arc (180:0:{\rad} and {\ecc}); 
\draw (P5) arc (180:360:{\rad} and {\ecc});
\draw (P5) arc (180:0:{\rad} and {\ecc}); 
\draw (P6) arc (180:360:{\rad} and {\ecc});
\draw[dashed] (P6) arc (180:0:{\rad} and {\ecc}); 

\node at ($(P1)+(\rad,-0.5*\hght)$) {$\OPQ_{120}^\PMC$};
\node at ($(P6)+(\rad,0.5*\hght)$) {$\OPQ_{210}^\PMC$};

\end{tikzpicture}}\quad+\quad\vcenterline{%auto-ignore
\begin{tikzpicture}
\coordinate (P1) at (0,0);
\coordinate (P2) at ($(P1)+(0,-\hght)$);
\coordinate (P3) at ($(P2)+(0,-\dist)$);
\coordinate (P4) at ($(P3)+(0,-\hght)$);

\draw (P2) arc (180:360:{\rad} and {\ecc});
\draw[dashed] (P2) arc (180:0:{\rad} and {\ecc});
\draw (P4) arc (180:360:{\rad} and {\ecc});
\draw[dashed] (P4) arc (180:0:{\rad} and {\ecc});
\draw (P1) arc (180:360:{\rad} and {\ecc});
\draw (P1) arc (180:0:{\rad} and {\ecc});
\draw (P3) arc (180:360:{\rad} and {\ecc});
\draw (P3) arc (180:0:{\rad} and {\ecc}); 

\draw (P1)--(P2);
\draw (P3)--(P4);

\draw ($(P1)+(2*\rad,0)$)--($(P2)+(2*\rad,0)$);
\draw ($(P3)+(2*\rad,0)$)--($(P4)+(2*\rad,0)$);

  
\node at ($(P1)+(\rad,-0.5*\hght)$) {$\OPQ_{110}^\PMC$};
\node at ($(P3)+(\rad,-0.37*\hght)$) {$\OPQ_{111}^\PMC$};


\coordinate (PG1) at ($(P3)+(\rad-\genrad,-0.72*\hght)$);
 \draw (PG1) to[out=-\genecc,in=180+\genecc] coordinate[pos=\gencanc] (PG11) coordinate[pos=1-\gencanc] (PG12) ($(PG1) + (2*\genrad,0)$) ;
 \draw (PG11) to[out=\genecc,in=180-\genecc] (PG12);
\end{tikzpicture}}\quad+\quad\vcenterline{%auto-ignore
\begin{tikzpicture}
\coordinate (P3) at (0,0);
\coordinate (P2) at ($(P3)+(0,-\hght)$);
\coordinate (P1) at ($(P2)+(0,-\dist)$);
\coordinate (P4) at ($(P1)+(0,-\hght)$);

\draw (P2) arc (180:360:{\rad} and {\ecc});
\draw[dashed] (P2) arc (180:0:{\rad} and {\ecc});
\draw (P4) arc (180:360:{\rad} and {\ecc});
\draw[dashed] (P4) arc (180:0:{\rad} and {\ecc});
\draw (P1) arc (180:360:{\rad} and {\ecc});
\draw (P1) arc (180:0:{\rad} and {\ecc});
\draw (P3) arc (180:360:{\rad} and {\ecc});
\draw (P3) arc (180:0:{\rad} and {\ecc}); 

\draw (P1)--(P2);
\draw (P3)--(P4);

\draw ($(P1)+(2*\rad,0)$)--($(P2)+(2*\rad,0)$);
\draw ($(P3)+(2*\rad,0)$)--($(P4)+(2*\rad,0)$);

  
\node at ($(P1)+(\rad,-0.5*\hght)$) {$\OPQ_{110}^\PMC$};
\node at ($(P3)+(\rad,-0.37*\hght)$) {$\OPQ_{111}^\PMC$};


\coordinate (PG1) at ($(P3)+(\rad-\genrad,-0.72*\hght)$);
 \draw (PG1) to[out=-\genecc,in=180+\genecc] coordinate[pos=\gencanc] (PG11) coordinate[pos=1-\gencanc] (PG12) ($(PG1) + (2*\genrad,0)$) ;
 \draw (PG11) to[out=\genecc,in=180-\genecc] (PG12);
\end{tikzpicture}}.
\end{align*}
\endgroup}
\end{Proposition}
\begin{proof}
The proof is clear by specializing \eqref{Eq:IBLInfRel}, \eqref{Eq:MaurerCartanEquation} and \eqref{Eq:TwistedOperations}.
\end{proof}

\begin{Remark}[Higher operations]\phantomsection\label{Rem:Higher}
\begin{RemarkList}
\item We see from Proposition~\ref{Prop:dIBL} that if $\OPQ_{120}^\PMC \circ_1 \OPQ_{120}^\PMC = 0$ and $\OPQ_{210}^\PMC \circ_{2} \OPQ_{120}^\PMC = 0$, then $[\Bdd^\PMC, \OPQ_{130}^\PMC] = 0$ and $[\Bdd^\PMC, \OPQ_{111}^\PMC] = 0$, respectively, and hence the operations $\OPQ_{130}^\PMC: \hat{\Ext}_1\HIBL^\PMC\rightarrow \hat{\Ext}_3\HIBL^\PMC$ and $\OPQ_{111}^\PMC: \hat{\Ext}_1\HIBL^\PMC\rightarrow \hat{\Ext}_1\HIBL^\PMC$ are well-defined (provided that the assumption of Definition \ref{Def:HomIBL} holds). Likewise, the higher operation~$\OPQ_{1lg}^\PMC$ defines a map $\hat{\Ext}_1\HIBL^\PMC \rightarrow \hat{\Ext}_l\HIBL^\PMC$, provided that the following equation holds:
\[ \OPQ^\PMC_{210}\circ_2 \OPQ_{1,l+1,g-1}^\PMC + \sum_{\substack{l_1, l_2 \ge 1 \\ g_1, g_2 \ge 0 \\ l_1 + l_2 = l+1 \\ g_1 + g_2 = g \\ (l_i,g_i)\neq (1,0)}} \OPQ^\PMC_{1l_1 g_1}\circ_1 \OPQ^\PMC_{1 l_2 g_2} = 0. \]
This expression is just the left-over after subtracting the commutator $[\OPQ_{1lg}^\PMC,\OPQ_{110}^\PMC] = \OPQ_{110}^\PMC \circ_1 \OPQ_{1lg}^\PMC + \OPQ_{1lg}^\PMC \circ_1 \OPQ_{110}^\PMC$ from \eqref{Eq:IBLInftydIBL}.
%Later, when the twist realizes a homotopy transfered $\IBLInfty$-algebra. Note that this is a similar situation as with higher $\AInfty$-Massey product. 
\item In the genus-$0$ case, i.e., $\OPQ_{1lg}^\PMC = 0$ whenever $g\ge 1$, relations \eqref{Eq:IBLInftydIBL} reduce to
\begin{align*}
 0 & = \OPQ_{210}\circ_1\OPQ_{210}, \\
 0 & = \OPQ_{1lg}^\PMC\circ_1\OPQ_{210} + \OPQ_{210}\circ_1\OPQ_{1lg}^\PMC, \\
 0 & = \sum_{\substack{l_1,l_2\ge 1\\ g_1, g_2 \ge 0 \\ l_1 + l_2 = l + 1 \\ g_1 + g_2 = g}} \OPQ_{1l_1 g_1}^\PMC \circ_1 \OPQ_{1l_2 g_2}^\PMC.
\end{align*}
The first relation is the Jacobi identity for $\OPQ_{210}$, the second relation is a generalization of the Drinfeld identity to higher coproducts $\OPQ_{1l0}^\PMC$, and the third relations are \emph{$\CoLInfty$-relations} for $\OPQ_{1lg}^\PMC$.
Therefore, allowing $g\ge 0$, we see that the twisted $\dIBL$-algebra $\OPQ_{110}^\PMC$, $\OPQ_{210}$, $(\OPQ_{1lg}^\PMC)$ is, in fact, a \emph{quantum $\CoLInfty$-algebra.}\qedhere
\end{RemarkList}
\end{Remark}

%It corresponds to complete gluings of a surface of signature $(k_1, l_1, g_1)$ at its $h$ outgoing ends, plus an appropriate number of trivial cylinders, to $h$ ingoing ends of a surface of signature $(k_2,l_2,g_2)$, plus an appropriate number of trivial cylinders. The signature consists of the number of incoming ends, outgoing ends and genus.
%
%
%It corresponds to complete gluings of $r$ connected surfaces of signatures $(0,l_i,g_i)$, plus $k$ trivial cylinders, at their outgoing ends to the ingoing ends of a connected surface of signature $(k',l',g')$, plus an appropriate number of trivial cylinders, to obtain a connected surface of signature $(k,l,g)$.
%
%It corresponds to complete gluings of $r$ connected surfaces of signatures $(0,l_i,g_i)$ at their $h_i$ outgoing ends to the ingoing ends of a connected surface of signature $(k',l',g')$, plus an appropriate number of trivial cylinders, to obtain a connected surface of signature $(0,l,g)$.

\end{document}
